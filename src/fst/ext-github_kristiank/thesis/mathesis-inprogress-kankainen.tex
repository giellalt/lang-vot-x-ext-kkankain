\documentclass[12pt,a4paper]{article}

\usepackage[top=4cm, bottom=3cm, left=4cm, right=2.5cm]{geometry}
\usepackage{setspace}

% litsentsi pdf-i lisamiseks
\usepackage{pdfpages}

\usepackage{float}
\usepackage{placeins} % defines \FloatBarrier

\usepackage{amsmath} % align ja gather

\usepackage{enumerate}
\usepackage{enumitem}

\usepackage{booktabs}

% polyglossia
\usepackage{polyglossia}
\usepackage{fontspec}
\usepackage{xunicode}
\usepackage{xltxtra}
\usepackage{url}
\usepackage{expex}

% \texttildelow to get a nice vertically centered tilde
\usepackage{textcomp}

% \mathfraktur jaoks
\usepackage{amsfonts}

% Use a Free/Libre font with Finnish–Hungarian-Cyrillic-UPA coverage
\setmainfont[Mapping=tex-text]{Linux Libertine O}
% set languages to use
\setmainlanguage{estonian}
\setotherlanguages{english}

% tsitaatide juttumärkide jaoks
\usepackage{csquotes}

% Polyglossias miskipärast ei tööta poolitamine
% \usepackage{polyglossia}
% \setdefaultlanguage{german}

% @todo: vormistada kirjanduse loetelu õigesti
% @done: vormistada tekstis viited õigesti (nt lk kooloni järel jne)
\usepackage[backend=biber,
style=authoryear-ibid-ut,
labeldateparts,
dashed=false,  % don't dash out repeated author names
isbn=false,    % don't print isbn
%doi=false,     % don't print doi
%url=false,     % don't print url
%useeditor=false,
hyperref=true,
date=year      % only print the year, not full dates
]{biblatex}

\addbibresource{bibliography-mathesis.bib}

% muuda tsiteerimisi, et oleks aasta ja leheküljenumbri vahel ainult koolon
% õigem oleks vist siiski teha wrapper-funktsioon nimega \citeUT
\DeclareDelimFormat{postnotedelim}{}
\DefineBibliographyStrings{estonian}{%
  page = {:~}
}

\AtEveryBibitem{%
  % remove doi and url if not an online resource
  \ifentrytype{online}
  {}
  {
    \clearfield{doi}
    \clearfield{url}
    \clearfield{urldate} % why this doesn't work here?
  }
  
  % do this for all entry types
  % remove pagetotal
  \clearfield{pagetotal}
  % remove page if it is not a range
  \iffieldnum{pages}
  {\clearfield{pages}}
  {}
  % remove notes
  \clearfield{note}
  % clear date when retrieved url
  \clearfield{urldate}
}


\AtBeginBibliography{%
  \renewcommand*{\bibfont}{\small}
%  \renewcommand*{\mkbibnamefirst}[1]{\textbf{#1}}
%  \renewcommand*{\mkbibnamelast}[1]{\textbf{#1}}
%  \renewcommand*{\mkbibnameprefix}[1]{\textbf{#1}}
%  \renewcommand*{\mkbibnameaffix}[1]{\textbf{#1}}
%  \renewcommand{\finentrypunct}{} % remove final punctuation
%  \DeclareFieldFormat{year}{\textbf{#1}}
}


% Agnese kommentaarid
% poolituse ette ja taha vähemalt 3 tähte
% tabelite reavahetustele luhvti juurde
% ekraanikuvale kas raam või hall läbipaistev kast peale
% enumerated teha 1) 2) mitte 1. 2.
% indent pikemaks, 1.5 korda see mis praegu
% pealkirjadele mitte 1.5 linespacing ning vähem laiust kui sisutekstil





\usepackage[hidelinks]{hyperref}
\urlstyle{}

% koodiplokkide esitamiseks
\usepackage{minted}

% Sõnastik
\usepackage[nopostdot,xindy]{glossaries}
\makeglossaries

% tabelite pööramiseks
\usepackage{rotating}

% float ei uju kõrgemale
\usepackage{flafter}

% pealdised tabularidele
\usepackage{caption}

% rohkem õhku (padding) tabularidele
%\usepackage{verbatimbox}
\usepackage{scrextend}

% omad mallid
\newcommand{\vadja}[1]{\textit{#1}}
\newcommand{\msd}[1]{\textsc{#1}}


% sisukord
% Bakalaureusetöös on soovitatav piirduda kolmeastmelise hierarhiaga, magistritöös neljaastmelisega
\setcounter{tocdepth}{4}


% kasutatud mõisted ja lühendid

%% tüüpsõna on mingil põhjusel valitud sõna, millega tähistatakse üht sõnatüüpi (kusjuures tüüpsõna peab kuuluma samasse sõnatüüpi)
%% sõnatüüp on ühtviisi käituvate sõnade hulk (vt Viks)

% TODO lisada lähtekeelne termin, nagu (concatenation)
\newglossaryentry{muuttüüp}{
  name=Muuttüüp,
  description={on morfoloogilise klassifikatsiooni elementaar\-üksus. See on sõnaklass, mis erineb teistest sõnaklassidest mingite erijoonte poolest sõnade muutmisel.}
}
\newglossaryentry{tüüpsõnamall}{
  name=Tüüpsõnamall,
  description={on ekstraktmorfoloogiaga leitud tüüpsõna paradigma kirjeldus, mis koosneb iga muutvormi koostamismallidest ehk muutvormimallidest. Tüüpsõnamall on relatsioon tehnilise tüve ja kõigi selle paradigmasse kuuluvate muutvormide vahel.}
}
\newglossaryentry{muutvormimall}{
  name=Muutvormimall,
  description={kirjeldab üksiku muutvormi koostamisskeemi ja kannab selle grammatilised tunnused. On integraalne osa tüüpsõnamallist. Koostamisskeem koosneb muutujatest ja konstantidest, mille tähtkoostised lükitakse üks\-teise järele. Muutujate tähtkoostised võivad olla mingil moel piiratud.}
}
\newglossaryentry{lemma}{
  name=Lemma,
  description={on suvaliselt valitud grammatiliste tunnuste komplekt, mida kasutatakse lekseemi viitamiseks.}
}
\newglossaryentry{tehniline-tüvi}{
  name=Tehniline tüvi,
  description={on tähtkoostiste järjend, millega saab tüüpsõnamalli muutvormide muutujad asendada elik väärtustada ja niiviisi koostada ühe konkreetse sõna kõik vormid.}
}
\newglossaryentry{mikrostruktuur}{
  name=Mikrostruktuur,
  description={on sõnastiku sõnaartikli sisemine struktuur.}
}
\newglossaryentry{konkatenatsioon}{
  name=Konkatenatsioon,
  description={ehk \ensuremath{\oplus} on tähtede ja täht\-koostise kokku\-viimine ehk lükkimine teine\-teise järele, et moodustada uus täht\-koostis. Näiteks \textit{aa} \ensuremath{\oplus} \textit{be} moodustab \textit{aabe}.},
  symbol=ensuremath{\oplus}
}



\begin{document}


\begin{titlepage}
  \centering
  % \includegraphics[width=0.15\textwidth]{example-image-1x1}\par\vspace{1cm}
  {\scshape\LARGE \textsc{Tartu Ülikool \\
      Humanitaarteaduste ja kunstide valdkond \\
      Eesti ja üldkeeleteaduse instituut \\
      Eesti keele osakond
    } \par}
  \vspace{1.5cm}
  {\Large Kristian Kankainen\par}
  \vspace{1.5cm}
  {\huge\bfseries Ekstraktmorfoloogia meetodiga tuletatud keeletehnoloogia \\ vadja noomeni vormisõnastiku näitel\par}
  \vspace{2cm}
  {\Large Magistritöö\par}
  \vfill
  {\Large Juhendajad dotsent Heinike Heinsoo ja PhD Külli Prillop\par}
  
  \vfill
  
% Bottom of the page
  {\large \textsc{Tartu 2019}\par}
\end{titlepage}

\newpage
\tableofcontents


% kust pärineb järgmine tsitaat? "computational methodology for linguistic analysis is not the same thing as computational linguistics" ()


\newpage
\spacing{1.5}
\section{Sissejuhatus}
\label{sissejuhatus}

% kirjutaja peamine idee ja kuhu ta pürgib
% selgesti uurimuse kesksed lahendused: uurimuse eesmärk ja kasutatud meetodid
% kokkuvõtvalt probleemi tausta
% lühidalt kogu töö seisukohalt kesksed mõisted ning aine piirid (teema piiritlus)

Magistritöö eesmärk on luua arvuti\-süsteem, mille abil saab koostada vadja kirja\-keele noomenite jaoks morfoloogilist sõnastikku ehk vormisõnastikku. Koostatud vormisõnastiku kuju peaks olema võimalikult paindlik ja täitma kolme suuremat eesmärki: 1)~see oleks keeleõppes kasutatav teksti moodustamist toetava sõnaraamatuna, mis esitab sõnade kõiki morfoloogilisi vorme, 2)~selle sisu peaks olema automaatselt teisendatav programm\-koodi,
% järgmine on värdjas lause
mis integreerib vormisõnastikus kätketud vadja keele morfoloogia keele\-tehnoloogilistesse rakendustesse ja -taristutesse ning 3)~vormisõnastik peaks olema lihtsasti täiendatav ja muudetav ilma programmeerimis\-oskusteta.

Esimese eesmärgi täitmiseks on vormi\-sõnastikku lisatud Heinike Heinsoo \enquote{\mbox{Vad̕d̕a} sõnakopittõja} (\citeyear{heinsoo_vadsonakopittoja_2015}) sõnastikust kõik noomenid (u~420 sõna), millele on juurde lisatud sõnavara teistest sõnaraamatutest (u~460 sõna). Igale sõnale on see\-järel lisatud nende põhivormid Tsvetkovi vadja keele Jõgõperä murde sõnaraamatust (\cite{laakso_vatjan_1989}) ja põhi\-vormide põhjal on moodustatud ja vormi\-sõnastikku lisatud kõik ülejäänud käändevormid. Käänete valikus järgitakse levinud vadja keele õpikuid (\cites{heinsoo_vadsonakopittoja_2015}{konkova_vaddceeli_2014}), mille kirja\-viisi on muudetud ja ühtlustatud Heinsoo kirjaviisile vastavaks. Magistritöö vältel koostatud morfoloogiline andmestik on kirjeldava iseloomuga ja ei pretendeeri lõplikule morfoloogia esitusele (vrd eesmärk~3).

Teiseks eesmärgiks on vormisõnastiku automaatne teisendatavus keeletehnoloogilisteks rakendusteks. Ainult morfoloogilisest sõnastikust ei piisa, vaid on vaja rakenduslikku väljundit, mida saaks kasutada vadja keele taas\-elustamisel.  Vadja keele õppijale ja kirjutajale on ehk kõige praktilisema väärtusega õigekirja\-kontrollijaks teisendatud vormi\-sõnastik. Arvuti\-lingvistile ja programmeerijale on oluline  vormi\-sõnastiku teisendatavus ka programm\-koodi kujule, mis täidab vadja keele morfoloogilise toe kahes keele\-tehnoloogilises taristus (Grammatical Framework ja Giella taristu). Vormi\-sõnastiku integreeritavuse tõttu on nendes taristutes tulevikus võimalik luua rohkem keeletehnoloogiat vadja keelele, nt keeleõppe\-rakendusi. Keele\-teadlasele on abiks vadja keele morfoloogiline analüsaator, mille abil on võimalik märgendada vadja kirja\-keelseid teksti\-korpuseid.

Kolmandaks eesmärgiks on loodava süsteemi kasutaja\-sõbralikkus. Vormi\-sõnastik peab olema täiendatav ja muudetav lihtsal viisil, mis ei eelda erilisi oskusi peale sõnade muut\-vormide tundmise. Uue sõna sisestamine vormi\-sõnastikku peaks olema nii lihtne, et kasutaja ainult valib õige tüüpsõna, mille järgi sisestatav sõna morfoloogiliselt käitub. Muutmise korral peaks kasutaja ainult ümber kirjutama vastavad muut\-vormid. Teisiti öeldes tahab magistritöö autor loodava süsteemiga anda vadja keelele kala asemel õnge,
% see on, keele\-tehnoloogia asemel vormisõnastiku,
et edaspidine vadja keele arendamine võiks toimida teiste eest\-vedamisel, mitte jääda arvutus\-lingvisti või keele\-tehnoloogi taha toppama.

Magistritöö uurib ekstrakt\-morfoloogia meetodi (\cite{ahlberg_semi-supervised_2014}) kasutatavust püstitatud eesmärkide täitmiseks. Meetod sobib ülesandele~(3) hästi, kuna ei põhine morfo(fono)loogilistel reeglitel, vaid ainult sõnade muut\-vormidel, millest meetod eraldab (ekstraheerib) tüüpsõna\-malli. Kuna eraldatud tüüpsõna\-mallid on teisendatavad programm\-koodi, siis toimib tüüp\-sõna\-mall reeglite asemel. % ja sõnavormide muutmine loodavas süsteemis muudab ka tüüpsõna\-malli.
Reegli\-põhiseid meetodeid on välditud, kuna reeglite kirjutamine tahest-tahtmata eeldaks suurt tehnilist oskust ja ülalpidamist reeglite lisamise ja muutmise korral. See on midagi, mida väikesel keele\-kasutajas\-konnal on raske endale lubada. Loodava süsteemi puhul muudaks sõnavormide muutmine vormisõnastikus sõnast eraldatud tüüpsõna\-malli ja seekaudu ka programm\-koodi.
%Sõnavormide muutmine muudab seega muuttüübi kirjeldust.


Valitud meetodi kompromissiks on tüüpsõnade paljusus: sõna \vadja{čiutto} (ee särk) käändub tüüpsõna \vadja{hattu} (ee müts) järgi, aga sõna \vadja{rätte} (ee rätt, rätik) käändub hoopis nagu tüüpsõna \vadja{tüttö} (ee tüdruk). Tüüp\-sõnade ainus erinevus on vokaalharmoonias. Ekstrakt\-morfoloogia meetodiga eraldatud tüüpsõnad sarnanevad Eesti leksikograafia\-traditsioonis Elmar Muugi \enquote{Väiksele õigekeelsus-sõnaraamatu} (\citeyear{muuk_vaike_1933}) muuttüüpidele, mille muuttüübistik hõlmab 895~eesti keele noomeni tüüpsõna. Aja jooksul on Eesti leksikograafiline traditsioon leidnud erinevaid viise kompaktsema muuttüübistiku koostamiseks ja Ülle Viks ei ole Muugi muuttüüpe pidanud enam muuttüüpidekski, vaid on nimetanud neid struktuuri\-tüüpideks (\cite[282]{viks_verbide_1976}). Magistritöö jooksul koostatud vadja vormisõnastik hõlmab 231~noomeni tüüpsõna. 

Vadja vormisõnastiku tüüpsõnade arv on ehk suur, ent mõistlik, kui arvestada lääne\-mere\-soome keeltega üldiselt. Näiteks on Viitso ja Ernštreiti liivi keele sõnastikus (\citeyear{viitso_livoki-estiki-tkisonarontoz_2012}) loetletud 242 tüüpsõna.

%% Ekstraktmorfoloogia meetod sobitub magistritöö ülesandele hästi mitmel põhjusel:
%% \begin{itemize}
%% \item meetod põhineb otse sõnavormidel ja ei vaja reegleid (vadja keelel puudub morfoloogiat piisavalt täpselt kirjeldav grammatika)
%% \item tekitatud sõnastik koos sõnatüübistikuga on võimalik automaatselt ümber kodeerida keeletehnoloogiasse (mh õigekirjakontrollija)
%% \item loodud sõnastik on iseenesest kasutajaliides, mis võimaldab vadja keele huvilistel endil oma sõnavara rikastada (ei vaja programmeerimis- ega keeleteaduslike oskusi)

Magistritöö on struktureeritud järgmiselt. 
Peatükis~\ref{teoreetilised-lähtekohad} kirjeldatakse vormisõnastiku koostamise teoreetilisi lähtekohti. Töö asetust korpuse planeerimises~(ptk \ref{korpuse-planeerimine}); vormisõnastiku sõnavara valikut, käänete valikut, paradigmade moodustamise ja ühtlustamise põhimõtteid~(ptk \ref{sec:sõnavara-valik}); ekstrakt\-morfo\-loogia meetodist tulenevat morfeemi definitsiooni~(ptk \ref{sec:morfeemi-staatus}) ja morfeemi definitsiooni sarnasust klassikalise paradigmaatilise morfoloogia käsitlusega~(ptk \ref{klassikaline-paradigmaatiline}). 

Peatükis~\ref{sec:vormisõnastiku-sisestamine} tutvustatakse vormisõnastiku kasutajaliidest Morfoloogia\-labor. 
Peatükis~\ref{sec:ekstraktmorfoloogia-meetod} kirjeldatakse ekstraktmorfoloogia meetodit. 
Peatükis~\ref{sec:analüüs} analüüsitakse meetodiga saadud muuttüübistikku käändkonniti. 
Peatükis~\ref{sec:programmkoodi-tuletamine} kirjeldatakse vormisõnastiku teisendamist keele\-tehnoloogiaks. Teisendamiseks vajalikku keskset kirjeldust kirjeldatakse peatükis~\ref{sec:lmf} ning programm\-koodi generaatoreid, mis teisendavad vormi\-sõnastiku programm\-koodiks ja integreerivad vormisõnastiku kahte taristusse (ptk~\ref{sec:gf} Grammatical Framework ja ptk~\ref{sec:giella} Giella taristu ja õigekirjakontrollija).
Peatükis~\ref{sec:kokkuvõte} võetakse töö kokku.


%% Tööl on neli suuremat eesmärki ehk teesi:
%% \begin{enumerate}
%% \item näidata, et ekstraktmorfoloogia on intuitiivne viis luua arvutimorfoloogia
%% (võimaldab mitte-(arvutus-)lingvistil luua ja muuta andmestikku)
%% \item ekstraktmorfoloogia meetodiga loodud arvutimorfoloogia on võimalik integreerida automaatse teisendusega erinevatesse keeletehnoloogiatesse
%% \item ekstraktmorfoloogia meetodiga luuakse arhiveeritava ja püsivama väärtusega andmestik % kui seda on keeletehnoloogia
%% \item kõrvutada ja ühitada vadja noomeni ekstraktmorfoloogia leiud traditsioonilise
%% morfoloogiakäsitlusega
%% \item on laiendatav teise sõnaliigiga sõnade sisestamiseks
%% \item on laiendatav tõlkevastustega %  neid kahte ma väga ei "tõesta"
%% \end{enumerate}
%% 
%% 
%% Magistritöö eesmärk on luua vadja keelele morfoloogiline sõnastik sellisel moel, et keeletehnoloogilised komponendid, mh õigekirjakontrollija, on sellest automaatselt tuletatavad. Sõnastik oleks loodava vadja kirjakeele õigekeelsuse baasiks ja aitaks keeleõppes ja revitaliseerimises kaasa.
%% 
%% Morfoloogiline sõnastik on süsteem, mis sisaldab kõigi sõnaartiklite kõiki muutvorme. Töö andmestik piirdub vadja noomeniga. Töö andmestik hõlmab u~900~noomenit, mis on valdavas osas pärit Vad̕d̕a Sõnakopittõja sõnastikust (\cite{heinsoo_vadsonakopittoja_2015}) koos täiendustega teistest sõnaraamatutest.
%% 
%% Sõnade paradigmad, ehk käändvormide tüvemuutused, on saadud Tsvetkovi sõnaraamatust (\cite{laakso_vatjan_1989}) milledele on lisatud muutelõpud levinud vadja keele õpikust (\cite{_vadceeli:_2014}). Sõnade kirjapilti on ühtlustatud vadja loodava kirjakeele järgi (Heinsoo, isiklik kommunikatsioon).
%% 
%% Loodud sõnaraamatu kirjed on kõrvuti morfoloogiliste muutvormidega varustatud ka sõnatüübi\-tähisega. Sõnatüübid leitakse ekstraktmorfoloogia meetodi abil automaatselt ja sõnatüübistik uuendatakse iga kord sõnaraamatu kirjeid muudetakse.
%%  
%% Ekstrakt-morfoloogiline klassifikatsioon põhineb ainult sõnade muutvormide tähtkoostisel ja sõnatüübistik on maksimaalne selles suhtes, et igal tüvevokaalil ja astmevahelduslikul tähel on omaette sõnatüüp.
%% 
%% Kõrvuti eksisteeriv sõnatüübistik võimaldab arvutimorfoloogia automaatse loomise ja niiviisi on sõnaraamatusse uute kirjete lisamise juures süsteemil võimalik genereerida ja kasutajale kuvada sõna potentsiaalseid muutvorme. Samuti on võimalik sõnatüübistiku kirjelduse põhjal automaatselt luua programm\-koodi, mis täidab morfoloogia komponendi ülesannet ühes või mitmes programmeerimis\-keeles.
%% 
%% Ekstraktmorfoloogia meetod sobitub magistritöö ülesandele hästi mitmel põhjusel:
%% \begin{itemize}
%% \item meetod põhineb otse sõnavormidel ja ei vaja reegleid (vadja keelel puudub morfoloogiat piisavalt täpselt kirjeldav grammatika)
%% \item tekitatud sõnastik koos sõnatüübistikuga on võimalik automaatselt ümber kodeerida keeletehnoloogiasse (mh õigekirjakontrollija)
%% \item loodud sõnastik on iseenesest kasutajaliides, mis võimaldab vadja keele huvilistel endil oma sõnavara rikastada (ei vaja programmeerimis- ega keeleteaduslike oskusi)
%% \item sõnastikus esinevad muutvormide artikleid on võimalik varustada korpusesinemuste informatsiooniga ja õigekeelsuslike kommentaaridega
%% \end{itemize}


%Magistritöö on struktureeritud järgmiselt:
%   2 Morfoloogilise sõnaraamatu teoreetilised lähtekohad
%        2.1 Vadja kirjakeel ja korpusplaneerimine
%        2.2 Arvutimorfoloogia eesmärk ja lingvistiline motiveeritus
%        2.3 Paradigmaatiline morfoloogia
%        2.4 Vadja kirjakeel ja normatiiv
%        2.5 Sõnavara
%        2.6 Ortograafia
%        2.7 Morfofonoloogia
%        2.8 Klassikaline paradigmaatiline morfoloogia
%        2.9 Morfeemi staatus ja definitsioon
%        2.10 Muuttüüp, tüüpsõna ja muutkond
%      3 Ekstraktmorfoloogia meetod
%        3.1 
%      4 Vadja morfoloogiliste tüüpsõnade analüüs
%        4.16 Ekstraktmorfoloogia üldistatud muuttüüpide algoritm
%        4.17 Põhivormid ja analoogiavormid
%      5 Programmkoodi tuletamine
%        5.1 Keskne kirjeldus Lexical Markup Framework vormingus
%        5.2 Grammatical Framework morfoloogiakomponent
%        5.3 Integreerimine Giella-taristuga
%      6 Arutelu
%      7 Kokkuvõte
%      8 Põhimõisted ja lühendid
%      9 Kirjandus
%      10 The use of Extract Morphology for Automatic Derivation of Language Technology for Votic
%      11 Lisad

% morfeemi staatus -- morfeem pole siin töös traditsioonilisel kujul kui väikseim tähenduslik vormiline üksus vaid hoopis mistahes-tähtkoostise-muutusena (vrd Baer)
% morfeemi puudumine töös toob kaasa ka (morfoloogiliste) reeglite puudumise. see on üsna ebatraditsiooniline, aga põhjendame siinkohal ajalooliselt siiski ära
% ekstraktmorfoloogia organiseeritus sarnaneb kõige lähedamini klassikalisele paradigmaatilisele morfoloogiakäsitlusele. Hockett mainib seda 'vanema ja väärikama teooriana' oma tänaseks klassikaks saanud artiklis kahe morf teooria üle. Ta räägib IP ja IA lähenemisest ja toob välja ajastu formaliseerimislembuse ning kuna IA oli juba formaliseeritud, oli tema meelest tookord aeg formaliseerida IP. Karlsson ütleb kokkuvõtvalt, "et sellest sai levinuim lähenemine". Tänapäeval on hakatud formaliseerima ka paradigmaatilist lähenemist, aga peamiselt kahes liinis, millest üks on rohkem reeglitel põhinev (nn uus) ja teine on vähem reeglitel põhinev (nn klassikaline). Arvutimorfoloogid on siiski hiljuti toonud välja olukorra, et paradigmaatilist lähenemist on võimalik modelleerida kasutades sama formalismi, millega IP morfoologiatki, nimelt lõplike automaatidega.
% reeglite lembusest


%% Magistritöö esimene eesmärk on luua H. Heinsoo Sõnakopittöjas esitatud sõnavarast morfoloogiline sõnastik, mis sisaldab sõnavara kõiki muutvorme. Selleks vajalik arvutimorfoloogiline kirjeldus ehitatakse sellisel moel, et see taandub tüüpsõnade muutvormitabelite esitamisele, mitte grammatiliste reeglite esitamisele. Niiviisi ehitatud teooriavaba(m) arvutimorfoloogiline kirjeldus võimaldab luua erinevaid keeletehnoloogiaid automaatselt programmkoodi tuletamise teel. Esitatakse kolme tehnoloogia automaatset tuletamist: 1)~ühe keeletehnoloogilise taristusse integreerimise kaudu õigekirjakontrollija, 2)~vadja keele arvutimorfoloogia moodul ühe loomulike keelte grammatikate koostamiseks mõeldud programmeerimiskeelele ja 3)~morfoloogia tehnoloogiaülene kirjeldus ühe rahvusvahelise standardi abil.

%% Kuna kõik tuletatud keeletehnoloogia edaspidine täiendamine ja täpsustamine käib ainult lekseemide muutvormitabelite täiendamise ja täpsustamise kaudu, peab esimese eesmärgi juurde lisama seda, et magistritöös loodud leksikograafiline süsteem võimaldab keeleaktivistide rühmal töötada oma sõnavara ja keeletehnoloogia kallal edaspidi ka ilma spetsialistist keeleteadlase ja keeletehnoloogi abil. Kas seda vadja keele puhul ka juhtub, jääb tuleviku näidata.

%% Magistritöö teine eesmärk on analüüsida leitud tüüpsõnad mitmel viisil: 1)~kirjeldada nende morfofonoloogiat keeleajalooliste arengute taustal, 2)~leida tüüpsõnade põhivormid ja analoogiavormid, 3)~esitada üks võimalik muuttüüpide süsteem ja võrrelda seda seni esitatutega ja viimalt 4)~analüüsida muuttüüpide produktiivsust.


%% ---


%% Magistritöö loob viisi ehitada arvutimorfoloogia puhtalt lekseemide sõnavormide esitamise teel ning teisendada ehitatud arvutimorfoloogilise mudeli automaatselt kahte keeletehnoloogilisse raamistikku.

%% Magistritöö kasutab loodud süsteemi selleks, et kirjeldada vadja keele normatiivsed morfoloogilised tüüpsõnad.

%% Tööd ajendab mõtteviis minimeerida tööd: loodud normatiivne morfoloogiline tüübistik on aluseks automaatselt tuletatud keeletehnoloogiale, kui normatiiv muutub, muutub ka keeletehnoloogia. Töö paneb leksikaalse ressursi esikohale ja kõik leitud sisulised vead õiendatakse otse ressursis, mitte keeletehnoloogilistes tarkvarades eraldi.




\newpage
\section{Vormisõnastiku koostamise teoreetilised lähtekohad}
\label{teoreetilised-lähtekohad}

Siin peatükis kirjeldatakse vormisõnastiku koostamise ja vadja kirjakeelsete muutvormide moodustamise põhimõtteid. Andmete sisestamist nn Morfoloogia\-labori rakendusega kirjeldatakse peatükis~\ref{sec:vormisõnastiku-sisestamine} ja talletatud andmete kuju kirjeldatakse peatükis~\ref{sec:lmf}.

Vormisõnastiku definitsiooniks magistritöös on võetud Ülle Viksi järgi:

\spacing{1}
\begin{quote}
  \enquote{Üks täielik vormisõnastik peaks esitama kõigi sõnade kõik muutevormid koos vastava grammatilise iseloomustusega. Ainult siis saab kasutaja sõnastikust ilma mingi vaevata ja täiesti kindlalt teada, milline on vajalik vorm antud sõnast või millise sõna millise vormiga on tegemist tundmatu sõnavormi puhul.} (\cite[7]{viks_vaike_1992}).
\end{quote}
\spacing{1.5}



%  siia lisada Viksi definitsioon vormisõnastikust

%Kuna töö opereerib arvutilingvistika, deskriptiivse ja dokumentaalse lingvistika ääremail, peame selgitama töö teoreetilised lähtekohad. Siinsele kompendiumiks on ka põhimõisted seletatud pt~\ref{sec:põhimõisted} \nameref{sec:põhimõisted}.

%Töö püüdleb olla võimalikult teooriavaba, lastes vadja lekseemide sõnavormide tähtkoostised ise määrata nende paradigmade koostamis\-reeglid. (See on olla deduktiivne esialgse morfoloogia postuleerimises, vastandudes induktiivsele, s.o mingist grammatilisest kirjeldusest lähtudes.)

%Tööl on siiski teoreetilised lähtekohad, mis tulenevad ühelt poolt arvutimorfoloogia nõuetest ja teisalt klassikalisest paradigmaatilisest morfoloogiakäsitlusest. Järgmiselt püüan argumenteerida, et arvutimorfoloogia ei pea olema mingist lingvistisest teooriast ajendatud. Seejärel tutvustan tööle kõige lähedamini asetsevat morfoloogilist käsitlust.

%  EELMINE JUTT TÄIELIKULT ÜLE VAADATA!

\subsection{Vadja ühiskeel ja korpuse planeerimine}
\label{korpuse-planeerimine}

% TODO lisada ühiskeele murdeline alus on Vaipooli (Li-Lu), sest see on mida "kõneldakse veel". Rohkem on uuritud ja kirjeldatud Kattila(?), mida Ernits on ka soovitanud ühiskeele aluseks valida.

% Selles peatükis seletatakse vadja ühiskeeleks valitud aluseid ning korpuse planeerimise mõistet.
Vadja keelele ei loodud Nõukogude Liidus 1930.\,aastail ühiskeelt, nagu seda tehti näiteks karjala, vepsa ja isuri keele jaoks. Vadja kirjakeele loomise pürgimistest pärast Nõukogude Liidu lagunemist on kirjutanud lähemalt Enn Ernits (\citeyear{ernits_vadja_2006}) ja vadjalaste keele normi tunnetusest (ilma kirja\-keeleta) on rääkinud Elena Markus ja Fedor Rozhanskiy (\citeyear{markus_concept_2013}).

% korpusplaneerimine on alamosa keeleplaneerimisest?
%Korpusplaneerimine on mis, miks seda tehakse ja kes seda teevad (kasuta Cooper 1996: 45).
Korpusplaneerimine on keele\-planeerimise üks osa ja koosneb Klossi (\citeyear{kloss_research_1970}) järgi kolmest osast:

\begin{enumerate}[noitemsep]
\item kirjamine (ingl. \textit{graphization}) ehk kirjaviisi määramine
\item morfoloogia ühtlustamine ja standardiseerimine
\item sõnavara moderniseerimine ja rikastamine
\end{enumerate}


Neid osi ei pea vaatama järjestikku etappidena keele moderniseerimise poole, Coulmas (\citeyear{coulmas_language_1989}) on nimetanud seda pidevaks adapteerimise protsessiks.

% alustame KIRJAVIISI kohta, seejärel morfoloogia ja sõnavara
Käesolev magistritöö jääb morfoloogia ühtlustamise ja standardiseerimise alla, ent tahest-tahtmata on töös kasutatud kirjaviisiga tehtud valik, kus on järgitud Heinike Heinsoo (\citeyear{heinsoo_vadsonakopittoja_2015}) õppe\-materjalide kirjaviisi. Valitud kirjaviis on siiski automaatselt teisendatav teise levinud, Olga Konkova (\citeyear{konkova_vaddceeli_2014}) õppematerjalides kasutatud kirja\-viisile vastavaks, mis\-tõttu peaks magistri\-töös loodud materjal olema kasulik mõlema õppe\-materjali jaoks. Ainsad erinevused kahe kirja\-viisi vahel on asendused ü~\textrightarrow~y ja č~\textrightarrow~c. % kas see peaks olema ainult sissejuhatuses või mis sellega viga on?

% klaviatuur on seotud kirjaviisiga
Soodustamaks vadja keeles kirjutamist ja seda\-kaudu vadja keele elavdamist, on magistritöö kõrval ilmunud mõlemaid ortograafiaid toetav klaviatuuri\-paigutis arvutile ja nuti\-seadmetele, mida autor on postrina esitanud (\cite[41]{toth_folklore_2019}). Klaviatuuri\-paigutis on internetis saadaval\footnote{\url{https://gtsvn.uit.no/langtech/trunk/keyboards/vot/}}. Vadja keele puhul kasutatud erinevatest kirjaviisidest on pikemalt kirjutanud Enn Ernits (\citeyear{ernits_vadja_2010}).


% MORFOLOOGIA, seejärel sõnavara rikastamise kohta 
Morfoloogia ühtlustamist iseloomustab siinses töös sõnavormide rekonstrueerimine Jõgõperä murdeliste muutvormide põhjal Vaipooli murdealale iseäralikele joontele vastavaks. Rekonstrueerimise kui standardiseerimise suhtes ei võta magistri\-töö süsteemset seisukohta, vaid on oma valikutes pigem kirjeldava iseloomuga. Jõgõperä ja Luuditsa murrete erinevustest on kirjutanud Rozhanskiy ja Markus (\citeyear{rozhanskiy_dialectal_2015}). Morfoloogia ühtlustamist kirjeldatakse lähemalt järgmises alajaotuses.

% SÕNAVARA moderniseerimine ja rikastamine 
Antud magistritöö ei pürgi vadja keele sõnavara moderniseerimisele ega rikastamisele -- hõlmatud sõnavara on varem avaldatud. Võimalikest vadja kirja\-keele sõnavara rikastamise viisidest on lähemalt kirjutanud Ernits (\citeyear{ernits_vadja_2010}). 
% TODO tõlgi artikli pealkiri ja kirjuta lahti siin
Vene laensõnade adapteerimisest Jõgõperä murdes ja selle problemaatikast on lähemalt kirjutanud Rozhanskiy (\citeyear{rozhanskiy_zaimstvovannyje_2009}).

% kas see jutt kuulub siia?
%\spacing{0}
%\begin{enumerate}
%\item lekseemide Jõgõperä-murdeliste muutvormide sisestamine ekstraktmorfoloogia %süsteemi, millest on saadud sõnatüübid
%\item saadud sõnatüüpide grupeerimine käändkondadesse
%\item käändkonniti sõnatüüpide ühtlustamine
%\end{enumerate}
%\spacing{1.5}

% TODO kuidas see ümberkirjutada ja selgeks teha? TODO see on ju kordus
Kuna magistritöös koostatud süsteem loob ühtse töökeskkonna, mille abil määratakse morfoloogia ainult sõnade muut\-vormide esitamise kaudu ja mis ei vaja program\-meerimis\-oskusi, siis loodab töö autor sellele, et edaspidine korpuse planeerimine võiks toimuda vadjalaste endi ja vadja keele huviliste eestvõttel -- lisa või muuda muutvorme vormi\-sõnastikus ja see muutub ka keele\-tehnoloogias.



\subsection{Sõnavara valik ja paradigmade moodustamine}
\label{sec:sõnavara-valik}

Sõnavara valik hõlmab Heinsoo (\citeyear{heinsoo_vadsonakopittoja_2015}) sõnastikus esitatud nimi\-sõnu ja adjektiive (u~420 sõna). Sõnavara on veel laiendatud autori enda silma järgi Tsvetkovi sõnastikust (\cite{laakso_vatjan_1989}) ettesattunuga (u~460 sõna, mh~\vadja{asfal̕tti}, \vadja{bibli} ja \vadja{biblioteekkõ}). Kokku on vormi\-sõnastikku koondatud 882 sõna\-artiklit.

Sõnaartiklitele on lisatud Tsvetkovi sõnaraamatus esitatud põhi\-käände\-vormid (\msd{sg nom}, \msd{sg gen}, \msd{sg par}, \msd{sg ill} ja  \msd{pl nom}, \msd{pl gen}, \msd{pl par}, harva ka \msd{pl ill}). Juhul kui Tsvetkovi sõna\-raamatus on esitatud mitu paralleelset põhikäändevormi, on eelistatud pikemaid vorme, nt \msd{sg ill} \vadja{asfal̕ttisõ} lühikese illatiivi asemel \vadja{asfal̕tti}. Niiviisi on morfoloogiliste tüüpsõnade kaardistamise protsessis välditud paralleelvorme, kuigi loodud lõpp\-lahendus ehk arvutimorfoloogiline süsteem võimaldab sõna\-artiklitel esitada ka paralleelseid tüüpsõnu. % TODO nagu Eesti arvutimorfoloogia traditsioonis on tehtud \cite[278]{viks_verbide_1976} ja SAMEST projektis aditiiv ja illatiiv lahku löödud
% TODO eelmisesse lisada ka pl part
% TODO kuigi see on avatud küsimus ja ilmselt muutub
% TODO ekstraktmorfoloogia meetod võimaldab paralleelvormide esitamise


% Alljärgnevalt seletatakse põhikäändevormide ühtlustamise ja analoogiavormide koostamise põhimõtted. % TODO ja mis veel seletatakse?
%Põhikäändevormidele on lisatud analoogiavormid \msd{sg ela} ... \msd{pl kom} Konkova õppematerjalides esitatut käändeid katmaks. Teised uurijad on ...


\subsection{Noomeni käänded ja ühtlustamine}


Käänete valiku ja muutelõppude puhul on järgitud Konkova õpikus (\citeyear[10]{konkova_vaddceeli_2014}) esitatut, mis langeb kokku Heinsoo (\citeyear[88]{heinsoo_vadsonakopittoja_2015}) õppe\-materjalide põhimõtetega. Käänded on näitlikustatud tabelis~\ref{tab:noomeni-käänded}.

\begin{table}[ht]
  \centering
  \begin{tabular}[t]{r l l l l l}
    % TODO vali paremad sõnad!
    % DONE vadja sõnad kursiivi!
     kääne & eespoolne & eespoolne & tagapoolne & tagapoolne & tagapoolne \\
    \hline
    \msd{sg nom} & \vadja{pää    } & \vadja{ärče     } & \vadja{lafkõ     } & \vadja{lammõz    } & \vadja{ivuz} \\
    \msd{sg gen} & \vadja{pää    } & \vadja{ärjä     } & \vadja{lavga     } & \vadja{lampa     } & \vadja{ivusõ} \\
    \msd{sg par} & \vadja{pääte  } & \vadja{ärčä     } & \vadja{lafka     } & \vadja{lammassõ  } & \vadja{ivussõ} \\
    \msd{sg ill} & \vadja{pähhe  } & \vadja{ärčäse   } & \vadja{lafkasõ   } & \vadja{lampasõ   } & \vadja{ivussõsõ} \\
    \msd{sg ine} & \vadja{pääz   } & \vadja{ärjez    } & \vadja{lavgõz    } & \vadja{lampaz    } & \vadja{ivusõz} \\
    \msd{sg ela} & \vadja{pääss  } & \vadja{ärjess   } & \vadja{lavgõss   } & \vadja{lampass   } & \vadja{ivusõss} \\
    \msd{sg all} & \vadja{päälle } & \vadja{ärjelle  } & \vadja{lavgõllõ  } & \vadja{lampallõ  } & \vadja{ivusõllõ} \\
    \msd{sg ade} & \vadja{pääll  } & \vadja{ärjell   } & \vadja{lavgõll   } & \vadja{lampall   } & \vadja{ivusõll} \\
    \msd{sg abl} & \vadja{päält  } & \vadja{ärjelt   } & \vadja{lavgõlt   } & \vadja{lampalt   } & \vadja{ivusõlt} \\
    \msd{sg tra} & \vadja{päässi } & \vadja{ärjessi  } & \vadja{lavgõssi  } & \vadja{lampassi  } & \vadja{ivusõssi} \\
    \msd{sg ter} & \vadja{päässaa} & \vadja{ärjessaa } & \vadja{lavgõssaa } & \vadja{lampassaa } & \vadja{ivusõssaa} \\
    \msd{sg com} & \vadja{pääka  } & \vadja{ärjeka   } & \vadja{lavgõka   } & \vadja{lampaka   } & \vadja{ivusõka} \\
    \msd{pl nom} & \vadja{pääd   } & \vadja{ärjed    } & \vadja{lavgõd    } & \vadja{lampad    } & \vadja{ivusõd} \\
    \msd{pl gen} & \vadja{päije  } & \vadja{ärčije   } & \vadja{lafkojõ   } & \vadja{lampajõ   } & \vadja{ivussijõ} \\
    \msd{pl par} & \vadja{päit   } & \vadja{ärčiit   } & \vadja{lafkoit   } & \vadja{lampait   } & \vadja{ivussiit} \\
    \msd{pl ill} & \vadja{päise  } & \vadja{ärčiise  } & \vadja{lafkoisõ  } & \vadja{lampaisõ  } & \vadja{ivussiisõ} \\
    \msd{pl ine} & \vadja{päiz   } & \vadja{ärčiiz   } & \vadja{lafkoiz   } & \vadja{lampaiz   } & \vadja{ivussiiz} \\
    \msd{pl ela} & \vadja{päiss  } & \vadja{ärčiiss  } & \vadja{lafkoiss  } & \vadja{lampaiss  } & \vadja{ivussiiss} \\
    \msd{pl all} & \vadja{päille } & \vadja{ärčiille } & \vadja{lafkoillõ } & \vadja{lampaillõ } & \vadja{ivussiillõ} \\
    \msd{pl ade} & \vadja{päill  } & \vadja{ärčiill  } & \vadja{lafkoill  } & \vadja{lampaill  } & \vadja{ivussiill} \\
    \msd{pl abl} & \vadja{päilt  } & \vadja{ärčiilt  } & \vadja{lafkoilt  } & \vadja{lampailt  } & \vadja{ivussiilt} \\
    \msd{pl tra} & \vadja{päissi } & \vadja{ärčiissi } & \vadja{lafkoissi } & \vadja{lampaissi } & \vadja{ivussiissi} \\
    \msd{pl ter} & \vadja{päissaa} & \vadja{ärčiissaa} & \vadja{lafkoissaa} & \vadja{lampaissaa} & \vadja{ivussiissaa} \\
    \msd{pl com} & \vadja{päika  } & \vadja{ärčijka  } & \vadja{lafkoika  } & \vadja{lampaika  } & \vadja{ivussijka} \\
  \end{tabular}
  \caption{Noomeni käänded koos käändelõppudega ees- ja tagapoolse vokalismi kujul.}
  \label{tab:noomeni-käänded}
\end{table}

Hõlmatud käändeid on seega 24: ainsuse ja mitmuse nominatiiv, genitiiv, partitiiv, illatiiv, inessiiv, elatiiv, allatiiv, adessiiv, ablatiiv, translatiiv, terminatiiv ja komitatiiv.

% mis käänded on välja jäetud
Välja on jäetud essiivi, abessiivi, ekstsessiivi ja instruktiivi käänded, mida Ariste (\citeyear[17]{ariste_grammar_1968}) on pidanud produktiivseteks kääneteks. Markus ja Rozhanskiy grammatika (\citeyear{markus_sovremennyi_2011}) ei käsitle terminatiivi ja komitatiivi käänetena vaid käände ja järelasendi vahe\-pealsetena, mille analüüsi nad on põhjendanud pikemalt artiklis (\cite{markus_comitative_2014}).

Alljärgnevalt seletatakse, kuidas Tsvetkovi sõnastikust saadud põhi\-vormid on ühtlustatud ja mille põhjal analoogia\-käänded on moodustatud.



\paragraph*{Nominatiiv}
Tsvetkovi sõnaraamatus antud vormile on tüve\-lõpu\-vokaali puudumise korral tavaliselt lisatud lühike vokaal vastavalt vokalismile Heinsoo ja Konkova eeskuju järgi (\cites[88]{heinsoo_vadsonakopittoja_2015}[10]{konkova_vaddceeli_2014}).


\paragraph*{Genitiiv}
Tsvetkovi sõnaraamatus antud vormi lõpuvokaal on kas ühtlustatud vastavalt Heinsoo ja Konkova sõnastikes esitatule või rekonstrueeritud vastavalt sõna vokalismile.


\paragraph*{Partitiiv}
Tsvetkovi antud vormi lõpuvokaal on ühtlustatud Heinsoo ja Konkova sõnastikele vastavalt või rekonstrueeritud vastavalt sõna vokalismile. Mitmuse partitiivi puhul on eelistatud pikemat, muute\-lõpuga varianti puhta -\textit{i}-mitmusliku tüve asemel. Mõlemad vormid esinevad paralleelsete variantidena Heinsoo ja Konkova õppe\-materjalides (\cites[88]{heinsoo_vadsonakopittoja_2015}[10]{konkova_vaddceeli_2014}).


\paragraph*{Illatiiv}
Loodava kirjakeele ühtlustamise huvides on vormisõnastiku koostamisel\linebreak eelistatud läbinähtava käändelõpuga vormi \textit{-se/-sõ} lühikese illatiivi asemel. Selline eelistus on mõnevõrra vastu\-olus olukorraga tänapäeva Luuditsa murdes, kus Markuse ja Rozhanskiy (\citeyear[247]{markus_comitative_2014}) järgi esineb tavaliselt lühike vorm ja pikk vorm esineb väga harva. Heinsoo ja Konkova õppe\-materjalides on mõlemad vormid esitatud rööpselt (\cites[88]{heinsoo_vadsonakopittoja_2015}[10]{konkova_vaddceeli_2014}).


\paragraph*{Inessiiv}
Inessiiv on analoogiavorm ja selle muutevormid ainsuses on moodustatud \msd{pl nom} vormi alusel, kusjuures muutelõpp -\textit{d} on asendatud lõpuga -\textit{z}. Mitmuse vormid on moodustatud \msd{pl ill} põhjal.

% ``A characteristic feature of the Votic inessive is the fact that geminate stops -kk-, -pp-, -tt-, the geminate affricates -tts-, -ttš-, the geminate -ss-, and the consonant cluster -hs- always are in the strong grade before this case marker''
Ariste (\citeyear[23]{ariste_grammar_1968}) järgi on vadja keele inessiivile omaseks tunnuseks see, et geminaat\-klusiilid -\textit{kk}-, -\textit{pp}- ja -\textit{tt}-, geminaat\-afrikaadid -\textit{tts}- ja -\textit{ttš}-, geminaat -\textit{ss}- ning konsonant\-kluster -\textit{hs}- esinevad alati tugevas astmes. Seda printsiipi järgides on \msd{sg ine} käände\-vormi vastavad tüved moodustatud tugeva\-astmelisteks.


\paragraph*{Elatiiv}
Elatiiv on analoogiavorm ja selle muutevormid ainsuses on moodustatud \msd{pl nom} vormi alusel, kusjuures muutelõpp -\textit{d} on asendatud lõpuga -\textit{ss}. Mitmuse vormid on moodustatud \msd{pl ill} põhjal.


\paragraph*{Allatiiv}
Allatiiv on analoogiavorm ja selle muutevormid ainsuses on moodustatud \msd{pl nom} vormi alusel, kusjuures muutelõpp -\textit{d} on asendatud lõpuga -\textit{lle} või -\textit{llõ} vastavalt sõna vokalismile. Mitmuse vormid on sarnaselt moodustatud \msd{pl ill} vormi põhjal.


\paragraph*{Adessiiv}
Adessiiv on analoogiavorm ja selle muutevormid ainsuses on moodustatud \msd{pl nom} vormi alusel, kusjuures muutelõpp -\textit{d} on asendatud lõpuga -\textit{ll}. Mitmuse vormid on moodustatud \msd{pl ill} põhjal.


\paragraph*{Ablatiiv}
Ablatiiv on analoogiavorm ja selle muutevormid ainsuses on moodustatud \msd{pl nom} vormi alusel, kusjuures muutelõpp -\textit{d} on asendatud lõpuga -\textit{lt}. Mitmuse vormid on moodustatud \msd{pl ill} põhjal.


\paragraph*{Translatiiv}
Translatiiv on analoogiavorm ja selle muutevormid ainsuses on moodustatud \msd{pl nom} vormi alusel, kusjuures muutelõpp -\textit{d} on asendatud lõpuga -\textit{ssi}. Mitmuse vormid on moodustatud \msd{pl ill} põhjal.


\paragraph*{Terminatiiv}
Terminatiiv on analoogiavorm ja selle muutevormid ainsuses on moodustatud \msd{sg ine} vormi alusel, kusjuures muutelõpp -\textit{z} on asendatud lõpuga -\textit{ssaa}. Mitmuse vormid on moodustatud \msd{pl ill} põhjal.

Ariste (\citeyear[34]{ariste_grammar_1968}) järgi võib terminatiiv põhineda illatiivi või allatiivi tüvel. Markuse ja Rozhanskiy (\citeyear[247]{markus_comitative_2014}) järgi põhineb see tänapäeva keeles genitiivi tüvel ja harva allatiivi tüvel. Seetõttu võib magistritöös valitud \msd{sg ine} tüve põhjal konstrueeritud muutevorm olla vale, kuigi valik ühtib Konkova ja Heinsoo õppe\-materjalides esitatuga, so tugeva\-astmelise geminaadiga.


\paragraph*{Komitatiiv}
Komitatiiv on analoogiavorm ja selle muutevormid ainsuses on moodustatud \msd{pl nom} vormi alusel, kusjuures muutelõpp -\textit{d} on asendatud lõpuga -\textit{ka}. Mitmuse vormid on moodustatud \msd{pl ill} põhjal, kusjuures -\textit{i}-tüveliste sõnade puhul on moodustunud pikk \textit{-ii-} muudetud \textit{-ij-} vastavalt Konkova (\citeyear[10]{konkova_vaddceeli_2014}) esitatud vormile (-\textit{-ijka}).






\subsection{Morfeemi staatus ja definitsioon}
\label{sec:morfeemi-staatus}


% Magistritöös ei käsitleta morfeemi levinud lingvistilise seisukoha järgi kui \textit{väikseimat tähenduslikku üksust}, vaid töö järgib definitsiooni, et morfeem on lekseemi \textit{mistahes fonoloogilise kuju muutust, millega kaasneb tähenduslik muutus} (\cites[31]{beard_morpheme_1987}[49]{beard_lexeme-morpheme_1995}), %Morfeemipõhist suunda ajab nt \cite{stump_inflectional_2001}.
%mis on sarnane klassikalise paradigmaatilise morfoloogia lähenemisele.

Magistritöös ei käsitleta morfeemi levinud lingvistilise seisukoha järgi kui \textit{väikseimat tähenduslikku üksust}. Töös kasutatud ekstrakt\-morfoloogia meetodi käsitlus morfeemist langeb kokku Robert Beardi lekseemi-morfeemi põhise morfoloogia teooriaga (\textit{Lexeme-Morpheme Base Morphology}). 
Beardi teoorias ei ole morfeem grammatiliselt tähenduslik, vaid defineeritud kui mistahes muutusena lekseemi fonoloogilises kujus, millega kaasneb tähenduse muutus (\cites[31]{beard_morpheme_1987}[49]{beard_lexeme-morpheme_1995}). Tema teooria järgi on ainult lekseem tervikuna tähenduslik ning grammatilised afiksid (morfeemid) on seda vaid sattumuslikult (\cite[17]{beard_morpheme_1987}).
Selline definitsioon sarnaneb klassikalise paradigmaatilise morfoloogia lähenemisele.


\subsection{Klassikaline paradigmaatiline morfoloogia}
\label{klassikaline-paradigmaatiline}

%Nii Beard'i teoorias kui ka klassikalises paradigmaatilises lähenemises on ainult lekseemid tähenduslikud märgid ning grammatilised afiksid (morfeemid) on seda vaid sattumuslikult (\cite[17]{beard_morpheme_1987}, \cite[189]{matthews_morphology_1991}).

Peter Matthews (\citeyear[186]{matthews_morphology_1991}) tõstab esile kaks paradigmaatilise morfoloogia käsitlust, klassikalise ja uue ehk strukturalistliku. 
Mõlemad käsitlused mõistavad sõna kui selle sõnavormide komplekti ehk paradigmat (\citeyear[186]{matthews_morphology_1991}). 
Kahe käsitluse peamiseks erinevuseks on see, et strukturalistlikus käsitluses koosneb sõna morfeemidest, aga klassikalises käsitluses pole sõnast väiksemat tähenduslikku üksust (\citeyear[196]{matthews_morphology_1991}). Sõnavormide grammatilised tähendused on klassikalise teooria järgi vaid sattumused (ingl. \textit{accidents}) (\citeyear[189]{matthews_morphology_1991}). % \cite[17]{beard_morpheme_1987}


Vanades ladina keele õpikutes ja grammatikates võidi esitada reegleid, mille abil ühe lekseemi kõik muutvormid moodustada. Reeglid opereerisid ainult sõnavormide tähtkoostisel. Näiteks võis reegel ühe sõnavormi lõputähti asendada muude tähtedega, et saada teine sõnavorm. Et asendatavatele tähtkoostistele ei omandatud mingit tähendust, näitlikustab ka see, et mõne reegli algvormiks võidi valida suvaline sõnavorm, mis oma tähtkoostise poolest kõige paremini sobis. (\citeyear[195-196]{matthews_morphology_1991})


Magistritöös rakendatud ekstrakt\-morfoloogia meetod koosneb sarnastest sõna\-vormi koostamis\-reeglitest, nn muutvormi\-mallidest. Meetodit tutvustatakse lähemalt pt~\ref{sec:ekstraktmorfoloogia-meetod}.

%Siiski on vadja keelel hulganisti lingvistilisi kirjeldusi, nagu grammatikaid (mh \cites{ahlqvist_wotisk_1856}{airila_vatjan_1934}{tsvetkov_vadja_2008}{ariste_grammar_1968}{__2011}), sõnaraamatuid (mh \cites{tsvetkov_vatjan_1995}{ariste_vadja_1943}{laakso_vatjan_1989}{raag_dictionary_1982}{pomberg_vadja_1991}{grunberg_vadja_2013}{heinsoo_vadsonakopittoja_2015}) ja ka etnograafilisi töid (mh \cites{kass_kasitoo-_1961}{malk_vadja_1977}).


%Käesolev töö ei pürgi looma lõplikku normatiivi, kuivõrd ta loob süsteemi, mis oskab vastata morfoloogilistele küsimustele. Aga loodud süsteemi peamine eesmärk on siiski võimaldada muuta ja jätkata tööd normatiivi arendamiseks ja mille ümber saaks keeleaktivistid ise koonduda, ilma et selleks oleks niivõrd vaja ei lingvistilist ega keeletehnoloogilist spetsialisti.

%Püüd luua vadja morfoloogiale normatiivne alus lihtsustab paljudele küsimustele vastusi leida, nt mis käändeid arvestada. Siiski on tööga loodud \textit{keele\-tehnoloogia tuletamise süsteem} avatud ka teistsugustele lähenemistele keeleainesele.




%% \subsection{Arvutimorfoloogia eesmärk ja lingvistiline motiveeritus}
%% 
%% % TODO leia allpool õige koht sellele: Kui lingvistika üldine eesmärk on leida ja kirjeldada keelenähtuste reeglipärasusi, siis jääb lahtiseks küsimus mis on arvuti\-lingvistika eesmärk -- kas see on formaliseerida lingvistika poolt leitud reeglipärasused või võib see mahutada ka nende reeglipärasuste leidmist? See töö lähtub arusaamast, et arvuti on abivahend lingvistile, mitte ei ole ainult lingvisti(ka) formalisatsioon.
%% % Seda tehakse kahel põhjusel -- esiteks puudub täielik morfoloogiakirjeldus vadja keelele. Teine põhjus on arvutiinsenerlik põhjus. Siinne töö maht ei luba täielikku morfoloogiakirjeldust, pealegi on keeled ajas muutuvad ja tuleb täiendusi teha. Seetõttu ei taha siinne töö esitada sellist formaalset morfoloogiakirjeldust, mida keegi teine ei saaks täiendada. Iga formalismi puhul kaasneb õpikõver, et sellest üldse aru saada, siinne töö esitab arvutimorfoloogia võimalikult formalismivabalt --- vormisõnastiku kujul --- milles sisalduva sõnavara on võimalik igal-ühel täiendada ja muuta ainult sellekaudu, et muuta konkreetse sõna sõnavormi.
%% % Tehnoloogiavaba(m) kirjeldus on standartne, ja keeletehnoloog, kes tahab lisada uue ... võib seda teha, kirjelduse formaat on kirjeldatud rahvusvahelise standardi dokumentatsioonis.
%% 
%% % Ehk alustada sellega: Kuna töö kasutab läbivalt sõna 'morfoloogia' natuke teistsuguses tähenduses kui see tavaliselt keeleteaduses kannab, tuleb selle tähendust kõigepealt lahti seletada. . Töö loobub morfeemist kui keele väikseim tähenduslik üksus ja seega ei tähenda siin töös morfoloogia morfeemilist morfoloogiat. Aronoff on pööranud tähelepanu tõigale, et 'morfoloogia' kannab teistsugust tähendust keeleteaduses kui ta seda kannab teistes teadusharudes. Enamik teadusharudes tähendab 'morfoloogia' umbes "võimalike vormide uurimine-kaardistamine". Keeleteaduses hoopis ...
%% 
%% %Aronoff on pööranud tähelepanu tõigale, et termin \emph{morfoloogia} on keeleteaduses alati erinenud selle tavatähendusest 'õpetus vormidest', mida termin kannab nt bioloogias ja geoloogias. Terminil on kitsam tähendus keele\-teaduses: 'grammatika see osa, mis tegeleb sõnavormide muutmisega ja sõna\-moodustamisega' ja ka filoloogias: 'keele grammatilise struktuuri üldised reeglid'. (\cite[1]{aronoff_morphology_1996})
%% % Aronoffi puänt on see, et miks kitsamalt ainult \emph{sõnade} morfoloogia, miks mitte ülejäänud keeleline vorm?
%% 
%% % Elektroonne vormisõnastik moodustab seega iga lekseemi jaoks relatsiooni ehk seose $(tsitaatvorm, {muutevormid koos vastava grammatilise iseloomustusega})$.
%% %Sellist vormisõnastikku võib moodustada erinevatel viisidel. Näiteks leksikaalse andme\-baasina, kus iga lekseemi puhul on nenditud kõik selle muutevormid koos vastava grammatilise iseloomustusega, või näiteks reeglite komplektina, mida rakendades saab koostada lekseemi muutvorme vastavalt nende grammatilistele iseloomustustele.
%% 
%% %Matemaatilises mõttes kujutab vormisõnastik vaid \textit{seost} muutevormide ja nende vastavate grammatiliste iseloomustuste vahel. 
%% 
%% % arvutimorfoloogia kui arbitraarne valik realiseerimaks vormisõnastiku sisu (kas info pakkimismehanism või lingvistiliselt motiveeritud)
%% %Arvutimorfoloogiad võivad seda seost (või vormisõnastiku funktsionaalsust) realiseerida arvutuslikult erinevatel viisidel ja ei pea olema lingvistilis-grammatiliselt motiveeritud. Kuna üks täielik vormisõnastik on mahult niivõrd suur (kui mitte lõpmatult suur), on selle mahu kompaktsem ja ülevaatlikum esitus peamiseks motivatsiooniks organiseerida selle koostamise reeglite abil, mis on ühel või teisel moel põhjendatud lingvistiliste-grammatiliste reeglipärasustega.
%% 
%% 
%% %\subsubsection{Morfoloogia formaalsete teooriate lingvistiline motiveeritus}
%% % pilk ajalukku (mida hiljem ümberlükata kui strukturalistlik-morfeemiline, aga mis on jätnud jälje ka arvutimorfoloogiate mõtemaailmale kui lingvistiliselt realistlikud)
%% %Eelmise sajandi keskpaiku jagas Charles Francis Hockett kõik seni Ameerikas sajandi algusest saadik ilmunud grammatikad kahe üldise mudeli järgi, IA (ingl. \textit{Item-and-Arrangement}, üksus ja distributsioon v järjestus v korraldus) ja IP (ingl. \textit{Item-and-Process}, üksus ja protsess ehk protsessi\-morfoloogia). Kõrvalmärkusena tõi ta välja ka kolmanda, ``vanema ja väärikama'' mudeli, WP (ingl. \textit{Word-and-Paradigm}, sõna ja paradigma), aga jättis selle oma käsitlusest välja (\cite[210]{hockett_two_1954}). Hockett võrdleb IA ja IP mudelite eeliseid ja argumenteerib, et IA toonane populaarsus seisneb eeskätt selles, et ajastu eelistab formaalseid mudeleid. Kuna IA-mudel oli juba formaliseeritud tahtis Hockett nüüd formaliseerida sellest vanema IP-mudeli (\cite[214]{hockett_two_1954}) ning sellest sai hiljem, Fred Karlssoni sõnade järgi, generatiivse lingvistika peamiseks mudeliks (\cite[126]{karlsson_uldkeeleteadus_2002}).
%% 
%% %IP-mudel põhineb (morfoloogilise) protsessi mõistel, millega ühest algvormiks valitud kujust (ingl. \textit{base}) luuakse teine vorm (\cite[210]{hockett_two_1954}). IA tekkis vastureaktsioonina IP protsessi\-mõiste suunalisusele -- enam ei tahetud tõsta esile üht vormi algsemaks teistest vormidest (\cite[211]{hockett_two_1954}). IA põhineb morfeemi mõistel, mida Hockett iseloomustab kui keele väikseimat grammatiliselt olulist üksust, ja selle distributsiooni määramisel (\cite[212]{hockett_two_1954}). Hocket nendib, et ka IA mudeli puhul tuleb siiski teha kohati suvalisi valikuid selle üle, mis kuulub morfeemi tasandile ja mis kuulub distributsiooni tasandile (\cite[212]{hockett_two_1954}).
%% 
%% % Stumpi neljamõõtmeline jaotus
%% Gregory Stump on arendanud Hocketti IP ja IA kaheks\-jagamise klassifikatsiooni edasi tänapäevaste morfoloogiliste teooriate põhjal. Nimetades IAd ümber leksikaalseks (ingl. \textit{lexical}) ja IPd inferentsiaalseks (ingl. \textit{inferential}) lisab ta klassifikatsioonile veel sisemise telje: inkrementaalsed (ingl. \textit{incremental}) ja realiseerivad (ingl \textit{realizational}) teooriad. (\cite{stump_inflectional_2001}, lk 1-2)
%% 
%% Inkrementaalsete teooriate järgi lisandub iga (olgu IA puhul leksikaalselt loetletud või IP puhul inferentsiaalse reegliga tuletatud) morfosüntaktilise tunnuse puhul sõnale ka selle vormiline eksponent (\cite[2]{stump_inflectional_2001}). Vormilised eksponendid on üks-üheses seoses grammatiliste tunnustega ja need väljenduvad ükshaaval elik inkrementaalselt.
%% 
%% Realiseerivate teooriate juures ei pea vormiline eksponent iga morfosüntaktilise tunnuse puhul eraldi ja koheselt väljenduma, vaid vormiline väljendus võib realiseeruda tunnuste suuremate komplektide puhul või üldse kui sõna kõik tunnused on teada (\cite[2]{stump_inflectional_2001}).
%% 
%% Realiseerivad teooriad võimaldavad niiviisi suurema paindlikkuse vormiliste väljendujate \textit{realiseerimisel}, loobudes vormiliste väljendujate üks-ühesest seosest morfosüntaktiliste tunnustega.
%% 
%% Stumpi jagab oma klassifikatsiooni järgi Lieberi morfoloogilise teooria leksikaalseks ja inkementaalseks. Halle ja Marantzi distributsioonilise morfoloogia teooria leksikaalseks ja realiseerivaks. Steele'i artikuleeritud morfoloogia teooria esindab inferentsiaalset ja inkrementaalset suunda. (\cite[2--3]{stump_inflectional_2001}).
%% 
%% Stumpi enda ja Matthewsi, Zwicky ning Andersoni teooriaid nimetab ta WP teooriateks, mis on inferentsiaalsed ja realiseerivad (\cite[3]{stump_inflectional_2001}).
%% % TODO siin oleks hea pöörata KLASSIKALISELE PARADIGMAATILISELE MORFOLE
%% 
%% % Finally, Word-and-Paradigm theories of inflection (e.g. those proposed
%% % by Matthews (), Zwicky (a), and Anderson ()) are of the
%% % inferential–realizational type. In inferential–realizational theories, an
%% % inflected word’s association with a particular set of morphosyntactic prop-
%% % erties licenses the application of rules determining the word’s inflectional
%% % form; likes, for example, arises by means of a rule appending -s to any verb
%% % stem associated with the properties ‘sg subject agreement’, ‘present tense’,
%% % and ‘indicative mood’.
%% 
%% Robert Beard on nimetanud ülaltoodud viimaste autorite arendatud realiseerivaid teooriaid eru-morfoloogiaks (ingl. \textit{'split' morphology}) (\cite[20]{beard_morpheme_1987}) ja pakkunud välja morfoloogia veel võimsama eraldamise, mis põhineb tema morfoloogia lahususe hüpoteesil (ingl. \textit{Separation Hypothesis}) (\cite{beard_lexeme-morpheme_1995}).
%% 
%% % ird-morfoloogia VS morfoloogia lahususe hüpotees
%% Morfoloogia lahususe hüpoteesil põhinevate teooriate ja realiseerivate (eru-)morfoloogia\-teooriate vahe on fundamentaalne ja lähtub nende käsitlusest süntaksi ja semantika vahekorrast. Kõige ilmekalt paistab nende vahe morfeemi definitsioonis, küsimuses kas morfeem on keele väikseim vormiline tähenduslik üksus või mitte.
%% 
%% Beardi teoorias ei ole morfeem grammatiliselt tähenduslik, vaid defineeritud kui mistahes muutusena lekseemi fonoloogilises kujus (\cite[31]{beard_morpheme_1987}). Seega on tema teoorias ainult lekseemid tähenduslikud märgid ning grammatilised afiksid (morfeemid) on seda vaid sattumuslikult (\cite[17]{beard_morpheme_1987}).
%% 
%% Käesolevas magistritöös rakendatud ekstraktmorfoloogia on oma organisatsiooni suhtes sõna ja paradigma mudel, aga selle käsitus morfeemist on lähedasem Beardi teooriale.
%% 
%% \subsubsection{Arvutimorfoloogiate lingvistiline motiveeritus}
%% 
%% % Karttust ja Koskenniemit tuua mängu sisse alles siis, kui on vaja näidata kuidas arvuti poolel on hakatud asju tõlgendama enda mõtteviisi järgi
%% Arvutilingvistikas on arvutimorfoloogiat üldiselt organiseeritud klassikalise morfeemi\-käsituse järgi. Seda ilmestab hästi 
%% % Karttuneni väljakutsed arvutimorfoloogias
%% Lauri Karttunen, kes nendib inimkeele mudeldamise puhul arvutimorfoloogias kaks väljakutset: 1)\nobreakspace morfotaktika ehk sõnast väiksemate üksuste kombineerumine ja 2)\nobreakspace morfoloogilised vaheldused ehk sõnast väiksemate üksuste kuju olenemine nende ümbritsevast kontekstist (\cite{karttunen_computing_2003}).
%% 
%% Mille mõlemad väljakutsed viitavad otseselt klassikalisele morfeemi\-käsitusele.
%% 
%% Karttuneni artikkel on vastus Stumpi teooriale ja ta näitlikustab selles kuidas Stumpi teooria on võimalik rakendada kasutades lõplike automaatide formalismi.
%% 
%% Karttunen toob välja olukorra, et arvuti\-morfoloogiad põhinevad arvutuslikel formalismidel, millega nad implementeerivad morfoloogiaid ja mitte ei põhine otse mingil lingvistilisel teoorial. Ta ütleb et morfoloogia\-uurija üllitiste peamine eesmärk on olla veenev, et tema teooria annab läbinägelikuma (ingl. \textit{insightful}) ja elegantsema kirjelduse kui teised teooriad ja formalismid (\cite[2]{karttunen_computing_2003}). Praktilised küsimused nagu sõnavaraline katvus, arvutus\-kiirus ja mälu\-maht ei ole relevantsed akadeemilisele morfoloogia\-uurijale (\cite[2]{karttunen_computing_2003}).
%% 
%% Seega võib öelda, et arvutimorfoloogia on laiem kui lingvistiline morfoloogia, kuna esimest ei piira mitte teooria, vaid arvutusliku meetodi võimsus. Karttunen tõestab artiklis, et Stumpi inferentsiaalne-realiseeriv teooria on taandatav lõplike automaatide formalismi arvutusvõimsusele.% (\cite{karttunen_computing_2003}). % TODO vaata kas viide tõesti hõlmab tervet paragrahvi
%% 
%% Sellest võib järeldada, et arvutilingvistikas on lingvistilise teooria roll pigem olla ajendiks kui tõetruuks postulaadiks, kuigi kindlasti on teooria ja selle implementatsioonilise praktika vahekord raskesti eraldatavad ja ajas muutuvad. Kuigi tendentsi tõetruuduse vähenemisele võib siiski täheldada tänapäeval ka Kimmo Koskenniemi töös, kus ta on hiljuti oma taandatud kahetasemelises morfoloogiamudelis püüdnud morfofoneemi mõiste juures loobuda selle tähendusliku külje lingvistilisest realismist, omastades seda puhtalt vormile:
%% \begin{quote}
%%   ``\textit{Morphophonemes} are represented just as the \textit{combinations of the corresponding letters} (or phonemes) which we can observe in the surface forms. On the one hand, such an interpretation of morphophonemes is crude, but on the other hand, it is a fact that anybody can observe.'' (\cite[157]{koskenniemi_informal_2013})
%% \end{quote}
%% 
%% \subsubsection{Sügavam epistomoloogiline põhjus: formaalse lingvistika lingvistiline motiveeritus}
%% % induktsioon ja deduktsioon
%% Sügavama epistemoloogilise põhjuse, miks arvutimorfoloogiaid on ajendanud pigem lingvistiline motivatsioon ja mitte arvutusteoreetilised võimalused, arvab siinkirjutaja leiduvat strukturaalse lingvistika formaliseerimisperioodi alguses, mis algas enne arvutusmasinate leiutamist (1940.--1960.-ndateil aastatel) ja ammu enne arvutite arvutus- ja mälumahtuvuse võimsuse plahvatuslikku suurenemist (1980.--2000.-ndail). % TODO lisada Karttuneni ajalooline ülevaade fst morfoloogia arengutest
%% 
%% % lingvistilise teooria formaliseerimine
%% Formaalseid teooriaid ja seega teooriate formaliseerimist peetakse teaduse lipulaevaks (\cite[2026]{auroux_history_2006}). Teooriate formaliseerimis\-protsessi jagab Pieter Seuren neljaks etapiks, kus esimene koosneb uuritava ainese tüüpide (ehk kategooriate) leidmisest ning nendele esitus\-kuju määramisest (\cite[2027]{auroux_history_2006}). (Teisisõnu tegeleb see \textit{type-token distinction}'i probleemiga). Teine etapp käib sageli käsi-käes esimese etapiga ja hõlmab tüüpide taksonoomia määramist, ehk selle määramist, mis andmed kuuluvad mis tüübi alla millal ja mis tingimustes (\cite[2027]{auroux_history_2006}). Kolmas etapp koosneb struktuuri määramisest tüüpide esinemisele, elik kuidas kategooriaid on võimalik omavahel kombineerida (\cite[2027 jj]{auroux_history_2006}) näiteks puu- või sõltuvus\-struktuuride abil. Neljas ja viimane etapp koosneb ühe ennustava ja kirjeldava väärtusega formaalse teooria ülesseadmisest algoritmina ehk sammsammulise tegevusjuhisena (\cite[2031]{auroux_history_2006}).
%% %final Stage 4, which consists in the setting up of a formal predictive and explanatory theory that has the precision of an algorithmic procedure.
%% 
%% % arvutimorfoloogia on arvutiprogramm mis on formaalne aparaat
%% Arvutimorfoloogia on arvutiprogramm (või mitme programmi komplekt), mis tahest-tahtmata hõlmab seelaadset formaalset sammsammulist tegevusjuhist.
%% 
%% % probleem asub 3 ja 4 etapi vahel -- kas teha deduktiivselt või induktiivselt?
%% Probleem, miks arvutimorfoloogiad juhinduvad lingvistilistest teooriatest ja mitte puht-arvutuslikest võimalustest asub formaliseerimis\-protsessi 3. ja 4. etapi vahel. Millisel viisil tuleb põhjendada struktuuri määravaid reegleid?
%% 
%% % zellig
%% % Generatiivse lingvistika suurkuju Noam Chomsky juhendaja Zellig Harris kirjutab
%% Zellig Harris (kes oli Noam Chomsky juhendaja) kirjeldab oma \textit{magnum opus} teoses grammatika formaliseerimise lähenemist, mis põhjendab strukturaalsete reeglite määramise ühe formaalse avastamis\-menetluse abil keeleainese korpus\-esinemustest. See on, formaalse teooria sammsammulised reeglid tuletatakse puhtalt struktuuride esinemistest korpusanalüüsi teel. Selline väga töömahukas grammatika loomise menetlusviis sai tema kaasaegsetelt kõva kriitikat olles nii ilmselgelt ebarealistlik ja ebapraktiline. Harris oli tundlik kriitikale ja mainib oma raamatu lõpus viisi, kuidas korpus\-esinemustest eraldi püstitatud reegleid saab hoopis vastupidises suunas \textit{testida} korpustekstide peal. See pani aluse generatiivsele grammatikale, mida arendas edasi tema kasvandik Noam Chomsky teoses \textit{Syntactic Structures} (\citeyear{chomsky_syntactic_1957}). (\cite[2031]{auroux_history_2006}).
%% 
%% % seletav tekst miks ma jauran
%% Eelnevaga olen ma tahtnud öelda seda, et arvutimorfoloogiate koostamis\-põhimõtted põhineda morfeemil ja morfotaktilistel reeglitel ja mitte puhtalt muutvormide nentimisel korpuse põhjal, on eeskätt ajalooliste traditsioonide järjepidevus. Käesolev töö ei järgi neid traditsioone.
%% 
%% See traditsioon on kristalliseerunud ka pealkirjas ``This volume grows out of a special session that we organized at the January 2009 Annual Meeting of the Linguistic Society of America entitled ``Computational Linguistics: Implementation of Analyses against Data''.'' (\cite{bender_computational_2010}).
%% 
%% % Matthews kirjeldab WP ``mudelit'' kui ... ja Karttuneni järgi kronoloogia Zwicky kaudu Stumpini, aga me ei peagi laskuma WP mudeli arvutusliku külge juurde -- see on väga lihtsal moel lahendatud ekstraktmorfoloogias. Ja Karttuneni konstateering, et Stumpi mudel on arvutuslikult samaväärne FSTga. Ja Roark ja Sproat argumenteerivad, et X ja Y on samuti taandatavad FST-le ja seega samaväärsed. Olen valmis oma argumentatsiooniga siin peatükis, et arvutimorfoloogiad võivad vormisõnastiku funktsionaalsuse realiseerida ilma lingvistilise motivatsioonita. Nagu Stump arvab, et tuleb eelistada ``A theory of inflectional morphology must be preferred to the extent that it minimizes any dependence on theoretical distinctions which are not empirically motivated.'' (\cite{stump_inflectional_2001} lk 9).
%% 



%\subsection{Muuttüüp, tüüpsõna ja sõnatüüp}

%Selles töös kasutatakse termineid \emph{muuttüüp}, \emph{tüüpsõna} ja \emph{sõnatüüp} sama tähendusega ja osutab lekseemi kõiki muutvorme. \emph{Tüüpsõna} on valitud lekseem, millega nimetatakse üht \emph{sõnatüüpi}. \emph{Sõnatüüp} on seega tüüpsõnast üldisem mõiste ja tähistab muutvormide koostamismalle, millega saab moodustada sellesse sõnatüüpi kuuluvate sõnade kõiki muutvorme.
%% 
%% % seda selle tõttu, sest töö põhined muuttüübil
%% % heal lapsel mitu nime, paar sõna 
%% 
%% Eesti traditsiooni järgi on muuttüüp tüüpsõnast üldisem. Kuidas siin töös terminoloogiliselt ümber käia, kas \textit{muuttüüp} või \textit{tüüpsõna\-mall}?
%% 
%% Muuttüübistik sõltub selle aluseks võetud klassifikatsioonist, ekstrakt\-morfoloogiat võiks vaadata kui lihtsalt üht väga formaalselt defineeritud muuttüübistikku.
%% 
%% Huldenil on omakorda üks väga formaalne viis, kuidas vähendada ekstrakt\-morfoloogiaga leitud muuttüüpide arvu. Kas see on hoopis muuttüübistik?
%% 
%% 





\newpage

\section{Vormisõnastiku sisestamine Morfoloogialabori rakenduse abil}
\label{sec:vormisõnastiku-sisestamine}

Morfologilabbet\footnote{(\url{https://spraakbanken.gu.se/morfologilabbet/?lexicon=votiska&lang=eng})} ehk Morfoloogialabor on eraldiseisev veebi\-rakendus morfoloogiliste sõnaraamatute koostamiseks. Rakendus on loodud prototüübina Språkbanken'i ehk Rootsi keele\-ressursside keskuse juures, et integreerida ekstrakt\-morfoloogia meetodi nende leksikaalsesse taristusse Karp.

Morfoloogialabori rakendus kasutab ekstraktmorfoloogia meetodit (mida tutvustatakse peatükis~\ref{sec:ekstraktmorfoloogia-meetod}) ja lihtsustab uute sõnade lisamist vormisõnastikku. Rakendus ennustab sisestatud sõnale sobivaid tüüpsõnu ja moodustab tüüpsõnale vastavalt kõik muutvormid. Kasutaja peab kontrollima, kas kõik moodustatud muutvormid on õiged või valima teise muuttüübi.  Rakendus on ühendatud korpuspäringute süsteemiga KORP ja näitab muutvormi juures kasutajale selle esinemis\-arvu korpustes. Joonisel~\ref{fig:morfologilabbet} näidatakse rakenduse ekraani\-kuva sõna \vadja{čiutto} sisestamise juures.

\begin{figure}[h]
  \center
  \setlength{\fboxsep}{0pt}
  %\setlength{\fboxrule}{5pt}
  \fbox{\includegraphics[width=.8\textwidth]{morfologilabbet}}
  \caption{Morfoloogialabori sisestamise vaade sõna \vadja{čiutto} tüüpsõna \vadja{hattu} järgi moodustatud muutvormide ja nende korpus\-esinemiste arvuga.}
  \label{fig:morfologilabbet}
\end{figure}

Magistritöö jooksul on koostatud ja kasutatud kaht vadja korpust: kirjakeelset korpust, mis koosneb Heinsoo õppe\-materjalide tekstidest ja murdeid kajastavat korpust, mis koosneb \enquote{Vadja keele sõnaraamatu} (\cite{grunberg_vadja_2013}) näitelausetest. Korpused on päritavad Eesti Keeleressursside Keskuse korpuspäringute süsteemist KORP\footnote{\url{https://korp.keeleressursid.ee/?mode=vadja}} ja avaldatud internetis (\cite{kankainen_keeleleek/votic-corpora:_2018}).

%Morfoloogialaboriga salvestatud andmed ja nende kuju vastab suuresti pt~\ref{sec:lmf} kirjeldatule.
Siin\-kirjutajal oli au viibida Språkbanken'is morfoloogia\-labori ehitamise ajal (2018.~a kevadel) ja sai panustada rakendusega salvestatud andmete kuju ja vormistada seda vastamaks rahvus\-vahelisele standardile Lexical Markup Framework, mida kirjeldatakse peatükis~\ref{sec:lmf}.


Magistritöö vadja vormisõnastiku sisestamise töövoog on olnud järgmine:

\begin{enumerate}[noitemsep]
\item sisestada lekseemi põhivorm või teine muutvorm
\item valida olemasolevatest tüüpsõnadest õige
\item tüüpsõna puudumisel sisestada lekseemi kõik muutvormid
\end{enumerate}


Punkti (3) juhul loob rakendus ekstrakt\-morfoloogia abil uue sõnatüübi, mis on juba järgmise sisestatava lekseemi puhul süsteemi poolt ennustatav ja valitav.

Töö algusfaasis ei olnud võimalik valida lekseemidele õiget tüüpsõna selle puudumise tõttu ning on pidanud sisestama kõik muutvormid. Töö jooksul on tüüpsõnade arv kasvanud ja rakenduse abil on saanud valida õige tüüpsõna, mis on töö\-protsessi kiirendanud. Tihti on ka olnud abiks võimalus valida nt vokalismi suhtes vale tüüpsõna ja genereeritud muutvormides ainult korrigeerida vokalismi kajastavad osad.

Kuivõrd magistritöö jooksul on vadja tüüpsõnu ühtlustatud, on seda tehtud tagant-järgi valides juba salvestatud sõnadele uue tüüpsõna. Kuna seda on pidanud tegema kõikide sõnade puhul, mis kuuluvad uue, ühtlustatud tüüpsõna alla, on see olnud aega\-nõudev protsess. Ühtlustamise ajal tekkinud mõtted, kuidas seda protsessi lihtsustada ja kiirendada, on edastatud Språkbanken'i arendajatele.


\newpage
\section{Ekstraktmorfoloogia meetod}
\label{sec:ekstraktmorfoloogia-meetod}

%Ekstraktmorfoloogia on juhendatud masinõppe meetod, mis (\cite[14]{forsberg_what_2016}) on välja pakkunud kui lihtsama ja loomulikuma viisina lingvistil määratleda arvutimorfoloogia, kui seda on juhinduda morfoloogilistest-grammatilistest kirjeldustest.
% Meetodit on lähemalt tutvustanud \cite{ahlberg_semi-supervised_2014} ja \cite{ahlberg_paradigm_2015}. % need viited tulevad ju allpool välja

%Meetod koosneb kahest komponendist: \emph{sõnatüübi eraldamise meetodist} ja tundmatule sõnale õige \emph{sõnatüübi ennustamise meetodist}.
% TODO kuhu? Magistritöös rakendatakse sõnatüübi eraldamise osa ja ennustamist ainult kaudselt

% ------

% mis on ekstmorf
Ekstraktmorfoloogia on juhendatud masinõppe meetod, mis koosneb kahest komponendist: \emph{tüüpsõna eraldamise meetodist} ja tundmatule sõnale õige \emph{tüüpsõna ennustamise meetodist}. Meetod on \textit{juhendatud}, sest sisestatud andmed ehk muutvormi\-tabelid peavad olema korrektsed.

% TODO mis on sõnatüüp? (tegelt juba pt 2.6)

Meetod on välja pakutud kui lihtsama ja loomulikuma viisina lingvistil määratleda arvutimorfoloogia, kuna see põhineb lekseemide sõnavormide esitamisel ja ei juhindu morfoloogiliste-grammatiliste reeglite kirjutamisest (\cite[14]{forsberg_what_2016}).

% mis on eraldamine
Tüüpsõna eraldamise (ekstraheerimise) korral eraldab meetod lekseemi muutvormide tabelist selle \glslink{tehniline-tüvi}{tehnilise tüve} ja \glslink{tüüpsõnamall}{tüüpsõnamalli}. Meetodi seda osa kirjeldatakse lähemalt pt-s~\ref{sec:ekstraktmorfoloogia-eraldamine}.
%Eraldatud sõna\-tüübi\-mallide põhjal tuletatakse magistritöös programmkoodi, mida kirjeldatakse lähemalt pt-s~\ref{sec:programmkoodi-tuletamine}.

% mis on ennustamine
Eraldatud tüüpsõnamallide ja tehniliste tüvede põhjal on võimalik üldistada nende iseärasusi ja luua statistiline ennustus\-mudel. Ennustus\-mudeliga on võimalik määrata tundmatu sõnavormi kuuluvust ühe või teise tüüpsõna alla. Meetodi seda osa kirjeldatakse lähemalt pt-s~\ref{sec:ekstraktmorfoloogia-ennustamine}.

Magistritöös kasutatakse ekstraktmorfoloogia meetodit eesmärgiga, et luua vadja keelele keele\-tehnoloogiat. Selleks otstarbeks tuletatakse eraldatud tüüpsõna\-mallide põhjal programm\-koodi, mis teostab mallidest tuleneva sõnavormide analüüsimise ja sünteesimise võime. Programm\-koodi automaatset tuletamist kirjeldatakse pt-s~\ref{sec:programmkoodi-tuletamine}.


% See osa kirjeldab töös rakendatud ekstraktmorfoloogia meetodit. Töö kasutab ekstraktmorfoloogiat kaheks otstarbeks, esiteks vadja keele morfoloogiliste tüüpsõnade väljaselgitamiseks ja kirjeldamiseks ja teisalt programmkoodi automaatseks tuletamiseks saadud kirjelduse põhjal. Neid kahte rakendust kirjeldatatakse lähemalt vastavates peatükkides \textit{\nameref{sec:analüüs}} ja \textit{\nameref{sec:programmkoodi-tuletamine}}.




\subsection{Tüüpsõnamalli eraldamine}
\label{sec:ekstraktmorfoloogia-eraldamine}
% võiks lihtsalt mainida, et opereerib tähtkoostisel
Tüüpsõna eraldamise mehhanism põhineb tehnilise tüve tuvastamisel. Tehniline tüvi moodustub nendest tähtedest, mis ilmnevad lekseemi igas muutvormis ja võib olla mitme-osaline.

Veel ilma detailidesse takerdumata näitlikustatakse siinkohal lugejale meetodi sisendit ja väljundit. Sisendiks on ühe lekseemi muutvormitabel tervikuna (tabel~\ref{tab:sisendtabel-katto}). Väljundiks on meetodi poolt leitud tehniline tüvi ja tüüp\-sõna\-mall, mille põhjal on võimalik moodustada sisendi muutvormi\-tabel (tabel~\ref{tab:väljundtabel-katto}). Tabelitele viidatakse alljärgnevas tekstis mitmel korral.

\spacing{1.0}
\begin{table}[H] %[!htbp] % kuvab tabelit definiitselt enne neile järgnevat teksti
      \footnotesize
  \begin{minipage}[t]{.40\textwidth}
%    \centering
    \begin{tabular}[t]{l l}
      muutvorm            & vormiinfo \\ \hline
      \textit{hattu}      & \textsc{sg nom} \\
      \textit{hatud}      & \textsc{pl nom} \\
      \textit{hatu}       & \textsc{sg gen} \\
      \textit{hattujõ}    & \textsc{pl gen} \\
      \textit{hattua}     & \textsc{sg part} \\
      \textit{hattuit}    & \textsc{pl part} \\
      \textit{hattusõ}    & \textsc{sg ill} \\
      \textit{hattuisõ}   & \textsc{pl ill} \\
      \textit{hattuz}     & \textsc{sg ine} \\
      \textit{hattuiz}    & \textsc{pl ine} \\
      \textit{hatuss}    & \textsc{sg ela} \\
      \textit{hattuiss}  & \textsc{pl ela} \\
      \textit{hatullõ}    & \textsc{sg all} \\
      \textit{hattuillõ}  & \textsc{pl all} \\
      \textit{hatull}      & \textsc{sg ade} \\
      \textit{hattuil}    & \textsc{pl ade} \\
      \textit{hatult}    & \textsc{sg abl} \\
      \textit{hattuilt}  & \textsc{pl abl} \\
      \textit{hatussi}    & \textsc{sg tran} \\
      \textit{hattuissi}  & \textsc{pl tran} \\
      \textit{hattussaa}  & \textsc{sg term} \\
      \textit{hattuissaa} & \textsc{pl term} \\
      \textit{hatuka}     & \textsc{sg com} \\
      \textit{hattuika}   & \textsc{pl com} \\
    \end{tabular}
    \caption{Sisendi muutvormide tabel koos morfo\-loogilise informatsiooniga.}
    \label{tab:sisendtabel-katto}
  \end{minipage}
  \hfill
  \begin{minipage}[t]{.55\textwidth}
    \centering
    \begin{tabular}[t]{l l l}
      tehniline tüvi                        & muutvormi\-mall                                & vormiinfo \\
      \hline
      \underline{hat} t \underline{u}       & $x_1$ $\oplus$ t $\oplus$ $x_2$                & \textsc{sg nom} \\
      \underline{hat}   \underline{u} d     & $x_1$ $\oplus$ $x_2$ $\oplus$ d                & \textsc{pl nom} \\
      \underline{hat}   \underline{u}       & $x_1$ $\oplus$ $x_2$                           & \textsc{sg gen} \\
      \underline{hat} t \underline{u} jõ    & $x_1$ $\oplus$ t $\oplus$ $x_2$ $\oplus$ jõ    & \textsc{pl gen} \\
      \underline{hat} t \underline{u} a     & $x_1$ $\oplus$ t $\oplus$ $x_2$ $\oplus$ a     & \textsc{sg part} \\
      \underline{hat} t \underline{u} it    & $x_1$ $\oplus$ t $\oplus$ $x_2$ $\oplus$ it    & \textsc{pl part} \\
      \underline{hat} t \underline{u} sõ    & $x_1$ $\oplus$ t $\oplus$ $x_2$ $\oplus$ sõ    & \textsc{sg ill} \\
      \underline{hat} t \underline{u} isõ   & $x_1$ $\oplus$ t $\oplus$ $x_2$ $\oplus$ isõ   & \textsc{pl ill} \\
      \underline{hat} t \underline{u} z     & $x_1$ $\oplus$ t $\oplus$ $x_2$ $\oplus$ z     & \textsc{sg ine} \\
      \underline{hat} t \underline{u} iz    & $x_1$ $\oplus$ t $\oplus$ $x_2$ $\oplus$ iz    & \textsc{pl ine} \\
      \underline{hat}   \underline{u} ss    & $x_1$ $\oplus$ $x_2$ $\oplus$ ss               & \textsc{sg ela} \\
      \underline{hat} t \underline{u} iss   & $x_1$ $\oplus$ t $\oplus$ $x_2$ $\oplus$ iss   & \textsc{pl ela} \\
      \underline{hat}   \underline{u} llõ   & $x_1$ $\oplus$ $x_2$ $\oplus$ llõ              & \textsc{sg all} \\
      \underline{hat} t \underline{u} illõ  & $x_1$ $\oplus$ t $\oplus$ $x_2$ $\oplus$ illõ  & \textsc{pl all} \\
      \underline{hat}   \underline{u} ll    & $x_1$ $\oplus$ $x_2$ $\oplus$ ll               & \textsc{sg ade} \\
      \underline{hat} t \underline{u} ill   & $x_1$ $\oplus$ t $\oplus$ $x_2$ $\oplus$ ill   & \textsc{pl ade} \\
      \underline{hat}   \underline{u} lt    & $x_1$ $\oplus$ $x_2$ $\oplus$ lt               & \textsc{sg abl} \\
      \underline{hat} t \underline{u} ilt   & $x_1$ $\oplus$ t $\oplus$ $x_2$ $\oplus$ ilt   & \textsc{pl abl} \\
      \underline{hat}   \underline{u} ssi   & $x_1$ $\oplus$ $x_2$ $\oplus$ ssi              & \textsc{sg tran} \\
      \underline{hat} t \underline{u} issi  & $x_1$ $\oplus$ t $\oplus$ $x_2$ $\oplus$ issi  & \textsc{pl tran} \\
      \underline{hat} t \underline{u} ssaa  & $x_1$ $\oplus$ t $\oplus$ $x_2$ $\oplus$ ssaa  & \textsc{sg term} \\
      \underline{hat} t \underline{u} issaa & $x_1$ $\oplus$ t $\oplus$ $x_2$ $\oplus$ issaa & \textsc{pl term} \\
      \underline{hat}   \underline{u} ka    & $x_1$ $\oplus$ $x_2$ $\oplus$ ka               & \textsc{sg com} \\
      \underline{hat} t \underline{u} ika   & $x_1$ $\oplus$ t $\oplus$ $x_2$ $\oplus$ ika   & \textsc{pl com} \\
    \end{tabular}
    \caption{Väljundi tüüpsõnamall (kus\-juures muutujad $x_1 = $ \textit{hat} ja $x_2 = $ \textit{u} vastab sisendist leitud, allajoonitud tehnilisele tüvele). Tähisega $\oplus$ märgitakse täht\-koostiste \glslink{konkatenatsioon}{kokku\-viimist}.}
    \label{tab:väljundtabel-katto}
  \end{minipage}
\end{table}
\spacing{1.5}

% tähtkoostise põhjal
Ekstraktmorfoloogia meetod eraldab sisendtabelist tüüp\-sõna\-malli ja tehnilise tüve lekseemi muutvormide tähtkoostise põhjal, s.o sõnavormide tähtede põhjal. Meetod arvestab palatalisatsiooniga kuivõird vadja ortograafias seda märgitakse, rõhuga meetod ei arvesta, kuna on sisendis märkimata. Silpide arvuga meetod ei arvesta. % TODO kas saaks paremini see s.o seletus?

% tehniline tüvi
Tehniliseks tüveks loetakse sõna need tähtede jadad, mis esinevad (korduvad) üle kõigi selle muutvormide (allajoonitud tähed tabelis~\ref{tab:väljundtabel-katto}, veerg~1). Tehniline tüvi koosneb ühest või enamast osast (tähtede jadast), antud näite puhul kahest. Sama tüüpsõna alla kuuluvad lekseemid erinevad vaid oma tehnilise tüve poolest. % TODO kas morfeemi definitsiooniga võrdlemine ka kopeerida siia?
Tüüpsõna muutvorme eristab teine-teisest see tähtkoostis, mis ei kordu üle kõigi muutvormide (vrd morfeemi definitsioon pt~\ref{sec:morfeemi-staatus}).

Tehnilise tüve leidmine koosneb kahest etapist: 1)~pikima ühisosajada eraldamine ning 2)~ühisosajada ühene jaotamine muutvormide vahel tehniliseks tüveks. Tüüpsõna koosneb muutvormimallidest ja koostatakse tehnilise tüve jaotuse põhjal, kusjuures üldistatakse tehnilise tüve osad muutujateks ja tüüpsõna funktsiooniks.


\subsubsection{Tehnilise tüve eraldamine}

% tehniline tüvi == LCS
Tehniline tüvi on ekstraktmorfoloogias defineeritud kordse pikima ühisosajadana (\textit{Multiple Longest Common Subsequence}). %, mille tuvastamiseks rakendatakse lõplike automaatide formalismi (\cite{hulden_generalizing_2014}).
Näites (tabel~\ref{tab:sisendtabel-katto}) esinevad üle kõigi muutvormide järgmised tähed: \vadja{k, a, t} ja \vadja{o}. Aga kuna \textit{t}-sid esineb eri muutvormides rohkem kui üks, ei ole selge millele neist vastab pikima ühisosajada täht \textit{t}.  Näite puhul on võimalik kaks pikima ühisosajada jaotust: \textit{hat}~ja~\textit{u} või \textit{ha}~ja~\textit{tu}.

Võimalike mitmesuste lahendamiseks kasutatakse ekstraktmorfoloogias ühestamise heuristikat. %, mida on lähemalt kirjeldanud \cite{hulden_generalizing_2014}.
Ühestamine eelistab mh pikemat esiosa (\cite[33]{hulden_generalizing_2014}), mistõttu valitakse näite puhul tehniliseks tüveks osad \textit{hat}~ja~\textit{u}.
% Üheseks tehniliseks tüveks mitme võimaliku variandi puhul valitakse 
% LCS ambigious
% Muutvormidest võib leiduda mitu erinevat ühisosajada ja nende vahel üheselt  tuleb  ja nendest ühe valimise


\subsubsection{Üldistamine tüüpsõna funktsiooniks}

Tehnilise tüve fikseerimise järel moodustatakse muutvormi\-mallid. Muutvormides asendatakse tehnilise tüve osad muutujatega (tabelis~\ref{tab:väljundtabel-katto}, veerg 2). Näiteks
\begin{align*}
  \textit{hattu} &\rightarrow \underline{\textit{hat}} \oplus \textit{t} \oplus \underline{\textit{u}} \rightarrow x_1 \oplus t \oplus x_2 \nonumber \\
  \textit{hatud} &\rightarrow \underline{\textit{hat}} \oplus \underline{\textit{u}} \oplus \textit{d} \rightarrow x_1 \oplus x_2 \oplus d \nonumber
\end{align*}
vastavalt \msd{sg nom} ja \msd{sg gen} vormidele.

On ilmne et, kui asendatakse muutvormimallides muutujad vastavate tehnilise tüve osadega, siis taas-moodustatakse muutvormid (\vadja{hattu} ja \vadja{hatud}). Muutujaid aga asendades teiste tähtedega, % moodustatakse muude (potentsiaalsete) lekseemide muutvormid,
nt \textit{čiut}~ja~\textit{o}, moodustatakse teise lekseemi, \vadja{čiutto} käändetabel. Näiteks
\begin{align*}
  x_1 \oplus t \oplus x_2 &\rightarrow \underline{\textit{čiut}} \oplus \textit{t} \oplus \underline{\textit{o}} \rightarrow \textit{čiutto} \nonumber \\
  x_1 \oplus x_2 \oplus d &\rightarrow \underline{\textit{čiut}} \oplus \underline{\textit{o}} \oplus \textit{d} \rightarrow \textit{čiutod} \nonumber
\end{align*}
vastavalt \msd{sg nom} ja \msd{sg gen} muutvormi\-mallidele.

Järelikult kuuluvad mõlemad lekseemid sama tüüpsõna alla ja erinevad ainult oma tehnilise tüve poolest.

Lekseemi iseloomustab ekstraktmorfoloogias tüüpsõna ja tehniline tüvi. Tüüpsõna on funktsioon, mille muutuja(te)ks on tehniline tüvi. Tüüpsõna muutvorme eristab teine-teisest see tähtkoostis, mis ei kordu üle kõigi muutvormide (vrd morfeemi definitsioon pt~\ref{sec:morfeemi-staatus}).

Kuna see vajab spetsiifilist teadmist tehnilise tüve moodustamise kohta, ei ole tehnilisel tüvel opereeriv funktsioon sobilik praktiliseks inim-kasutamiseks. On võimalik moodustada kasutamist hõlbustav funktsioon, mille sisendiks on terve sõnavorm.

Tehniline tüvi ei sõltu definitsiooni järgi ühest konkreetsest muutvormist, vaid lekseemi tervest muutvormi\-tabelist (paradigmast). Seetõttu on ekstrakt\-morfoloogia lemma suhtes neutraalne. Lemma valik on teise\-järguline ja seda on võimalik hiljem vahetada. % TODO rohkem sellest kuskil?

Kui lemmaks on valitud kindel muutvorm (nt \msd{sg nom}), on võimalik koostada funktsioon, mille sisendiks on muutvorm tervikuna ja väljundiks on tehniline tüvi. Funktsiooni koostamise aluseks on lemmaks valitud vormi muutvormi\-mall, mille järgi sisend-sõnavormi tähtkoostis jaotatakse ühestamisheuristika põhjal (\cite[572]{ahlberg_semi-supervised_2014}). Niiviisi saab aheldada lemmavormi sisendiks võtva funktsioon kokku tehnilise tüve sisendiks võtva funktsiooniga ja moodustada kasutust hõlbustava funktsiooni, mille sisendiks on sõna lemmavorm ja väljundiks muutvormide tabel. Seda lähenemisviisi kasutatakse magistritöös Grammatical Framework programmkoodi tuletamise juures (pt~\ref{sec:gf}). % TODO aga Giella puhul? kas FST ei tee mitte täpselt seda sama? ... aga pole funktsioon?


%% ----- 
%% ülejäänud tähtkoostis iseloomustab muutvorme ja võrdlus morfeemi definitsiooniga
%%Lekseemi sõnatüüpi iseloomustab sellest eraldatud sõna\-tüübi\-mall, mis koostatakse tehnilise tüve ja muutvormide ülejäänud tähtkoostise põhjal.
%% LIIGUTA Tähtkoostis, mis ei esine üle kõigi muutvormide, loetakse muutvorme teine-teisest eristavaks muutvormi\-malliks (vrd morfeemi definitsioon pt~\ref{sec:morfeemi-staatus}).
%
%% Kuidas sõnatüübimallid koostatakse?
%Sõnatüübi\-mall koosneb muutvormimallidest. Iga muutvormi mall koostatakse eraldatud tehnilise tüve jaotamise
%
%Iga muutvormimall on omakorda moodustatud tehnilise tüve ja ülejäänud tähtkoostise jadadest sel moel, et tehnilise tüve osad on abstraheeritud muutujateks (tabeli~\ref{tab:väljundtabel-katto}, veerg 2).
%
%%  sõnatüübimalli võib vaadata funktsioonina
%Sõnatüüpi ja muutvormimalle võib vaadata funktsioonina, mille muutuja(te)ks on tehniline tüvi. Kui muut\-voormi\-mallides asendada muutujad eraldatud tehnilise tüvega, rekonstrueeritakse sõna muut\-vormid. %Kui muut\-vormi\-malli muutujad asendatakse tehnilise tüvega, siis taastatakse lekseemi muutvorm (nt       \underline{kat} t \underline{o} je    & $x_1$ + t + $x_2$ + je    & \textsc{pl gen}
%
%
%% sama sõnatüübi lekseemid erinevad vaid tüve poolest
%Sama sõnatüübi alla kuuluvad lekseemid erinevad vaid oma tehnilise tüve poolest. %Teised lekseemid, mis käituvad sama sõnatüübi järgi, erinevad ainult tehnilise tüve poolest.
%Näiteks \vadja{katto} (ee~katus) ja \vadja{čiutto} (ee~särk) tehnilised tüved on vastavavalt $x_1 = $ \textit{kat}, $x_2 = $ \textit{o} ja $x_1 = $ \textit{čiut}, $x_2 = $ \textit{o} ning moodustavad nt \msd{pl gen} muutvormimalli järgi sõnavormid.
%
%% 
%\begin{figure}[H]
%  \centering
%  % tee graaf nii, et mall on jagatud mõlemaga: mall < üleval ja all sõnavormid
%
%  $ \mathfrak{katto}(x_1, x_2) = x_1 + \mathtt{t} + x_2 + \mathtt{j\~{o}} = \textup{kat} + \mathtt{t} + \textup{o} + \mathtt{j\~{o}} = \mathit{kattoj\~{o}} $
%  
%  $ \mathfrak{katto}(x_1, x_2) = x_1 + \mathtt{t} + x_2 + \mathtt{j\~{o}} = \textup{čiut} + \mathtt{t} + \textup{o} + \mathtt{j\~{o}} = \mathit{čiuttoj\~{o}} $
%  \caption{Lekseemide \vadja{katto} ja \vadja{čiutto} \msd{sg gen} vormide rekonstrueerimine muutvormimalli ja tehnilise tüvega.}
%  \label{fig:muutvormimalli-rakendamine}
%\end{figure}
%
%
%% meetod eraldab maksimaalseid tüüpe, nt vokalism 'tüttö'
%Kuivõrd sõna \vadja{tüttö} käitub morfoloogiliselt väga sarnaselt siin näitlikustatud sõnatüübiga, eraldatakse selleks siiski omaette sõnatüüp ekstrakt\-morfoloogias. Põhjus on vokalismis, mistõttu erinevad ees- ja tagapoolsed muutvormimallid oma täht\-koostise poolest ja seega ei saa kaks sõnatüüpi kokku langeda (\msd{pl gen} muut\-vormi\-malli puhul \vadja{je} ja \vadja{jõ}).
%
%% maksimaalne ja konkretiseeritud üksikhäälikuteni
%kuna sisendiks on võetud ortograafiline vormisõnastik, kajastab tähtkoostise põhjal eraldatud tähed ka ortograafiat -- nt palatalisatsiooni on, rõhku pole. % TODO
%Ekstraktmorfoloogia opereerib tähtkoostisel ja seetõttu on ka saadud (eraldatud) sõnatüübistik konkretiseeritud üksikhäälikuteni, mida \cite[282]{viks_verbide_1976} peab sõnade muuttüüpide seisukohalt täiesti tarbetuks. Sõnatüübistik sarnaneb E.~Muugi 1933. a VÕSis esitatud tüübistikuga, mis hõlmab 895 eesti keele noomeni\-tüüpi (\cite[282]{viks_verbide_1976}, \cite[282]{viks_muuttüübid_}). Viks ei näe põhjust neid pidada muuttüüpideks, vaid nimetab neid pigem struktuuritüüpideks \cite[282]{viks_verbide_1976}.
%% TODO välja arvatud see, et Muuk arvestab silpide arvu
%
%% info ei kadu -- tehniline tüvi + mall rekonstrueerib sisendi (infot ei kaotata, sobitub mälupõhisele)
%Kõik ekstraktmorfoloogiaga eraldatud informatsioon, s.o sõnatüübi mall ja tehniline tüvi, salvestatakse andmebaasi, mille kuju kirjeldatakse lähemalt pt-s~\ref{sec:lmf}. Kuna salvestatud informatsiooni põhjal on võimalik kõik sisend\-materjal taastada, on ekstrakt\-morfoloogia meetod n.n mitte-destruktiivne. %  või iseloomustada mälupõhisena?
%
%%  kuhu panna viiteid nt Huldeni fst algoritmile ja üldse lähteartiklitele?
%% kuhu panna jutt mallide loetavuse kohta?
%
%% into järgmisse peatükki
%Järgmine peatükk kirjeldab kuidas tehnilise tüve põhjal on võimalik luua statistiline mudel, mille abil saab määrata tundmatu sõnavormi kuuluvust ühe või teise sõnatüübi alla. %  pole stat mudel vaid statistikaga rehkendatud regulaaravaldis!
%
%
%%  tehniline tüvi on käändevormist sõltumata ja seega on ka lemmavorm vabalt valitav

\subsection{Tüüpsõnamalli ennustamine}
\label{sec:ekstraktmorfoloogia-ennustamine}

% eraldamisest eraldiseisev
Tüüpsõna ennustamine on eraldiseisev protsess ja põhineb ainult eraldatud tehniliste tüvede muutujate tähtkoostise analüüsimisel. Ennustamine koosneb piirangute seadmisest tehnilise tüve muutujate kujule.

% kasutatakse ainult morfoloogialabori veebirakenduses, kuigi oleks võimalik keeletehnoloogiasse sisse kodeerida. seda ei tehta, sest esialgu on piirdutud õigekirjakontrollija GF sõnastiku loomisega. kitsendused sarnanevad GF raamistikus levinud Smart Paradigm funktsiooni mõistega ja oleks üks võimalik edasiarenduse suund.
Tüüpsõna ennustamismeetodit on magistritöös kasutatud vaid kaudselt, Morfoloogialabori veebiliidese kaudu vormisõnastiku koostamisel. Seda on tehtud kuna magistritöö eesmärgiks on olnud esmane integreerimine keeletehnoloogilistesse raamistikesse ja õigekirjakontrollija loomine, kus ennustamisvõimet ei ole otseselt tarvis. Ennustamismeetod sarnaneb \textit{Grammatical Framework}'i \textit{Smart Paradigm}-funktsiooni mõistega (\cite{detrez_smart_2012}) ja on magistritöö üks võimalik edasiarenduse suund. Forsberg ja Hulden (\citeyear{forsberg_deriving_2016}) on rakendanud meetodit tundmatu muutvormi tüüpsõna ennustaja (ingl. \textit{morphological guesser}) ehitamiseks, mida Morfoloogialabori veebiliides kasutab uute sõnade sisestamisel neile pakutud tüüpsõnade järjestamiseks kasutajale.

% tehnilise tüve akumuleerimine
Vormisõnastikku salvestatud lekseemid kannavad informatsiooni tehnilise tüve ja tüüpsõnamalli kohta. Mida rohkem lekseeme jagavad üht ja sama tüüpsõna malli, seda rohkem on informatsiooni selle tüüpsõna tehniliste tüvede kujude kohta.

% constrained by reflect morphophonological phenomena and orthographic conventions
Tehnilised tüved ei jaotu tüüpsõnade vahel arbitraarselt, vaid nende kuju peegeldab tihti morfo\-fonoloogilisi nähtuseid ja orto\-graafilisi tavasid. Seda informatsiooni on võimalik kasutada selleks, et seada tüüp\-sõna\-mallide muutujatele kitsendusi ja piirata selle ühtivate tehniliste tüvede hulka, ning tundmatu sõnavormi puhul ennustada selle kuulumist ühe või teise tüüpsõna alla. (\cite[2579]{forsberg_deriving_2016})

Forsberg ja Hulden kirjeldavad statistilist viisi, millega nad loovad tehnilise tüve kuju tähtkoostist kitsendava regulaarse grammatika. Statistik arvutab salvestatud andmete põhjal piirmäära, mida mitte ületades loetakse tehnilise tüve tähtkoostise algus- või lõpposa (või terviklikult) suletud klassiks. Suletud klassi puhul peab leitud tähtkoostis antud positsioonis esinema, et lubada tundmatul sõnavormil ühtida tüüpsõnaga. (\cite[2580]{forsberg_deriving_2016})

Meetodi näitlikustamiseks võib vormisõnastiku põhjal täheldada seda, et kõik \textit{hattu}-tüüpsõna alla kuuluvad lekseemid moodustaksid suletud klassi: nende tehnilise tüve esimene muutuja lõpeb \textit{t}-tähega: \vadja{\textbf{jut}t\textbf{u}, \textbf{hat}t\textbf{u}, \textbf{kat}t\textbf{o}, \textbf{kut}t\textbf{su}, \textbf{laat}t\textbf{o}, \textbf{lant}t\textbf{u}, \textbf{pal̕t}t\textbf{o}, \textbf{port}t\textbf{u}, \textbf{Tart}t\textbf{u}, \textbf{čiut}t\textbf{o}}.


%%\subsection{Võrdlus Eesti leksikograafilise traditsiooniga}
% kuhu alla see paremini paigutada? ARVUTIMORFOLOOGIA EESMÄRGI ALLA?

%% Tavapärane on arvutimorfoloogia koostada reeglite kirjutamise kaudu, seejuures teha kõigepealt klassifikatsioon, mille järgi on leida kõik eri klassid elik üksused milledele rakenduvad samad reeglid.
%% Seda on teinud nt Viks eesti arvutimorfoloogia koostamiseks (mh \cite{viks_verbide_1976}) ja sarnase viisi metodoloogiat on püüdnud formaliseerida Koskenniemi (\cite{koskenniemi_informal_2013}).



\newpage
\section{Vadja ekstraktmorfoloogiliste tüüpsõnamallide esitus ja analüüs}
\label{sec:analüüs}

See osa kirjeldab ekstraktmorfoloogiaga leitud vadja keele morfoloogilised tüüpsõna\-mallid ja jaotab need käänd\-kondadesse. Käändkondade süsteemiks on kasutatud Ariste (\citeyear{ariste_grammar_1968}) grammatikas esitatud põhimõtteid.
%Magistri\-töös on tüüpsõna\-mallid jagatud käändkondadesse ülevaatlikuma esituse nimel, mistõttu on paljud TODO selle võiks lisada!!

Ariste (\citeyear[42]{ariste_grammar_1968}) käändkondade süsteem põhineb muutustel, mis kajastuvad sõnade \msd{nom}, \msd{gen}, \msd{par}, \msd{ill} ja \msd{ela} käändeis. %: \msd{sg nom} ja \msd{pl nom}, \msd{sg gen} ja \msd{pl gen}, \msd{sg par} ja \msd{pl par}, \msd{sg ill} ja \msd{pl ill} ning \msd{sg ela} ja \msd{pl ela}.

Käändkondadesse klassifitseerimine ei ole olnud range, vaid selle eesmärgiks on ühelt poolt olnud koondada kokku sarnaselt käituvaid tüüpsõna\-malle gruppideks ja saavutada üle\-vaatlikuma esituse. Teisalt on see aidanud kaasa ühtlustada Tsvetkovi sõna\-raamatus esinevat suurt varieerumist.  % Seejärel on sõnu gruppide\-siseselt püütud analoogia põhjal ühtlustada Tsvetkovi sõnaraamatus esinevat variatsiooni kirjakeele lihtsama õppimise eesmärgil.
Need sõnad, mida ei ole suudetud klassifitseerida Ariste käändkondadesse, esitatakse eraldi.

%Peatüki viimases osas analüüsitakse kui hästi \cite{silfverberg_computational_2018} esitatud ekstraktmorfoloogia üldiste muuttüüpide algoritm kajastab käändkondi. %  lisa see osa!

Ekstraktmorfoloogia meetodiga on vormisõnastiku 882~sõnast eraldatud 231~erinevat tüüpsõna. Võrdluses Viitso ja Ernštreiti liivi keele sõnastikus (\citeyear{viitso_livoki-estiki-tkisonarontoz_2012}) loetletud 242 tüüpsõnaga, tundub vadja tüüpsõnade arv mõistlik.


%\subsection{Kirjakeele ühtlustamine}
Kõikides käändkondades on ühtlustatud peamiselt lõpuvokaali õ:a vaheldumist ja lõpukao tõttu puuduvad vokaalid on püütud rekonstrueerida. Allpool on detailsemalt kirjeldatud iga käändkonna juures eraldi ühtlustatud komponente.

%  tee endale selgeks mis asi on noomen ja kas komparatiivid kuuluvad selle alla
% TODO sõnavara sisaldab ka pronoomeneid, millel pole defektsed paradigmad
%Noomeni sõnavarast on välja jäetud komparatiivsed vormid (mõlõpi, vanepi). 

% 
Alljärgnevate kirjelduste juurde on lisatud tüüpsõnade muutvormimallid käände\-tabelina, milles on alla\-joonitud tehniline tüvi. Selline esitus võimaldab lugejal paremini näha läbi ekstrakt\-morfoloogia muutvormimalli funktsioneerimist ja moodustada teiste sama tüüsõna alla kuuluvate sõnade käändvorme. Tähisega $\oplus$ märgitakse täht\-koostiste kokku\-viimist.

\subsection{\RN{1} käändkond}

Esimesse käändkonda kuuluvad Ariste (\citeyear[40]{ariste_grammar_1968}) järgi ühetüvelised, ühesilbilised sõnad. Ekstraktmorfoloogia eraldab kõik sellesse käändkonda kuuluvad tüüpsõnad, aga koondab ka mitmesilbilised liitsõnad, mille järelkomponent kuulub siia käändkonda (nt \vadja{pihlpuu} nagu \vadja{puu} ja \vadja{bulipää} nagu \vadja{pää}).

\msd{sg par} vormidele on lisatud lõpuhäälik \textit{-õ} või \textit{-e} vastavalt sõna vokalismile.

Koostatud vormisõnastiku sõnade \msd{sg ill} vorme on esialgselt üsna kunstlikult ühtlustatud kirjakeele jaoks: \textit{pää:pähhe}, \textit{puu:puhhu} ja \textit{maa:mahha} ning diftongiga sõnade puhul \textit{või:võisõ}, \textit{täi:täise}. Erandiks on tüüpsõna \textit{sää}, mis ei ole ühtlustatud nagu \textit{pää} vaid \textit{sää:sähäse}.

Näide Tsvetkovi sõnaraamatus esinevast \msd{sg ill} vormi variatsioonist: \vadja{kuu}:\vadja{kuusõ}, \vadja{üü}:\vadja{ühese}, \vadja{püü}:\vadja{pühe}\textasciitilde \vadja{pühese} ja \vadja{vüü}:\vadja{vühe}\textasciitilde \vadja{vühese}\textasciitilde \vadja{vüüse}.

Ekstraktmorfoloogia eraldatud tüüpsõnade paljusus on ajendatud ühelt poolt vokalismist (muutelõpu erinevus \textit{või}:\textit{võisõ} ja \textit{täi}:\textit{täise}). Teisalt aga pika tüve\-vokaali lühenemise tõttu i-mitmuse vormides (\textit{pääd}:\textit{päije} ja \textit{tüüd}:\textit{tüije}), mistõttu jääb ühe tüüpsõna alla -\textit{ä}-lised muutelõpud ja teise -\textit{ü}-lised.

Üks tähelepanek on \msd{sg} ja \msd{pl} vormide kokkulangemine diftongiga sõnade puhul.

\subsubsection*{Ekstraktmorfoloogia tüüpsõnad}

% counter tüüpsõnamallide numbreerimiseks
\newcounter{mallinumber}
%\addtocounter{mallinumber}

\spacing{1.1}
\vspace{-3.5em}


\vspace{3.5em}
\noindent \begin{minipage}{\textwidth}
\stepcounter{mallinumber}
\noindent \textbf{Tüüpsõnamall \arabic{mallinumber}\,\vadja{maa}}\\

\begin{sideways}
\begin{tabular}{l l}
muutvormimall & tunnused \\
\hline
\underline{ma}\,$\oplus$\,a & \textsc{ sg nom } \\
\underline{ma}\,$\oplus$\,a & \textsc{ sg gen } \\
\underline{ma}\,$\oplus$\,atõ & \textsc{ sg par } \\
\underline{ma}\,$\oplus$\,hha & \textsc{ sg ill } \\
\underline{ma}\,$\oplus$\,az & \textsc{ sg ine } \\
\underline{ma}\,$\oplus$\,ass & \textsc{ sg ela } \\
\underline{ma}\,$\oplus$\,allõ & \textsc{ sg all } \\
\underline{ma}\,$\oplus$\,all & \textsc{ sg ade } \\
\underline{ma}\,$\oplus$\,alt & \textsc{ sg abl } \\
\underline{ma}\,$\oplus$\,assi & \textsc{ sg tra } \\
\underline{ma}\,$\oplus$\,assaa & \textsc{ sg ter } \\
\underline{ma}\,$\oplus$\,aka & \textsc{ sg com } \\
\underline{ma}\,$\oplus$\,ad & \textsc{ pl nom } \\
\underline{ma}\,$\oplus$\,ijõ & \textsc{ pl gen } \\
\underline{ma}\,$\oplus$\,it & \textsc{ pl par } \\
\underline{ma}\,$\oplus$\,isõ & \textsc{ pl ill } \\
\underline{ma}\,$\oplus$\,iz & \textsc{ pl ine } \\
\underline{ma}\,$\oplus$\,iss & \textsc{ pl ela } \\
\underline{ma}\,$\oplus$\,illõ & \textsc{ pl all } \\
\underline{ma}\,$\oplus$\,ill & \textsc{ pl ade } \\
\underline{ma}\,$\oplus$\,ilt & \textsc{ pl abl } \\
\underline{ma}\,$\oplus$\,issi & \textsc{ pl tra } \\
\underline{ma}\,$\oplus$\,issaa & \textsc{ pl ter } \\
\underline{ma}\,$\oplus$\,ika & \textsc{ pl com } \\
\end{tabular}
\end{sideways}
\captionof{table}{Tüüpsõnamall \arabic{mallinumber}\,\vadja{maa} ekstraheeritud muutvormimallid.}
\label{tab:tüüpsõnamall-maa}

\end{minipage}

 
\vspace{1em}
\noindent Tüüpsõnamall \vadja{maa} ei hõlma teisi lekseeme vormi\-sõnastikus.

Tüüpsõna\-mall kirjeldab tagapoolseid -\textit{aa}-lõpulisi sõnu.


\vspace{3.5em}
\noindent \begin{minipage}{\textwidth}
\stepcounter{mallinumber}
\noindent \textbf{Tüüpsõnamall \arabic{mallinumber}\,\vadja{puu}}\\

\begin{sideways}
\begin{tabular}{l l}
muutvormimall & tunnused \\
\hline
\underline{pu}\,$\oplus$\,u & \textsc{ sg nom } \\
\underline{pu}\,$\oplus$\,u & \textsc{ sg gen } \\
\underline{pu}\,$\oplus$\,utõ & \textsc{ sg par } \\
\underline{pu}\,$\oplus$\,hhu & \textsc{ sg ill } \\
\underline{pu}\,$\oplus$\,uz & \textsc{ sg ine } \\
\underline{pu}\,$\oplus$\,uss & \textsc{ sg ela } \\
\underline{pu}\,$\oplus$\,ullõ & \textsc{ sg all } \\
\underline{pu}\,$\oplus$\,ull & \textsc{ sg ade } \\
\underline{pu}\,$\oplus$\,ult & \textsc{ sg abl } \\
\underline{pu}\,$\oplus$\,ussi & \textsc{ sg tra } \\
\underline{pu}\,$\oplus$\,ussaa & \textsc{ sg ter } \\
\underline{pu}\,$\oplus$\,uka & \textsc{ sg com } \\
\underline{pu}\,$\oplus$\,ud & \textsc{ pl nom } \\
\underline{pu}\,$\oplus$\,ijõ & \textsc{ pl gen } \\
\underline{pu}\,$\oplus$\,it & \textsc{ pl par } \\
\underline{pu}\,$\oplus$\,isõ & \textsc{ pl ill } \\
\underline{pu}\,$\oplus$\,iz & \textsc{ pl ine } \\
\underline{pu}\,$\oplus$\,iss & \textsc{ pl ela } \\
\underline{pu}\,$\oplus$\,illõ & \textsc{ pl all } \\
\underline{pu}\,$\oplus$\,ill & \textsc{ pl ade } \\
\underline{pu}\,$\oplus$\,ilt & \textsc{ pl abl } \\
\underline{pu}\,$\oplus$\,issi & \textsc{ pl tra } \\
\underline{pu}\,$\oplus$\,issaa & \textsc{ pl ter } \\
\underline{pu}\,$\oplus$\,ika & \textsc{ pl com } \\
\end{tabular}
\end{sideways}
\captionof{table}{Tüüpsõnamall \arabic{mallinumber}\,\vadja{puu} ekstraheeritud muutvormimallid.}
\label{tab:tüüpsõnamall-puu}

\end{minipage}

 
\vspace{1em}
\noindent Tüüpsõnamall \vadja{puu} hõlmab vormisõnastikus 5 lekseemi: \vadja{\underline{pu}u, \underline{ku}u, \underline{lu}u, \underline{su}u} ja \vadja{\underline{pihlpu}u}.

Tüüpsõna\-mall kirjeldab tagapoolseid -\textit{uu}-lõpulisi sõnu.


\vspace{3.5em}
\noindent \begin{minipage}{\textwidth}
\stepcounter{mallinumber}
\noindent \textbf{Tüüpsõnamall \arabic{mallinumber}\,\vadja{pii}}\\

\begin{sideways}
\begin{tabular}{l l}
muutvormimall & tunnused \\
\hline
\underline{pi}\,$\oplus$\,i & \textsc{ sg nom } \\
\underline{pi}\,$\oplus$\,i & \textsc{ sg gen } \\
\underline{pi}\,$\oplus$\,ite & \textsc{ sg par } \\
\underline{pi}\,$\oplus$\,hhe & \textsc{ sg ill } \\
\underline{pi}\,$\oplus$\,iz & \textsc{ sg ine } \\
\underline{pi}\,$\oplus$\,iss & \textsc{ sg ela } \\
\underline{pi}\,$\oplus$\,ille & \textsc{ sg all } \\
\underline{pi}\,$\oplus$\,ill & \textsc{ sg ade } \\
\underline{pi}\,$\oplus$\,ilt & \textsc{ sg abl } \\
\underline{pi}\,$\oplus$\,issi & \textsc{ sg tra } \\
\underline{pi}\,$\oplus$\,issaa & \textsc{ sg ter } \\
\underline{pi}\,$\oplus$\,ika & \textsc{ sg com } \\
\underline{pi}\,$\oplus$\,id & \textsc{ pl nom } \\
\underline{pi}\,$\oplus$\,ije & \textsc{ pl gen } \\
\underline{pi}\,$\oplus$\,it & \textsc{ pl par } \\
\underline{pi}\,$\oplus$\,ise & \textsc{ pl ill } \\
\underline{pi}\,$\oplus$\,iz & \textsc{ pl ine } \\
\underline{pi}\,$\oplus$\,iss & \textsc{ pl ela } \\
\underline{pi}\,$\oplus$\,ille & \textsc{ pl all } \\
\underline{pi}\,$\oplus$\,ill & \textsc{ pl ade } \\
\underline{pi}\,$\oplus$\,ilt & \textsc{ pl abl } \\
\underline{pi}\,$\oplus$\,issi & \textsc{ pl tra } \\
\underline{pi}\,$\oplus$\,issaa & \textsc{ pl ter } \\
\underline{pi}\,$\oplus$\,jka & \textsc{ pl com } \\
\end{tabular}
\end{sideways}
\captionof{table}{Tüüpsõnamall \arabic{mallinumber}\,\vadja{pii} ekstraheeritud muutvormimallid.}
\label{tab:tüüpsõnamall-pii}

\end{minipage}

 
\vspace{1em}
\noindent Tüüpsõnamall \vadja{pii} ei hõlma teisi lekseeme vormi\-sõnastikus.

Tüüpsõna\-mall kirjeldab eespoolseid -\textit{ii}-lõpulisi sõnu.


\vspace{3.5em}
\noindent \begin{minipage}{\textwidth}
\stepcounter{mallinumber}
\noindent \textbf{Tüüpsõnamall \arabic{mallinumber}\,\vadja{pää}}\\

\begin{sideways}
\begin{tabular}{l l}
muutvormimall & tunnused \\
\hline
\underline{pä}\,$\oplus$\,ä & \textsc{ sg nom } \\
\underline{pä}\,$\oplus$\,ä & \textsc{ sg gen } \\
\underline{pä}\,$\oplus$\,äte & \textsc{ sg par } \\
\underline{pä}\,$\oplus$\,hhe & \textsc{ sg ill } \\
\underline{pä}\,$\oplus$\,äz & \textsc{ sg ine } \\
\underline{pä}\,$\oplus$\,äss & \textsc{ sg ela } \\
\underline{pä}\,$\oplus$\,älle & \textsc{ sg all } \\
\underline{pä}\,$\oplus$\,äll & \textsc{ sg ade } \\
\underline{pä}\,$\oplus$\,ält & \textsc{ sg abl } \\
\underline{pä}\,$\oplus$\,ässi & \textsc{ sg tra } \\
\underline{pä}\,$\oplus$\,ässaa & \textsc{ sg ter } \\
\underline{pä}\,$\oplus$\,äka & \textsc{ sg com } \\
\underline{pä}\,$\oplus$\,äd & \textsc{ pl nom } \\
\underline{pä}\,$\oplus$\,ije & \textsc{ pl gen } \\
\underline{pä}\,$\oplus$\,it & \textsc{ pl par } \\
\underline{pä}\,$\oplus$\,ise & \textsc{ pl ill } \\
\underline{pä}\,$\oplus$\,iz & \textsc{ pl ine } \\
\underline{pä}\,$\oplus$\,iss & \textsc{ pl ela } \\
\underline{pä}\,$\oplus$\,ille & \textsc{ pl all } \\
\underline{pä}\,$\oplus$\,ill & \textsc{ pl ade } \\
\underline{pä}\,$\oplus$\,ilt & \textsc{ pl abl } \\
\underline{pä}\,$\oplus$\,issi & \textsc{ pl tra } \\
\underline{pä}\,$\oplus$\,issaa & \textsc{ pl ter } \\
\underline{pä}\,$\oplus$\,ika & \textsc{ pl com } \\
\end{tabular}
\end{sideways}
\captionof{table}{Tüüpsõnamall \arabic{mallinumber}\,\vadja{pää} ekstraheeritud muutvormimallid.}
\label{tab:tüüpsõnamall-pää}

\end{minipage}

 
\vspace{1em}
\noindent Tüüpsõnamall \vadja{pää} hõlmab vormisõnastikus 2 lekseemi: \vadja{\underline{pä}ä} ja \vadja{\underline{bulipä}ä}.

Tüüpsõna\-mall kirjeldab eespoolseid -\textit{ää}-lõpulisi sõnu, mille \msd{sg ill} vorm on (\textit{pähhe}).


\vspace{3.5em}
\noindent \begin{minipage}{\textwidth}
\stepcounter{mallinumber}
\noindent \textbf{Tüüpsõnamall \arabic{mallinumber}\,\vadja{sää}}\\

\begin{sideways}
\begin{tabular}{l l}
muutvormimall & tunnused \\
\hline
\underline{sä}\,$\oplus$\,ä & \textsc{ sg nom } \\
\underline{sä}\,$\oplus$\,ä & \textsc{ sg gen } \\
\underline{sä}\,$\oplus$\,äte & \textsc{ sg par } \\
\underline{sä}\,$\oplus$\,häse & \textsc{ sg ill } \\
\underline{sä}\,$\oplus$\,äz & \textsc{ sg ine } \\
\underline{sä}\,$\oplus$\,äss & \textsc{ sg ela } \\
\underline{sä}\,$\oplus$\,älle & \textsc{ sg all } \\
\underline{sä}\,$\oplus$\,äll & \textsc{ sg ade } \\
\underline{sä}\,$\oplus$\,ält & \textsc{ sg abl } \\
\underline{sä}\,$\oplus$\,ässi & \textsc{ sg tra } \\
\underline{sä}\,$\oplus$\,ässaa & \textsc{ sg ter } \\
\underline{sä}\,$\oplus$\,äka & \textsc{ sg com } \\
\underline{sä}\,$\oplus$\,äd & \textsc{ pl nom } \\
\underline{sä}\,$\oplus$\,ije & \textsc{ pl gen } \\
\underline{sä}\,$\oplus$\,it & \textsc{ pl par } \\
\underline{sä}\,$\oplus$\,ise & \textsc{ pl ill } \\
\underline{sä}\,$\oplus$\,iz & \textsc{ pl ine } \\
\underline{sä}\,$\oplus$\,iss & \textsc{ pl ela } \\
\underline{sä}\,$\oplus$\,ille & \textsc{ pl all } \\
\underline{sä}\,$\oplus$\,ill & \textsc{ pl ade } \\
\underline{sä}\,$\oplus$\,ilt & \textsc{ pl abl } \\
\underline{sä}\,$\oplus$\,issi & \textsc{ pl tra } \\
\underline{sä}\,$\oplus$\,issaa & \textsc{ pl ter } \\
\underline{sä}\,$\oplus$\,ika & \textsc{ pl com } \\
\end{tabular}
\end{sideways}
\captionof{table}{Tüüpsõnamall \arabic{mallinumber}\,\vadja{sää} ekstraheeritud muutvormimallid.}
\label{tab:tüüpsõnamall-sää}

\end{minipage}

 
\vspace{1em}
\noindent Tüüpsõnamall \vadja{sää} ei hõlma teisi lekseeme vormi\-sõnastikus.

Tüüpsõna\-mall kirjeldab eespoolseid -\textit{ää}-lõpulisi sõnu, mille \msd{sg ill} vorm on (\textit{pähäse}).


\vspace{3.5em}
\noindent \begin{minipage}{\textwidth}
\stepcounter{mallinumber}
\noindent \textbf{Tüüpsõnamall \arabic{mallinumber}\,\vadja{soo}}\\

\begin{sideways}
\begin{tabular}{l l}
muutvormimall & tunnused \\
\hline
\underline{so}\,$\oplus$\,o & \textsc{ sg nom } \\
\underline{so}\,$\oplus$\,o & \textsc{ sg gen } \\
\underline{so}\,$\oplus$\,otõ & \textsc{ sg par } \\
\underline{so}\,$\oplus$\,hho & \textsc{ sg ill } \\
\underline{so}\,$\oplus$\,oz & \textsc{ sg ine } \\
\underline{so}\,$\oplus$\,oss & \textsc{ sg ela } \\
\underline{so}\,$\oplus$\,ollõ & \textsc{ sg all } \\
\underline{so}\,$\oplus$\,oll & \textsc{ sg ade } \\
\underline{so}\,$\oplus$\,olt & \textsc{ sg abl } \\
\underline{so}\,$\oplus$\,ossi & \textsc{ sg tra } \\
\underline{so}\,$\oplus$\,ossaa & \textsc{ sg ter } \\
\underline{so}\,$\oplus$\,oka & \textsc{ sg com } \\
\underline{so}\,$\oplus$\,od & \textsc{ pl nom } \\
\underline{so}\,$\oplus$\,ijõ & \textsc{ pl gen } \\
\underline{so}\,$\oplus$\,it & \textsc{ pl par } \\
\underline{so}\,$\oplus$\,isõ & \textsc{ pl ill } \\
\underline{so}\,$\oplus$\,iz & \textsc{ pl ine } \\
\underline{so}\,$\oplus$\,iss & \textsc{ pl ela } \\
\underline{so}\,$\oplus$\,illõ & \textsc{ pl all } \\
\underline{so}\,$\oplus$\,ill & \textsc{ pl ade } \\
\underline{so}\,$\oplus$\,ilt & \textsc{ pl abl } \\
\underline{so}\,$\oplus$\,issi & \textsc{ pl tra } \\
\underline{so}\,$\oplus$\,issaa & \textsc{ pl ter } \\
\underline{so}\,$\oplus$\,ika & \textsc{ pl com } \\
\end{tabular}
\end{sideways}
\captionof{table}{Tüüpsõnamall \arabic{mallinumber}\,\vadja{soo} ekstraheeritud muutvormimallid.}
\label{tab:tüüpsõnamall-soo}

\end{minipage}

 
\vspace{1em}
\noindent Tüüpsõnamall \vadja{soo} ei hõlma teisi lekseeme vormi\-sõnastikus.

Tüüpsõna\-mall kirjeldab tagapoolseid -\textit{oo}-lõpulisi sõnu.


\vspace{3.5em}
\noindent \begin{minipage}{\textwidth}
\stepcounter{mallinumber}
\noindent \textbf{Tüüpsõnamall \arabic{mallinumber}\,\vadja{tee}}\\

\begin{sideways}
\begin{tabular}{l l}
muutvormimall & tunnused \\
\hline
\underline{te}\,$\oplus$\,e & \textsc{ sg nom } \\
\underline{te}\,$\oplus$\,e & \textsc{ sg gen } \\
\underline{te}\,$\oplus$\,ete & \textsc{ sg par } \\
\underline{te}\,$\oplus$\,hhe & \textsc{ sg ill } \\
\underline{te}\,$\oplus$\,ez & \textsc{ sg ine } \\
\underline{te}\,$\oplus$\,ess & \textsc{ sg ela } \\
\underline{te}\,$\oplus$\,elle & \textsc{ sg all } \\
\underline{te}\,$\oplus$\,ell & \textsc{ sg ade } \\
\underline{te}\,$\oplus$\,elt & \textsc{ sg abl } \\
\underline{te}\,$\oplus$\,essi & \textsc{ sg tra } \\
\underline{te}\,$\oplus$\,essaa & \textsc{ sg ter } \\
\underline{te}\,$\oplus$\,eka & \textsc{ sg com } \\
\underline{te}\,$\oplus$\,ed & \textsc{ pl nom } \\
\underline{te}\,$\oplus$\,ije & \textsc{ pl gen } \\
\underline{te}\,$\oplus$\,it & \textsc{ pl par } \\
\underline{te}\,$\oplus$\,ise & \textsc{ pl ill } \\
\underline{te}\,$\oplus$\,iz & \textsc{ pl ine } \\
\underline{te}\,$\oplus$\,iss & \textsc{ pl ela } \\
\underline{te}\,$\oplus$\,ille & \textsc{ pl all } \\
\underline{te}\,$\oplus$\,ill & \textsc{ pl ade } \\
\underline{te}\,$\oplus$\,ilt & \textsc{ pl abl } \\
\underline{te}\,$\oplus$\,issi & \textsc{ pl tra } \\
\underline{te}\,$\oplus$\,issaa & \textsc{ pl ter } \\
\underline{te}\,$\oplus$\,ika & \textsc{ pl com } \\
\end{tabular}
\end{sideways}
\captionof{table}{Tüüpsõnamall \arabic{mallinumber}\,\vadja{tee} ekstraheeritud muutvormimallid.}
\label{tab:tüüpsõnamall-tee}

\end{minipage}

 
\vspace{1em}
\noindent Tüüpsõnamall \vadja{tee} ei hõlma teisi lekseeme vormi\-sõnastikus.

Tüüpsõna\-mall kirjeldab eespoolseid -\textit{ee}-lõpulisi sõnu.


\vspace{3.5em}
\noindent \begin{minipage}{\textwidth}
\stepcounter{mallinumber}
\noindent \textbf{Tüüpsõnamall \arabic{mallinumber}\,\vadja{tüü}}\\

\begin{sideways}
\begin{tabular}{l l}
muutvormimall & tunnused \\
\hline
\underline{tü}\,$\oplus$\,ü & \textsc{ sg nom } \\
\underline{tü}\,$\oplus$\,ü & \textsc{ sg gen } \\
\underline{tü}\,$\oplus$\,üte & \textsc{ sg par } \\
\underline{tü}\,$\oplus$\,hhe & \textsc{ sg ill } \\
\underline{tü}\,$\oplus$\,üz & \textsc{ sg ine } \\
\underline{tü}\,$\oplus$\,üss & \textsc{ sg ela } \\
\underline{tü}\,$\oplus$\,ülle & \textsc{ sg all } \\
\underline{tü}\,$\oplus$\,üll & \textsc{ sg ade } \\
\underline{tü}\,$\oplus$\,ült & \textsc{ sg abl } \\
\underline{tü}\,$\oplus$\,üssi & \textsc{ sg tra } \\
\underline{tü}\,$\oplus$\,üssaa & \textsc{ sg ter } \\
\underline{tü}\,$\oplus$\,üka & \textsc{ sg com } \\
\underline{tü}\,$\oplus$\,üd & \textsc{ pl nom } \\
\underline{tü}\,$\oplus$\,ije & \textsc{ pl gen } \\
\underline{tü}\,$\oplus$\,it & \textsc{ pl par } \\
\underline{tü}\,$\oplus$\,ise & \textsc{ pl ill } \\
\underline{tü}\,$\oplus$\,iz & \textsc{ pl ine } \\
\underline{tü}\,$\oplus$\,iss & \textsc{ pl ela } \\
\underline{tü}\,$\oplus$\,ille & \textsc{ pl all } \\
\underline{tü}\,$\oplus$\,ill & \textsc{ pl ade } \\
\underline{tü}\,$\oplus$\,ilt & \textsc{ pl abl } \\
\underline{tü}\,$\oplus$\,issi & \textsc{ pl tra } \\
\underline{tü}\,$\oplus$\,issaa & \textsc{ pl ter } \\
\underline{tü}\,$\oplus$\,ika & \textsc{ pl com } \\
\end{tabular}
\end{sideways}
\captionof{table}{Tüüpsõnamall \arabic{mallinumber}\,\vadja{tüü} ekstraheeritud muutvormimallid.}
\label{tab:tüüpsõnamall-tüü}

\end{minipage}

 
\vspace{1em}
\noindent Tüüpsõnamall \vadja{tüü} hõlmab vormisõnastikus 4 lekseemi: \vadja{\underline{tü}ü, \underline{vü}ü, \underline{ü}ü} ja \vadja{\underline{pü}ü}.

Tüüpsõna\-mall kirjeldab eespoolseid -\textit{üü}-lõpulisi sõnu.


\vspace{3.5em}
\noindent \begin{minipage}{\textwidth}
\stepcounter{mallinumber}
\noindent \textbf{Tüüpsõnamall \arabic{mallinumber}\,\vadja{täi}}\\

\begin{sideways}
\begin{tabular}{l l}
muutvormimall & tunnused \\
\hline
\underline{täi} & \textsc{ sg nom } \\
\underline{täi} & \textsc{ sg gen } \\
\underline{täi}\,$\oplus$\,te & \textsc{ sg par } \\
\underline{täi}\,$\oplus$\,se & \textsc{ sg ill } \\
\underline{täi}\,$\oplus$\,z & \textsc{ sg ine } \\
\underline{täi}\,$\oplus$\,ss & \textsc{ sg ela } \\
\underline{täi}\,$\oplus$\,lle & \textsc{ sg all } \\
\underline{täi}\,$\oplus$\,ll & \textsc{ sg ade } \\
\underline{täi}\,$\oplus$\,lt & \textsc{ sg abl } \\
\underline{täi}\,$\oplus$\,ssi & \textsc{ sg tra } \\
\underline{täi}\,$\oplus$\,ssaa & \textsc{ sg ter } \\
\underline{täi}\,$\oplus$\,ka & \textsc{ sg com } \\
\underline{täi}\,$\oplus$\,d & \textsc{ pl nom } \\
\underline{täi}\,$\oplus$\,je & \textsc{ pl gen } \\
\underline{täi}\,$\oplus$\,t & \textsc{ pl par } \\
\underline{täi}\,$\oplus$\,se & \textsc{ pl ill } \\
\underline{täi}\,$\oplus$\,z & \textsc{ pl ine } \\
\underline{täi}\,$\oplus$\,ss & \textsc{ pl ela } \\
\underline{täi}\,$\oplus$\,lle & \textsc{ pl all } \\
\underline{täi}\,$\oplus$\,ll & \textsc{ pl ade } \\
\underline{täi}\,$\oplus$\,lt & \textsc{ pl abl } \\
\underline{täi}\,$\oplus$\,ssi & \textsc{ pl tra } \\
\underline{täi}\,$\oplus$\,ssaa & \textsc{ pl ter } \\
\underline{täi}\,$\oplus$\,ka & \textsc{ pl com } \\
\end{tabular}
\end{sideways}
\captionof{table}{Tüüpsõnamall \arabic{mallinumber}\,\vadja{täi} ekstraheeritud muutvormimallid.}
\label{tab:tüüpsõnamall-täi}

\end{minipage}

 
\vspace{1em}
\noindent Tüüpsõnamall \vadja{täi} ei hõlma teisi lekseeme vormi\-sõnastikus.

Tüüpsõna\-mall kirjeldab eespoolseid diftongiga lõppevaid sõnu.


\vspace{3.5em}
\noindent \begin{minipage}{\textwidth}
\stepcounter{mallinumber}
\noindent \textbf{Tüüpsõnamall \arabic{mallinumber}\,\vadja{või}}\\

\begin{sideways}
\begin{tabular}{l l}
muutvormimall & tunnused \\
\hline
\underline{või} & \textsc{ sg nom } \\
\underline{või} & \textsc{ sg gen } \\
\underline{või}\,$\oplus$\,tõ & \textsc{ sg par } \\
\underline{või}\,$\oplus$\,sõ & \textsc{ sg ill } \\
\underline{või}\,$\oplus$\,z & \textsc{ sg ine } \\
\underline{või}\,$\oplus$\,ss & \textsc{ sg ela } \\
\underline{või}\,$\oplus$\,llõ & \textsc{ sg all } \\
\underline{või}\,$\oplus$\,ll & \textsc{ sg ade } \\
\underline{või}\,$\oplus$\,lt & \textsc{ sg abl } \\
\underline{või}\,$\oplus$\,ssi & \textsc{ sg tra } \\
\underline{või}\,$\oplus$\,ssaa & \textsc{ sg ter } \\
\underline{või}\,$\oplus$\,ka & \textsc{ sg com } \\
\underline{või}\,$\oplus$\,d & \textsc{ pl nom } \\
\underline{või}\,$\oplus$\,jõ & \textsc{ pl gen } \\
\underline{või}\,$\oplus$\,t & \textsc{ pl par } \\
\underline{või}\,$\oplus$\,sõ & \textsc{ pl ill } \\
\underline{või}\,$\oplus$\,z & \textsc{ pl ine } \\
\underline{või}\,$\oplus$\,ss & \textsc{ pl ela } \\
\underline{või}\,$\oplus$\,llõ & \textsc{ pl all } \\
\underline{või}\,$\oplus$\,ll & \textsc{ pl ade } \\
\underline{või}\,$\oplus$\,lt & \textsc{ pl abl } \\
\underline{või}\,$\oplus$\,ssi & \textsc{ pl tra } \\
\underline{või}\,$\oplus$\,ssaa & \textsc{ pl ter } \\
\underline{või}\,$\oplus$\,ka & \textsc{ pl com } \\
\end{tabular}
\end{sideways}
\captionof{table}{Tüüpsõnamall \arabic{mallinumber}\,\vadja{või} ekstraheeritud muutvormimallid.}
\label{tab:tüüpsõnamall-või}

\end{minipage}

 
\vspace{1em}
\noindent Tüüpsõnamall \vadja{või} hõlmab vormisõnastikus 2 lekseemi: \vadja{\underline{või}} ja \vadja{\underline{koi}}.

Tüüpsõna\-mall kirjeldab tagapoolseid diftongiga lõppevaid sõnu.
\spacing{1.5}


\subsection{\RN{2} käändkond}

Teise käändkonda kuuluvad kahesilbilised sõnad, mille tüvevokaal on \vadja{-o}, \vadja{-u}, \vadja{-ü}, \vadja{-i} või \vadja{-õ} ning rohkem silpidega sõnad, mille tüvevokaal on \vadja{-o} (\cite[42]{ariste_grammar_1968}).

Sellesse käändkonda kuuluvad paljud vene laensõnad. Kirja\-keele puhul puudub ülevaade vene laensõnade mugandamisest. Vene laen\-sõnade mugandamis\-strateegiatest ja problemaatikast vadja Jõgõperä murdes on kirjutatud artikkel (\cite{rozhanskiy_zaimstvovannyje_2009}). % TODO aga seda ei järgita, seleta


\subsubsection*{Ekstraktmorfoloogia tüüpsõnad}
\spacing{1.1}
\vspace{-3.5em}


\vspace{3.5em}
\noindent \begin{minipage}{\textwidth}
\stepcounter{mallinumber}
\noindent \textbf{Tüüpsõnamall \arabic{mallinumber}\,\vadja{auči}}\\

\begin{sideways}
\begin{tabular}{l l}
muutvormimall & tunnused \\
\hline
\underline{au}\,$\oplus$\,č\,$\oplus$\,\underline{i} & \textsc{ sg nom } \\
\underline{au}\,$\oplus$\,dž\,$\oplus$\,\underline{i} & \textsc{ sg gen } \\
\underline{au}\,$\oplus$\,č\,$\oplus$\,\underline{i}\,$\oplus$\,a & \textsc{ sg par } \\
\underline{au}\,$\oplus$\,č\,$\oplus$\,\underline{i}\,$\oplus$\,sõ & \textsc{ sg ill } \\
\underline{au}\,$\oplus$\,dž\,$\oplus$\,\underline{i}\,$\oplus$\,z & \textsc{ sg ine } \\
\underline{au}\,$\oplus$\,dž\,$\oplus$\,\underline{i}\,$\oplus$\,ss & \textsc{ sg ela } \\
\underline{au}\,$\oplus$\,dž\,$\oplus$\,\underline{i}\,$\oplus$\,llõ & \textsc{ sg all } \\
\underline{au}\,$\oplus$\,dž\,$\oplus$\,\underline{i}\,$\oplus$\,ll & \textsc{ sg ade } \\
\underline{au}\,$\oplus$\,dž\,$\oplus$\,\underline{i}\,$\oplus$\,lt & \textsc{ sg abl } \\
\underline{au}\,$\oplus$\,dž\,$\oplus$\,\underline{i}\,$\oplus$\,ssi & \textsc{ sg tra } \\
\underline{au}\,$\oplus$\,dž\,$\oplus$\,\underline{i}\,$\oplus$\,ssaa & \textsc{ sg ter } \\
\underline{au}\,$\oplus$\,dž\,$\oplus$\,\underline{i}\,$\oplus$\,ka & \textsc{ sg com } \\
\underline{au}\,$\oplus$\,dž\,$\oplus$\,\underline{i}\,$\oplus$\,d & \textsc{ pl nom } \\
\underline{au}\,$\oplus$\,č\,$\oplus$\,\underline{i}\,$\oplus$\,jõ & \textsc{ pl gen } \\
\underline{au}\,$\oplus$\,č\,$\oplus$\,\underline{i}\,$\oplus$\,it & \textsc{ pl par } \\
\underline{au}\,$\oplus$\,č\,$\oplus$\,\underline{i}\,$\oplus$\,isõ & \textsc{ pl ill } \\
\underline{au}\,$\oplus$\,č\,$\oplus$\,\underline{i}\,$\oplus$\,iz & \textsc{ pl ine } \\
\underline{au}\,$\oplus$\,č\,$\oplus$\,\underline{i}\,$\oplus$\,iss & \textsc{ pl ela } \\
\underline{au}\,$\oplus$\,č\,$\oplus$\,\underline{i}\,$\oplus$\,illõ & \textsc{ pl all } \\
\underline{au}\,$\oplus$\,č\,$\oplus$\,\underline{i}\,$\oplus$\,ill & \textsc{ pl ade } \\
\underline{au}\,$\oplus$\,č\,$\oplus$\,\underline{i}\,$\oplus$\,ilt & \textsc{ pl abl } \\
\underline{au}\,$\oplus$\,č\,$\oplus$\,\underline{i}\,$\oplus$\,issi & \textsc{ pl tra } \\
\underline{au}\,$\oplus$\,č\,$\oplus$\,\underline{i}\,$\oplus$\,issaa & \textsc{ pl ter } \\
\underline{au}\,$\oplus$\,č\,$\oplus$\,\underline{i}\,$\oplus$\,jka & \textsc{ pl com } \\
\end{tabular}
\end{sideways}
\captionof{table}{Tüüpsõnamall \arabic{mallinumber}\,\vadja{auči} ekstraheeritud muutvormimallid.}
\label{tab:tüüpsõnamall-auči}

\end{minipage}

 
\vspace{1em}
\noindent Tüüpsõnamall \vadja{auči} ei hõlma teisi lekseeme vormi\-sõnastikus.

Tüüpsõna\-mall kirjeldab tagapoolseid sõnu tüvemuutusega č:dž.


\vspace{3.5em}
\noindent \begin{minipage}{\textwidth}
\stepcounter{mallinumber}
\noindent \textbf{Tüüpsõnamall \arabic{mallinumber}\,\vadja{süüčči}}\\

\begin{sideways}
\begin{tabular}{l l}
muutvormimall & tunnused \\
\hline
\underline{süüč}\,$\oplus$\,č\,$\oplus$\,\underline{i} & \textsc{ sg nom } \\
\underline{süüč}\,$\oplus$\,\underline{i} & \textsc{ sg gen } \\
\underline{süüč}\,$\oplus$\,č\,$\oplus$\,\underline{i}\,$\oplus$\,ä & \textsc{ sg par } \\
\underline{süüč}\,$\oplus$\,č\,$\oplus$\,\underline{i}\,$\oplus$\,se & \textsc{ sg ill } \\
\underline{süüč}\,$\oplus$\,\underline{i}\,$\oplus$\,z & \textsc{ sg ine } \\
\underline{süüč}\,$\oplus$\,\underline{i}\,$\oplus$\,ss & \textsc{ sg ela } \\
\underline{süüč}\,$\oplus$\,\underline{i}\,$\oplus$\,lle & \textsc{ sg all } \\
\underline{süüč}\,$\oplus$\,\underline{i}\,$\oplus$\,ll & \textsc{ sg ade } \\
\underline{süüč}\,$\oplus$\,\underline{i}\,$\oplus$\,lt & \textsc{ sg abl } \\
\underline{süüč}\,$\oplus$\,\underline{i}\,$\oplus$\,ssi & \textsc{ sg tra } \\
\underline{süüč}\,$\oplus$\,\underline{i}\,$\oplus$\,ssaa & \textsc{ sg ter } \\
\underline{süüč}\,$\oplus$\,\underline{i}\,$\oplus$\,ka & \textsc{ sg com } \\
\underline{süüč}\,$\oplus$\,\underline{i}\,$\oplus$\,d & \textsc{ pl nom } \\
\underline{süüč}\,$\oplus$\,č\,$\oplus$\,\underline{i}\,$\oplus$\,je & \textsc{ pl gen } \\
\underline{süüč}\,$\oplus$\,č\,$\oplus$\,\underline{i}\,$\oplus$\,it & \textsc{ pl par } \\
\underline{süüč}\,$\oplus$\,č\,$\oplus$\,\underline{i}\,$\oplus$\,ise & \textsc{ pl ill } \\
\underline{süüč}\,$\oplus$\,č\,$\oplus$\,\underline{i}\,$\oplus$\,iz & \textsc{ pl ine } \\
\underline{süüč}\,$\oplus$\,č\,$\oplus$\,\underline{i}\,$\oplus$\,iss & \textsc{ pl ela } \\
\underline{süüč}\,$\oplus$\,č\,$\oplus$\,\underline{i}\,$\oplus$\,ille & \textsc{ pl all } \\
\underline{süüč}\,$\oplus$\,č\,$\oplus$\,\underline{i}\,$\oplus$\,ill & \textsc{ pl ade } \\
\underline{süüč}\,$\oplus$\,č\,$\oplus$\,\underline{i}\,$\oplus$\,ilt & \textsc{ pl abl } \\
\underline{süüč}\,$\oplus$\,č\,$\oplus$\,\underline{i}\,$\oplus$\,issi & \textsc{ pl tra } \\
\underline{süüč}\,$\oplus$\,č\,$\oplus$\,\underline{i}\,$\oplus$\,issaa & \textsc{ pl ter } \\
\underline{süüč}\,$\oplus$\,č\,$\oplus$\,\underline{i}\,$\oplus$\,jka & \textsc{ pl com } \\
\end{tabular}
\end{sideways}
\captionof{table}{Tüüpsõnamall \arabic{mallinumber}\,\vadja{süüčči} ekstraheeritud muutvormimallid.}
\label{tab:tüüpsõnamall-süüčči}

\end{minipage}

 
\vspace{1em}
\noindent Tüüpsõnamall \vadja{süüčči} ei hõlma teisi lekseeme vormi\-sõnastikus.

Tüüp\-sõna\-mall kirjeldab eespoolseid sõnu tüvemuutusega čč:č.


\vspace{3.5em}
\noindent \begin{minipage}{\textwidth}
\stepcounter{mallinumber}
\noindent \textbf{Tüüpsõnamall \arabic{mallinumber}\,\vadja{järčü}}\\

\begin{sideways}
\begin{tabular}{l l}
muutvormimall & tunnused \\
\hline
\underline{jär}\,$\oplus$\,č\,$\oplus$\,\underline{ü} & \textsc{ sg nom } \\
\underline{jär}\,$\oplus$\,j\,$\oplus$\,\underline{ü} & \textsc{ sg gen } \\
\underline{jär}\,$\oplus$\,č\,$\oplus$\,\underline{ü}\,$\oplus$\,ä & \textsc{ sg par } \\
\underline{jär}\,$\oplus$\,č\,$\oplus$\,\underline{ü}\,$\oplus$\,se & \textsc{ sg ill } \\
\underline{jär}\,$\oplus$\,j\,$\oplus$\,\underline{ü}\,$\oplus$\,z & \textsc{ sg ine } \\
\underline{jär}\,$\oplus$\,j\,$\oplus$\,\underline{ü}\,$\oplus$\,ss & \textsc{ sg ela } \\
\underline{jär}\,$\oplus$\,j\,$\oplus$\,\underline{ü}\,$\oplus$\,lle & \textsc{ sg all } \\
\underline{jär}\,$\oplus$\,j\,$\oplus$\,\underline{ü}\,$\oplus$\,ll & \textsc{ sg ade } \\
\underline{jär}\,$\oplus$\,j\,$\oplus$\,\underline{ü}\,$\oplus$\,lt & \textsc{ sg abl } \\
\underline{jär}\,$\oplus$\,j\,$\oplus$\,\underline{ü}\,$\oplus$\,ssi & \textsc{ sg tra } \\
\underline{jär}\,$\oplus$\,j\,$\oplus$\,\underline{ü}\,$\oplus$\,ssaa & \textsc{ sg ter } \\
\underline{jär}\,$\oplus$\,j\,$\oplus$\,\underline{ü}\,$\oplus$\,ka & \textsc{ sg com } \\
\underline{jär}\,$\oplus$\,j\,$\oplus$\,\underline{ü}\,$\oplus$\,d & \textsc{ pl nom } \\
\underline{jär}\,$\oplus$\,č\,$\oplus$\,\underline{ü}\,$\oplus$\,je & \textsc{ pl gen } \\
\underline{jär}\,$\oplus$\,č\,$\oplus$\,\underline{ü}\,$\oplus$\,it & \textsc{ pl par } \\
\underline{jär}\,$\oplus$\,č\,$\oplus$\,\underline{ü}\,$\oplus$\,ise & \textsc{ pl ill } \\
\underline{jär}\,$\oplus$\,č\,$\oplus$\,\underline{ü}\,$\oplus$\,iz & \textsc{ pl ine } \\
\underline{jär}\,$\oplus$\,č\,$\oplus$\,\underline{ü}\,$\oplus$\,iss & \textsc{ pl ela } \\
\underline{jär}\,$\oplus$\,č\,$\oplus$\,\underline{ü}\,$\oplus$\,ille & \textsc{ pl all } \\
\underline{jär}\,$\oplus$\,č\,$\oplus$\,\underline{ü}\,$\oplus$\,ill & \textsc{ pl ade } \\
\underline{jär}\,$\oplus$\,č\,$\oplus$\,\underline{ü}\,$\oplus$\,ilt & \textsc{ pl abl } \\
\underline{jär}\,$\oplus$\,č\,$\oplus$\,\underline{ü}\,$\oplus$\,issi & \textsc{ pl tra } \\
\underline{jär}\,$\oplus$\,č\,$\oplus$\,\underline{ü}\,$\oplus$\,issaa & \textsc{ pl ter } \\
\underline{jär}\,$\oplus$\,č\,$\oplus$\,\underline{ü}\,$\oplus$\,ika & \textsc{ pl com } \\
\end{tabular}
\end{sideways}
\captionof{table}{Tüüpsõnamall \arabic{mallinumber}\,\vadja{järčü} ekstraheeritud muutvormimallid.}
\label{tab:tüüpsõnamall-järčü}

\end{minipage}

 
\vspace{1em}
\noindent Tüüpsõnamall \vadja{järčü} ei hõlma teisi lekseeme vormi\-sõnastikus.

Tüüp\-sõna\-mall kirjeldab eespoolseid sõnu tüvemuutusega rč:rj.


\vspace{3.5em}
\noindent \begin{minipage}{\textwidth}
\stepcounter{mallinumber}
\noindent \textbf{Tüüpsõnamall \arabic{mallinumber}\,\vadja{koffi}}\\

\begin{sideways}
\begin{tabular}{l l}
muutvormimall & tunnused \\
\hline
\underline{kof}\,$\oplus$\,f\,$\oplus$\,\underline{i} & \textsc{ sg nom } \\
\underline{kof}\,$\oplus$\,\underline{i} & \textsc{ sg gen } \\
\underline{kof}\,$\oplus$\,f\,$\oplus$\,\underline{i}\,$\oplus$\,a & \textsc{ sg par } \\
\underline{kof}\,$\oplus$\,f\,$\oplus$\,\underline{i}\,$\oplus$\,sõ & \textsc{ sg ill } \\
\underline{kof}\,$\oplus$\,\underline{i}\,$\oplus$\,z & \textsc{ sg ine } \\
\underline{kof}\,$\oplus$\,\underline{i}\,$\oplus$\,ss & \textsc{ sg ela } \\
\underline{kof}\,$\oplus$\,\underline{i}\,$\oplus$\,llõ & \textsc{ sg all } \\
\underline{kof}\,$\oplus$\,\underline{i}\,$\oplus$\,ll & \textsc{ sg ade } \\
\underline{kof}\,$\oplus$\,\underline{i}\,$\oplus$\,lt & \textsc{ sg abl } \\
\underline{kof}\,$\oplus$\,\underline{i}\,$\oplus$\,ssi & \textsc{ sg tra } \\
\underline{kof}\,$\oplus$\,\underline{i}\,$\oplus$\,ssaa & \textsc{ sg ter } \\
\underline{kof}\,$\oplus$\,\underline{i}\,$\oplus$\,ka & \textsc{ sg com } \\
\underline{kof}\,$\oplus$\,\underline{i}\,$\oplus$\,d & \textsc{ pl nom } \\
\underline{kof}\,$\oplus$\,f\,$\oplus$\,\underline{i}\,$\oplus$\,jõ & \textsc{ pl gen } \\
\underline{kof}\,$\oplus$\,f\,$\oplus$\,\underline{i}\,$\oplus$\,it & \textsc{ pl par } \\
\underline{kof}\,$\oplus$\,f\,$\oplus$\,\underline{i}\,$\oplus$\,isõ & \textsc{ pl ill } \\
\underline{kof}\,$\oplus$\,f\,$\oplus$\,\underline{i}\,$\oplus$\,iz & \textsc{ pl ine } \\
\underline{kof}\,$\oplus$\,f\,$\oplus$\,\underline{i}\,$\oplus$\,iss & \textsc{ pl ela } \\
\underline{kof}\,$\oplus$\,f\,$\oplus$\,\underline{i}\,$\oplus$\,illõ & \textsc{ pl all } \\
\underline{kof}\,$\oplus$\,f\,$\oplus$\,\underline{i}\,$\oplus$\,ill & \textsc{ pl ade } \\
\underline{kof}\,$\oplus$\,f\,$\oplus$\,\underline{i}\,$\oplus$\,ilt & \textsc{ pl abl } \\
\underline{kof}\,$\oplus$\,f\,$\oplus$\,\underline{i}\,$\oplus$\,issi & \textsc{ pl tra } \\
\underline{kof}\,$\oplus$\,f\,$\oplus$\,\underline{i}\,$\oplus$\,issaa & \textsc{ pl ter } \\
\underline{kof}\,$\oplus$\,f\,$\oplus$\,\underline{i}\,$\oplus$\,jka & \textsc{ pl com } \\
\end{tabular}
\end{sideways}
\captionof{table}{Tüüpsõnamall \arabic{mallinumber}\,\vadja{koffi} ekstraheeritud muutvormimallid.}
\label{tab:tüüpsõnamall-koffi}

\end{minipage}

 
\vspace{1em}
\noindent Tüüpsõnamall \vadja{koffi} ei hõlma teisi lekseeme vormi\-sõnastikus.

Tüüp\-sõna\-mall kirjeldab tagapoolseid sõnu tüvemuutusega ff:f, mille lõpuvokaal on \textit{i}. 


\vspace{3.5em}
\noindent \begin{minipage}{\textwidth}
\stepcounter{mallinumber}
\noindent \textbf{Tüüpsõnamall \arabic{mallinumber}\,\vadja{suuto}}\\

\begin{sideways}
\begin{tabular}{l l}
muutvormimall & tunnused \\
\hline
\underline{suu}\,$\oplus$\,t\,$\oplus$\,\underline{o} & \textsc{ sg nom } \\
\underline{suu}\,$\oplus$\,\underline{o} & \textsc{ sg gen } \\
\underline{suu}\,$\oplus$\,t\,$\oplus$\,\underline{o}\,$\oplus$\,a & \textsc{ sg par } \\
\underline{suu}\,$\oplus$\,t\,$\oplus$\,\underline{o}\,$\oplus$\,sõ & \textsc{ sg ill } \\
\underline{suu}\,$\oplus$\,\underline{o}\,$\oplus$\,z & \textsc{ sg ine } \\
\underline{suu}\,$\oplus$\,\underline{o}\,$\oplus$\,ss & \textsc{ sg ela } \\
\underline{suu}\,$\oplus$\,\underline{o}\,$\oplus$\,llõ & \textsc{ sg all } \\
\underline{suu}\,$\oplus$\,\underline{o}\,$\oplus$\,ll & \textsc{ sg ade } \\
\underline{suu}\,$\oplus$\,\underline{o}\,$\oplus$\,lt & \textsc{ sg abl } \\
\underline{suu}\,$\oplus$\,\underline{o}\,$\oplus$\,ssi & \textsc{ sg tra } \\
\underline{suu}\,$\oplus$\,\underline{o}\,$\oplus$\,ssaa & \textsc{ sg ter } \\
\underline{suu}\,$\oplus$\,\underline{o}\,$\oplus$\,ka & \textsc{ sg com } \\
\underline{suu}\,$\oplus$\,\underline{o}\,$\oplus$\,d & \textsc{ pl nom } \\
\underline{suu}\,$\oplus$\,t\,$\oplus$\,\underline{o}\,$\oplus$\,jõ & \textsc{ pl gen } \\
\underline{suu}\,$\oplus$\,t\,$\oplus$\,\underline{o}\,$\oplus$\,it & \textsc{ pl par } \\
\underline{suu}\,$\oplus$\,t\,$\oplus$\,\underline{o}\,$\oplus$\,isõ & \textsc{ pl ill } \\
\underline{suu}\,$\oplus$\,t\,$\oplus$\,\underline{o}\,$\oplus$\,iz & \textsc{ pl ine } \\
\underline{suu}\,$\oplus$\,t\,$\oplus$\,\underline{o}\,$\oplus$\,iss & \textsc{ pl ela } \\
\underline{suu}\,$\oplus$\,t\,$\oplus$\,\underline{o}\,$\oplus$\,illõ & \textsc{ pl all } \\
\underline{suu}\,$\oplus$\,t\,$\oplus$\,\underline{o}\,$\oplus$\,ill & \textsc{ pl ade } \\
\underline{suu}\,$\oplus$\,t\,$\oplus$\,\underline{o}\,$\oplus$\,ilt & \textsc{ pl abl } \\
\underline{suu}\,$\oplus$\,t\,$\oplus$\,\underline{o}\,$\oplus$\,issi & \textsc{ pl tra } \\
\underline{suu}\,$\oplus$\,t\,$\oplus$\,\underline{o}\,$\oplus$\,issaa & \textsc{ pl ter } \\
\underline{suu}\,$\oplus$\,t\,$\oplus$\,\underline{o}\,$\oplus$\,ika & \textsc{ pl com } \\
\end{tabular}
\end{sideways}
\captionof{table}{Tüüpsõnamall \arabic{mallinumber}\,\vadja{suuto} ekstraheeritud muutvormimallid.}
\label{tab:tüüpsõnamall-suuto}

\end{minipage}

 
\vspace{1em}
\noindent Tüüpsõnamall \vadja{suuto} hõlmab vormisõnastikus 3 lekseemi: \vadja{\underline{suu}t\underline{o}, \underline{vaah}t\underline{o}} ja \vadja{\underline{leh}t\underline{o}}.

Tüüp\-sõna\-mall kirjeldab tagapoolseid sõnu tüvemuutusega t:∅.


\vspace{3.5em}
\noindent \begin{minipage}{\textwidth}
\stepcounter{mallinumber}
\noindent \textbf{Tüüpsõnamall \arabic{mallinumber}\,\vadja{vahti}}\\

\begin{sideways}
\begin{tabular}{l l}
muutvormimall & tunnused \\
\hline
\underline{vah}\,$\oplus$\,t\,$\oplus$\,\underline{i} & \textsc{ sg nom } \\
\underline{vah}\,$\oplus$\,\underline{i} & \textsc{ sg gen } \\
\underline{vah}\,$\oplus$\,t\,$\oplus$\,\underline{i}\,$\oplus$\,a & \textsc{ sg par } \\
\underline{vah}\,$\oplus$\,t\,$\oplus$\,\underline{i}\,$\oplus$\,sõ & \textsc{ sg ill } \\
\underline{vah}\,$\oplus$\,\underline{i}\,$\oplus$\,z & \textsc{ sg ine } \\
\underline{vah}\,$\oplus$\,\underline{i}\,$\oplus$\,ss & \textsc{ sg ela } \\
\underline{vah}\,$\oplus$\,\underline{i}\,$\oplus$\,llõ & \textsc{ sg all } \\
\underline{vah}\,$\oplus$\,\underline{i}\,$\oplus$\,ll & \textsc{ sg ade } \\
\underline{vah}\,$\oplus$\,\underline{i}\,$\oplus$\,lt & \textsc{ sg abl } \\
\underline{vah}\,$\oplus$\,\underline{i}\,$\oplus$\,ssi & \textsc{ sg tra } \\
\underline{vah}\,$\oplus$\,\underline{i}\,$\oplus$\,ssaa & \textsc{ sg ter } \\
\underline{vah}\,$\oplus$\,\underline{i}\,$\oplus$\,ka & \textsc{ sg com } \\
\underline{vah}\,$\oplus$\,\underline{i}\,$\oplus$\,d & \textsc{ pl nom } \\
\underline{vah}\,$\oplus$\,t\,$\oplus$\,\underline{i}\,$\oplus$\,jõ & \textsc{ pl gen } \\
\underline{vah}\,$\oplus$\,t\,$\oplus$\,\underline{i}\,$\oplus$\,it & \textsc{ pl par } \\
\underline{vah}\,$\oplus$\,t\,$\oplus$\,\underline{i}\,$\oplus$\,isõ & \textsc{ pl ill } \\
\underline{vah}\,$\oplus$\,t\,$\oplus$\,\underline{i}\,$\oplus$\,iz & \textsc{ pl ine } \\
\underline{vah}\,$\oplus$\,t\,$\oplus$\,\underline{i}\,$\oplus$\,iss & \textsc{ pl ela } \\
\underline{vah}\,$\oplus$\,t\,$\oplus$\,\underline{i}\,$\oplus$\,illõ & \textsc{ pl all } \\
\underline{vah}\,$\oplus$\,t\,$\oplus$\,\underline{i}\,$\oplus$\,ill & \textsc{ pl ade } \\
\underline{vah}\,$\oplus$\,t\,$\oplus$\,\underline{i}\,$\oplus$\,ilt & \textsc{ pl abl } \\
\underline{vah}\,$\oplus$\,t\,$\oplus$\,\underline{i}\,$\oplus$\,issi & \textsc{ pl tra } \\
\underline{vah}\,$\oplus$\,t\,$\oplus$\,\underline{i}\,$\oplus$\,issaa & \textsc{ pl ter } \\
\underline{vah}\,$\oplus$\,t\,$\oplus$\,\underline{i}\,$\oplus$\,jka & \textsc{ pl com } \\
\end{tabular}
\end{sideways}
\captionof{table}{Tüüpsõnamall \arabic{mallinumber}\,\vadja{vahti} ekstraheeritud muutvormimallid.}
\label{tab:tüüpsõnamall-vahti}

\end{minipage}

 
\vspace{1em}
\noindent Tüüpsõnamall \vadja{vahti} ei hõlma teisi lekseeme vormi\-sõnastikus.

Tüüp\-sõna\-mall kirjeldab tagapoolseid sõnu tüvemuutusega t:∅, mille lõpuvokaal on \textit{i}. 


\vspace{3.5em}
\noindent \begin{minipage}{\textwidth}
\stepcounter{mallinumber}
\noindent \textbf{Tüüpsõnamall \arabic{mallinumber}\,\vadja{alku}}\\

\begin{sideways}
\begin{tabular}{l l}
muutvormimall & tunnused \\
\hline
\underline{al}\,$\oplus$\,k\,$\oplus$\,\underline{u} & \textsc{ sg nom } \\
\underline{al}\,$\oplus$\,g\,$\oplus$\,\underline{u} & \textsc{ sg gen } \\
\underline{al}\,$\oplus$\,k\,$\oplus$\,\underline{u}\,$\oplus$\,a & \textsc{ sg par } \\
\underline{al}\,$\oplus$\,k\,$\oplus$\,\underline{u}\,$\oplus$\,sõ & \textsc{ sg ill } \\
\underline{al}\,$\oplus$\,g\,$\oplus$\,\underline{u}\,$\oplus$\,z & \textsc{ sg ine } \\
\underline{al}\,$\oplus$\,g\,$\oplus$\,\underline{u}\,$\oplus$\,ss & \textsc{ sg ela } \\
\underline{al}\,$\oplus$\,g\,$\oplus$\,\underline{u}\,$\oplus$\,llõ & \textsc{ sg all } \\
\underline{al}\,$\oplus$\,g\,$\oplus$\,\underline{u}\,$\oplus$\,ll & \textsc{ sg ade } \\
\underline{al}\,$\oplus$\,g\,$\oplus$\,\underline{u}\,$\oplus$\,lt & \textsc{ sg abl } \\
\underline{al}\,$\oplus$\,g\,$\oplus$\,\underline{u}\,$\oplus$\,ssi & \textsc{ sg tra } \\
\underline{al}\,$\oplus$\,g\,$\oplus$\,\underline{u}\,$\oplus$\,ssaa & \textsc{ sg ter } \\
\underline{al}\,$\oplus$\,g\,$\oplus$\,\underline{u}\,$\oplus$\,ka & \textsc{ sg com } \\
\underline{al}\,$\oplus$\,g\,$\oplus$\,\underline{u}\,$\oplus$\,d & \textsc{ pl nom } \\
\underline{al}\,$\oplus$\,k\,$\oplus$\,\underline{u}\,$\oplus$\,jõ & \textsc{ pl gen } \\
\underline{al}\,$\oplus$\,k\,$\oplus$\,\underline{u}\,$\oplus$\,it & \textsc{ pl par } \\
\underline{al}\,$\oplus$\,k\,$\oplus$\,\underline{u}\,$\oplus$\,isõ & \textsc{ pl ill } \\
\underline{al}\,$\oplus$\,k\,$\oplus$\,\underline{u}\,$\oplus$\,iz & \textsc{ pl ine } \\
\underline{al}\,$\oplus$\,k\,$\oplus$\,\underline{u}\,$\oplus$\,iss & \textsc{ pl ela } \\
\underline{al}\,$\oplus$\,k\,$\oplus$\,\underline{u}\,$\oplus$\,illõ & \textsc{ pl all } \\
\underline{al}\,$\oplus$\,k\,$\oplus$\,\underline{u}\,$\oplus$\,ill & \textsc{ pl ade } \\
\underline{al}\,$\oplus$\,k\,$\oplus$\,\underline{u}\,$\oplus$\,ilt & \textsc{ pl abl } \\
\underline{al}\,$\oplus$\,k\,$\oplus$\,\underline{u}\,$\oplus$\,issi & \textsc{ pl tra } \\
\underline{al}\,$\oplus$\,k\,$\oplus$\,\underline{u}\,$\oplus$\,issaa & \textsc{ pl ter } \\
\underline{al}\,$\oplus$\,k\,$\oplus$\,\underline{u}\,$\oplus$\,ika & \textsc{ pl com } \\
\end{tabular}
\end{sideways}
\captionof{table}{Tüüpsõnamall \arabic{mallinumber}\,\vadja{alku} ekstraheeritud muutvormimallid.}
\label{tab:tüüpsõnamall-alku}

\end{minipage}

 
\vspace{1em}
\noindent Tüüpsõnamall \vadja{alku} hõlmab vormisõnastikus 8 lekseemi: \vadja{\underline{al}k\underline{u}, \underline{loh}k\underline{o}, \underline{peh}k\underline{o}, \underline{pleh}k\underline{u}, \underline{tou}k\underline{o}, \underline{vih}k\underline{o}, \underline{vin}k\underline{u}} ja \vadja{\underline{al}k\underline{o}}.

Tüüp\-sõna\-mall kirjeldab tagapoolseid sõnu tüvemuutusega k:g, kusjuures tüvemuutus esineb konsonant\-klustris, mistõttu gemineerumist ei toimu \msd{sg par} ja \msd{sg ill} tüvedes.


\vspace{3.5em}
\noindent \begin{minipage}{\textwidth}
\stepcounter{mallinumber}
\noindent \textbf{Tüüpsõnamall \arabic{mallinumber}\,\vadja{lako}}\\

\begin{sideways}
\begin{tabular}{l l}
muutvormimall & tunnused \\
\hline
\underline{la}\,$\oplus$\,k\,$\oplus$\,\underline{o} & \textsc{ sg nom } \\
\underline{la}\,$\oplus$\,g\,$\oplus$\,\underline{o} & \textsc{ sg gen } \\
\underline{la}\,$\oplus$\,kk\,$\oplus$\,\underline{o}\,$\oplus$\,a & \textsc{ sg par } \\
\underline{la}\,$\oplus$\,kk\,$\oplus$\,\underline{o}\,$\oplus$\,sõ & \textsc{ sg ill } \\
\underline{la}\,$\oplus$\,g\,$\oplus$\,\underline{o}\,$\oplus$\,z & \textsc{ sg ine } \\
\underline{la}\,$\oplus$\,g\,$\oplus$\,\underline{o}\,$\oplus$\,ss & \textsc{ sg ela } \\
\underline{la}\,$\oplus$\,g\,$\oplus$\,\underline{o}\,$\oplus$\,llõ & \textsc{ sg all } \\
\underline{la}\,$\oplus$\,g\,$\oplus$\,\underline{o}\,$\oplus$\,ll & \textsc{ sg ade } \\
\underline{la}\,$\oplus$\,g\,$\oplus$\,\underline{o}\,$\oplus$\,lt & \textsc{ sg abl } \\
\underline{la}\,$\oplus$\,g\,$\oplus$\,\underline{o}\,$\oplus$\,ssi & \textsc{ sg tra } \\
\underline{la}\,$\oplus$\,g\,$\oplus$\,\underline{o}\,$\oplus$\,ssaa & \textsc{ sg ter } \\
\underline{la}\,$\oplus$\,g\,$\oplus$\,\underline{o}\,$\oplus$\,ka & \textsc{ sg com } \\
\underline{la}\,$\oplus$\,g\,$\oplus$\,\underline{o}\,$\oplus$\,d & \textsc{ pl nom } \\
\underline{la}\,$\oplus$\,k\,$\oplus$\,\underline{o}\,$\oplus$\,jõ & \textsc{ pl gen } \\
\underline{la}\,$\oplus$\,k\,$\oplus$\,\underline{o}\,$\oplus$\,it & \textsc{ pl par } \\
\underline{la}\,$\oplus$\,k\,$\oplus$\,\underline{o}\,$\oplus$\,isõ & \textsc{ pl ill } \\
\underline{la}\,$\oplus$\,k\,$\oplus$\,\underline{o}\,$\oplus$\,iz & \textsc{ pl ine } \\
\underline{la}\,$\oplus$\,k\,$\oplus$\,\underline{o}\,$\oplus$\,iss & \textsc{ pl ela } \\
\underline{la}\,$\oplus$\,k\,$\oplus$\,\underline{o}\,$\oplus$\,illõ & \textsc{ pl all } \\
\underline{la}\,$\oplus$\,k\,$\oplus$\,\underline{o}\,$\oplus$\,ill & \textsc{ pl ade } \\
\underline{la}\,$\oplus$\,k\,$\oplus$\,\underline{o}\,$\oplus$\,ilt & \textsc{ pl abl } \\
\underline{la}\,$\oplus$\,k\,$\oplus$\,\underline{o}\,$\oplus$\,issi & \textsc{ pl tra } \\
\underline{la}\,$\oplus$\,k\,$\oplus$\,\underline{o}\,$\oplus$\,issaa & \textsc{ pl ter } \\
\underline{la}\,$\oplus$\,k\,$\oplus$\,\underline{o}\,$\oplus$\,ika & \textsc{ pl com } \\
\end{tabular}
\end{sideways}
\captionof{table}{Tüüpsõnamall \arabic{mallinumber}\,\vadja{lako} ekstraheeritud muutvormimallid.}
\label{tab:tüüpsõnamall-lako}

\end{minipage}

 
\vspace{1em}
\noindent Tüüpsõnamall \vadja{lako} hõlmab vormisõnastikus 7 lekseemi: \vadja{\underline{la}k\underline{o}, \underline{lu}k\underline{u}, \underline{ma}k\underline{o}, \underline{ma}k\underline{u}, \underline{su}k\underline{u}, \underline{va}k\underline{o}} ja \vadja{\underline{ča}k\underline{o}}.

Tüüp\-sõna\-mall kirjeldab tagapoolseid sõnu tüvemuutusega k:g, kusjuures tüvi gemineerub \msd{sg par} ja \msd{sg ill} vormides.


\vspace{3.5em}
\noindent \begin{minipage}{\textwidth}
\stepcounter{mallinumber}
\noindent \textbf{Tüüpsõnamall \arabic{mallinumber}\,\vadja{läikki}}\\

\begin{sideways}
\begin{tabular}{l l}
muutvormimall & tunnused \\
\hline
\underline{läik}\,$\oplus$\,k\,$\oplus$\,\underline{i} & \textsc{ sg nom } \\
\underline{läik}\,$\oplus$\,\underline{i} & \textsc{ sg gen } \\
\underline{läik}\,$\oplus$\,k\,$\oplus$\,\underline{i}\,$\oplus$\,ä & \textsc{ sg par } \\
\underline{läik}\,$\oplus$\,k\,$\oplus$\,\underline{i}\,$\oplus$\,se & \textsc{ sg ill } \\
\underline{läik}\,$\oplus$\,k\,$\oplus$\,\underline{i}\,$\oplus$\,z & \textsc{ sg ine } \\
\underline{läik}\,$\oplus$\,\underline{i}\,$\oplus$\,ss & \textsc{ sg ela } \\
\underline{läik}\,$\oplus$\,\underline{i}\,$\oplus$\,lle & \textsc{ sg all } \\
\underline{läik}\,$\oplus$\,\underline{i}\,$\oplus$\,ll & \textsc{ sg ade } \\
\underline{läik}\,$\oplus$\,\underline{i}\,$\oplus$\,lt & \textsc{ sg abl } \\
\underline{läik}\,$\oplus$\,\underline{i}\,$\oplus$\,ssi & \textsc{ sg tra } \\
\underline{läik}\,$\oplus$\,k\,$\oplus$\,\underline{i}\,$\oplus$\,ssaa & \textsc{ sg ter } \\
\underline{läik}\,$\oplus$\,\underline{i}\,$\oplus$\,ka & \textsc{ sg com } \\
\underline{läik}\,$\oplus$\,\underline{i}\,$\oplus$\,d & \textsc{ pl nom } \\
\underline{läik}\,$\oplus$\,k\,$\oplus$\,\underline{i}\,$\oplus$\,je & \textsc{ pl gen } \\
\underline{läik}\,$\oplus$\,ke\,$\oplus$\,\underline{i}\,$\oplus$\,t & \textsc{ pl par } \\
\underline{läik}\,$\oplus$\,ke\,$\oplus$\,\underline{i}\,$\oplus$\,se & \textsc{ pl ill } \\
\underline{läik}\,$\oplus$\,ke\,$\oplus$\,\underline{i}\,$\oplus$\,z & \textsc{ pl ine } \\
\underline{läik}\,$\oplus$\,ke\,$\oplus$\,\underline{i}\,$\oplus$\,ss & \textsc{ pl ela } \\
\underline{läik}\,$\oplus$\,ke\,$\oplus$\,\underline{i}\,$\oplus$\,lle & \textsc{ pl all } \\
\underline{läik}\,$\oplus$\,ke\,$\oplus$\,\underline{i}\,$\oplus$\,ll & \textsc{ pl ade } \\
\underline{läik}\,$\oplus$\,ke\,$\oplus$\,\underline{i}\,$\oplus$\,lt & \textsc{ pl abl } \\
\underline{läik}\,$\oplus$\,ke\,$\oplus$\,\underline{i}\,$\oplus$\,ssi & \textsc{ pl tra } \\
\underline{läik}\,$\oplus$\,ke\,$\oplus$\,\underline{i}\,$\oplus$\,ssaa & \textsc{ pl ter } \\
\underline{läik}\,$\oplus$\,ke\,$\oplus$\,\underline{i}\,$\oplus$\,ka & \textsc{ pl com } \\
\end{tabular}
\end{sideways}
\captionof{table}{Tüüpsõnamall \arabic{mallinumber}\,\vadja{läikki} ekstraheeritud muutvormimallid.}
\label{tab:tüüpsõnamall-läikki}

\end{minipage}

 
\vspace{1em}
\noindent Tüüpsõnamall \vadja{läikki} ei hõlma teisi lekseeme vormi\-sõnastikus.

Tüüp\-sõna\-mall kirjeldab eespoolseid sõnu tüvemuutusega kk:k, mille lõpuvokaal on \textit{i}. 


\vspace{3.5em}
\noindent \begin{minipage}{\textwidth}
\stepcounter{mallinumber}
\noindent \textbf{Tüüpsõnamall \arabic{mallinumber}\,\vadja{tükkü}}\\

\begin{sideways}
\begin{tabular}{l l}
muutvormimall & tunnused \\
\hline
\underline{tük}\,$\oplus$\,k\,$\oplus$\,\underline{ü} & \textsc{ sg nom } \\
\underline{tük}\,$\oplus$\,\underline{ü} & \textsc{ sg gen } \\
\underline{tük}\,$\oplus$\,k\,$\oplus$\,\underline{ü}\,$\oplus$\,ä & \textsc{ sg par } \\
\underline{tük}\,$\oplus$\,k\,$\oplus$\,\underline{ü}\,$\oplus$\,se & \textsc{ sg ill } \\
\underline{tük}\,$\oplus$\,k\,$\oplus$\,\underline{ü}\,$\oplus$\,z & \textsc{ sg ine } \\
\underline{tük}\,$\oplus$\,\underline{ü}\,$\oplus$\,ss & \textsc{ sg ela } \\
\underline{tük}\,$\oplus$\,\underline{ü}\,$\oplus$\,lle & \textsc{ sg all } \\
\underline{tük}\,$\oplus$\,\underline{ü}\,$\oplus$\,ll & \textsc{ sg ade } \\
\underline{tük}\,$\oplus$\,\underline{ü}\,$\oplus$\,lt & \textsc{ sg abl } \\
\underline{tük}\,$\oplus$\,\underline{ü}\,$\oplus$\,ssi & \textsc{ sg tra } \\
\underline{tük}\,$\oplus$\,k\,$\oplus$\,\underline{ü}\,$\oplus$\,ssaa & \textsc{ sg ter } \\
\underline{tük}\,$\oplus$\,\underline{ü}\,$\oplus$\,ka & \textsc{ sg com } \\
\underline{tük}\,$\oplus$\,\underline{ü}\,$\oplus$\,d & \textsc{ pl nom } \\
\underline{tük}\,$\oplus$\,k\,$\oplus$\,\underline{ü}\,$\oplus$\,je & \textsc{ pl gen } \\
\underline{tük}\,$\oplus$\,k\,$\oplus$\,\underline{ü}\,$\oplus$\,it & \textsc{ pl par } \\
\underline{tük}\,$\oplus$\,k\,$\oplus$\,\underline{ü}\,$\oplus$\,ise & \textsc{ pl ill } \\
\underline{tük}\,$\oplus$\,k\,$\oplus$\,\underline{ü}\,$\oplus$\,iz & \textsc{ pl ine } \\
\underline{tük}\,$\oplus$\,k\,$\oplus$\,\underline{ü}\,$\oplus$\,iss & \textsc{ pl ela } \\
\underline{tük}\,$\oplus$\,k\,$\oplus$\,\underline{ü}\,$\oplus$\,ille & \textsc{ pl all } \\
\underline{tük}\,$\oplus$\,k\,$\oplus$\,\underline{ü}\,$\oplus$\,ill & \textsc{ pl ade } \\
\underline{tük}\,$\oplus$\,k\,$\oplus$\,\underline{ü}\,$\oplus$\,ilt & \textsc{ pl abl } \\
\underline{tük}\,$\oplus$\,k\,$\oplus$\,\underline{ü}\,$\oplus$\,issi & \textsc{ pl tra } \\
\underline{tük}\,$\oplus$\,k\,$\oplus$\,\underline{ü}\,$\oplus$\,issaa & \textsc{ pl ter } \\
\underline{tük}\,$\oplus$\,k\,$\oplus$\,\underline{ü}\,$\oplus$\,ika & \textsc{ pl com } \\
\end{tabular}
\end{sideways}
\captionof{table}{Tüüpsõnamall \arabic{mallinumber}\,\vadja{tükkü} ekstraheeritud muutvormimallid.}
\label{tab:tüüpsõnamall-tükkü}

\end{minipage}

 
\vspace{1em}
\noindent Tüüpsõnamall \vadja{tükkü} ei hõlma teisi lekseeme vormi\-sõnastikus.

Tüüp\-sõna\-mall kirjeldab eespoolseid sõnu tüvemuutusega kk:k.


\vspace{3.5em}
\noindent \begin{minipage}{\textwidth}
\stepcounter{mallinumber}
\noindent \textbf{Tüüpsõnamall \arabic{mallinumber}\,\vadja{viki}}\\

\begin{sideways}
\begin{tabular}{l l}
muutvormimall & tunnused \\
\hline
\underline{vik}\,$\oplus$\,\underline{i} & \textsc{ sg nom } \\
\underline{vik}\,$\oplus$\,\underline{i} & \textsc{ sg gen } \\
\underline{vik}\,$\oplus$\,k\,$\oplus$\,\underline{i}\,$\oplus$\,ä & \textsc{ sg par } \\
\underline{vik}\,$\oplus$\,k\,$\oplus$\,\underline{i}\,$\oplus$\,se & \textsc{ sg ill } \\
\underline{vik}\,$\oplus$\,\underline{i}\,$\oplus$\,z & \textsc{ sg ine } \\
\underline{vik}\,$\oplus$\,\underline{i}\,$\oplus$\,ss & \textsc{ sg ela } \\
\underline{vik}\,$\oplus$\,\underline{i}\,$\oplus$\,lle & \textsc{ sg all } \\
\underline{vik}\,$\oplus$\,\underline{i}\,$\oplus$\,ll & \textsc{ sg ade } \\
\underline{vik}\,$\oplus$\,\underline{i}\,$\oplus$\,lt & \textsc{ sg abl } \\
\underline{vik}\,$\oplus$\,\underline{i}\,$\oplus$\,ssi & \textsc{ sg tra } \\
\underline{vik}\,$\oplus$\,\underline{i}\,$\oplus$\,ssaa & \textsc{ sg ter } \\
\underline{vik}\,$\oplus$\,\underline{i}\,$\oplus$\,ka & \textsc{ sg com } \\
\underline{vik}\,$\oplus$\,\underline{i}\,$\oplus$\,d & \textsc{ pl nom } \\
\underline{vik}\,$\oplus$\,k\,$\oplus$\,\underline{i}\,$\oplus$\,je & \textsc{ pl gen } \\
\underline{vik}\,$\oplus$\,k\,$\oplus$\,\underline{i}\,$\oplus$\,it & \textsc{ pl par } \\
\underline{vik}\,$\oplus$\,k\,$\oplus$\,\underline{i}\,$\oplus$\,ise & \textsc{ pl ill } \\
\underline{vik}\,$\oplus$\,k\,$\oplus$\,\underline{i}\,$\oplus$\,iz & \textsc{ pl ine } \\
\underline{vik}\,$\oplus$\,k\,$\oplus$\,\underline{i}\,$\oplus$\,iss & \textsc{ pl ela } \\
\underline{vik}\,$\oplus$\,k\,$\oplus$\,\underline{i}\,$\oplus$\,ille & \textsc{ pl all } \\
\underline{vik}\,$\oplus$\,k\,$\oplus$\,\underline{i}\,$\oplus$\,ill & \textsc{ pl ade } \\
\underline{vik}\,$\oplus$\,k\,$\oplus$\,\underline{i}\,$\oplus$\,ilt & \textsc{ pl abl } \\
\underline{vik}\,$\oplus$\,k\,$\oplus$\,\underline{i}\,$\oplus$\,issi & \textsc{ pl tra } \\
\underline{vik}\,$\oplus$\,k\,$\oplus$\,\underline{i}\,$\oplus$\,issaa & \textsc{ pl ter } \\
\underline{vik}\,$\oplus$\,k\,$\oplus$\,\underline{i}\,$\oplus$\,jka & \textsc{ pl com } \\
\end{tabular}
\end{sideways}
\captionof{table}{Tüüpsõnamall \arabic{mallinumber}\,\vadja{viki} ekstraheeritud muutvormimallid.}
\label{tab:tüüpsõnamall-viki}

\end{minipage}

 
\vspace{1em}
\noindent Tüüpsõnamall \vadja{viki} ei hõlma teisi lekseeme vormi\-sõnastikus.

Tüüp\-sõna\-mall kirjeldab eespoolseid sõnu tüvemuutusega kk:k, mille lõpuvokaal on \textit{i}. 


\vspace{3.5em}
\noindent \begin{minipage}{\textwidth}
\stepcounter{mallinumber}
\noindent \textbf{Tüüpsõnamall \arabic{mallinumber}\,\vadja{flakku}}\\

\begin{sideways}
\begin{tabular}{l l}
muutvormimall & tunnused \\
\hline
\underline{flak}\,$\oplus$\,k\,$\oplus$\,\underline{u} & \textsc{ sg nom } \\
\underline{flak}\,$\oplus$\,\underline{u} & \textsc{ sg gen } \\
\underline{flak}\,$\oplus$\,k\,$\oplus$\,\underline{u}\,$\oplus$\,a & \textsc{ sg par } \\
\underline{flak}\,$\oplus$\,k\,$\oplus$\,\underline{u}\,$\oplus$\,sõ & \textsc{ sg ill } \\
\underline{flak}\,$\oplus$\,k\,$\oplus$\,\underline{u}\,$\oplus$\,z & \textsc{ sg ine } \\
\underline{flak}\,$\oplus$\,\underline{u}\,$\oplus$\,ss & \textsc{ sg ela } \\
\underline{flak}\,$\oplus$\,\underline{u}\,$\oplus$\,llõ & \textsc{ sg all } \\
\underline{flak}\,$\oplus$\,\underline{u}\,$\oplus$\,ll & \textsc{ sg ade } \\
\underline{flak}\,$\oplus$\,\underline{u}\,$\oplus$\,lt & \textsc{ sg abl } \\
\underline{flak}\,$\oplus$\,\underline{u}\,$\oplus$\,ssi & \textsc{ sg tra } \\
\underline{flak}\,$\oplus$\,k\,$\oplus$\,\underline{u}\,$\oplus$\,ssaa & \textsc{ sg ter } \\
\underline{flak}\,$\oplus$\,\underline{u}\,$\oplus$\,ka & \textsc{ sg com } \\
\underline{flak}\,$\oplus$\,\underline{u}\,$\oplus$\,d & \textsc{ pl nom } \\
\underline{flak}\,$\oplus$\,k\,$\oplus$\,\underline{u}\,$\oplus$\,jõ & \textsc{ pl gen } \\
\underline{flak}\,$\oplus$\,k\,$\oplus$\,\underline{u}\,$\oplus$\,it & \textsc{ pl par } \\
\underline{flak}\,$\oplus$\,k\,$\oplus$\,\underline{u}\,$\oplus$\,isõ & \textsc{ pl ill } \\
\underline{flak}\,$\oplus$\,k\,$\oplus$\,\underline{u}\,$\oplus$\,iz & \textsc{ pl ine } \\
\underline{flak}\,$\oplus$\,k\,$\oplus$\,\underline{u}\,$\oplus$\,iss & \textsc{ pl ela } \\
\underline{flak}\,$\oplus$\,k\,$\oplus$\,\underline{u}\,$\oplus$\,illõ & \textsc{ pl all } \\
\underline{flak}\,$\oplus$\,k\,$\oplus$\,\underline{u}\,$\oplus$\,ill & \textsc{ pl ade } \\
\underline{flak}\,$\oplus$\,k\,$\oplus$\,\underline{u}\,$\oplus$\,ilt & \textsc{ pl abl } \\
\underline{flak}\,$\oplus$\,k\,$\oplus$\,\underline{u}\,$\oplus$\,issi & \textsc{ pl tra } \\
\underline{flak}\,$\oplus$\,k\,$\oplus$\,\underline{u}\,$\oplus$\,issaa & \textsc{ pl ter } \\
\underline{flak}\,$\oplus$\,k\,$\oplus$\,\underline{u}\,$\oplus$\,ika & \textsc{ pl com } \\
\end{tabular}
\end{sideways}
\captionof{table}{Tüüpsõnamall \arabic{mallinumber}\,\vadja{flakku} ekstraheeritud muutvormimallid.}
\label{tab:tüüpsõnamall-flakku}

\end{minipage}

 
\vspace{1em}
\noindent Tüüpsõnamall \vadja{flakku} hõlmab vormisõnastikus 26 lekseemi: \vadja{\underline{flak}k\underline{u}, \underline{herk}k\underline{u}, \underline{jõkilik}k\underline{o}, \underline{kak}k\underline{o}, \underline{kak}k\underline{u}, \underline{kiik}k\underline{u}, \underline{kolk}k\underline{u}, \underline{kuk}k\underline{o}, \underline{kurk}k\underline{u}, \underline{kuuzik}k\underline{o}, \underline{lepik}k\underline{o}, \underline{liivik}k\underline{o}, \underline{luik}k\underline{o}, \underline{luk}k\underline{u}, \underline{lõõk}k\underline{u}, \underline{majak}k\underline{o}, \underline{musik}k\underline{o}, \underline{mäčizik}k\underline{o}, \underline{naizik}k\underline{o}, \underline{oomnik}k\underline{o}, \underline{pettelik}k\underline{o}, \underline{rehtelkak}k\underline{u}, \underline{seuk}k\underline{o}, \underline{võrk}k\underline{o}, \underline{õzrik}k\underline{o}} ja \vadja{\underline{čerik}k\underline{o}}.

Tüüp\-sõna\-mall kirjeldab tagapoolseid sõnu tüvemuutusega kk:k.


\vspace{3.5em}
\noindent \begin{minipage}{\textwidth}
\stepcounter{mallinumber}
\noindent \textbf{Tüüpsõnamall \arabic{mallinumber}\,\vadja{galstukki}}\\

\begin{sideways}
\begin{tabular}{l l}
muutvormimall & tunnused \\
\hline
\underline{galstuk}\,$\oplus$\,k\,$\oplus$\,\underline{i} & \textsc{ sg nom } \\
\underline{galstuk}\,$\oplus$\,\underline{i} & \textsc{ sg gen } \\
\underline{galstuk}\,$\oplus$\,k\,$\oplus$\,\underline{i}\,$\oplus$\,a & \textsc{ sg par } \\
\underline{galstuk}\,$\oplus$\,k\,$\oplus$\,\underline{i}\,$\oplus$\,sõ & \textsc{ sg ill } \\
\underline{galstuk}\,$\oplus$\,k\,$\oplus$\,\underline{i}\,$\oplus$\,z & \textsc{ sg ine } \\
\underline{galstuk}\,$\oplus$\,\underline{i}\,$\oplus$\,ss & \textsc{ sg ela } \\
\underline{galstuk}\,$\oplus$\,\underline{i}\,$\oplus$\,llõ & \textsc{ sg all } \\
\underline{galstuk}\,$\oplus$\,\underline{i}\,$\oplus$\,ll & \textsc{ sg ade } \\
\underline{galstuk}\,$\oplus$\,\underline{i}\,$\oplus$\,lt & \textsc{ sg abl } \\
\underline{galstuk}\,$\oplus$\,\underline{i}\,$\oplus$\,ssi & \textsc{ sg tra } \\
\underline{galstuk}\,$\oplus$\,k\,$\oplus$\,\underline{i}\,$\oplus$\,ssaa & \textsc{ sg ter } \\
\underline{galstuk}\,$\oplus$\,\underline{i}\,$\oplus$\,ka & \textsc{ sg com } \\
\underline{galstuk}\,$\oplus$\,\underline{i}\,$\oplus$\,d & \textsc{ pl nom } \\
\underline{galstuk}\,$\oplus$\,k\,$\oplus$\,\underline{i}\,$\oplus$\,jõ & \textsc{ pl gen } \\
\underline{galstuk}\,$\oplus$\,k\,$\oplus$\,\underline{i}\,$\oplus$\,it & \textsc{ pl par } \\
\underline{galstuk}\,$\oplus$\,k\,$\oplus$\,\underline{i}\,$\oplus$\,isõ & \textsc{ pl ill } \\
\underline{galstuk}\,$\oplus$\,k\,$\oplus$\,\underline{i}\,$\oplus$\,iz & \textsc{ pl ine } \\
\underline{galstuk}\,$\oplus$\,k\,$\oplus$\,\underline{i}\,$\oplus$\,iss & \textsc{ pl ela } \\
\underline{galstuk}\,$\oplus$\,k\,$\oplus$\,\underline{i}\,$\oplus$\,illõ & \textsc{ pl all } \\
\underline{galstuk}\,$\oplus$\,k\,$\oplus$\,\underline{i}\,$\oplus$\,ill & \textsc{ pl ade } \\
\underline{galstuk}\,$\oplus$\,k\,$\oplus$\,\underline{i}\,$\oplus$\,ilt & \textsc{ pl abl } \\
\underline{galstuk}\,$\oplus$\,k\,$\oplus$\,\underline{i}\,$\oplus$\,issi & \textsc{ pl tra } \\
\underline{galstuk}\,$\oplus$\,k\,$\oplus$\,\underline{i}\,$\oplus$\,issaa & \textsc{ pl ter } \\
\underline{galstuk}\,$\oplus$\,k\,$\oplus$\,\underline{i}\,$\oplus$\,jka & \textsc{ pl com } \\
\end{tabular}
\end{sideways}
\captionof{table}{Tüüpsõnamall \arabic{mallinumber}\,\vadja{galstukki} ekstraheeritud muutvormimallid.}
\label{tab:tüüpsõnamall-galstukki}

\end{minipage}

 
\vspace{1em}
\noindent Tüüpsõnamall \vadja{galstukki} hõlmab vormisõnastikus 7 lekseemi: \vadja{\underline{galstuk}k\underline{i}, \underline{kok}k\underline{i}, \underline{kolk}k\underline{i}, \underline{luuk}k\underline{i}, \underline{puk}k\underline{i}, \underline{vok}k\underline{i}} ja \vadja{\underline{fraak}k\underline{i}}.

Tüüp\-sõna\-mall kirjeldab tagapoolseid sõnu tüvemuutusega kk:k, mille lõpuvokaal on \textit{i}. 


\vspace{3.5em}
\noindent \begin{minipage}{\textwidth}
\stepcounter{mallinumber}
\noindent \textbf{Tüüpsõnamall \arabic{mallinumber}\,\vadja{põlto}}\\

\begin{sideways}
\begin{tabular}{l l}
muutvormimall & tunnused \\
\hline
\underline{põl}\,$\oplus$\,t\,$\oplus$\,\underline{o} & \textsc{ sg nom } \\
\underline{põl}\,$\oplus$\,l\,$\oplus$\,\underline{o} & \textsc{ sg gen } \\
\underline{põl}\,$\oplus$\,t\,$\oplus$\,\underline{o}\,$\oplus$\,a & \textsc{ sg par } \\
\underline{põl}\,$\oplus$\,t\,$\oplus$\,\underline{o}\,$\oplus$\,sõ & \textsc{ sg ill } \\
\underline{põl}\,$\oplus$\,l\,$\oplus$\,\underline{o}\,$\oplus$\,z & \textsc{ sg ine } \\
\underline{põl}\,$\oplus$\,l\,$\oplus$\,\underline{o}\,$\oplus$\,ss & \textsc{ sg ela } \\
\underline{põl}\,$\oplus$\,l\,$\oplus$\,\underline{o}\,$\oplus$\,llõ & \textsc{ sg all } \\
\underline{põl}\,$\oplus$\,l\,$\oplus$\,\underline{o}\,$\oplus$\,ll & \textsc{ sg ade } \\
\underline{põl}\,$\oplus$\,l\,$\oplus$\,\underline{o}\,$\oplus$\,lt & \textsc{ sg abl } \\
\underline{põl}\,$\oplus$\,l\,$\oplus$\,\underline{o}\,$\oplus$\,ssi & \textsc{ sg tra } \\
\underline{põl}\,$\oplus$\,l\,$\oplus$\,\underline{o}\,$\oplus$\,ssaa & \textsc{ sg ter } \\
\underline{põl}\,$\oplus$\,l\,$\oplus$\,\underline{o}\,$\oplus$\,ka & \textsc{ sg com } \\
\underline{põl}\,$\oplus$\,l\,$\oplus$\,\underline{o}\,$\oplus$\,d & \textsc{ pl nom } \\
\underline{põl}\,$\oplus$\,t\,$\oplus$\,\underline{o}\,$\oplus$\,jõ & \textsc{ pl gen } \\
\underline{põl}\,$\oplus$\,t\,$\oplus$\,\underline{o}\,$\oplus$\,it & \textsc{ pl par } \\
\underline{põl}\,$\oplus$\,t\,$\oplus$\,\underline{o}\,$\oplus$\,isõ & \textsc{ pl ill } \\
\underline{põl}\,$\oplus$\,t\,$\oplus$\,\underline{o}\,$\oplus$\,iz & \textsc{ pl ine } \\
\underline{põl}\,$\oplus$\,t\,$\oplus$\,\underline{o}\,$\oplus$\,iss & \textsc{ pl ela } \\
\underline{põl}\,$\oplus$\,t\,$\oplus$\,\underline{o}\,$\oplus$\,illõ & \textsc{ pl all } \\
\underline{põl}\,$\oplus$\,t\,$\oplus$\,\underline{o}\,$\oplus$\,ill & \textsc{ pl ade } \\
\underline{põl}\,$\oplus$\,t\,$\oplus$\,\underline{o}\,$\oplus$\,ilt & \textsc{ pl abl } \\
\underline{põl}\,$\oplus$\,t\,$\oplus$\,\underline{o}\,$\oplus$\,issi & \textsc{ pl tra } \\
\underline{põl}\,$\oplus$\,t\,$\oplus$\,\underline{o}\,$\oplus$\,issaa & \textsc{ pl ter } \\
\underline{põl}\,$\oplus$\,t\,$\oplus$\,\underline{o}\,$\oplus$\,ika & \textsc{ pl com } \\
\end{tabular}
\end{sideways}
\captionof{table}{Tüüpsõnamall \arabic{mallinumber}\,\vadja{põlto} ekstraheeritud muutvormimallid.}
\label{tab:tüüpsõnamall-põlto}

\end{minipage}

 
\vspace{1em}
\noindent Tüüpsõnamall \vadja{põlto} hõlmab vormisõnastikus 2 lekseemi: \vadja{\underline{põl}t\underline{o}} ja \vadja{\underline{mõl}t\underline{o}}.

Tüüp\-sõna\-mall kirjeldab tagapoolseid sõnu tüvemuutusega lt:ll.


\vspace{3.5em}
\noindent \begin{minipage}{\textwidth}
\stepcounter{mallinumber}
\noindent \textbf{Tüüpsõnamall \arabic{mallinumber}\,\vadja{greebeni}}\\

\begin{sideways}
\begin{tabular}{l l}
muutvormimall & tunnused \\
\hline
\underline{greebeni} & \textsc{ sg nom } \\
\underline{greebeni} & \textsc{ sg gen } \\
\underline{greebeni}\,$\oplus$\,ä & \textsc{ sg par } \\
\underline{greebeni}\,$\oplus$\,se & \textsc{ sg ill } \\
\underline{greebeni}\,$\oplus$\,z & \textsc{ sg ine } \\
\underline{greebeni}\,$\oplus$\,ss & \textsc{ sg ela } \\
\underline{greebeni}\,$\oplus$\,lle & \textsc{ sg all } \\
\underline{greebeni}\,$\oplus$\,ll & \textsc{ sg ade } \\
\underline{greebeni}\,$\oplus$\,lt & \textsc{ sg abl } \\
\underline{greebeni}\,$\oplus$\,ssi & \textsc{ sg tra } \\
\underline{greebeni}\,$\oplus$\,ssaa & \textsc{ sg ter } \\
\underline{greebeni}\,$\oplus$\,ka & \textsc{ sg com } \\
\underline{greebeni}\,$\oplus$\,d & \textsc{ pl nom } \\
\underline{greebeni}\,$\oplus$\,je & \textsc{ pl gen } \\
\underline{greebeni}\,$\oplus$\,it & \textsc{ pl par } \\
\underline{greebeni}\,$\oplus$\,ise & \textsc{ pl ill } \\
\underline{greebeni}\,$\oplus$\,iz & \textsc{ pl ine } \\
\underline{greebeni}\,$\oplus$\,iss & \textsc{ pl ela } \\
\underline{greebeni}\,$\oplus$\,ille & \textsc{ pl all } \\
\underline{greebeni}\,$\oplus$\,ill & \textsc{ pl ade } \\
\underline{greebeni}\,$\oplus$\,ilt & \textsc{ pl abl } \\
\underline{greebeni}\,$\oplus$\,issi & \textsc{ pl tra } \\
\underline{greebeni}\,$\oplus$\,issaa & \textsc{ pl ter } \\
\underline{greebeni}\,$\oplus$\,jka & \textsc{ pl com } \\
\end{tabular}
\end{sideways}
\captionof{table}{Tüüpsõnamall \arabic{mallinumber}\,\vadja{greebeni} ekstraheeritud muutvormimallid.}
\label{tab:tüüpsõnamall-greebeni}

\end{minipage}

 
\vspace{1em}
\noindent Tüüpsõnamall \vadja{greebeni} hõlmab vormisõnastikus 15 lekseemi: \vadja{\underline{greebeni}, \underline{Helsengi}, \underline{jevi}, \underline{kiikeri}, \underline{kiisseli}, \underline{meebeli}, \underline{nätel̕i}, \underline{Reeveli}, \underline{retsepti}, \underline{rööveli}, \underline{špeili}, \underline{väli}, \underline{vääri}, \underline{ängeli}} ja \vadja{\underline{bibli}}.

Tüüp\-sõna\-mall kirjeldab eespoolseid tüvemuutuseta sõnu, mille lõpuvokaal on \textit{i}. 


\vspace{3.5em}
\noindent \begin{minipage}{\textwidth}
\stepcounter{mallinumber}
\noindent \textbf{Tüüpsõnamall \arabic{mallinumber}\,\vadja{löülü}}\\

\begin{sideways}
\begin{tabular}{l l}
muutvormimall & tunnused \\
\hline
\underline{löülü} & \textsc{ sg nom } \\
\underline{löülü} & \textsc{ sg gen } \\
\underline{löülü}\,$\oplus$\,ä & \textsc{ sg par } \\
\underline{löülü}\,$\oplus$\,se & \textsc{ sg ill } \\
\underline{löülü}\,$\oplus$\,z & \textsc{ sg ine } \\
\underline{löülü}\,$\oplus$\,ss & \textsc{ sg ela } \\
\underline{löülü}\,$\oplus$\,lle & \textsc{ sg all } \\
\underline{löülü}\,$\oplus$\,ll & \textsc{ sg ade } \\
\underline{löülü}\,$\oplus$\,lt & \textsc{ sg abl } \\
\underline{löülü}\,$\oplus$\,ssi & \textsc{ sg tra } \\
\underline{löülü}\,$\oplus$\,ssaa & \textsc{ sg ter } \\
\underline{löülü}\,$\oplus$\,ka & \textsc{ sg com } \\
\underline{löülü}\,$\oplus$\,d & \textsc{ pl nom } \\
\underline{löülü}\,$\oplus$\,je & \textsc{ pl gen } \\
\underline{löülü}\,$\oplus$\,it & \textsc{ pl par } \\
\underline{löülü}\,$\oplus$\,ise & \textsc{ pl ill } \\
\underline{löülü}\,$\oplus$\,iz & \textsc{ pl ine } \\
\underline{löülü}\,$\oplus$\,iss & \textsc{ pl ela } \\
\underline{löülü}\,$\oplus$\,ille & \textsc{ pl all } \\
\underline{löülü}\,$\oplus$\,ill & \textsc{ pl ade } \\
\underline{löülü}\,$\oplus$\,ilt & \textsc{ pl abl } \\
\underline{löülü}\,$\oplus$\,issi & \textsc{ pl tra } \\
\underline{löülü}\,$\oplus$\,issaa & \textsc{ pl ter } \\
\underline{löülü}\,$\oplus$\,ika & \textsc{ pl com } \\
\end{tabular}
\end{sideways}
\captionof{table}{Tüüpsõnamall \arabic{mallinumber}\,\vadja{löülü} ekstraheeritud muutvormimallid.}
\label{tab:tüüpsõnamall-löülü}

\end{minipage}

 
\vspace{1em}
\noindent Tüüpsõnamall \vadja{löülü} hõlmab vormisõnastikus 3 lekseemi: \vadja{\underline{löülü}, \underline{süčüzü}} ja \vadja{\underline{jürü}}.

Tüüp\-sõna\-mall kirjeldab eespoolseid tüvemuutuseta sõnu.


\vspace{3.5em}
\noindent \begin{minipage}{\textwidth}
\stepcounter{mallinumber}
\noindent \textbf{Tüüpsõnamall \arabic{mallinumber}\,\vadja{airo}}\\

\begin{sideways}
\begin{tabular}{l l}
muutvormimall & tunnused \\
\hline
\underline{airo} & \textsc{ sg nom } \\
\underline{airo} & \textsc{ sg gen } \\
\underline{airo}\,$\oplus$\,a & \textsc{ sg par } \\
\underline{airo}\,$\oplus$\,sõ & \textsc{ sg ill } \\
\underline{airo}\,$\oplus$\,z & \textsc{ sg ine } \\
\underline{airo}\,$\oplus$\,ss & \textsc{ sg ela } \\
\underline{airo}\,$\oplus$\,llõ & \textsc{ sg all } \\
\underline{airo}\,$\oplus$\,ll & \textsc{ sg ade } \\
\underline{airo}\,$\oplus$\,lt & \textsc{ sg abl } \\
\underline{airo}\,$\oplus$\,ssi & \textsc{ sg tra } \\
\underline{airo}\,$\oplus$\,ssaa & \textsc{ sg ter } \\
\underline{airo}\,$\oplus$\,ka & \textsc{ sg com } \\
\underline{airo}\,$\oplus$\,d & \textsc{ pl nom } \\
\underline{airo}\,$\oplus$\,jõ & \textsc{ pl gen } \\
\underline{airo}\,$\oplus$\,it & \textsc{ pl par } \\
\underline{airo}\,$\oplus$\,isõ & \textsc{ pl ill } \\
\underline{airo}\,$\oplus$\,iz & \textsc{ pl ine } \\
\underline{airo}\,$\oplus$\,iss & \textsc{ pl ela } \\
\underline{airo}\,$\oplus$\,illõ & \textsc{ pl all } \\
\underline{airo}\,$\oplus$\,ill & \textsc{ pl ade } \\
\underline{airo}\,$\oplus$\,ilt & \textsc{ pl abl } \\
\underline{airo}\,$\oplus$\,issi & \textsc{ pl tra } \\
\underline{airo}\,$\oplus$\,issaa & \textsc{ pl ter } \\
\underline{airo}\,$\oplus$\,ika & \textsc{ pl com } \\
\end{tabular}
\end{sideways}
\captionof{table}{Tüüpsõnamall \arabic{mallinumber}\,\vadja{airo} ekstraheeritud muutvormimallid.}
\label{tab:tüüpsõnamall-airo}

\end{minipage}

 
\vspace{1em}
\noindent Tüüpsõnamall \vadja{airo} hõlmab vormisõnastikus 42 lekseemi: \vadja{\underline{airo}, \underline{aju}, \underline{anõ}, \underline{čaaju}, \underline{čello}, \underline{elo}, \underline{haadu}, \underline{heeno}, \underline{hlaamu}, \underline{ilo}, \underline{javo}, \underline{jõulu}, \underline{kahu}, \underline{kalmo}, \underline{karu}, \underline{kehno}, \underline{kirstu}, \underline{koivu}, \underline{konno}, \underline{laulu}, \underline{lello}, \underline{morško}, \underline{muru}, \underline{nagru}, \underline{ohtõgo}, \underline{paju}, \underline{paksu}, \underline{pallo}, \underline{passibo}, \underline{pojo}, \underline{saadu}, \underline{savvu}, \underline{siivo}, \underline{škoulu}, \underline{talo}, \underline{varjo}, \underline{vello}, \underline{vilu}, \underline{viro}, \underline{vooro}, \underline{õhtõgo}} ja \vadja{\underline{ahjo}}.

Tüüp\-sõna\-mall kirjeldab tagapoolseid tüvemuutuseta sõnu.


\vspace{3.5em}
\noindent \begin{minipage}{\textwidth}
\stepcounter{mallinumber}
\noindent \textbf{Tüüpsõnamall \arabic{mallinumber}\,\vadja{bagaži}}\\

\begin{sideways}
\begin{tabular}{l l}
muutvormimall & tunnused \\
\hline
\underline{bagaži} & \textsc{ sg nom } \\
\underline{bagaži} & \textsc{ sg gen } \\
\underline{bagaži}\,$\oplus$\,a & \textsc{ sg par } \\
\underline{bagaži}\,$\oplus$\,sõ & \textsc{ sg ill } \\
\underline{bagaži}\,$\oplus$\,z & \textsc{ sg ine } \\
\underline{bagaži}\,$\oplus$\,ss & \textsc{ sg ela } \\
\underline{bagaži}\,$\oplus$\,llõ & \textsc{ sg all } \\
\underline{bagaži}\,$\oplus$\,ll & \textsc{ sg ade } \\
\underline{bagaži}\,$\oplus$\,lt & \textsc{ sg abl } \\
\underline{bagaži}\,$\oplus$\,ssi & \textsc{ sg tra } \\
\underline{bagaži}\,$\oplus$\,ssaa & \textsc{ sg ter } \\
\underline{bagaži}\,$\oplus$\,ka & \textsc{ sg com } \\
\underline{bagaži}\,$\oplus$\,d & \textsc{ pl nom } \\
\underline{bagaži}\,$\oplus$\,jõ & \textsc{ pl gen } \\
\underline{bagaži}\,$\oplus$\,it & \textsc{ pl par } \\
\underline{bagaži}\,$\oplus$\,isõ & \textsc{ pl ill } \\
\underline{bagaži}\,$\oplus$\,iz & \textsc{ pl ine } \\
\underline{bagaži}\,$\oplus$\,iss & \textsc{ pl ela } \\
\underline{bagaži}\,$\oplus$\,illõ & \textsc{ pl all } \\
\underline{bagaži}\,$\oplus$\,ill & \textsc{ pl ade } \\
\underline{bagaži}\,$\oplus$\,ilt & \textsc{ pl abl } \\
\underline{bagaži}\,$\oplus$\,issi & \textsc{ pl tra } \\
\underline{bagaži}\,$\oplus$\,issaa & \textsc{ pl ter } \\
\underline{bagaži}\,$\oplus$\,jka & \textsc{ pl com } \\
\end{tabular}
\end{sideways}
\captionof{table}{Tüüpsõnamall \arabic{mallinumber}\,\vadja{bagaži} ekstraheeritud muutvormimallid.}
\label{tab:tüüpsõnamall-bagaži}

\end{minipage}

 
\vspace{1em}
\noindent Tüüpsõnamall \vadja{bagaži} hõlmab vormisõnastikus 39 lekseemi: \vadja{\underline{bagaži}, \underline{balhoni}, \underline{baroni}, \underline{bil̕jardi}, \underline{bobuli}, \underline{bul̕joni}, \underline{d̕ivani}, \underline{dohtõri}, \underline{farfori}, \underline{flaneli}, \underline{gimnazi}, \underline{gitari}, \underline{glazi}, \underline{haili}, \underline{inspektori}, \underline{itkuri}, \underline{jaani}, \underline{kammõri}, \underline{kongressi}, \underline{kuhni}, \underline{lusti}, \underline{makarooni}, \underline{mal̕ari}, \underline{mandõri}, \underline{naapuri}, \underline{nojaabri}, \underline{nuumõri}, \underline{paperi}, \underline{plaastõri}, \underline{pošti}, \underline{stooli}, \underline{suukkuri}, \underline{taari}, \underline{tormi}, \underline{tunni}, \underline{vagzõli}, \underline{vari}, \underline{vinkuri}} ja \vadja{\underline{almõzi}}.

Tüüp\-sõna\-mall kirjeldab tagapoolseid tüvemuutuseta sõnu, mille lõpuvokaal on \textit{i}. 


\vspace{3.5em}
\noindent \begin{minipage}{\textwidth}
\stepcounter{mallinumber}
\noindent \textbf{Tüüpsõnamall \arabic{mallinumber}\,\vadja{poštaljon}}\\

\begin{sideways}
\begin{tabular}{l l}
muutvormimall & tunnused \\
\hline
\underline{poštaljon} & \textsc{ sg nom } \\
\underline{poštaljon}\,$\oplus$\,i & \textsc{ sg gen } \\
\underline{poštaljon}\,$\oplus$\,ia & \textsc{ sg par } \\
\underline{poštaljon}\,$\oplus$\,isõ & \textsc{ sg ill } \\
\underline{poštaljon}\,$\oplus$\,iz & \textsc{ sg ine } \\
\underline{poštaljon}\,$\oplus$\,iss & \textsc{ sg ela } \\
\underline{poštaljon}\,$\oplus$\,illõ & \textsc{ sg all } \\
\underline{poštaljon}\,$\oplus$\,ill & \textsc{ sg ade } \\
\underline{poštaljon}\,$\oplus$\,ilt & \textsc{ sg abl } \\
\underline{poštaljon}\,$\oplus$\,issi & \textsc{ sg tra } \\
\underline{poštaljon}\,$\oplus$\,issaa & \textsc{ sg ter } \\
\underline{poštaljon}\,$\oplus$\,ika & \textsc{ sg com } \\
\underline{poštaljon}\,$\oplus$\,id & \textsc{ pl nom } \\
\underline{poštaljon}\,$\oplus$\,ijõ & \textsc{ pl gen } \\
\underline{poštaljon}\,$\oplus$\,iit & \textsc{ pl par } \\
\underline{poštaljon}\,$\oplus$\,iisõ & \textsc{ pl ill } \\
\underline{poštaljon}\,$\oplus$\,iiz & \textsc{ pl ine } \\
\underline{poštaljon}\,$\oplus$\,iiss & \textsc{ pl ela } \\
\underline{poštaljon}\,$\oplus$\,iillõ & \textsc{ pl all } \\
\underline{poštaljon}\,$\oplus$\,iill & \textsc{ pl ade } \\
\underline{poštaljon}\,$\oplus$\,iilt & \textsc{ pl abl } \\
\underline{poštaljon}\,$\oplus$\,iissi & \textsc{ pl tra } \\
\underline{poštaljon}\,$\oplus$\,iissaa & \textsc{ pl ter } \\
\underline{poštaljon}\,$\oplus$\,ijka & \textsc{ pl com } \\
\end{tabular}
\end{sideways}
\captionof{table}{Tüüpsõnamall \arabic{mallinumber}\,\vadja{poštaljon} ekstraheeritud muutvormimallid.}
\label{tab:tüüpsõnamall-poštaljon}

\end{minipage}

 
\vspace{1em}
\noindent Tüüpsõnamall \vadja{poštaljon} hõlmab vormisõnastikus 2 lekseemi: \vadja{\underline{poštaljon}} ja \vadja{\underline{parad}}.

Tüüp\-sõna\-mall kirjeldab tagapoolseid tüvemuutuseta sõnu.


\vspace{3.5em}
\noindent \begin{minipage}{\textwidth}
\stepcounter{mallinumber}
\noindent \textbf{Tüüpsõnamall \arabic{mallinumber}\,\vadja{sünti}}\\

\begin{sideways}
\begin{tabular}{l l}
muutvormimall & tunnused \\
\hline
\underline{sün}\,$\oplus$\,t\,$\oplus$\,\underline{i} & \textsc{ sg nom } \\
\underline{sün}\,$\oplus$\,n\,$\oplus$\,\underline{i} & \textsc{ sg gen } \\
\underline{sün}\,$\oplus$\,t\,$\oplus$\,\underline{i}\,$\oplus$\,ä & \textsc{ sg par } \\
\underline{sün}\,$\oplus$\,t\,$\oplus$\,\underline{i}\,$\oplus$\,se & \textsc{ sg ill } \\
\underline{sün}\,$\oplus$\,n\,$\oplus$\,\underline{i}\,$\oplus$\,z & \textsc{ sg ine } \\
\underline{sün}\,$\oplus$\,n\,$\oplus$\,\underline{i}\,$\oplus$\,ss & \textsc{ sg ela } \\
\underline{sün}\,$\oplus$\,n\,$\oplus$\,\underline{i}\,$\oplus$\,lle & \textsc{ sg all } \\
\underline{sün}\,$\oplus$\,n\,$\oplus$\,\underline{i}\,$\oplus$\,ll & \textsc{ sg ade } \\
\underline{sün}\,$\oplus$\,n\,$\oplus$\,\underline{i}\,$\oplus$\,lt & \textsc{ sg abl } \\
\underline{sün}\,$\oplus$\,n\,$\oplus$\,\underline{i}\,$\oplus$\,ssi & \textsc{ sg tra } \\
\underline{sün}\,$\oplus$\,n\,$\oplus$\,\underline{i}\,$\oplus$\,ssaa & \textsc{ sg ter } \\
\underline{sün}\,$\oplus$\,n\,$\oplus$\,\underline{i}\,$\oplus$\,ka & \textsc{ sg com } \\
\underline{sün}\,$\oplus$\,n\,$\oplus$\,\underline{i}\,$\oplus$\,d & \textsc{ pl nom } \\
\underline{sün}\,$\oplus$\,t\,$\oplus$\,\underline{i}\,$\oplus$\,je & \textsc{ pl gen } \\
\underline{sün}\,$\oplus$\,t\,$\oplus$\,\underline{i}\,$\oplus$\,it & \textsc{ pl par } \\
\underline{sün}\,$\oplus$\,t\,$\oplus$\,\underline{i}\,$\oplus$\,ise & \textsc{ pl ill } \\
\underline{sün}\,$\oplus$\,t\,$\oplus$\,\underline{i}\,$\oplus$\,iz & \textsc{ pl ine } \\
\underline{sün}\,$\oplus$\,t\,$\oplus$\,\underline{i}\,$\oplus$\,iss & \textsc{ pl ela } \\
\underline{sün}\,$\oplus$\,t\,$\oplus$\,\underline{i}\,$\oplus$\,ille & \textsc{ pl all } \\
\underline{sün}\,$\oplus$\,t\,$\oplus$\,\underline{i}\,$\oplus$\,ill & \textsc{ pl ade } \\
\underline{sün}\,$\oplus$\,t\,$\oplus$\,\underline{i}\,$\oplus$\,ilt & \textsc{ pl abl } \\
\underline{sün}\,$\oplus$\,t\,$\oplus$\,\underline{i}\,$\oplus$\,issi & \textsc{ pl tra } \\
\underline{sün}\,$\oplus$\,t\,$\oplus$\,\underline{i}\,$\oplus$\,issaa & \textsc{ pl ter } \\
\underline{sün}\,$\oplus$\,t\,$\oplus$\,\underline{i}\,$\oplus$\,jka & \textsc{ pl com } \\
\end{tabular}
\end{sideways}
\captionof{table}{Tüüpsõnamall \arabic{mallinumber}\,\vadja{sünti} ekstraheeritud muutvormimallid.}
\label{tab:tüüpsõnamall-sünti}

\end{minipage}

 
\vspace{1em}
\noindent Tüüpsõnamall \vadja{sünti} ei hõlma teisi lekseeme vormi\-sõnastikus.

Tüüp\-sõna\-mall kirjeldab eespoolseid sõnu tüvemuutusega nt:nn, mille lõpuvokaal on \textit{i}.


\vspace{3.5em}
\noindent \begin{minipage}{\textwidth}
\stepcounter{mallinumber}
\noindent \textbf{Tüüpsõnamall \arabic{mallinumber}\,\vadja{lento}}\\

\begin{sideways}
\begin{tabular}{l l}
muutvormimall & tunnused \\
\hline
\underline{len}\,$\oplus$\,t\,$\oplus$\,\underline{o} & \textsc{ sg nom } \\
\underline{len}\,$\oplus$\,n\,$\oplus$\,\underline{o} & \textsc{ sg gen } \\
\underline{len}\,$\oplus$\,t\,$\oplus$\,\underline{o}\,$\oplus$\,a & \textsc{ sg par } \\
\underline{len}\,$\oplus$\,t\,$\oplus$\,\underline{o}\,$\oplus$\,sõ & \textsc{ sg ill } \\
\underline{len}\,$\oplus$\,n\,$\oplus$\,\underline{o}\,$\oplus$\,z & \textsc{ sg ine } \\
\underline{len}\,$\oplus$\,n\,$\oplus$\,\underline{o}\,$\oplus$\,ss & \textsc{ sg ela } \\
\underline{len}\,$\oplus$\,n\,$\oplus$\,\underline{o}\,$\oplus$\,llõ & \textsc{ sg all } \\
\underline{len}\,$\oplus$\,n\,$\oplus$\,\underline{o}\,$\oplus$\,ll & \textsc{ sg ade } \\
\underline{len}\,$\oplus$\,n\,$\oplus$\,\underline{o}\,$\oplus$\,lt & \textsc{ sg abl } \\
\underline{len}\,$\oplus$\,n\,$\oplus$\,\underline{o}\,$\oplus$\,ssi & \textsc{ sg tra } \\
\underline{len}\,$\oplus$\,n\,$\oplus$\,\underline{o}\,$\oplus$\,ssaa & \textsc{ sg ter } \\
\underline{len}\,$\oplus$\,n\,$\oplus$\,\underline{o}\,$\oplus$\,ka & \textsc{ sg com } \\
\underline{len}\,$\oplus$\,n\,$\oplus$\,\underline{o}\,$\oplus$\,d & \textsc{ pl nom } \\
\underline{len}\,$\oplus$\,t\,$\oplus$\,\underline{o}\,$\oplus$\,jõ & \textsc{ pl gen } \\
\underline{len}\,$\oplus$\,t\,$\oplus$\,\underline{o}\,$\oplus$\,it & \textsc{ pl par } \\
\underline{len}\,$\oplus$\,t\,$\oplus$\,\underline{o}\,$\oplus$\,isõ & \textsc{ pl ill } \\
\underline{len}\,$\oplus$\,t\,$\oplus$\,\underline{o}\,$\oplus$\,iz & \textsc{ pl ine } \\
\underline{len}\,$\oplus$\,t\,$\oplus$\,\underline{o}\,$\oplus$\,iss & \textsc{ pl ela } \\
\underline{len}\,$\oplus$\,t\,$\oplus$\,\underline{o}\,$\oplus$\,illõ & \textsc{ pl all } \\
\underline{len}\,$\oplus$\,t\,$\oplus$\,\underline{o}\,$\oplus$\,ill & \textsc{ pl ade } \\
\underline{len}\,$\oplus$\,t\,$\oplus$\,\underline{o}\,$\oplus$\,ilt & \textsc{ pl abl } \\
\underline{len}\,$\oplus$\,t\,$\oplus$\,\underline{o}\,$\oplus$\,issi & \textsc{ pl tra } \\
\underline{len}\,$\oplus$\,t\,$\oplus$\,\underline{o}\,$\oplus$\,issaa & \textsc{ pl ter } \\
\underline{len}\,$\oplus$\,t\,$\oplus$\,\underline{o}\,$\oplus$\,ika & \textsc{ pl com } \\
\end{tabular}
\end{sideways}
\captionof{table}{Tüüpsõnamall \arabic{mallinumber}\,\vadja{lento} ekstraheeritud muutvormimallid.}
\label{tab:tüüpsõnamall-lento}

\end{minipage}

 
\vspace{1em}
\noindent Tüüpsõnamall \vadja{lento} hõlmab vormisõnastikus 4 lekseemi: \vadja{\underline{len}t\underline{o}, \underline{lin}t\underline{u}, \underline{rokkalin}t\underline{u}} ja \vadja{\underline{kan}t\underline{o}}.

Tüüp\-sõna\-mall kirjeldab tagapoolseid sõnu tüvemuutusega nt:nn.


\vspace{3.5em}
\noindent \begin{minipage}{\textwidth}
\stepcounter{mallinumber}
\noindent \textbf{Tüüpsõnamall \arabic{mallinumber}\,\vadja{vipu}}\\

\begin{sideways}
\begin{tabular}{l l}
muutvormimall & tunnused \\
\hline
\underline{vi}\,$\oplus$\,p\,$\oplus$\,\underline{u} & \textsc{ sg nom } \\
\underline{vi}\,$\oplus$\,v\,$\oplus$\,\underline{u} & \textsc{ sg gen } \\
\underline{vi}\,$\oplus$\,pp\,$\oplus$\,\underline{u}\,$\oplus$\,a & \textsc{ sg par } \\
\underline{vi}\,$\oplus$\,pp\,$\oplus$\,\underline{u}\,$\oplus$\,sõ & \textsc{ sg ill } \\
\underline{vi}\,$\oplus$\,v\,$\oplus$\,\underline{u}\,$\oplus$\,z & \textsc{ sg ine } \\
\underline{vi}\,$\oplus$\,v\,$\oplus$\,\underline{u}\,$\oplus$\,ss & \textsc{ sg ela } \\
\underline{vi}\,$\oplus$\,v\,$\oplus$\,\underline{u}\,$\oplus$\,llõ & \textsc{ sg all } \\
\underline{vi}\,$\oplus$\,v\,$\oplus$\,\underline{u}\,$\oplus$\,ll & \textsc{ sg ade } \\
\underline{vi}\,$\oplus$\,v\,$\oplus$\,\underline{u}\,$\oplus$\,lt & \textsc{ sg abl } \\
\underline{vi}\,$\oplus$\,v\,$\oplus$\,\underline{u}\,$\oplus$\,ssi & \textsc{ sg tra } \\
\underline{vi}\,$\oplus$\,v\,$\oplus$\,\underline{u}\,$\oplus$\,ssaa & \textsc{ sg ter } \\
\underline{vi}\,$\oplus$\,v\,$\oplus$\,\underline{u}\,$\oplus$\,ka & \textsc{ sg com } \\
\underline{vi}\,$\oplus$\,v\,$\oplus$\,\underline{u}\,$\oplus$\,d & \textsc{ pl nom } \\
\underline{vi}\,$\oplus$\,p\,$\oplus$\,\underline{u}\,$\oplus$\,jõ & \textsc{ pl gen } \\
\underline{vi}\,$\oplus$\,p\,$\oplus$\,\underline{u}\,$\oplus$\,it & \textsc{ pl par } \\
\underline{vi}\,$\oplus$\,p\,$\oplus$\,\underline{u}\,$\oplus$\,isõ & \textsc{ pl ill } \\
\underline{vi}\,$\oplus$\,p\,$\oplus$\,\underline{u}\,$\oplus$\,iz & \textsc{ pl ine } \\
\underline{vi}\,$\oplus$\,p\,$\oplus$\,\underline{u}\,$\oplus$\,iss & \textsc{ pl ela } \\
\underline{vi}\,$\oplus$\,p\,$\oplus$\,\underline{u}\,$\oplus$\,illõ & \textsc{ pl all } \\
\underline{vi}\,$\oplus$\,p\,$\oplus$\,\underline{u}\,$\oplus$\,ill & \textsc{ pl ade } \\
\underline{vi}\,$\oplus$\,p\,$\oplus$\,\underline{u}\,$\oplus$\,ilt & \textsc{ pl abl } \\
\underline{vi}\,$\oplus$\,p\,$\oplus$\,\underline{u}\,$\oplus$\,issi & \textsc{ pl tra } \\
\underline{vi}\,$\oplus$\,p\,$\oplus$\,\underline{u}\,$\oplus$\,issaa & \textsc{ pl ter } \\
\underline{vi}\,$\oplus$\,p\,$\oplus$\,\underline{u}\,$\oplus$\,ika & \textsc{ pl com } \\
\end{tabular}
\end{sideways}
\captionof{table}{Tüüpsõnamall \arabic{mallinumber}\,\vadja{vipu} ekstraheeritud muutvormimallid.}
\label{tab:tüüpsõnamall-vipu}

\end{minipage}

 
\vspace{1em}
\noindent Tüüpsõnamall \vadja{vipu} ei hõlma teisi lekseeme vormi\-sõnastikus.

Tüüp\-sõna\-mall kirjeldab tagapoolseid sõnu tüvemuutusega p:v.


\vspace{3.5em}
\noindent \begin{minipage}{\textwidth}
\stepcounter{mallinumber}
\noindent \textbf{Tüüpsõnamall \arabic{mallinumber}\,\vadja{hapo}}\\

\begin{sideways}
\begin{tabular}{l l}
muutvormimall & tunnused \\
\hline
\underline{hap}\,$\oplus$\,\underline{o} & \textsc{ sg nom } \\
\underline{hap}\,$\oplus$\,\underline{o} & \textsc{ sg gen } \\
\underline{hap}\,$\oplus$\,p\,$\oplus$\,\underline{o}\,$\oplus$\,a & \textsc{ sg par } \\
\underline{hap}\,$\oplus$\,p\,$\oplus$\,\underline{o}\,$\oplus$\,sõ & \textsc{ sg ill } \\
\underline{hap}\,$\oplus$\,\underline{o}\,$\oplus$\,z & \textsc{ sg ine } \\
\underline{hap}\,$\oplus$\,\underline{o}\,$\oplus$\,ss & \textsc{ sg ela } \\
\underline{hap}\,$\oplus$\,\underline{o}\,$\oplus$\,llõ & \textsc{ sg all } \\
\underline{hap}\,$\oplus$\,\underline{o}\,$\oplus$\,ll & \textsc{ sg ade } \\
\underline{hap}\,$\oplus$\,\underline{o}\,$\oplus$\,lt & \textsc{ sg abl } \\
\underline{hap}\,$\oplus$\,\underline{o}\,$\oplus$\,ssi & \textsc{ sg tra } \\
\underline{hap}\,$\oplus$\,\underline{o}\,$\oplus$\,ssaa & \textsc{ sg ter } \\
\underline{hap}\,$\oplus$\,\underline{o}\,$\oplus$\,ka & \textsc{ sg com } \\
\underline{hap}\,$\oplus$\,\underline{o}\,$\oplus$\,d & \textsc{ pl nom } \\
\underline{hap}\,$\oplus$\,p\,$\oplus$\,\underline{o}\,$\oplus$\,jõ & \textsc{ pl gen } \\
\underline{hap}\,$\oplus$\,p\,$\oplus$\,\underline{o}\,$\oplus$\,it & \textsc{ pl par } \\
\underline{hap}\,$\oplus$\,p\,$\oplus$\,\underline{o}\,$\oplus$\,isõ & \textsc{ pl ill } \\
\underline{hap}\,$\oplus$\,p\,$\oplus$\,\underline{o}\,$\oplus$\,iz & \textsc{ pl ine } \\
\underline{hap}\,$\oplus$\,p\,$\oplus$\,\underline{o}\,$\oplus$\,iss & \textsc{ pl ela } \\
\underline{hap}\,$\oplus$\,p\,$\oplus$\,\underline{o}\,$\oplus$\,illõ & \textsc{ pl all } \\
\underline{hap}\,$\oplus$\,p\,$\oplus$\,\underline{o}\,$\oplus$\,ill & \textsc{ pl ade } \\
\underline{hap}\,$\oplus$\,p\,$\oplus$\,\underline{o}\,$\oplus$\,ilt & \textsc{ pl abl } \\
\underline{hap}\,$\oplus$\,p\,$\oplus$\,\underline{o}\,$\oplus$\,issi & \textsc{ pl tra } \\
\underline{hap}\,$\oplus$\,p\,$\oplus$\,\underline{o}\,$\oplus$\,issaa & \textsc{ pl ter } \\
\underline{hap}\,$\oplus$\,p\,$\oplus$\,\underline{o}\,$\oplus$\,ika & \textsc{ pl com } \\
\end{tabular}
\end{sideways}
\captionof{table}{Tüüpsõnamall \arabic{mallinumber}\,\vadja{hapo} ekstraheeritud muutvormimallid.}
\label{tab:tüüpsõnamall-hapo}

\end{minipage}

 
\vspace{1em}
\noindent Tüüpsõnamall \vadja{hapo} ei hõlma teisi lekseeme vormi\-sõnastikus.

Tüüp\-sõna\-mall kirjeldab tagapoolseid sõnu tüvemuutusega p:pp.


\vspace{3.5em}
\noindent \begin{minipage}{\textwidth}
\stepcounter{mallinumber}
\noindent \textbf{Tüüpsõnamall \arabic{mallinumber}\,\vadja{vilppi}}\\

\begin{sideways}
\begin{tabular}{l l}
muutvormimall & tunnused \\
\hline
\underline{vilp}\,$\oplus$\,p\,$\oplus$\,\underline{i} & \textsc{ sg nom } \\
\underline{vilp}\,$\oplus$\,\underline{i} & \textsc{ sg gen } \\
\underline{vilp}\,$\oplus$\,p\,$\oplus$\,\underline{i}\,$\oplus$\,ä & \textsc{ sg par } \\
\underline{vilp}\,$\oplus$\,p\,$\oplus$\,\underline{i}\,$\oplus$\,se & \textsc{ sg ill } \\
\underline{vilp}\,$\oplus$\,p\,$\oplus$\,\underline{i}\,$\oplus$\,z & \textsc{ sg ine } \\
\underline{vilp}\,$\oplus$\,\underline{i}\,$\oplus$\,ss & \textsc{ sg ela } \\
\underline{vilp}\,$\oplus$\,\underline{i}\,$\oplus$\,lle & \textsc{ sg all } \\
\underline{vilp}\,$\oplus$\,\underline{i}\,$\oplus$\,ll & \textsc{ sg ade } \\
\underline{vilp}\,$\oplus$\,\underline{i}\,$\oplus$\,lt & \textsc{ sg abl } \\
\underline{vilp}\,$\oplus$\,\underline{i}\,$\oplus$\,ssi & \textsc{ sg tra } \\
\underline{vilp}\,$\oplus$\,p\,$\oplus$\,\underline{i}\,$\oplus$\,ssaa & \textsc{ sg ter } \\
\underline{vilp}\,$\oplus$\,\underline{i}\,$\oplus$\,ka & \textsc{ sg com } \\
\underline{vilp}\,$\oplus$\,\underline{i}\,$\oplus$\,d & \textsc{ pl nom } \\
\underline{vilp}\,$\oplus$\,p\,$\oplus$\,\underline{i}\,$\oplus$\,je & \textsc{ pl gen } \\
\underline{vilp}\,$\oplus$\,p\,$\oplus$\,\underline{i}\,$\oplus$\,it & \textsc{ pl par } \\
\underline{vilp}\,$\oplus$\,p\,$\oplus$\,\underline{i}\,$\oplus$\,ise & \textsc{ pl ill } \\
\underline{vilp}\,$\oplus$\,p\,$\oplus$\,\underline{i}\,$\oplus$\,iz & \textsc{ pl ine } \\
\underline{vilp}\,$\oplus$\,p\,$\oplus$\,\underline{i}\,$\oplus$\,iss & \textsc{ pl ela } \\
\underline{vilp}\,$\oplus$\,p\,$\oplus$\,\underline{i}\,$\oplus$\,ille & \textsc{ pl all } \\
\underline{vilp}\,$\oplus$\,p\,$\oplus$\,\underline{i}\,$\oplus$\,ill & \textsc{ pl ade } \\
\underline{vilp}\,$\oplus$\,p\,$\oplus$\,\underline{i}\,$\oplus$\,ilt & \textsc{ pl abl } \\
\underline{vilp}\,$\oplus$\,p\,$\oplus$\,\underline{i}\,$\oplus$\,issi & \textsc{ pl tra } \\
\underline{vilp}\,$\oplus$\,p\,$\oplus$\,\underline{i}\,$\oplus$\,issaa & \textsc{ pl ter } \\
\underline{vilp}\,$\oplus$\,p\,$\oplus$\,\underline{i}\,$\oplus$\,jka & \textsc{ pl com } \\
\end{tabular}
\end{sideways}
\captionof{table}{Tüüpsõnamall \arabic{mallinumber}\,\vadja{vilppi} ekstraheeritud muutvormimallid.}
\label{tab:tüüpsõnamall-vilppi}

\end{minipage}

 
\vspace{1em}
\noindent Tüüpsõnamall \vadja{vilppi} hõlmab vormisõnastikus 2 lekseemi: \vadja{\underline{vilp}p\underline{i}} ja \vadja{\underline{šlääp}p\underline{i}}.

Tüüp\-sõna\-mall kirjeldab eespoolseid sõnu tüvemuutusega pp:p, mille lõpuvokaal on \textit{i}.


\vspace{3.5em}
\noindent \begin{minipage}{\textwidth}
\stepcounter{mallinumber}
\noindent \textbf{Tüüpsõnamall \arabic{mallinumber}\,\vadja{hüppü}}\\

\begin{sideways}
\begin{tabular}{l l}
muutvormimall & tunnused \\
\hline
\underline{hüp}\,$\oplus$\,p\,$\oplus$\,\underline{ü} & \textsc{ sg nom } \\
\underline{hüp}\,$\oplus$\,\underline{ü} & \textsc{ sg gen } \\
\underline{hüp}\,$\oplus$\,p\,$\oplus$\,\underline{ü}\,$\oplus$\,ä & \textsc{ sg par } \\
\underline{hüp}\,$\oplus$\,p\,$\oplus$\,\underline{ü}\,$\oplus$\,se & \textsc{ sg ill } \\
\underline{hüp}\,$\oplus$\,p\,$\oplus$\,\underline{ü}\,$\oplus$\,z & \textsc{ sg ine } \\
\underline{hüp}\,$\oplus$\,\underline{ü}\,$\oplus$\,ss & \textsc{ sg ela } \\
\underline{hüp}\,$\oplus$\,\underline{ü}\,$\oplus$\,lle & \textsc{ sg all } \\
\underline{hüp}\,$\oplus$\,\underline{ü}\,$\oplus$\,ll & \textsc{ sg ade } \\
\underline{hüp}\,$\oplus$\,\underline{ü}\,$\oplus$\,lt & \textsc{ sg abl } \\
\underline{hüp}\,$\oplus$\,\underline{ü}\,$\oplus$\,ssi & \textsc{ sg tra } \\
\underline{hüp}\,$\oplus$\,p\,$\oplus$\,\underline{ü}\,$\oplus$\,ssaa & \textsc{ sg ter } \\
\underline{hüp}\,$\oplus$\,\underline{ü}\,$\oplus$\,ka & \textsc{ sg com } \\
\underline{hüp}\,$\oplus$\,\underline{ü}\,$\oplus$\,d & \textsc{ pl nom } \\
\underline{hüp}\,$\oplus$\,p\,$\oplus$\,\underline{ü}\,$\oplus$\,je & \textsc{ pl gen } \\
\underline{hüp}\,$\oplus$\,p\,$\oplus$\,\underline{ü}\,$\oplus$\,it & \textsc{ pl par } \\
\underline{hüp}\,$\oplus$\,p\,$\oplus$\,\underline{ü}\,$\oplus$\,ise & \textsc{ pl ill } \\
\underline{hüp}\,$\oplus$\,p\,$\oplus$\,\underline{ü}\,$\oplus$\,iz & \textsc{ pl ine } \\
\underline{hüp}\,$\oplus$\,p\,$\oplus$\,\underline{ü}\,$\oplus$\,iss & \textsc{ pl ela } \\
\underline{hüp}\,$\oplus$\,p\,$\oplus$\,\underline{ü}\,$\oplus$\,ille & \textsc{ pl all } \\
\underline{hüp}\,$\oplus$\,p\,$\oplus$\,\underline{ü}\,$\oplus$\,ill & \textsc{ pl ade } \\
\underline{hüp}\,$\oplus$\,p\,$\oplus$\,\underline{ü}\,$\oplus$\,ilt & \textsc{ pl abl } \\
\underline{hüp}\,$\oplus$\,p\,$\oplus$\,\underline{ü}\,$\oplus$\,issi & \textsc{ pl tra } \\
\underline{hüp}\,$\oplus$\,p\,$\oplus$\,\underline{ü}\,$\oplus$\,issaa & \textsc{ pl ter } \\
\underline{hüp}\,$\oplus$\,p\,$\oplus$\,\underline{ü}\,$\oplus$\,ika & \textsc{ pl com } \\
\end{tabular}
\end{sideways}
\captionof{table}{Tüüpsõnamall \arabic{mallinumber}\,\vadja{hüppü} ekstraheeritud muutvormimallid.}
\label{tab:tüüpsõnamall-hüppü}

\end{minipage}

 
\vspace{1em}
\noindent Tüüpsõnamall \vadja{hüppü} ei hõlma teisi lekseeme vormi\-sõnastikus.

Tüüp\-sõna\-mall kirjeldab eespoolseid sõnu tüvemuutusega pp:p.


\vspace{3.5em}
\noindent \begin{minipage}{\textwidth}
\stepcounter{mallinumber}
\noindent \textbf{Tüüpsõnamall \arabic{mallinumber}\,\vadja{lippu}}\\

\begin{sideways}
\begin{tabular}{l l}
muutvormimall & tunnused \\
\hline
\underline{lip}\,$\oplus$\,p\,$\oplus$\,\underline{u} & \textsc{ sg nom } \\
\underline{lip}\,$\oplus$\,\underline{u} & \textsc{ sg gen } \\
\underline{lip}\,$\oplus$\,p\,$\oplus$\,\underline{u}\,$\oplus$\,a & \textsc{ sg par } \\
\underline{lip}\,$\oplus$\,p\,$\oplus$\,\underline{u}\,$\oplus$\,sõ & \textsc{ sg ill } \\
\underline{lip}\,$\oplus$\,p\,$\oplus$\,\underline{u}\,$\oplus$\,z & \textsc{ sg ine } \\
\underline{lip}\,$\oplus$\,\underline{u}\,$\oplus$\,ss & \textsc{ sg ela } \\
\underline{lip}\,$\oplus$\,\underline{u}\,$\oplus$\,llõ & \textsc{ sg all } \\
\underline{lip}\,$\oplus$\,\underline{u}\,$\oplus$\,ll & \textsc{ sg ade } \\
\underline{lip}\,$\oplus$\,\underline{u}\,$\oplus$\,lt & \textsc{ sg abl } \\
\underline{lip}\,$\oplus$\,\underline{u}\,$\oplus$\,ssi & \textsc{ sg tra } \\
\underline{lip}\,$\oplus$\,p\,$\oplus$\,\underline{u}\,$\oplus$\,ssaa & \textsc{ sg ter } \\
\underline{lip}\,$\oplus$\,\underline{u}\,$\oplus$\,ka & \textsc{ sg com } \\
\underline{lip}\,$\oplus$\,\underline{u}\,$\oplus$\,d & \textsc{ pl nom } \\
\underline{lip}\,$\oplus$\,p\,$\oplus$\,\underline{u}\,$\oplus$\,jõ & \textsc{ pl gen } \\
\underline{lip}\,$\oplus$\,p\,$\oplus$\,\underline{u}\,$\oplus$\,it & \textsc{ pl par } \\
\underline{lip}\,$\oplus$\,p\,$\oplus$\,\underline{u}\,$\oplus$\,isõ & \textsc{ pl ill } \\
\underline{lip}\,$\oplus$\,p\,$\oplus$\,\underline{u}\,$\oplus$\,iz & \textsc{ pl ine } \\
\underline{lip}\,$\oplus$\,p\,$\oplus$\,\underline{u}\,$\oplus$\,iss & \textsc{ pl ela } \\
\underline{lip}\,$\oplus$\,p\,$\oplus$\,\underline{u}\,$\oplus$\,illõ & \textsc{ pl all } \\
\underline{lip}\,$\oplus$\,p\,$\oplus$\,\underline{u}\,$\oplus$\,ill & \textsc{ pl ade } \\
\underline{lip}\,$\oplus$\,p\,$\oplus$\,\underline{u}\,$\oplus$\,ilt & \textsc{ pl abl } \\
\underline{lip}\,$\oplus$\,p\,$\oplus$\,\underline{u}\,$\oplus$\,issi & \textsc{ pl tra } \\
\underline{lip}\,$\oplus$\,p\,$\oplus$\,\underline{u}\,$\oplus$\,issaa & \textsc{ pl ter } \\
\underline{lip}\,$\oplus$\,p\,$\oplus$\,\underline{u}\,$\oplus$\,ika & \textsc{ pl com } \\
\end{tabular}
\end{sideways}
\captionof{table}{Tüüpsõnamall \arabic{mallinumber}\,\vadja{lippu} ekstraheeritud muutvormimallid.}
\label{tab:tüüpsõnamall-lippu}

\end{minipage}

 
\vspace{1em}
\noindent Tüüpsõnamall \vadja{lippu} hõlmab vormisõnastikus 4 lekseemi: \vadja{\underline{lip}p\underline{u}, \underline{lõp}p\underline{u}, \underline{puip}p\underline{u}} ja \vadja{\underline{kip}p\underline{u}}.

Tüüp\-sõna\-mall kirjeldab tagapoolseid sõnu tüvemuutusega pp:p.


\vspace{3.5em}
\noindent \begin{minipage}{\textwidth}
\stepcounter{mallinumber}
\noindent \textbf{Tüüpsõnamall \arabic{mallinumber}\,\vadja{lamppi}}\\

\begin{sideways}
\begin{tabular}{l l}
muutvormimall & tunnused \\
\hline
\underline{lamp}\,$\oplus$\,p\,$\oplus$\,\underline{i} & \textsc{ sg nom } \\
\underline{lamp}\,$\oplus$\,\underline{i} & \textsc{ sg gen } \\
\underline{lamp}\,$\oplus$\,p\,$\oplus$\,\underline{i}\,$\oplus$\,a & \textsc{ sg par } \\
\underline{lamp}\,$\oplus$\,p\,$\oplus$\,\underline{i}\,$\oplus$\,sõ & \textsc{ sg ill } \\
\underline{lamp}\,$\oplus$\,p\,$\oplus$\,\underline{i}\,$\oplus$\,z & \textsc{ sg ine } \\
\underline{lamp}\,$\oplus$\,\underline{i}\,$\oplus$\,ss & \textsc{ sg ela } \\
\underline{lamp}\,$\oplus$\,\underline{i}\,$\oplus$\,llõ & \textsc{ sg all } \\
\underline{lamp}\,$\oplus$\,\underline{i}\,$\oplus$\,ll & \textsc{ sg ade } \\
\underline{lamp}\,$\oplus$\,\underline{i}\,$\oplus$\,lt & \textsc{ sg abl } \\
\underline{lamp}\,$\oplus$\,\underline{i}\,$\oplus$\,ssi & \textsc{ sg tra } \\
\underline{lamp}\,$\oplus$\,p\,$\oplus$\,\underline{i}\,$\oplus$\,ssaa & \textsc{ sg ter } \\
\underline{lamp}\,$\oplus$\,\underline{i}\,$\oplus$\,ka & \textsc{ sg com } \\
\underline{lamp}\,$\oplus$\,\underline{i}\,$\oplus$\,d & \textsc{ pl nom } \\
\underline{lamp}\,$\oplus$\,p\,$\oplus$\,\underline{i}\,$\oplus$\,jõ & \textsc{ pl gen } \\
\underline{lamp}\,$\oplus$\,p\,$\oplus$\,\underline{i}\,$\oplus$\,it & \textsc{ pl par } \\
\underline{lamp}\,$\oplus$\,p\,$\oplus$\,\underline{i}\,$\oplus$\,isõ & \textsc{ pl ill } \\
\underline{lamp}\,$\oplus$\,p\,$\oplus$\,\underline{i}\,$\oplus$\,iz & \textsc{ pl ine } \\
\underline{lamp}\,$\oplus$\,p\,$\oplus$\,\underline{i}\,$\oplus$\,iss & \textsc{ pl ela } \\
\underline{lamp}\,$\oplus$\,p\,$\oplus$\,\underline{i}\,$\oplus$\,illõ & \textsc{ pl all } \\
\underline{lamp}\,$\oplus$\,p\,$\oplus$\,\underline{i}\,$\oplus$\,ill & \textsc{ pl ade } \\
\underline{lamp}\,$\oplus$\,p\,$\oplus$\,\underline{i}\,$\oplus$\,ilt & \textsc{ pl abl } \\
\underline{lamp}\,$\oplus$\,p\,$\oplus$\,\underline{i}\,$\oplus$\,issi & \textsc{ pl tra } \\
\underline{lamp}\,$\oplus$\,p\,$\oplus$\,\underline{i}\,$\oplus$\,issaa & \textsc{ pl ter } \\
\underline{lamp}\,$\oplus$\,p\,$\oplus$\,\underline{i}\,$\oplus$\,jka & \textsc{ pl com } \\
\end{tabular}
\end{sideways}
\captionof{table}{Tüüpsõnamall \arabic{mallinumber}\,\vadja{lamppi} ekstraheeritud muutvormimallid.}
\label{tab:tüüpsõnamall-lamppi}

\end{minipage}

 
\vspace{1em}
\noindent Tüüpsõnamall \vadja{lamppi} hõlmab vormisõnastikus 5 lekseemi: \vadja{\underline{lamp}p\underline{i}, \underline{pap}p\underline{i}, \underline{sup}p\underline{i}, \underline{ukrop}p\underline{i}} ja \vadja{\underline{kaap}p\underline{i}}.

Tüüp\-sõna\-mall kirjeldab tagapoolseid sõnu tüvemuutusega pp:p, mille lõpuvokaal on \textit{i}.


\vspace{3.5em}
\noindent \begin{minipage}{\textwidth}
\stepcounter{mallinumber}
\noindent \textbf{Tüüpsõnamall \arabic{mallinumber}\,\vadja{sese}}\\

\begin{sideways}
\begin{tabular}{l l}
muutvormimall & tunnused \\
\hline
\underline{se}\,$\oplus$\,s\,$\oplus$\,\underline{e} & \textsc{ sg nom } \\
\underline{se}\,$\oplus$\,z\,$\oplus$\,\underline{e} & \textsc{ sg gen } \\
\underline{se}\,$\oplus$\,ss\,$\oplus$\,\underline{e}\,$\oplus$\,ä & \textsc{ sg par } \\
\underline{se}\,$\oplus$\,ss\,$\oplus$\,\underline{e}\,$\oplus$\,se & \textsc{ sg ill } \\
\underline{se}\,$\oplus$\,z\,$\oplus$\,\underline{e}\,$\oplus$\,z & \textsc{ sg ine } \\
\underline{se}\,$\oplus$\,z\,$\oplus$\,\underline{e}\,$\oplus$\,ss & \textsc{ sg ela } \\
\underline{se}\,$\oplus$\,z\,$\oplus$\,\underline{e}\,$\oplus$\,lle & \textsc{ sg all } \\
\underline{se}\,$\oplus$\,z\,$\oplus$\,\underline{e}\,$\oplus$\,ll & \textsc{ sg ade } \\
\underline{se}\,$\oplus$\,z\,$\oplus$\,\underline{e}\,$\oplus$\,lt & \textsc{ sg abl } \\
\underline{se}\,$\oplus$\,z\,$\oplus$\,\underline{e}\,$\oplus$\,ssi & \textsc{ sg tra } \\
\underline{se}\,$\oplus$\,z\,$\oplus$\,\underline{e}\,$\oplus$\,ssaa & \textsc{ sg ter } \\
\underline{se}\,$\oplus$\,z\,$\oplus$\,\underline{e}\,$\oplus$\,ka & \textsc{ sg com } \\
\underline{se}\,$\oplus$\,z\,$\oplus$\,\underline{e}\,$\oplus$\,d & \textsc{ pl nom } \\
\underline{se}\,$\oplus$\,s\,$\oplus$\,\underline{e}\,$\oplus$\,je & \textsc{ pl gen } \\
\underline{se}\,$\oplus$\,s\,$\oplus$\,\underline{e}\,$\oplus$\,it & \textsc{ pl par } \\
\underline{se}\,$\oplus$\,s\,$\oplus$\,\underline{e}\,$\oplus$\,ise & \textsc{ pl ill } \\
\underline{se}\,$\oplus$\,s\,$\oplus$\,\underline{e}\,$\oplus$\,iz & \textsc{ pl ine } \\
\underline{se}\,$\oplus$\,s\,$\oplus$\,\underline{e}\,$\oplus$\,iss & \textsc{ pl ela } \\
\underline{se}\,$\oplus$\,s\,$\oplus$\,\underline{e}\,$\oplus$\,ille & \textsc{ pl all } \\
\underline{se}\,$\oplus$\,s\,$\oplus$\,\underline{e}\,$\oplus$\,ill & \textsc{ pl ade } \\
\underline{se}\,$\oplus$\,s\,$\oplus$\,\underline{e}\,$\oplus$\,ilt & \textsc{ pl abl } \\
\underline{se}\,$\oplus$\,s\,$\oplus$\,\underline{e}\,$\oplus$\,issi & \textsc{ pl tra } \\
\underline{se}\,$\oplus$\,s\,$\oplus$\,\underline{e}\,$\oplus$\,issaa & \textsc{ pl ter } \\
\underline{se}\,$\oplus$\,s\,$\oplus$\,\underline{e}\,$\oplus$\,ika & \textsc{ pl com } \\
\end{tabular}
\end{sideways}
\captionof{table}{Tüüpsõnamall \arabic{mallinumber}\,\vadja{sese} ekstraheeritud muutvormimallid.}
\label{tab:tüüpsõnamall-sese}

\end{minipage}

 
\vspace{1em}
\noindent Tüüpsõnamall \vadja{sese} hõlmab vormisõnastikus 2 lekseemi: \vadja{\underline{se}s\underline{e}} ja \vadja{\underline{lä}s\underline{ü}}.

Tüüp\-sõna\-mall kirjeldab eespoolseid sõnu tüvemuutusega s:z.


\vspace{3.5em}
\noindent \begin{minipage}{\textwidth}
\stepcounter{mallinumber}
\noindent \textbf{Tüüpsõnamall \arabic{mallinumber}\,\vadja{siso}}\\

\begin{sideways}
\begin{tabular}{l l}
muutvormimall & tunnused \\
\hline
\underline{si}\,$\oplus$\,s\,$\oplus$\,\underline{o} & \textsc{ sg nom } \\
\underline{si}\,$\oplus$\,z\,$\oplus$\,\underline{o} & \textsc{ sg gen } \\
\underline{si}\,$\oplus$\,ss\,$\oplus$\,\underline{o}\,$\oplus$\,a & \textsc{ sg par } \\
\underline{si}\,$\oplus$\,ss\,$\oplus$\,\underline{o}\,$\oplus$\,sõ & \textsc{ sg ill } \\
\underline{si}\,$\oplus$\,z\,$\oplus$\,\underline{o}\,$\oplus$\,z & \textsc{ sg ine } \\
\underline{si}\,$\oplus$\,z\,$\oplus$\,\underline{o}\,$\oplus$\,ss & \textsc{ sg ela } \\
\underline{si}\,$\oplus$\,z\,$\oplus$\,\underline{o}\,$\oplus$\,llõ & \textsc{ sg all } \\
\underline{si}\,$\oplus$\,z\,$\oplus$\,\underline{o}\,$\oplus$\,ll & \textsc{ sg ade } \\
\underline{si}\,$\oplus$\,z\,$\oplus$\,\underline{o}\,$\oplus$\,lt & \textsc{ sg abl } \\
\underline{si}\,$\oplus$\,z\,$\oplus$\,\underline{o}\,$\oplus$\,ssi & \textsc{ sg tra } \\
\underline{si}\,$\oplus$\,z\,$\oplus$\,\underline{o}\,$\oplus$\,ssaa & \textsc{ sg ter } \\
\underline{si}\,$\oplus$\,z\,$\oplus$\,\underline{o}\,$\oplus$\,ka & \textsc{ sg com } \\
\underline{si}\,$\oplus$\,z\,$\oplus$\,\underline{o}\,$\oplus$\,d & \textsc{ pl nom } \\
\underline{si}\,$\oplus$\,s\,$\oplus$\,\underline{o}\,$\oplus$\,jõ & \textsc{ pl gen } \\
\underline{si}\,$\oplus$\,s\,$\oplus$\,\underline{o}\,$\oplus$\,it & \textsc{ pl par } \\
\underline{si}\,$\oplus$\,s\,$\oplus$\,\underline{o}\,$\oplus$\,isõ & \textsc{ pl ill } \\
\underline{si}\,$\oplus$\,s\,$\oplus$\,\underline{o}\,$\oplus$\,iz & \textsc{ pl ine } \\
\underline{si}\,$\oplus$\,s\,$\oplus$\,\underline{o}\,$\oplus$\,iss & \textsc{ pl ela } \\
\underline{si}\,$\oplus$\,s\,$\oplus$\,\underline{o}\,$\oplus$\,illõ & \textsc{ pl all } \\
\underline{si}\,$\oplus$\,s\,$\oplus$\,\underline{o}\,$\oplus$\,ill & \textsc{ pl ade } \\
\underline{si}\,$\oplus$\,s\,$\oplus$\,\underline{o}\,$\oplus$\,ilt & \textsc{ pl abl } \\
\underline{si}\,$\oplus$\,s\,$\oplus$\,\underline{o}\,$\oplus$\,issi & \textsc{ pl tra } \\
\underline{si}\,$\oplus$\,s\,$\oplus$\,\underline{o}\,$\oplus$\,issaa & \textsc{ pl ter } \\
\underline{si}\,$\oplus$\,s\,$\oplus$\,\underline{o}\,$\oplus$\,ika & \textsc{ pl com } \\
\end{tabular}
\end{sideways}
\captionof{table}{Tüüpsõnamall \arabic{mallinumber}\,\vadja{siso} ekstraheeritud muutvormimallid.}
\label{tab:tüüpsõnamall-siso}

\end{minipage}

 
\vspace{1em}
\noindent Tüüpsõnamall \vadja{siso} hõlmab vormisõnastikus 2 lekseemi: \vadja{\underline{si}s\underline{o}} ja \vadja{\underline{ni}s\underline{u}}.

Tüüp\-sõna\-mall kirjeldab tagapoolseid sõnu tüvemuutusega s:z.


\vspace{3.5em}
\noindent \begin{minipage}{\textwidth}
\stepcounter{mallinumber}
\noindent \textbf{Tüüpsõnamall \arabic{mallinumber}\,\vadja{mahsu}}\\

\begin{sideways}
\begin{tabular}{l l}
muutvormimall & tunnused \\
\hline
\underline{mah}\,$\oplus$\,s\,$\oplus$\,\underline{u} & \textsc{ sg nom } \\
\underline{mah}\,$\oplus$\,z\,$\oplus$\,\underline{u} & \textsc{ sg gen } \\
\underline{mah}\,$\oplus$\,s\,$\oplus$\,\underline{u}\,$\oplus$\,a & \textsc{ sg par } \\
\underline{mah}\,$\oplus$\,s\,$\oplus$\,\underline{u}\,$\oplus$\,sõ & \textsc{ sg ill } \\
\underline{mah}\,$\oplus$\,z\,$\oplus$\,\underline{u}\,$\oplus$\,z & \textsc{ sg ine } \\
\underline{mah}\,$\oplus$\,z\,$\oplus$\,\underline{u}\,$\oplus$\,ss & \textsc{ sg ela } \\
\underline{mah}\,$\oplus$\,z\,$\oplus$\,\underline{u}\,$\oplus$\,llõ & \textsc{ sg all } \\
\underline{mah}\,$\oplus$\,z\,$\oplus$\,\underline{u}\,$\oplus$\,ll & \textsc{ sg ade } \\
\underline{mah}\,$\oplus$\,z\,$\oplus$\,\underline{u}\,$\oplus$\,lt & \textsc{ sg abl } \\
\underline{mah}\,$\oplus$\,z\,$\oplus$\,\underline{u}\,$\oplus$\,ssi & \textsc{ sg tra } \\
\underline{mah}\,$\oplus$\,s\,$\oplus$\,\underline{u}\,$\oplus$\,ssaa & \textsc{ sg ter } \\
\underline{mah}\,$\oplus$\,z\,$\oplus$\,\underline{u}\,$\oplus$\,ka & \textsc{ sg com } \\
\underline{mah}\,$\oplus$\,z\,$\oplus$\,\underline{u}\,$\oplus$\,d & \textsc{ pl nom } \\
\underline{mah}\,$\oplus$\,s\,$\oplus$\,\underline{u}\,$\oplus$\,jõ & \textsc{ pl gen } \\
\underline{mah}\,$\oplus$\,s\,$\oplus$\,\underline{u}\,$\oplus$\,it & \textsc{ pl par } \\
\underline{mah}\,$\oplus$\,s\,$\oplus$\,\underline{u}\,$\oplus$\,isõ & \textsc{ pl ill } \\
\underline{mah}\,$\oplus$\,s\,$\oplus$\,\underline{u}\,$\oplus$\,iz & \textsc{ pl ine } \\
\underline{mah}\,$\oplus$\,s\,$\oplus$\,\underline{u}\,$\oplus$\,iss & \textsc{ pl ela } \\
\underline{mah}\,$\oplus$\,s\,$\oplus$\,\underline{u}\,$\oplus$\,illõ & \textsc{ pl all } \\
\underline{mah}\,$\oplus$\,s\,$\oplus$\,\underline{u}\,$\oplus$\,ill & \textsc{ pl ade } \\
\underline{mah}\,$\oplus$\,s\,$\oplus$\,\underline{u}\,$\oplus$\,ilt & \textsc{ pl abl } \\
\underline{mah}\,$\oplus$\,s\,$\oplus$\,\underline{u}\,$\oplus$\,issi & \textsc{ pl tra } \\
\underline{mah}\,$\oplus$\,s\,$\oplus$\,\underline{u}\,$\oplus$\,issaa & \textsc{ pl ter } \\
\underline{mah}\,$\oplus$\,s\,$\oplus$\,\underline{u}\,$\oplus$\,ika & \textsc{ pl com } \\
\end{tabular}
\end{sideways}
\captionof{table}{Tüüpsõnamall \arabic{mallinumber}\,\vadja{mahsu} ekstraheeritud muutvormimallid.}
\label{tab:tüüpsõnamall-mahsu}

\end{minipage}

 
\vspace{1em}
\noindent Tüüpsõnamall \vadja{mahsu} hõlmab vormisõnastikus 2 lekseemi: \vadja{\underline{mah}s\underline{u}} ja \vadja{\underline{hai}s\underline{u}}.

Tüüp\-sõna\-mall kirjeldab tagapoolseid sõnu tüvemuutusega s:z, mille tüvi konsonant\-klustri tõttu ei gemineeru.


\vspace{3.5em}
\noindent \begin{minipage}{\textwidth}
\stepcounter{mallinumber}
\noindent \textbf{Tüüpsõnamall \arabic{mallinumber}\,\vadja{kursi}}\\

\begin{sideways}
\begin{tabular}{l l}
muutvormimall & tunnused \\
\hline
\underline{kur}\,$\oplus$\,s\,$\oplus$\,\underline{i} & \textsc{ sg nom } \\
\underline{kur}\,$\oplus$\,z\,$\oplus$\,\underline{i} & \textsc{ sg gen } \\
\underline{kur}\,$\oplus$\,s\,$\oplus$\,\underline{i}\,$\oplus$\,a & \textsc{ sg par } \\
\underline{kur}\,$\oplus$\,s\,$\oplus$\,\underline{i}\,$\oplus$\,sõ & \textsc{ sg ill } \\
\underline{kur}\,$\oplus$\,z\,$\oplus$\,\underline{i}\,$\oplus$\,z & \textsc{ sg ine } \\
\underline{kur}\,$\oplus$\,z\,$\oplus$\,\underline{i}\,$\oplus$\,ss & \textsc{ sg ela } \\
\underline{kur}\,$\oplus$\,z\,$\oplus$\,\underline{i}\,$\oplus$\,llõ & \textsc{ sg all } \\
\underline{kur}\,$\oplus$\,z\,$\oplus$\,\underline{i}\,$\oplus$\,ll & \textsc{ sg ade } \\
\underline{kur}\,$\oplus$\,z\,$\oplus$\,\underline{i}\,$\oplus$\,lt & \textsc{ sg abl } \\
\underline{kur}\,$\oplus$\,z\,$\oplus$\,\underline{i}\,$\oplus$\,ssi & \textsc{ sg tra } \\
\underline{kur}\,$\oplus$\,s\,$\oplus$\,\underline{i}\,$\oplus$\,ssaa & \textsc{ sg ter } \\
\underline{kur}\,$\oplus$\,z\,$\oplus$\,\underline{i}\,$\oplus$\,ka & \textsc{ sg com } \\
\underline{kur}\,$\oplus$\,z\,$\oplus$\,\underline{i}\,$\oplus$\,d & \textsc{ pl nom } \\
\underline{kur}\,$\oplus$\,s\,$\oplus$\,\underline{i}\,$\oplus$\,jõ & \textsc{ pl gen } \\
\underline{kur}\,$\oplus$\,s\,$\oplus$\,\underline{i}\,$\oplus$\,it & \textsc{ pl par } \\
\underline{kur}\,$\oplus$\,s\,$\oplus$\,\underline{i}\,$\oplus$\,isõ & \textsc{ pl ill } \\
\underline{kur}\,$\oplus$\,s\,$\oplus$\,\underline{i}\,$\oplus$\,iz & \textsc{ pl ine } \\
\underline{kur}\,$\oplus$\,s\,$\oplus$\,\underline{i}\,$\oplus$\,iss & \textsc{ pl ela } \\
\underline{kur}\,$\oplus$\,s\,$\oplus$\,\underline{i}\,$\oplus$\,illõ & \textsc{ pl all } \\
\underline{kur}\,$\oplus$\,s\,$\oplus$\,\underline{i}\,$\oplus$\,ill & \textsc{ pl ade } \\
\underline{kur}\,$\oplus$\,s\,$\oplus$\,\underline{i}\,$\oplus$\,ilt & \textsc{ pl abl } \\
\underline{kur}\,$\oplus$\,s\,$\oplus$\,\underline{i}\,$\oplus$\,issi & \textsc{ pl tra } \\
\underline{kur}\,$\oplus$\,s\,$\oplus$\,\underline{i}\,$\oplus$\,issaa & \textsc{ pl ter } \\
\underline{kur}\,$\oplus$\,s\,$\oplus$\,\underline{i}\,$\oplus$\,jka & \textsc{ pl com } \\
\end{tabular}
\end{sideways}
\captionof{table}{Tüüpsõnamall \arabic{mallinumber}\,\vadja{kursi} ekstraheeritud muutvormimallid.}
\label{tab:tüüpsõnamall-kursi}

\end{minipage}

 
\vspace{1em}
\noindent Tüüpsõnamall \vadja{kursi} ei hõlma teisi lekseeme vormi\-sõnastikus.

Tüüp\-sõna\-mall kirjeldab tagapoolseid sõnu tüvemuutusega s:z, mille tüvi konsonant\-klustri tõttu ei gemineeru, mille lõpuvokaal on \textit{i}. 


\vspace{3.5em}
\noindent \begin{minipage}{\textwidth}
\stepcounter{mallinumber}
\noindent \textbf{Tüüpsõnamall \arabic{mallinumber}\,\vadja{rusko}}\\

\begin{sideways}
\begin{tabular}{l l}
muutvormimall & tunnused \\
\hline
\underline{ru}\,$\oplus$\,sk\,$\oplus$\,\underline{o} & \textsc{ sg nom } \\
\underline{ru}\,$\oplus$\,zg\,$\oplus$\,\underline{o} & \textsc{ sg gen } \\
\underline{ru}\,$\oplus$\,sk\,$\oplus$\,\underline{o}\,$\oplus$\,a & \textsc{ sg par } \\
\underline{ru}\,$\oplus$\,sk\,$\oplus$\,\underline{o}\,$\oplus$\,sõ & \textsc{ sg ill } \\
\underline{ru}\,$\oplus$\,zg\,$\oplus$\,\underline{o}\,$\oplus$\,z & \textsc{ sg ine } \\
\underline{ru}\,$\oplus$\,zg\,$\oplus$\,\underline{o}\,$\oplus$\,ss & \textsc{ sg ela } \\
\underline{ru}\,$\oplus$\,zg\,$\oplus$\,\underline{o}\,$\oplus$\,llõ & \textsc{ sg all } \\
\underline{ru}\,$\oplus$\,zg\,$\oplus$\,\underline{o}\,$\oplus$\,ll & \textsc{ sg ade } \\
\underline{ru}\,$\oplus$\,zg\,$\oplus$\,\underline{o}\,$\oplus$\,lt & \textsc{ sg abl } \\
\underline{ru}\,$\oplus$\,zg\,$\oplus$\,\underline{o}\,$\oplus$\,ssi & \textsc{ sg tra } \\
\underline{ru}\,$\oplus$\,zg\,$\oplus$\,\underline{o}\,$\oplus$\,ssaa & \textsc{ sg ter } \\
\underline{ru}\,$\oplus$\,zg\,$\oplus$\,\underline{o}\,$\oplus$\,ka & \textsc{ sg com } \\
\underline{ru}\,$\oplus$\,zg\,$\oplus$\,\underline{o}\,$\oplus$\,d & \textsc{ pl nom } \\
\underline{ru}\,$\oplus$\,sk\,$\oplus$\,\underline{o}\,$\oplus$\,jõ & \textsc{ pl gen } \\
\underline{ru}\,$\oplus$\,sk\,$\oplus$\,\underline{o}\,$\oplus$\,it & \textsc{ pl par } \\
\underline{ru}\,$\oplus$\,sk\,$\oplus$\,\underline{o}\,$\oplus$\,isõ & \textsc{ pl ill } \\
\underline{ru}\,$\oplus$\,sk\,$\oplus$\,\underline{o}\,$\oplus$\,iz & \textsc{ pl ine } \\
\underline{ru}\,$\oplus$\,sk\,$\oplus$\,\underline{o}\,$\oplus$\,iss & \textsc{ pl ela } \\
\underline{ru}\,$\oplus$\,sk\,$\oplus$\,\underline{o}\,$\oplus$\,illõ & \textsc{ pl all } \\
\underline{ru}\,$\oplus$\,sk\,$\oplus$\,\underline{o}\,$\oplus$\,ill & \textsc{ pl ade } \\
\underline{ru}\,$\oplus$\,sk\,$\oplus$\,\underline{o}\,$\oplus$\,ilt & \textsc{ pl abl } \\
\underline{ru}\,$\oplus$\,sk\,$\oplus$\,\underline{o}\,$\oplus$\,issi & \textsc{ pl tra } \\
\underline{ru}\,$\oplus$\,sk\,$\oplus$\,\underline{o}\,$\oplus$\,issaa & \textsc{ pl ter } \\
\underline{ru}\,$\oplus$\,sk\,$\oplus$\,\underline{o}\,$\oplus$\,ika & \textsc{ pl com } \\
\end{tabular}
\end{sideways}
\captionof{table}{Tüüpsõnamall \arabic{mallinumber}\,\vadja{rusko} ekstraheeritud muutvormimallid.}
\label{tab:tüüpsõnamall-rusko}

\end{minipage}

 
\vspace{1em}
\noindent Tüüpsõnamall \vadja{rusko} hõlmab vormisõnastikus 4 lekseemi: \vadja{\underline{ru}sk\underline{o}, \underline{tui}sk\underline{u}, \underline{u}sk\underline{o}} ja \vadja{\underline{pää}sk\underline{o}}.

Tüüp\-sõna\-mall kirjeldab tagapoolseid sõnu tüvemuutusega sk:zg.


\vspace{3.5em}
\noindent \begin{minipage}{\textwidth}
\stepcounter{mallinumber}
\noindent \textbf{Tüüpsõnamall \arabic{mallinumber}\,\vadja{rissi}}\\

\begin{sideways}
\begin{tabular}{l l}
muutvormimall & tunnused \\
\hline
\underline{ris}\,$\oplus$\,s\,$\oplus$\,\underline{i} & \textsc{ sg nom } \\
\underline{ris}\,$\oplus$\,\underline{i} & \textsc{ sg gen } \\
\underline{ris}\,$\oplus$\,s\,$\oplus$\,\underline{i}\,$\oplus$\,ä & \textsc{ sg par } \\
\underline{ris}\,$\oplus$\,s\,$\oplus$\,\underline{i}\,$\oplus$\,se & \textsc{ sg ill } \\
\underline{ris}\,$\oplus$\,s\,$\oplus$\,\underline{i}\,$\oplus$\,z & \textsc{ sg ine } \\
\underline{ris}\,$\oplus$\,\underline{i}\,$\oplus$\,ss & \textsc{ sg ela } \\
\underline{ris}\,$\oplus$\,\underline{i}\,$\oplus$\,lle & \textsc{ sg all } \\
\underline{ris}\,$\oplus$\,\underline{i}\,$\oplus$\,ll & \textsc{ sg ade } \\
\underline{ris}\,$\oplus$\,\underline{i}\,$\oplus$\,lt & \textsc{ sg abl } \\
\underline{ris}\,$\oplus$\,\underline{i}\,$\oplus$\,ssi & \textsc{ sg tra } \\
\underline{ris}\,$\oplus$\,s\,$\oplus$\,\underline{i}\,$\oplus$\,ssaa & \textsc{ sg ter } \\
\underline{ris}\,$\oplus$\,\underline{i}\,$\oplus$\,ka & \textsc{ sg com } \\
\underline{ris}\,$\oplus$\,\underline{i}\,$\oplus$\,d & \textsc{ pl nom } \\
\underline{ris}\,$\oplus$\,s\,$\oplus$\,\underline{i}\,$\oplus$\,je & \textsc{ pl gen } \\
\underline{ris}\,$\oplus$\,s\,$\oplus$\,\underline{i}\,$\oplus$\,it & \textsc{ pl par } \\
\underline{ris}\,$\oplus$\,s\,$\oplus$\,\underline{i}\,$\oplus$\,ise & \textsc{ pl ill } \\
\underline{ris}\,$\oplus$\,s\,$\oplus$\,\underline{i}\,$\oplus$\,iz & \textsc{ pl ine } \\
\underline{ris}\,$\oplus$\,s\,$\oplus$\,\underline{i}\,$\oplus$\,iss & \textsc{ pl ela } \\
\underline{ris}\,$\oplus$\,s\,$\oplus$\,\underline{i}\,$\oplus$\,ille & \textsc{ pl all } \\
\underline{ris}\,$\oplus$\,s\,$\oplus$\,\underline{i}\,$\oplus$\,ill & \textsc{ pl ade } \\
\underline{ris}\,$\oplus$\,s\,$\oplus$\,\underline{i}\,$\oplus$\,ilt & \textsc{ pl abl } \\
\underline{ris}\,$\oplus$\,s\,$\oplus$\,\underline{i}\,$\oplus$\,issi & \textsc{ pl tra } \\
\underline{ris}\,$\oplus$\,s\,$\oplus$\,\underline{i}\,$\oplus$\,issaa & \textsc{ pl ter } \\
\underline{ris}\,$\oplus$\,s\,$\oplus$\,\underline{i}\,$\oplus$\,jka & \textsc{ pl com } \\
\end{tabular}
\end{sideways}
\captionof{table}{Tüüpsõnamall \arabic{mallinumber}\,\vadja{rissi} ekstraheeritud muutvormimallid.}
\label{tab:tüüpsõnamall-rissi}

\end{minipage}

 
\vspace{1em}
\noindent Tüüpsõnamall \vadja{rissi} ei hõlma teisi lekseeme vormi\-sõnastikus.

Tüüp\-sõna\-mall kirjeldab eespoolseid sõnu tüvemuutusega ss:s, mille lõpuvokaal on \textit{i}. 


\vspace{3.5em}
\noindent \begin{minipage}{\textwidth}
\stepcounter{mallinumber}
\noindent \textbf{Tüüpsõnamall \arabic{mallinumber}\,\vadja{passi}}\\

\begin{sideways}
\begin{tabular}{l l}
muutvormimall & tunnused \\
\hline
\underline{pas}\,$\oplus$\,s\,$\oplus$\,\underline{i} & \textsc{ sg nom } \\
\underline{pas}\,$\oplus$\,\underline{i} & \textsc{ sg gen } \\
\underline{pas}\,$\oplus$\,s\,$\oplus$\,\underline{i}\,$\oplus$\,a & \textsc{ sg par } \\
\underline{pas}\,$\oplus$\,s\,$\oplus$\,\underline{i}\,$\oplus$\,sõ & \textsc{ sg ill } \\
\underline{pas}\,$\oplus$\,s\,$\oplus$\,\underline{i}\,$\oplus$\,z & \textsc{ sg ine } \\
\underline{pas}\,$\oplus$\,\underline{i}\,$\oplus$\,ss & \textsc{ sg ela } \\
\underline{pas}\,$\oplus$\,\underline{i}\,$\oplus$\,llõ & \textsc{ sg all } \\
\underline{pas}\,$\oplus$\,\underline{i}\,$\oplus$\,ll & \textsc{ sg ade } \\
\underline{pas}\,$\oplus$\,\underline{i}\,$\oplus$\,lt & \textsc{ sg abl } \\
\underline{pas}\,$\oplus$\,\underline{i}\,$\oplus$\,ssi & \textsc{ sg tra } \\
\underline{pas}\,$\oplus$\,s\,$\oplus$\,\underline{i}\,$\oplus$\,ssaa & \textsc{ sg ter } \\
\underline{pas}\,$\oplus$\,\underline{i}\,$\oplus$\,ka & \textsc{ sg com } \\
\underline{pas}\,$\oplus$\,\underline{i}\,$\oplus$\,d & \textsc{ pl nom } \\
\underline{pas}\,$\oplus$\,s\,$\oplus$\,\underline{i}\,$\oplus$\,jõ & \textsc{ pl gen } \\
\underline{pas}\,$\oplus$\,s\,$\oplus$\,\underline{i}\,$\oplus$\,it & \textsc{ pl par } \\
\underline{pas}\,$\oplus$\,s\,$\oplus$\,\underline{i}\,$\oplus$\,isõ & \textsc{ pl ill } \\
\underline{pas}\,$\oplus$\,s\,$\oplus$\,\underline{i}\,$\oplus$\,iz & \textsc{ pl ine } \\
\underline{pas}\,$\oplus$\,s\,$\oplus$\,\underline{i}\,$\oplus$\,iss & \textsc{ pl ela } \\
\underline{pas}\,$\oplus$\,s\,$\oplus$\,\underline{i}\,$\oplus$\,illõ & \textsc{ pl all } \\
\underline{pas}\,$\oplus$\,s\,$\oplus$\,\underline{i}\,$\oplus$\,ill & \textsc{ pl ade } \\
\underline{pas}\,$\oplus$\,s\,$\oplus$\,\underline{i}\,$\oplus$\,ilt & \textsc{ pl abl } \\
\underline{pas}\,$\oplus$\,s\,$\oplus$\,\underline{i}\,$\oplus$\,issi & \textsc{ pl tra } \\
\underline{pas}\,$\oplus$\,s\,$\oplus$\,\underline{i}\,$\oplus$\,issaa & \textsc{ pl ter } \\
\underline{pas}\,$\oplus$\,s\,$\oplus$\,\underline{i}\,$\oplus$\,jka & \textsc{ pl com } \\
\end{tabular}
\end{sideways}
\captionof{table}{Tüüpsõnamall \arabic{mallinumber}\,\vadja{passi} ekstraheeritud muutvormimallid.}
\label{tab:tüüpsõnamall-passi}

\end{minipage}

 
\vspace{1em}
\noindent Tüüpsõnamall \vadja{passi} hõlmab vormisõnastikus 2 lekseemi: \vadja{\underline{pas}s\underline{i}} ja \vadja{\underline{komis}s\underline{i}}.

Tüüp\-sõna\-mall kirjeldab tagapoolseid sõnu tüvemuutusega ss:s, mille lõpuvokaal on \textit{i}. 


\vspace{3.5em}
\noindent \begin{minipage}{\textwidth}
\stepcounter{mallinumber}
\noindent \textbf{Tüüpsõnamall \arabic{mallinumber}\,\vadja{karjušši}}\\

\begin{sideways}
\begin{tabular}{l l}
muutvormimall & tunnused \\
\hline
\underline{karjuš}\,$\oplus$\,š\,$\oplus$\,\underline{i} & \textsc{ sg nom } \\
\underline{karjuš}\,$\oplus$\,\underline{i} & \textsc{ sg gen } \\
\underline{karjuš}\,$\oplus$\,š\,$\oplus$\,\underline{i}\,$\oplus$\,a & \textsc{ sg par } \\
\underline{karjuš}\,$\oplus$\,š\,$\oplus$\,\underline{i}\,$\oplus$\,sõ & \textsc{ sg ill } \\
\underline{karjuš}\,$\oplus$\,\underline{i}\,$\oplus$\,z & \textsc{ sg ine } \\
\underline{karjuš}\,$\oplus$\,\underline{i}\,$\oplus$\,ss & \textsc{ sg ela } \\
\underline{karjuš}\,$\oplus$\,\underline{i}\,$\oplus$\,llõ & \textsc{ sg all } \\
\underline{karjuš}\,$\oplus$\,\underline{i}\,$\oplus$\,ll & \textsc{ sg ade } \\
\underline{karjuš}\,$\oplus$\,\underline{i}\,$\oplus$\,lt & \textsc{ sg abl } \\
\underline{karjuš}\,$\oplus$\,\underline{i}\,$\oplus$\,ssi & \textsc{ sg tra } \\
\underline{karjuš}\,$\oplus$\,\underline{i}\,$\oplus$\,ssaa & \textsc{ sg ter } \\
\underline{karjuš}\,$\oplus$\,\underline{i}\,$\oplus$\,ka & \textsc{ sg com } \\
\underline{karjuš}\,$\oplus$\,\underline{i}\,$\oplus$\,d & \textsc{ pl nom } \\
\underline{karjuš}\,$\oplus$\,š\,$\oplus$\,\underline{i}\,$\oplus$\,jõ & \textsc{ pl gen } \\
\underline{karjuš}\,$\oplus$\,š\,$\oplus$\,\underline{i}\,$\oplus$\,it & \textsc{ pl par } \\
\underline{karjuš}\,$\oplus$\,š\,$\oplus$\,\underline{i}\,$\oplus$\,isõ & \textsc{ pl ill } \\
\underline{karjuš}\,$\oplus$\,š\,$\oplus$\,\underline{i}\,$\oplus$\,iz & \textsc{ pl ine } \\
\underline{karjuš}\,$\oplus$\,š\,$\oplus$\,\underline{i}\,$\oplus$\,iss & \textsc{ pl ela } \\
\underline{karjuš}\,$\oplus$\,š\,$\oplus$\,\underline{i}\,$\oplus$\,illõ & \textsc{ pl all } \\
\underline{karjuš}\,$\oplus$\,š\,$\oplus$\,\underline{i}\,$\oplus$\,ill & \textsc{ pl ade } \\
\underline{karjuš}\,$\oplus$\,š\,$\oplus$\,\underline{i}\,$\oplus$\,ilt & \textsc{ pl abl } \\
\underline{karjuš}\,$\oplus$\,š\,$\oplus$\,\underline{i}\,$\oplus$\,issi & \textsc{ pl tra } \\
\underline{karjuš}\,$\oplus$\,š\,$\oplus$\,\underline{i}\,$\oplus$\,issaa & \textsc{ pl ter } \\
\underline{karjuš}\,$\oplus$\,š\,$\oplus$\,\underline{i}\,$\oplus$\,jka & \textsc{ pl com } \\
\end{tabular}
\end{sideways}
\captionof{table}{Tüüpsõnamall \arabic{mallinumber}\,\vadja{karjušši} ekstraheeritud muutvormimallid.}
\label{tab:tüüpsõnamall-karjušši}

\end{minipage}

 
\vspace{1em}
\noindent Tüüpsõnamall \vadja{karjušši} hõlmab vormisõnastikus 4 lekseemi: \vadja{\underline{karjuš}š\underline{i}, \underline{latõš}š\underline{i}, \underline{potaš}š\underline{i}} ja \vadja{\underline{fal̕š}š\underline{i}}.

Tüüp\-sõna\-mall kirjeldab tagapoolseid sõnu tüvemuutusega šš:š, mille lõpuvokaal on \textit{i}. 


\vspace{3.5em}
\noindent \begin{minipage}{\textwidth}
\stepcounter{mallinumber}
\noindent \textbf{Tüüpsõnamall \arabic{mallinumber}\,\vadja{täti}}\\

\begin{sideways}
\begin{tabular}{l l}
muutvormimall & tunnused \\
\hline
\underline{tä}\,$\oplus$\,t\,$\oplus$\,\underline{i} & \textsc{ sg nom } \\
\underline{tä}\,$\oplus$\,d\,$\oplus$\,\underline{i} & \textsc{ sg gen } \\
\underline{tä}\,$\oplus$\,tt\,$\oplus$\,\underline{i}\,$\oplus$\,ä & \textsc{ sg par } \\
\underline{tä}\,$\oplus$\,tt\,$\oplus$\,\underline{i}\,$\oplus$\,se & \textsc{ sg ill } \\
\underline{tä}\,$\oplus$\,d\,$\oplus$\,\underline{i}\,$\oplus$\,z & \textsc{ sg ine } \\
\underline{tä}\,$\oplus$\,d\,$\oplus$\,\underline{i}\,$\oplus$\,ss & \textsc{ sg ela } \\
\underline{tä}\,$\oplus$\,d\,$\oplus$\,\underline{i}\,$\oplus$\,lle & \textsc{ sg all } \\
\underline{tä}\,$\oplus$\,d\,$\oplus$\,\underline{i}\,$\oplus$\,ll & \textsc{ sg ade } \\
\underline{tä}\,$\oplus$\,d\,$\oplus$\,\underline{i}\,$\oplus$\,lt & \textsc{ sg abl } \\
\underline{tä}\,$\oplus$\,d\,$\oplus$\,\underline{i}\,$\oplus$\,ssi & \textsc{ sg tra } \\
\underline{tä}\,$\oplus$\,d\,$\oplus$\,\underline{i}\,$\oplus$\,ssaa & \textsc{ sg ter } \\
\underline{tä}\,$\oplus$\,d\,$\oplus$\,\underline{i}\,$\oplus$\,ka & \textsc{ sg com } \\
\underline{tä}\,$\oplus$\,d\,$\oplus$\,\underline{i}\,$\oplus$\,d & \textsc{ pl nom } \\
\underline{tä}\,$\oplus$\,t\,$\oplus$\,\underline{i}\,$\oplus$\,je & \textsc{ pl gen } \\
\underline{tä}\,$\oplus$\,t\,$\oplus$\,\underline{i}\,$\oplus$\,it & \textsc{ pl par } \\
\underline{tä}\,$\oplus$\,t\,$\oplus$\,\underline{i}\,$\oplus$\,ise & \textsc{ pl ill } \\
\underline{tä}\,$\oplus$\,t\,$\oplus$\,\underline{i}\,$\oplus$\,iz & \textsc{ pl ine } \\
\underline{tä}\,$\oplus$\,t\,$\oplus$\,\underline{i}\,$\oplus$\,iss & \textsc{ pl ela } \\
\underline{tä}\,$\oplus$\,t\,$\oplus$\,\underline{i}\,$\oplus$\,ille & \textsc{ pl all } \\
\underline{tä}\,$\oplus$\,t\,$\oplus$\,\underline{i}\,$\oplus$\,ill & \textsc{ pl ade } \\
\underline{tä}\,$\oplus$\,t\,$\oplus$\,\underline{i}\,$\oplus$\,ilt & \textsc{ pl abl } \\
\underline{tä}\,$\oplus$\,t\,$\oplus$\,\underline{i}\,$\oplus$\,issi & \textsc{ pl tra } \\
\underline{tä}\,$\oplus$\,t\,$\oplus$\,\underline{i}\,$\oplus$\,issaa & \textsc{ pl ter } \\
\underline{tä}\,$\oplus$\,t\,$\oplus$\,\underline{i}\,$\oplus$\,jka & \textsc{ pl com } \\
\end{tabular}
\end{sideways}
\captionof{table}{Tüüpsõnamall \arabic{mallinumber}\,\vadja{täti} ekstraheeritud muutvormimallid.}
\label{tab:tüüpsõnamall-täti}

\end{minipage}

 
\vspace{1em}
\noindent Tüüpsõnamall \vadja{täti} ei hõlma teisi lekseeme vormi\-sõnastikus.

Tüüp\-sõna\-mall kirjeldab eespoolseid sõnu tüvemuutusega t:d, mille lõpuvokaal on \textit{i}. 


\vspace{3.5em}
\noindent \begin{minipage}{\textwidth}
\stepcounter{mallinumber}
\noindent \textbf{Tüüpsõnamall \arabic{mallinumber}\,\vadja{kotko}}\\

\begin{sideways}
\begin{tabular}{l l}
muutvormimall & tunnused \\
\hline
\underline{ko}\,$\oplus$\,tk\,$\oplus$\,\underline{o} & \textsc{ sg nom } \\
\underline{ko}\,$\oplus$\,dg\,$\oplus$\,\underline{o} & \textsc{ sg gen } \\
\underline{ko}\,$\oplus$\,tk\,$\oplus$\,\underline{o}\,$\oplus$\,a & \textsc{ sg par } \\
\underline{ko}\,$\oplus$\,tk\,$\oplus$\,\underline{o}\,$\oplus$\,sõ & \textsc{ sg ill } \\
\underline{ko}\,$\oplus$\,dg\,$\oplus$\,\underline{o}\,$\oplus$\,z & \textsc{ sg ine } \\
\underline{ko}\,$\oplus$\,dg\,$\oplus$\,\underline{o}\,$\oplus$\,ss & \textsc{ sg ela } \\
\underline{ko}\,$\oplus$\,dg\,$\oplus$\,\underline{o}\,$\oplus$\,llõ & \textsc{ sg all } \\
\underline{ko}\,$\oplus$\,dg\,$\oplus$\,\underline{o}\,$\oplus$\,ll & \textsc{ sg ade } \\
\underline{ko}\,$\oplus$\,dg\,$\oplus$\,\underline{o}\,$\oplus$\,lt & \textsc{ sg abl } \\
\underline{ko}\,$\oplus$\,dg\,$\oplus$\,\underline{o}\,$\oplus$\,ssi & \textsc{ sg tra } \\
\underline{ko}\,$\oplus$\,dg\,$\oplus$\,\underline{o}\,$\oplus$\,ssaa & \textsc{ sg ter } \\
\underline{ko}\,$\oplus$\,dg\,$\oplus$\,\underline{o}\,$\oplus$\,ka & \textsc{ sg com } \\
\underline{ko}\,$\oplus$\,dg\,$\oplus$\,\underline{o}\,$\oplus$\,d & \textsc{ pl nom } \\
\underline{ko}\,$\oplus$\,tk\,$\oplus$\,\underline{o}\,$\oplus$\,jõ & \textsc{ pl gen } \\
\underline{ko}\,$\oplus$\,tk\,$\oplus$\,\underline{o}\,$\oplus$\,it & \textsc{ pl par } \\
\underline{ko}\,$\oplus$\,tk\,$\oplus$\,\underline{o}\,$\oplus$\,isõ & \textsc{ pl ill } \\
\underline{ko}\,$\oplus$\,tk\,$\oplus$\,\underline{o}\,$\oplus$\,iz & \textsc{ pl ine } \\
\underline{ko}\,$\oplus$\,tk\,$\oplus$\,\underline{o}\,$\oplus$\,iss & \textsc{ pl ela } \\
\underline{ko}\,$\oplus$\,tk\,$\oplus$\,\underline{o}\,$\oplus$\,illõ & \textsc{ pl all } \\
\underline{ko}\,$\oplus$\,tk\,$\oplus$\,\underline{o}\,$\oplus$\,ill & \textsc{ pl ade } \\
\underline{ko}\,$\oplus$\,tk\,$\oplus$\,\underline{o}\,$\oplus$\,ilt & \textsc{ pl abl } \\
\underline{ko}\,$\oplus$\,tk\,$\oplus$\,\underline{o}\,$\oplus$\,issi & \textsc{ pl tra } \\
\underline{ko}\,$\oplus$\,tk\,$\oplus$\,\underline{o}\,$\oplus$\,issaa & \textsc{ pl ter } \\
\underline{ko}\,$\oplus$\,tk\,$\oplus$\,\underline{o}\,$\oplus$\,ika & \textsc{ pl com } \\
\end{tabular}
\end{sideways}
\captionof{table}{Tüüpsõnamall \arabic{mallinumber}\,\vadja{kotko} ekstraheeritud muutvormimallid.}
\label{tab:tüüpsõnamall-kotko}

\end{minipage}

 
\vspace{1em}
\noindent Tüüpsõnamall \vadja{kotko} hõlmab vormisõnastikus 3 lekseemi: \vadja{\underline{ko}tk\underline{o}, \underline{laa}tk\underline{o}} ja \vadja{\underline{i}tk\underline{u}}.

Tüüp\-sõna\-mall kirjeldab tagapoolseid sõnu tüvemuutusega tk:dg.


\vspace{3.5em}
\noindent \begin{minipage}{\textwidth}
\stepcounter{mallinumber}
\noindent \textbf{Tüüpsõnamall \arabic{mallinumber}\,\vadja{kittsi}}\\

\begin{sideways}
\begin{tabular}{l l}
muutvormimall & tunnused \\
\hline
\underline{kit}\,$\oplus$\,t\,$\oplus$\,\underline{si} & \textsc{ sg nom } \\
\underline{kit}\,$\oplus$\,\underline{si} & \textsc{ sg gen } \\
\underline{kit}\,$\oplus$\,t\,$\oplus$\,\underline{si}\,$\oplus$\,ä & \textsc{ sg par } \\
\underline{kit}\,$\oplus$\,t\,$\oplus$\,\underline{si}\,$\oplus$\,se & \textsc{ sg ill } \\
\underline{kit}\,$\oplus$\,t\,$\oplus$\,\underline{si}\,$\oplus$\,z & \textsc{ sg ine } \\
\underline{kit}\,$\oplus$\,\underline{si}\,$\oplus$\,ss & \textsc{ sg ela } \\
\underline{kit}\,$\oplus$\,\underline{si}\,$\oplus$\,lle & \textsc{ sg all } \\
\underline{kit}\,$\oplus$\,\underline{si}\,$\oplus$\,ll & \textsc{ sg ade } \\
\underline{kit}\,$\oplus$\,\underline{si}\,$\oplus$\,lt & \textsc{ sg abl } \\
\underline{kit}\,$\oplus$\,\underline{si}\,$\oplus$\,ssi & \textsc{ sg tra } \\
\underline{kit}\,$\oplus$\,t\,$\oplus$\,\underline{si}\,$\oplus$\,ssaa & \textsc{ sg ter } \\
\underline{kit}\,$\oplus$\,\underline{si}\,$\oplus$\,ka & \textsc{ sg com } \\
\underline{kit}\,$\oplus$\,\underline{si}\,$\oplus$\,d & \textsc{ pl nom } \\
\underline{kit}\,$\oplus$\,t\,$\oplus$\,\underline{si}\,$\oplus$\,je & \textsc{ pl gen } \\
\underline{kit}\,$\oplus$\,t\,$\oplus$\,\underline{si}\,$\oplus$\,it & \textsc{ pl par } \\
\underline{kit}\,$\oplus$\,t\,$\oplus$\,\underline{si}\,$\oplus$\,ise & \textsc{ pl ill } \\
\underline{kit}\,$\oplus$\,t\,$\oplus$\,\underline{si}\,$\oplus$\,iz & \textsc{ pl ine } \\
\underline{kit}\,$\oplus$\,t\,$\oplus$\,\underline{si}\,$\oplus$\,iss & \textsc{ pl ela } \\
\underline{kit}\,$\oplus$\,t\,$\oplus$\,\underline{si}\,$\oplus$\,ille & \textsc{ pl all } \\
\underline{kit}\,$\oplus$\,t\,$\oplus$\,\underline{si}\,$\oplus$\,ill & \textsc{ pl ade } \\
\underline{kit}\,$\oplus$\,t\,$\oplus$\,\underline{si}\,$\oplus$\,ilt & \textsc{ pl abl } \\
\underline{kit}\,$\oplus$\,t\,$\oplus$\,\underline{si}\,$\oplus$\,issi & \textsc{ pl tra } \\
\underline{kit}\,$\oplus$\,t\,$\oplus$\,\underline{si}\,$\oplus$\,issaa & \textsc{ pl ter } \\
\underline{kit}\,$\oplus$\,t\,$\oplus$\,\underline{si}\,$\oplus$\,jka & \textsc{ pl com } \\
\end{tabular}
\end{sideways}
\captionof{table}{Tüüpsõnamall \arabic{mallinumber}\,\vadja{kittsi} ekstraheeritud muutvormimallid.}
\label{tab:tüüpsõnamall-kittsi}

\end{minipage}

 
\vspace{1em}
\noindent Tüüpsõnamall \vadja{kittsi} hõlmab vormisõnastikus 4 lekseemi: \vadja{\underline{kit}t\underline{si}, \underline{komferent}t\underline{si}, \underline{plet}t\underline{i}} ja \vadja{\underline{bilet}t\underline{i}}.

Tüüp\-sõna\-mall kirjeldab eespoolseid sõnu tüvemuutusega tt:t, mille lõpuvokaal on \textit{i}. 


\vspace{3.5em}
\noindent \begin{minipage}{\textwidth}
\stepcounter{mallinumber}
\noindent \textbf{Tüüpsõnamall \arabic{mallinumber}\,\vadja{rätte}}\\

\begin{sideways}
\begin{tabular}{l l}
muutvormimall & tunnused \\
\hline
\underline{rät}\,$\oplus$\,t\,$\oplus$\,\underline{e} & \textsc{ sg nom } \\
\underline{rät}\,$\oplus$\,\underline{e} & \textsc{ sg gen } \\
\underline{rät}\,$\oplus$\,t\,$\oplus$\,\underline{e}\,$\oplus$\,ä & \textsc{ sg par } \\
\underline{rät}\,$\oplus$\,t\,$\oplus$\,\underline{e}\,$\oplus$\,se & \textsc{ sg ill } \\
\underline{rät}\,$\oplus$\,t\,$\oplus$\,\underline{e}\,$\oplus$\,z & \textsc{ sg ine } \\
\underline{rät}\,$\oplus$\,\underline{e}\,$\oplus$\,ss & \textsc{ sg ela } \\
\underline{rät}\,$\oplus$\,\underline{e}\,$\oplus$\,lle & \textsc{ sg all } \\
\underline{rät}\,$\oplus$\,\underline{e}\,$\oplus$\,ll & \textsc{ sg ade } \\
\underline{rät}\,$\oplus$\,\underline{e}\,$\oplus$\,lt & \textsc{ sg abl } \\
\underline{rät}\,$\oplus$\,\underline{e}\,$\oplus$\,ssi & \textsc{ sg tra } \\
\underline{rät}\,$\oplus$\,t\,$\oplus$\,\underline{e}\,$\oplus$\,ssaa & \textsc{ sg ter } \\
\underline{rät}\,$\oplus$\,\underline{e}\,$\oplus$\,ka & \textsc{ sg com } \\
\underline{rät}\,$\oplus$\,\underline{e}\,$\oplus$\,d & \textsc{ pl nom } \\
\underline{rät}\,$\oplus$\,t\,$\oplus$\,\underline{e}\,$\oplus$\,je & \textsc{ pl gen } \\
\underline{rät}\,$\oplus$\,t\,$\oplus$\,\underline{e}\,$\oplus$\,it & \textsc{ pl par } \\
\underline{rät}\,$\oplus$\,t\,$\oplus$\,\underline{e}\,$\oplus$\,ise & \textsc{ pl ill } \\
\underline{rät}\,$\oplus$\,t\,$\oplus$\,\underline{e}\,$\oplus$\,iz & \textsc{ pl ine } \\
\underline{rät}\,$\oplus$\,t\,$\oplus$\,\underline{e}\,$\oplus$\,iss & \textsc{ pl ela } \\
\underline{rät}\,$\oplus$\,t\,$\oplus$\,\underline{e}\,$\oplus$\,ille & \textsc{ pl all } \\
\underline{rät}\,$\oplus$\,t\,$\oplus$\,\underline{e}\,$\oplus$\,ill & \textsc{ pl ade } \\
\underline{rät}\,$\oplus$\,t\,$\oplus$\,\underline{e}\,$\oplus$\,ilt & \textsc{ pl abl } \\
\underline{rät}\,$\oplus$\,t\,$\oplus$\,\underline{e}\,$\oplus$\,issi & \textsc{ pl tra } \\
\underline{rät}\,$\oplus$\,t\,$\oplus$\,\underline{e}\,$\oplus$\,issaa & \textsc{ pl ter } \\
\underline{rät}\,$\oplus$\,t\,$\oplus$\,\underline{e}\,$\oplus$\,ika & \textsc{ pl com } \\
\end{tabular}
\end{sideways}
\captionof{table}{Tüüpsõnamall \arabic{mallinumber}\,\vadja{rätte} ekstraheeritud muutvormimallid.}
\label{tab:tüüpsõnamall-rätte}

\end{minipage}

 
\vspace{1em}
\noindent Tüüpsõnamall \vadja{rätte} hõlmab vormisõnastikus 3 lekseemi: \vadja{\underline{rät}t\underline{e}, \underline{tüt}t\underline{ö}} ja \vadja{\underline{nenärät}t\underline{e}}.

Tüüp\-sõna\-mall kirjeldab eespoolseid sõnu tüvemuutusega tt:t.


\vspace{3.5em}
\noindent \begin{minipage}{\textwidth}
\stepcounter{mallinumber}
\noindent \textbf{Tüüpsõnamall \arabic{mallinumber}\,\vadja{hattu}}\\

\begin{sideways}
\begin{tabular}{l l}
muutvormimall & tunnused \\
\hline
\underline{hat}\,$\oplus$\,t\,$\oplus$\,\underline{u} & \textsc{ sg nom } \\
\underline{hat}\,$\oplus$\,\underline{u} & \textsc{ sg gen } \\
\underline{hat}\,$\oplus$\,t\,$\oplus$\,\underline{u}\,$\oplus$\,a & \textsc{ sg par } \\
\underline{hat}\,$\oplus$\,t\,$\oplus$\,\underline{u}\,$\oplus$\,sõ & \textsc{ sg ill } \\
\underline{hat}\,$\oplus$\,t\,$\oplus$\,\underline{u}\,$\oplus$\,z & \textsc{ sg ine } \\
\underline{hat}\,$\oplus$\,\underline{u}\,$\oplus$\,ss & \textsc{ sg ela } \\
\underline{hat}\,$\oplus$\,\underline{u}\,$\oplus$\,llõ & \textsc{ sg all } \\
\underline{hat}\,$\oplus$\,\underline{u}\,$\oplus$\,ll & \textsc{ sg ade } \\
\underline{hat}\,$\oplus$\,\underline{u}\,$\oplus$\,lt & \textsc{ sg abl } \\
\underline{hat}\,$\oplus$\,\underline{u}\,$\oplus$\,ssi & \textsc{ sg tra } \\
\underline{hat}\,$\oplus$\,t\,$\oplus$\,\underline{u}\,$\oplus$\,ssaa & \textsc{ sg ter } \\
\underline{hat}\,$\oplus$\,\underline{u}\,$\oplus$\,ka & \textsc{ sg com } \\
\underline{hat}\,$\oplus$\,\underline{u}\,$\oplus$\,d & \textsc{ pl nom } \\
\underline{hat}\,$\oplus$\,t\,$\oplus$\,\underline{u}\,$\oplus$\,jõ & \textsc{ pl gen } \\
\underline{hat}\,$\oplus$\,t\,$\oplus$\,\underline{u}\,$\oplus$\,it & \textsc{ pl par } \\
\underline{hat}\,$\oplus$\,t\,$\oplus$\,\underline{u}\,$\oplus$\,isõ & \textsc{ pl ill } \\
\underline{hat}\,$\oplus$\,t\,$\oplus$\,\underline{u}\,$\oplus$\,iz & \textsc{ pl ine } \\
\underline{hat}\,$\oplus$\,t\,$\oplus$\,\underline{u}\,$\oplus$\,iss & \textsc{ pl ela } \\
\underline{hat}\,$\oplus$\,t\,$\oplus$\,\underline{u}\,$\oplus$\,illõ & \textsc{ pl all } \\
\underline{hat}\,$\oplus$\,t\,$\oplus$\,\underline{u}\,$\oplus$\,ill & \textsc{ pl ade } \\
\underline{hat}\,$\oplus$\,t\,$\oplus$\,\underline{u}\,$\oplus$\,ilt & \textsc{ pl abl } \\
\underline{hat}\,$\oplus$\,t\,$\oplus$\,\underline{u}\,$\oplus$\,issi & \textsc{ pl tra } \\
\underline{hat}\,$\oplus$\,t\,$\oplus$\,\underline{u}\,$\oplus$\,issaa & \textsc{ pl ter } \\
\underline{hat}\,$\oplus$\,t\,$\oplus$\,\underline{u}\,$\oplus$\,ika & \textsc{ pl com } \\
\end{tabular}
\end{sideways}
\captionof{table}{Tüüpsõnamall \arabic{mallinumber}\,\vadja{hattu} ekstraheeritud muutvormimallid.}
\label{tab:tüüpsõnamall-hattu}

\end{minipage}

 
\vspace{1em}
\noindent Tüüpsõnamall \vadja{hattu} hõlmab vormisõnastikus 10 lekseemi: \vadja{\underline{hat}t\underline{u}, \underline{jut}t\underline{u}, \underline{kat}t\underline{o}, \underline{kut}t\underline{su}, \underline{laat}t\underline{o}, \underline{lant}t\underline{u}, \underline{pal̕t}t\underline{o}, \underline{port}t\underline{u}, \underline{Tart}t\underline{u}} ja \vadja{\underline{čiut}t\underline{o}}.

Tüüp\-sõna\-mall kirjeldab tagapoolseid sõnu tüvemuutusega tt:t.


\vspace{3.5em}
\noindent \begin{minipage}{\textwidth}
\stepcounter{mallinumber}
\noindent \textbf{Tüüpsõnamall \arabic{mallinumber}\,\vadja{dokumentti}}\\

\begin{sideways}
\begin{tabular}{l l}
muutvormimall & tunnused \\
\hline
\underline{dokument}\,$\oplus$\,t\,$\oplus$\,\underline{i} & \textsc{ sg nom } \\
\underline{dokument}\,$\oplus$\,\underline{i} & \textsc{ sg gen } \\
\underline{dokument}\,$\oplus$\,t\,$\oplus$\,\underline{i}\,$\oplus$\,a & \textsc{ sg par } \\
\underline{dokument}\,$\oplus$\,t\,$\oplus$\,\underline{i}\,$\oplus$\,sõ & \textsc{ sg ill } \\
\underline{dokument}\,$\oplus$\,t\,$\oplus$\,\underline{i}\,$\oplus$\,z & \textsc{ sg ine } \\
\underline{dokument}\,$\oplus$\,\underline{i}\,$\oplus$\,ss & \textsc{ sg ela } \\
\underline{dokument}\,$\oplus$\,\underline{i}\,$\oplus$\,llõ & \textsc{ sg all } \\
\underline{dokument}\,$\oplus$\,\underline{i}\,$\oplus$\,ll & \textsc{ sg ade } \\
\underline{dokument}\,$\oplus$\,\underline{i}\,$\oplus$\,lt & \textsc{ sg abl } \\
\underline{dokument}\,$\oplus$\,\underline{i}\,$\oplus$\,ssi & \textsc{ sg tra } \\
\underline{dokument}\,$\oplus$\,t\,$\oplus$\,\underline{i}\,$\oplus$\,ssaa & \textsc{ sg ter } \\
\underline{dokument}\,$\oplus$\,\underline{i}\,$\oplus$\,ka & \textsc{ sg com } \\
\underline{dokument}\,$\oplus$\,\underline{i}\,$\oplus$\,d & \textsc{ pl nom } \\
\underline{dokument}\,$\oplus$\,t\,$\oplus$\,\underline{i}\,$\oplus$\,jõ & \textsc{ pl gen } \\
\underline{dokument}\,$\oplus$\,t\,$\oplus$\,\underline{i}\,$\oplus$\,it & \textsc{ pl par } \\
\underline{dokument}\,$\oplus$\,t\,$\oplus$\,\underline{i}\,$\oplus$\,isõ & \textsc{ pl ill } \\
\underline{dokument}\,$\oplus$\,t\,$\oplus$\,\underline{i}\,$\oplus$\,iz & \textsc{ pl ine } \\
\underline{dokument}\,$\oplus$\,t\,$\oplus$\,\underline{i}\,$\oplus$\,iss & \textsc{ pl ela } \\
\underline{dokument}\,$\oplus$\,t\,$\oplus$\,\underline{i}\,$\oplus$\,illõ & \textsc{ pl all } \\
\underline{dokument}\,$\oplus$\,t\,$\oplus$\,\underline{i}\,$\oplus$\,ill & \textsc{ pl ade } \\
\underline{dokument}\,$\oplus$\,t\,$\oplus$\,\underline{i}\,$\oplus$\,ilt & \textsc{ pl abl } \\
\underline{dokument}\,$\oplus$\,t\,$\oplus$\,\underline{i}\,$\oplus$\,issi & \textsc{ pl tra } \\
\underline{dokument}\,$\oplus$\,t\,$\oplus$\,\underline{i}\,$\oplus$\,issaa & \textsc{ pl ter } \\
\underline{dokument}\,$\oplus$\,t\,$\oplus$\,\underline{i}\,$\oplus$\,jka & \textsc{ pl com } \\
\end{tabular}
\end{sideways}
\captionof{table}{Tüüpsõnamall \arabic{mallinumber}\,\vadja{dokumentti} ekstraheeritud muutvormimallid.}
\label{tab:tüüpsõnamall-dokumentti}

\end{minipage}

 
\vspace{1em}
\noindent Tüüpsõnamall \vadja{dokumentti} hõlmab vormisõnastikus 19 lekseemi: \vadja{\underline{dokument}t\underline{i}, \underline{fabrikant}t\underline{i}, \underline{Frant}t\underline{si}, \underline{fundament}t\underline{i}, \underline{kajut}t\underline{i}, \underline{kamet}t\underline{i}, \underline{kanfet}t\underline{i}, \underline{kat}t\underline{i}, \underline{komet}t\underline{i}, \underline{komfet}t\underline{i}, \underline{komnõt}t\underline{i}, \underline{laut}t\underline{i}, \underline{magnet}t\underline{i}, \underline{minut}t\underline{i}, \underline{muzõkant}t\underline{i}, \underline{protestant}t\underline{i}, \underline{protsent}t\underline{i}, \underline{Root}t\underline{si}} ja \vadja{\underline{bankrut}t\underline{i}}.

Tüüp\-sõna\-mall kirjeldab tagapoolseid sõnu tüvemuutusega tt:t, mille lõpuvokaal on \textit{i}.  


\vspace{3.5em}
\noindent \begin{minipage}{\textwidth}
\stepcounter{mallinumber}
\noindent \textbf{Tüüpsõnamall \arabic{mallinumber}\,\vadja{komit̕et}}\\

\begin{sideways}
\begin{tabular}{l l}
muutvormimall & tunnused \\
\hline
\underline{komit̕et} & \textsc{ sg nom } \\
\underline{komit̕et}\,$\oplus$\,i & \textsc{ sg gen } \\
\underline{komit̕et}\,$\oplus$\,tiä & \textsc{ sg par } \\
\underline{komit̕et}\,$\oplus$\,ti & \textsc{ sg ill } \\
\underline{komit̕et}\,$\oplus$\,iz & \textsc{ sg ine } \\
\underline{komit̕et}\,$\oplus$\,iss & \textsc{ sg ela } \\
\underline{komit̕et}\,$\oplus$\,ille & \textsc{ sg all } \\
\underline{komit̕et}\,$\oplus$\,ill & \textsc{ sg ade } \\
\underline{komit̕et}\,$\oplus$\,ilt & \textsc{ sg abl } \\
\underline{komit̕et}\,$\oplus$\,issi & \textsc{ sg tra } \\
\underline{komit̕et}\,$\oplus$\,issaa & \textsc{ sg ter } \\
\underline{komit̕et}\,$\oplus$\,ika & \textsc{ sg com } \\
\underline{komit̕et}\,$\oplus$\,id & \textsc{ pl nom } \\
\underline{komit̕et}\,$\oplus$\,tije & \textsc{ pl gen } \\
\underline{komit̕et}\,$\oplus$\,tiit & \textsc{ pl par } \\
\underline{komit̕et}\,$\oplus$\,tiise & \textsc{ pl ill } \\
\underline{komit̕et}\,$\oplus$\,tiiz & \textsc{ pl ine } \\
\underline{komit̕et}\,$\oplus$\,tiiss & \textsc{ pl ela } \\
\underline{komit̕et}\,$\oplus$\,tiille & \textsc{ pl all } \\
\underline{komit̕et}\,$\oplus$\,tiill & \textsc{ pl ade } \\
\underline{komit̕et}\,$\oplus$\,tiilt & \textsc{ pl abl } \\
\underline{komit̕et}\,$\oplus$\,tiissi & \textsc{ pl tra } \\
\underline{komit̕et}\,$\oplus$\,tiissaa & \textsc{ pl ter } \\
\underline{komit̕et}\,$\oplus$\,tijka & \textsc{ pl com } \\
\end{tabular}
\end{sideways}
\captionof{table}{Tüüpsõnamall \arabic{mallinumber}\,\vadja{komit̕et} ekstraheeritud muutvormimallid.}
\label{tab:tüüpsõnamall-komit̕et}

\end{minipage}

 
\vspace{1em}
\noindent Tüüpsõnamall \vadja{komit̕et} ei hõlma teisi lekseeme vormi\-sõnastikus.

Tüüp\-sõna\-mall kirjeldab tagapoolseid sõnu tüvemuutusega tt:t.
%

\vspace{3.5em}
\noindent \begin{minipage}{\textwidth}
\stepcounter{mallinumber}
\noindent \textbf{Tüüpsõnamall \arabic{mallinumber}\,\vadja{mato}}\\

\begin{sideways}
\begin{tabular}{l l}
muutvormimall & tunnused \\
\hline
\underline{ma}\,$\oplus$\,t\,$\oplus$\,\underline{o} & \textsc{ sg nom } \\
\underline{ma}\,$\oplus$\,\underline{o} & \textsc{ sg gen } \\
\underline{ma}\,$\oplus$\,tt\,$\oplus$\,\underline{o}\,$\oplus$\,a & \textsc{ sg par } \\
\underline{ma}\,$\oplus$\,tt\,$\oplus$\,\underline{o}\,$\oplus$\,sõ & \textsc{ sg ill } \\
\underline{ma}\,$\oplus$\,\underline{o}\,$\oplus$\,z & \textsc{ sg ine } \\
\underline{ma}\,$\oplus$\,\underline{o}\,$\oplus$\,ss & \textsc{ sg ela } \\
\underline{ma}\,$\oplus$\,\underline{o}\,$\oplus$\,llõ & \textsc{ sg all } \\
\underline{ma}\,$\oplus$\,\underline{o}\,$\oplus$\,ll & \textsc{ sg ade } \\
\underline{ma}\,$\oplus$\,\underline{o}\,$\oplus$\,lt & \textsc{ sg abl } \\
\underline{ma}\,$\oplus$\,\underline{o}\,$\oplus$\,ssi & \textsc{ sg tra } \\
\underline{ma}\,$\oplus$\,\underline{o}\,$\oplus$\,ssaa & \textsc{ sg ter } \\
\underline{ma}\,$\oplus$\,\underline{o}\,$\oplus$\,ka & \textsc{ sg com } \\
\underline{ma}\,$\oplus$\,\underline{o}\,$\oplus$\,d & \textsc{ pl nom } \\
\underline{ma}\,$\oplus$\,t\,$\oplus$\,\underline{o}\,$\oplus$\,jõ & \textsc{ pl gen } \\
\underline{ma}\,$\oplus$\,t\,$\oplus$\,\underline{o}\,$\oplus$\,it & \textsc{ pl par } \\
\underline{ma}\,$\oplus$\,t\,$\oplus$\,\underline{o}\,$\oplus$\,isõ & \textsc{ pl ill } \\
\underline{ma}\,$\oplus$\,t\,$\oplus$\,\underline{o}\,$\oplus$\,iz & \textsc{ pl ine } \\
\underline{ma}\,$\oplus$\,t\,$\oplus$\,\underline{o}\,$\oplus$\,iss & \textsc{ pl ela } \\
\underline{ma}\,$\oplus$\,t\,$\oplus$\,\underline{o}\,$\oplus$\,illõ & \textsc{ pl all } \\
\underline{ma}\,$\oplus$\,t\,$\oplus$\,\underline{o}\,$\oplus$\,ill & \textsc{ pl ade } \\
\underline{ma}\,$\oplus$\,t\,$\oplus$\,\underline{o}\,$\oplus$\,ilt & \textsc{ pl abl } \\
\underline{ma}\,$\oplus$\,t\,$\oplus$\,\underline{o}\,$\oplus$\,issi & \textsc{ pl tra } \\
\underline{ma}\,$\oplus$\,t\,$\oplus$\,\underline{o}\,$\oplus$\,issaa & \textsc{ pl ter } \\
\underline{ma}\,$\oplus$\,t\,$\oplus$\,\underline{o}\,$\oplus$\,ika & \textsc{ pl com } \\
\end{tabular}
\end{sideways}
\captionof{table}{Tüüpsõnamall \arabic{mallinumber}\,\vadja{mato} ekstraheeritud muutvormimallid.}
\label{tab:tüüpsõnamall-mato}

\end{minipage}

 
\vspace{1em}
\noindent Tüüpsõnamall \vadja{mato} hõlmab vormisõnastikus 5 lekseemi: \vadja{\underline{ma}t\underline{o}, \underline{na}t\underline{o}, \underline{sa}t\underline{o}, \underline{ve}t\underline{o}} ja \vadja{\underline{ko}t\underline{o}}.

%Tüüpsõna\-mall kirjeldab tagapoolseid sõnu tüvemuutusega .


\vspace{3.5em}
\noindent \begin{minipage}{\textwidth}
\stepcounter{mallinumber}
\noindent \textbf{Tüüpsõnamall \arabic{mallinumber}\,\vadja{famil̕i}}\\

\begin{sideways}
\begin{tabular}{l l}
muutvormimall & tunnused \\
\hline
\underline{famil̕}\,$\oplus$\,i & \textsc{ sg nom } \\
\underline{famil̕}\,$\oplus$\,a & \textsc{ sg gen } \\
\underline{famil̕}\,$\oplus$\,a & \textsc{ sg par } \\
\underline{famil̕}\,$\oplus$\,asõ & \textsc{ sg ill } \\
\underline{famil̕}\,$\oplus$\,õz & \textsc{ sg ine } \\
\underline{famil̕}\,$\oplus$\,õss & \textsc{ sg ela } \\
\underline{famil̕}\,$\oplus$\,õllõ & \textsc{ sg all } \\
\underline{famil̕}\,$\oplus$\,õll & \textsc{ sg ade } \\
\underline{famil̕}\,$\oplus$\,õlt & \textsc{ sg abl } \\
\underline{famil̕}\,$\oplus$\,õssi & \textsc{ sg tra } \\
\underline{famil̕}\,$\oplus$\,õssaa & \textsc{ sg ter } \\
\underline{famil̕}\,$\oplus$\,õka & \textsc{ sg com } \\
\underline{famil̕}\,$\oplus$\,õd & \textsc{ pl nom } \\
\underline{famil̕}\,$\oplus$\,ojõ & \textsc{ pl gen } \\
\underline{famil̕}\,$\oplus$\,oit & \textsc{ pl par } \\
\underline{famil̕}\,$\oplus$\,oisõ & \textsc{ pl ill } \\
\underline{famil̕}\,$\oplus$\,oiz & \textsc{ pl ine } \\
\underline{famil̕}\,$\oplus$\,oiss & \textsc{ pl ela } \\
\underline{famil̕}\,$\oplus$\,oillõ & \textsc{ pl all } \\
\underline{famil̕}\,$\oplus$\,oill & \textsc{ pl ade } \\
\underline{famil̕}\,$\oplus$\,oilt & \textsc{ pl abl } \\
\underline{famil̕}\,$\oplus$\,oissi & \textsc{ pl tra } \\
\underline{famil̕}\,$\oplus$\,oissaa & \textsc{ pl ter } \\
\underline{famil̕}\,$\oplus$\,oika & \textsc{ pl com } \\
\end{tabular}
\end{sideways}
\captionof{table}{Tüüpsõnamall \arabic{mallinumber}\,\vadja{famil̕i} ekstraheeritud muutvormimallid.}
\label{tab:tüüpsõnamall-famil̕i}

\end{minipage}

 
\vspace{1em}
\noindent Tüüpsõnamall \vadja{famil̕i} hõlmab vormisõnastikus 2 lekseemi: \vadja{\underline{famil̕}i} ja \vadja{\underline{bašn̕}i}.

Tüüp\-sõna\-mall kirjeldab tagapoolseid tüvemuutuseta sõnu, mis eeldatavasti palatalisatsiooni tõttu käituvad eriliselt.
\spacing{1.5}


\subsection{\RN{3} käändkond}

Kolmandasse käändkonda kuuluvad Ariste (\citeyear[42]{ariste_grammar_1968}) järgi kahesilbilised sõnad, mille tüvevokaal on \vadja{-a} ning rohkem silpidega sõnad, millel esineb esimeses silbis \vadja{-a-}, \vadja{-õ-} või \vadja{-i-}.

Vormisõnastikus esineb \RN{3} käändkonnas siiski palju sõnu, mille esimises silbis on \textit{-u-}, mis Ariste järgi peaks kuuluma hoopis \RN{5} käändkonda (\cite[44]{ariste_grammar_1968}). Need sõnad on üldjuhul jäetud muutmata kujule ja on järgitud Tsvetkovi tüvekujusid. Näiteks \textit{muuzikkõ}, \textit{vunukkõ} (tüüp\-sõna \textit{borovikkõ}); \textit{suma} (tüüp\-sõna \textit{baldõhina}); \textit{bruudõ}, \textit{pudrõ}, \textit{surmõ} (tüüp\-sõna \textit{az̕z̕õ}).

Tsvetkovi sõnaraamatus esineb rohkesti sõnu, mille mitmuse\-tüvi on \textit{-õi-}. Paljud nendest on vormi\-sõnastikus ühtlustatud \textit{-oi}-kujule.

Sellesse käändkonda kuuluvad paljud vene laen\-sõnad, mille lõpu\-vokaalide ühtlustamine on olnud keeruline. Üldiselt on nende kuju jäetud muutmata ja need esinevad samal kujul, nagu Tsvetkovi sõnaraamatus.

%% Avatuid küsimusi-tähelepanekuid:
%% \begin{itemize}
%% \item siin on sõnu millel on -u- 1. silbis, need peaksid käima hoopis \RN{5} alla
%% \item Tsvetkovil palju -õi-mitmusetüvega, need on ühtlustatud -oi-
%% \item paljud laensõnad kuuluvad siia alla, nende lõpuvokaalidega on häda
%% \end{itemize}

\subsubsection*{Ekstraktmorfoloogia tüüpsõnad}
\spacing{1.1}
\vspace{-3.5em}


\vspace{3.5em}
\noindent \begin{minipage}{\textwidth}
\stepcounter{mallinumber}
\noindent \textbf{Tüüpsõnamall \arabic{mallinumber}\,\vadja{tuhattõ}}\\

\begin{sideways}
\begin{tabular}{l l}
muutvormimall & tunnused \\
\hline
\underline{tuhat}\,$\oplus$\,tõ & \textsc{ sg nom } \\
\underline{tuhat}\,$\oplus$\,ta & \textsc{ sg gen } \\
\underline{tuhat}\,$\oplus$\,ta & \textsc{ sg par } \\
\underline{tuhat}\,$\oplus$\,tasõ & \textsc{ sg ill } \\
\underline{tuhat}\,$\oplus$\,tõz & \textsc{ sg ine } \\
\underline{tuhat}\,$\oplus$\,õss & \textsc{ sg ela } \\
\underline{tuhat}\,$\oplus$\,õllõ & \textsc{ sg all } \\
\underline{tuhat}\,$\oplus$\,õll & \textsc{ sg ade } \\
\underline{tuhat}\,$\oplus$\,õlt & \textsc{ sg abl } \\
\underline{tuhat}\,$\oplus$\,õssi & \textsc{ sg tra } \\
\underline{tuhat}\,$\oplus$\,tõssaa & \textsc{ sg ter } \\
\underline{tuhat}\,$\oplus$\,õka & \textsc{ sg com } \\
\underline{tuhat}\,$\oplus$\,õd & \textsc{ pl nom } \\
\underline{tuhat}\,$\oplus$\,tojõ & \textsc{ pl gen } \\
\underline{tuhat}\,$\oplus$\,toit & \textsc{ pl par } \\
\underline{tuhat}\,$\oplus$\,toisõ & \textsc{ pl ill } \\
\underline{tuhat}\,$\oplus$\,toiz & \textsc{ pl ine } \\
\underline{tuhat}\,$\oplus$\,toiss & \textsc{ pl ela } \\
\underline{tuhat}\,$\oplus$\,toillõ & \textsc{ pl all } \\
\underline{tuhat}\,$\oplus$\,toill & \textsc{ pl ade } \\
\underline{tuhat}\,$\oplus$\,toilt & \textsc{ pl abl } \\
\underline{tuhat}\,$\oplus$\,toissi & \textsc{ pl tra } \\
\underline{tuhat}\,$\oplus$\,toissaa & \textsc{ pl ter } \\
\underline{tuhat}\,$\oplus$\,toika & \textsc{ pl com } \\
\end{tabular}
\end{sideways}
\captionof{table}{Tüüpsõnamall \arabic{mallinumber}\,\vadja{tuhattõ} ekstraheeritud muutvormimallid.}
\label{tab:tüüpsõnamall-tuhattõ}

\end{minipage}

 
\vspace{1em}
\noindent Tüüpsõnamall \vadja{tuhattõ} ei hõlma teisi lekseeme vormi\-sõnastikus.

Tüüp\-sõna\-mall kirjeldab tagapoolseid tüvemuutuseta sõnu


\vspace{3.5em}
\noindent \begin{minipage}{\textwidth}
\stepcounter{mallinumber}
\noindent \textbf{Tüüpsõnamall \arabic{mallinumber}\,\vadja{lafkõ}}\\

\begin{sideways}
\begin{tabular}{l l}
muutvormimall & tunnused \\
\hline
\underline{la}\,$\oplus$\,fkõ & \textsc{ sg nom } \\
\underline{la}\,$\oplus$\,vga & \textsc{ sg gen } \\
\underline{la}\,$\oplus$\,fka & \textsc{ sg par } \\
\underline{la}\,$\oplus$\,fkasõ & \textsc{ sg ill } \\
\underline{la}\,$\oplus$\,vgõz & \textsc{ sg ine } \\
\underline{la}\,$\oplus$\,vgõss & \textsc{ sg ela } \\
\underline{la}\,$\oplus$\,vgõllõ & \textsc{ sg all } \\
\underline{la}\,$\oplus$\,vgõll & \textsc{ sg ade } \\
\underline{la}\,$\oplus$\,vgõlt & \textsc{ sg abl } \\
\underline{la}\,$\oplus$\,vgõssi & \textsc{ sg tra } \\
\underline{la}\,$\oplus$\,vgõssaa & \textsc{ sg ter } \\
\underline{la}\,$\oplus$\,vgõka & \textsc{ sg com } \\
\underline{la}\,$\oplus$\,vgõd & \textsc{ pl nom } \\
\underline{la}\,$\oplus$\,fkojõ & \textsc{ pl gen } \\
\underline{la}\,$\oplus$\,fkoit & \textsc{ pl par } \\
\underline{la}\,$\oplus$\,fkoisõ & \textsc{ pl ill } \\
\underline{la}\,$\oplus$\,fkoiz & \textsc{ pl ine } \\
\underline{la}\,$\oplus$\,fkoiss & \textsc{ pl ela } \\
\underline{la}\,$\oplus$\,fkoillõ & \textsc{ pl all } \\
\underline{la}\,$\oplus$\,fkoill & \textsc{ pl ade } \\
\underline{la}\,$\oplus$\,fkoilt & \textsc{ pl abl } \\
\underline{la}\,$\oplus$\,fkoissi & \textsc{ pl tra } \\
\underline{la}\,$\oplus$\,fkoissaa & \textsc{ pl ter } \\
\underline{la}\,$\oplus$\,fkoika & \textsc{ pl com } \\
\end{tabular}
\end{sideways}
\captionof{table}{Tüüpsõnamall \arabic{mallinumber}\,\vadja{lafkõ} ekstraheeritud muutvormimallid.}
\label{tab:tüüpsõnamall-lafkõ}

\end{minipage}

 
\vspace{1em}
\noindent Tüüpsõnamall \vadja{lafkõ} ei hõlma teisi lekseeme vormi\-sõnastikus.

Tüüp\-sõna\-mall kirjeldab tagapoolseid sõnu tüvemuutusega fk:vg. 


\vspace{3.5em}
\noindent \begin{minipage}{\textwidth}
\stepcounter{mallinumber}
\noindent \textbf{Tüüpsõnamall \arabic{mallinumber}\,\vadja{mahsõ}}\\

\begin{sideways}
\begin{tabular}{l l}
muutvormimall & tunnused \\
\hline
\underline{mah}\,$\oplus$\,sõ & \textsc{ sg nom } \\
\underline{mah}\,$\oplus$\,za & \textsc{ sg gen } \\
\underline{mah}\,$\oplus$\,sa & \textsc{ sg par } \\
\underline{mah}\,$\oplus$\,sasõ & \textsc{ sg ill } \\
\underline{mah}\,$\oplus$\,sõz & \textsc{ sg ine } \\
\underline{mah}\,$\oplus$\,zõss & \textsc{ sg ela } \\
\underline{mah}\,$\oplus$\,zõllõ & \textsc{ sg all } \\
\underline{mah}\,$\oplus$\,zõll & \textsc{ sg ade } \\
\underline{mah}\,$\oplus$\,zõlt & \textsc{ sg abl } \\
\underline{mah}\,$\oplus$\,zõssi & \textsc{ sg tra } \\
\underline{mah}\,$\oplus$\,sõssaa & \textsc{ sg ter } \\
\underline{mah}\,$\oplus$\,zõka & \textsc{ sg com } \\
\underline{mah}\,$\oplus$\,zõd & \textsc{ pl nom } \\
\underline{mah}\,$\oplus$\,sojõ & \textsc{ pl gen } \\
\underline{mah}\,$\oplus$\,soit & \textsc{ pl par } \\
\underline{mah}\,$\oplus$\,soisõ & \textsc{ pl ill } \\
\underline{mah}\,$\oplus$\,soiz & \textsc{ pl ine } \\
\underline{mah}\,$\oplus$\,soiss & \textsc{ pl ela } \\
\underline{mah}\,$\oplus$\,soillõ & \textsc{ pl all } \\
\underline{mah}\,$\oplus$\,soill & \textsc{ pl ade } \\
\underline{mah}\,$\oplus$\,soilt & \textsc{ pl abl } \\
\underline{mah}\,$\oplus$\,soissi & \textsc{ pl tra } \\
\underline{mah}\,$\oplus$\,soissaa & \textsc{ pl ter } \\
\underline{mah}\,$\oplus$\,soika & \textsc{ pl com } \\
\end{tabular}
\end{sideways}
\captionof{table}{Tüüpsõnamall \arabic{mallinumber}\,\vadja{mahsõ} ekstraheeritud muutvormimallid.}
\label{tab:tüüpsõnamall-mahsõ}

\end{minipage}

 
\vspace{1em}
\noindent Tüüpsõnamall \vadja{mahsõ} ei hõlma teisi lekseeme vormi\-sõnastikus.

Tüüp\-sõna\-mall kirjeldab tagapoolseid sõnu tüvemuutusega hs:hz.


\vspace{3.5em}
\noindent \begin{minipage}{\textwidth}
\stepcounter{mallinumber}
\noindent \textbf{Tüüpsõnamall \arabic{mallinumber}\,\vadja{vihtõ}}\\

\begin{sideways}
\begin{tabular}{l l}
muutvormimall & tunnused \\
\hline
\underline{vih}\,$\oplus$\,tõ & \textsc{ sg nom } \\
\underline{vih}\,$\oplus$\,a & \textsc{ sg gen } \\
\underline{vih}\,$\oplus$\,ta & \textsc{ sg par } \\
\underline{vih}\,$\oplus$\,tasõ & \textsc{ sg ill } \\
\underline{vih}\,$\oplus$\,õz & \textsc{ sg ine } \\
\underline{vih}\,$\oplus$\,õss & \textsc{ sg ela } \\
\underline{vih}\,$\oplus$\,õllõ & \textsc{ sg all } \\
\underline{vih}\,$\oplus$\,õll & \textsc{ sg ade } \\
\underline{vih}\,$\oplus$\,õlt & \textsc{ sg abl } \\
\underline{vih}\,$\oplus$\,õssi & \textsc{ sg tra } \\
\underline{vih}\,$\oplus$\,õssaa & \textsc{ sg ter } \\
\underline{vih}\,$\oplus$\,õka & \textsc{ sg com } \\
\underline{vih}\,$\oplus$\,õd & \textsc{ pl nom } \\
\underline{vih}\,$\oplus$\,tojõ & \textsc{ pl gen } \\
\underline{vih}\,$\oplus$\,toit & \textsc{ pl par } \\
\underline{vih}\,$\oplus$\,toisõ & \textsc{ pl ill } \\
\underline{vih}\,$\oplus$\,toiz & \textsc{ pl ine } \\
\underline{vih}\,$\oplus$\,toiss & \textsc{ pl ela } \\
\underline{vih}\,$\oplus$\,toillõ & \textsc{ pl all } \\
\underline{vih}\,$\oplus$\,toill & \textsc{ pl ade } \\
\underline{vih}\,$\oplus$\,toilt & \textsc{ pl abl } \\
\underline{vih}\,$\oplus$\,toissi & \textsc{ pl tra } \\
\underline{vih}\,$\oplus$\,toissaa & \textsc{ pl ter } \\
\underline{vih}\,$\oplus$\,toika & \textsc{ pl com } \\
\end{tabular}
\end{sideways}
\captionof{table}{Tüüpsõnamall \arabic{mallinumber}\,\vadja{vihtõ} ekstraheeritud muutvormimallid.}
\label{tab:tüüpsõnamall-vihtõ}

\end{minipage}

 
\vspace{1em}
\noindent Tüüpsõnamall \vadja{vihtõ} ei hõlma teisi lekseeme vormi\-sõnastikus.

Tüüp\-sõna\-mall kirjeldab tagapoolseid sõnu tüvemuutusega ht:h.


\vspace{3.5em}
\noindent \begin{minipage}{\textwidth}
\stepcounter{mallinumber}
\noindent \textbf{Tüüpsõnamall \arabic{mallinumber}\,\vadja{aitõ}}\\

\begin{sideways}
\begin{tabular}{l l}
muutvormimall & tunnused \\
\hline
\underline{a}\,$\oplus$\,itõ & \textsc{ sg nom } \\
\underline{a}\,$\oplus$\,jja & \textsc{ sg gen } \\
\underline{a}\,$\oplus$\,ita & \textsc{ sg par } \\
\underline{a}\,$\oplus$\,itasõ & \textsc{ sg ill } \\
\underline{a}\,$\oplus$\,ijõz & \textsc{ sg ine } \\
\underline{a}\,$\oplus$\,ijõss & \textsc{ sg ela } \\
\underline{a}\,$\oplus$\,ijõllõ & \textsc{ sg all } \\
\underline{a}\,$\oplus$\,ijõll & \textsc{ sg ade } \\
\underline{a}\,$\oplus$\,ijõlt & \textsc{ sg abl } \\
\underline{a}\,$\oplus$\,ijõssi & \textsc{ sg tra } \\
\underline{a}\,$\oplus$\,ijõssaa & \textsc{ sg ter } \\
\underline{a}\,$\oplus$\,ijõka & \textsc{ sg com } \\
\underline{a}\,$\oplus$\,ijõd & \textsc{ pl nom } \\
\underline{a}\,$\oplus$\,itojõ & \textsc{ pl gen } \\
\underline{a}\,$\oplus$\,itoit & \textsc{ pl par } \\
\underline{a}\,$\oplus$\,itoisõ & \textsc{ pl ill } \\
\underline{a}\,$\oplus$\,itoiz & \textsc{ pl ine } \\
\underline{a}\,$\oplus$\,itoiss & \textsc{ pl ela } \\
\underline{a}\,$\oplus$\,itoillõ & \textsc{ pl all } \\
\underline{a}\,$\oplus$\,itoill & \textsc{ pl ade } \\
\underline{a}\,$\oplus$\,itoilt & \textsc{ pl abl } \\
\underline{a}\,$\oplus$\,itoissi & \textsc{ pl tra } \\
\underline{a}\,$\oplus$\,itoissaa & \textsc{ pl ter } \\
\underline{a}\,$\oplus$\,itoika & \textsc{ pl com } \\
\end{tabular}
\end{sideways}
\captionof{table}{Tüüpsõnamall \arabic{mallinumber}\,\vadja{aitõ} ekstraheeritud muutvormimallid.}
\label{tab:tüüpsõnamall-aitõ}

\end{minipage}

 
\vspace{1em}
\noindent Tüüpsõnamall \vadja{aitõ} ei hõlma teisi lekseeme vormi\-sõnastikus.

Tüüp\-sõna\-mall kirjeldab tagapoolseid sõnu tüvemuutusega it:jj.


\vspace{3.5em}
\noindent \begin{minipage}{\textwidth}
\stepcounter{mallinumber}
\noindent \textbf{Tüüpsõnamall \arabic{mallinumber}\,\vadja{aikõ}}\\

\begin{sideways}
\begin{tabular}{l l}
muutvormimall & tunnused \\
\hline
\underline{ai}\,$\oplus$\,kõ & \textsc{ sg nom } \\
\underline{ai}\,$\oplus$\,ga & \textsc{ sg gen } \\
\underline{ai}\,$\oplus$\,ka & \textsc{ sg par } \\
\underline{ai}\,$\oplus$\,kasõ & \textsc{ sg ill } \\
\underline{ai}\,$\oplus$\,gõz & \textsc{ sg ine } \\
\underline{ai}\,$\oplus$\,gõss & \textsc{ sg ela } \\
\underline{ai}\,$\oplus$\,gõllõ & \textsc{ sg all } \\
\underline{ai}\,$\oplus$\,gõll & \textsc{ sg ade } \\
\underline{ai}\,$\oplus$\,gõlt & \textsc{ sg abl } \\
\underline{ai}\,$\oplus$\,gõssi & \textsc{ sg tra } \\
\underline{ai}\,$\oplus$\,gõssaa & \textsc{ sg ter } \\
\underline{ai}\,$\oplus$\,gõka & \textsc{ sg com } \\
\underline{ai}\,$\oplus$\,gõd & \textsc{ pl nom } \\
\underline{ai}\,$\oplus$\,kojõ & \textsc{ pl gen } \\
\underline{ai}\,$\oplus$\,koit & \textsc{ pl par } \\
\underline{ai}\,$\oplus$\,koisõ & \textsc{ pl ill } \\
\underline{ai}\,$\oplus$\,koiz & \textsc{ pl ine } \\
\underline{ai}\,$\oplus$\,koiss & \textsc{ pl ela } \\
\underline{ai}\,$\oplus$\,koillõ & \textsc{ pl all } \\
\underline{ai}\,$\oplus$\,koill & \textsc{ pl ade } \\
\underline{ai}\,$\oplus$\,koilt & \textsc{ pl abl } \\
\underline{ai}\,$\oplus$\,koissi & \textsc{ pl tra } \\
\underline{ai}\,$\oplus$\,koissaa & \textsc{ pl ter } \\
\underline{ai}\,$\oplus$\,koika & \textsc{ pl com } \\
\end{tabular}
\end{sideways}
\captionof{table}{Tüüpsõnamall \arabic{mallinumber}\,\vadja{aikõ} ekstraheeritud muutvormimallid.}
\label{tab:tüüpsõnamall-aikõ}

\end{minipage}

 
\vspace{1em}
\noindent Tüüpsõnamall \vadja{aikõ} hõlmab vormisõnastikus 10 lekseemi: \vadja{\underline{ai}kõ, \underline{jal}kõ, \underline{lii}kõ, \underline{lõn}kõ, \underline{nah}kõ, \underline{rah}kõ, \underline{vil}kõ, \underline{vin}kõ, \underline{võl}kõ} ja \vadja{\underline{aastai}kõ}.

Tüüp\-sõna\-mall kirjeldab tagapoolseid sõnu tüvemuutusega k:g.


\vspace{3.5em}
\noindent \begin{minipage}{\textwidth}
\stepcounter{mallinumber}
\noindent \textbf{Tüüpsõnamall \arabic{mallinumber}\,\vadja{sika}}\\

\begin{sideways}
\begin{tabular}{l l}
muutvormimall & tunnused \\
\hline
\underline{si}\,$\oplus$\,ka & \textsc{ sg nom } \\
\underline{si}\,$\oplus$\,ga & \textsc{ sg gen } \\
\underline{si}\,$\oplus$\,kka & \textsc{ sg par } \\
\underline{si}\,$\oplus$\,kkasõ & \textsc{ sg ill } \\
\underline{si}\,$\oplus$\,gaz & \textsc{ sg ine } \\
\underline{si}\,$\oplus$\,gass & \textsc{ sg ela } \\
\underline{si}\,$\oplus$\,gallõ & \textsc{ sg all } \\
\underline{si}\,$\oplus$\,gall & \textsc{ sg ade } \\
\underline{si}\,$\oplus$\,galt & \textsc{ sg abl } \\
\underline{si}\,$\oplus$\,gassi & \textsc{ sg tra } \\
\underline{si}\,$\oplus$\,gassaa & \textsc{ sg ter } \\
\underline{si}\,$\oplus$\,gaka & \textsc{ sg com } \\
\underline{si}\,$\oplus$\,gad & \textsc{ pl nom } \\
\underline{si}\,$\oplus$\,kojõ & \textsc{ pl gen } \\
\underline{si}\,$\oplus$\,koit & \textsc{ pl par } \\
\underline{si}\,$\oplus$\,koisõ & \textsc{ pl ill } \\
\underline{si}\,$\oplus$\,koiz & \textsc{ pl ine } \\
\underline{si}\,$\oplus$\,koiss & \textsc{ pl ela } \\
\underline{si}\,$\oplus$\,koillõ & \textsc{ pl all } \\
\underline{si}\,$\oplus$\,koill & \textsc{ pl ade } \\
\underline{si}\,$\oplus$\,koilt & \textsc{ pl abl } \\
\underline{si}\,$\oplus$\,koissi & \textsc{ pl tra } \\
\underline{si}\,$\oplus$\,koissaa & \textsc{ pl ter } \\
\underline{si}\,$\oplus$\,koika & \textsc{ pl com } \\
\end{tabular}
\end{sideways}
\captionof{table}{Tüüpsõnamall \arabic{mallinumber}\,\vadja{sika} ekstraheeritud muutvormimallid.}
\label{tab:tüüpsõnamall-sika}

\end{minipage}

 
\vspace{1em}
\noindent Tüüpsõnamall \vadja{sika} ei hõlma teisi lekseeme vormi\-sõnastikus.

Tüüp\-sõna\-mall kirjeldab tagapoolseid sõnu tüvemuutusega k:g, mis geminaadi tõttu esinevad \msd{sg ill} vormis tugevas astmes.


\vspace{3.5em}
\noindent \begin{minipage}{\textwidth}
\stepcounter{mallinumber}
\noindent \textbf{Tüüpsõnamall \arabic{mallinumber}\,\vadja{borovikkõ}}\\

\begin{sideways}
\begin{tabular}{l l}
muutvormimall & tunnused \\
\hline
\underline{borovik}\,$\oplus$\,kõ & \textsc{ sg nom } \\
\underline{borovik}\,$\oplus$\,a & \textsc{ sg gen } \\
\underline{borovik}\,$\oplus$\,ka & \textsc{ sg par } \\
\underline{borovik}\,$\oplus$\,kasõ & \textsc{ sg ill } \\
\underline{borovik}\,$\oplus$\,kõz & \textsc{ sg ine } \\
\underline{borovik}\,$\oplus$\,õss & \textsc{ sg ela } \\
\underline{borovik}\,$\oplus$\,õllõ & \textsc{ sg all } \\
\underline{borovik}\,$\oplus$\,õll & \textsc{ sg ade } \\
\underline{borovik}\,$\oplus$\,õlt & \textsc{ sg abl } \\
\underline{borovik}\,$\oplus$\,õssi & \textsc{ sg tra } \\
\underline{borovik}\,$\oplus$\,kõssaa & \textsc{ sg ter } \\
\underline{borovik}\,$\oplus$\,õka & \textsc{ sg com } \\
\underline{borovik}\,$\oplus$\,õd & \textsc{ pl nom } \\
\underline{borovik}\,$\oplus$\,kojõ & \textsc{ pl gen } \\
\underline{borovik}\,$\oplus$\,koit & \textsc{ pl par } \\
\underline{borovik}\,$\oplus$\,koisõ & \textsc{ pl ill } \\
\underline{borovik}\,$\oplus$\,koiz & \textsc{ pl ine } \\
\underline{borovik}\,$\oplus$\,koiss & \textsc{ pl ela } \\
\underline{borovik}\,$\oplus$\,koillõ & \textsc{ pl all } \\
\underline{borovik}\,$\oplus$\,koill & \textsc{ pl ade } \\
\underline{borovik}\,$\oplus$\,koilt & \textsc{ pl abl } \\
\underline{borovik}\,$\oplus$\,koissi & \textsc{ pl tra } \\
\underline{borovik}\,$\oplus$\,koissaa & \textsc{ pl ter } \\
\underline{borovik}\,$\oplus$\,koika & \textsc{ pl com } \\
\end{tabular}
\end{sideways}
\captionof{table}{Tüüpsõnamall \arabic{mallinumber}\,\vadja{borovikkõ} ekstraheeritud muutvormimallid.}
\label{tab:tüüpsõnamall-borovikkõ}

\end{minipage}

 
\vspace{1em}
\noindent Tüüpsõnamall \vadja{borovikkõ} hõlmab vormisõnastikus 41 lekseemi: \vadja{\underline{borovik}kõ, \underline{domovik}kõ, \underline{durak}kõ, \underline{fartuk}kõ, \underline{fiizik}kõ, \underline{fookusnik}kõ, \underline{frištik}kõ, \underline{gribanik}kõ, \underline{harak}kõ, \underline{itik}kõ, \underline{joožik}kõ, \underline{kaamenšik}kõ, \underline{kabak}kõ, \underline{kamal̕ik}kõ, \underline{katol̕ik}kõ, \underline{kelk}kõ, \underline{koomik}kõ, \underline{kopek}kõ, \underline{latik}kõ, \underline{luzik}kõ, \underline{luuk}kõ, \underline{mark}kõ, \underline{muuzik}kõ, \underline{mõiznik}kõ, \underline{noorik}kõ, \underline{nuužnik}kõ, \underline{obak}kõ, \underline{paik}kõ, \underline{palk}kõ, \underline{pinžak}kõ, \underline{podark}kõ, \underline{poštimark}kõ, \underline{rank}kõ, \underline{rohosirk}kõ, \underline{tark}kõ, \underline{tik}kõ, \underline{tubak}kõ, \underline{urok}kõ, \underline{vak}kõ, \underline{vunuk}kõ} ja \vadja{\underline{bašmuk}kõ}.

Tüüp\-sõna\-mall kirjeldab tagapoolseid sõnu tüvemuutusega kk:k.


\vspace{3.5em}
\noindent \begin{minipage}{\textwidth}
\stepcounter{mallinumber}
\noindent \textbf{Tüüpsõnamall \arabic{mallinumber}\,\vadja{siltõ}}\\

\begin{sideways}
\begin{tabular}{l l}
muutvormimall & tunnused \\
\hline
\underline{sil}\,$\oplus$\,tõ & \textsc{ sg nom } \\
\underline{sil}\,$\oplus$\,la & \textsc{ sg gen } \\
\underline{sil}\,$\oplus$\,ta & \textsc{ sg par } \\
\underline{sil}\,$\oplus$\,tasõ & \textsc{ sg ill } \\
\underline{sil}\,$\oplus$\,lõz & \textsc{ sg ine } \\
\underline{sil}\,$\oplus$\,lõss & \textsc{ sg ela } \\
\underline{sil}\,$\oplus$\,lõllõ & \textsc{ sg all } \\
\underline{sil}\,$\oplus$\,lõll & \textsc{ sg ade } \\
\underline{sil}\,$\oplus$\,lõlt & \textsc{ sg abl } \\
\underline{sil}\,$\oplus$\,lõssi & \textsc{ sg tra } \\
\underline{sil}\,$\oplus$\,lõssaa & \textsc{ sg ter } \\
\underline{sil}\,$\oplus$\,lõka & \textsc{ sg com } \\
\underline{sil}\,$\oplus$\,lõd & \textsc{ pl nom } \\
\underline{sil}\,$\oplus$\,tojõ & \textsc{ pl gen } \\
\underline{sil}\,$\oplus$\,toit & \textsc{ pl par } \\
\underline{sil}\,$\oplus$\,toisõ & \textsc{ pl ill } \\
\underline{sil}\,$\oplus$\,toiz & \textsc{ pl ine } \\
\underline{sil}\,$\oplus$\,toiss & \textsc{ pl ela } \\
\underline{sil}\,$\oplus$\,toillõ & \textsc{ pl all } \\
\underline{sil}\,$\oplus$\,toill & \textsc{ pl ade } \\
\underline{sil}\,$\oplus$\,toilt & \textsc{ pl abl } \\
\underline{sil}\,$\oplus$\,toissi & \textsc{ pl tra } \\
\underline{sil}\,$\oplus$\,toissaa & \textsc{ pl ter } \\
\underline{sil}\,$\oplus$\,toika & \textsc{ pl com } \\
\end{tabular}
\end{sideways}
\captionof{table}{Tüüpsõnamall \arabic{mallinumber}\,\vadja{siltõ} ekstraheeritud muutvormimallid.}
\label{tab:tüüpsõnamall-siltõ}

\end{minipage}

 
\vspace{1em}
\noindent Tüüpsõnamall \vadja{siltõ} ei hõlma teisi lekseeme vormi\-sõnastikus.

Tüüp\-sõna\-mall kirjeldab tagapoolseid sõnu tüvemuutusega lt:ll.


\vspace{3.5em}
\noindent \begin{minipage}{\textwidth}
\stepcounter{mallinumber}
\noindent \textbf{Tüüpsõnamall \arabic{mallinumber}\,\vadja{paha}}\\

\begin{sideways}
\begin{tabular}{l l}
muutvormimall & tunnused \\
\hline
\underline{pah}\,$\oplus$\,a & \textsc{ sg nom } \\
\underline{pah}\,$\oplus$\,a & \textsc{ sg gen } \\
\underline{pah}\,$\oplus$\,ha & \textsc{ sg par } \\
\underline{pah}\,$\oplus$\,hasõ & \textsc{ sg ill } \\
\underline{pah}\,$\oplus$\,õz & \textsc{ sg ine } \\
\underline{pah}\,$\oplus$\,õss & \textsc{ sg ela } \\
\underline{pah}\,$\oplus$\,õllõ & \textsc{ sg all } \\
\underline{pah}\,$\oplus$\,õll & \textsc{ sg ade } \\
\underline{pah}\,$\oplus$\,õlt & \textsc{ sg abl } \\
\underline{pah}\,$\oplus$\,õssi & \textsc{ sg tra } \\
\underline{pah}\,$\oplus$\,õssaa & \textsc{ sg ter } \\
\underline{pah}\,$\oplus$\,õka & \textsc{ sg com } \\
\underline{pah}\,$\oplus$\,õd & \textsc{ pl nom } \\
\underline{pah}\,$\oplus$\,hojõ & \textsc{ pl gen } \\
\underline{pah}\,$\oplus$\,hoit & \textsc{ pl par } \\
\underline{pah}\,$\oplus$\,hoisõ & \textsc{ pl ill } \\
\underline{pah}\,$\oplus$\,hoiz & \textsc{ pl ine } \\
\underline{pah}\,$\oplus$\,hoiss & \textsc{ pl ela } \\
\underline{pah}\,$\oplus$\,hoillõ & \textsc{ pl all } \\
\underline{pah}\,$\oplus$\,hoill & \textsc{ pl ade } \\
\underline{pah}\,$\oplus$\,hoilt & \textsc{ pl abl } \\
\underline{pah}\,$\oplus$\,hoissi & \textsc{ pl tra } \\
\underline{pah}\,$\oplus$\,hoissaa & \textsc{ pl ter } \\
\underline{pah}\,$\oplus$\,hoika & \textsc{ pl com } \\
\end{tabular}
\end{sideways}
\captionof{table}{Tüüpsõnamall \arabic{mallinumber}\,\vadja{paha} ekstraheeritud muutvormimallid.}
\label{tab:tüüpsõnamall-paha}

\end{minipage}

 
\vspace{1em}
\noindent Tüüpsõnamall \vadja{paha} ei hõlma teisi lekseeme vormi\-sõnastikus.

Tüüp\-sõna\-mall kirjeldab tagapoolseid tüvemuutuseta sõnu, mille lõpuvokaal on \textit{a} ning mis on \msd{sg par} ja \msd{sg ill} vormis tugevas astmes.


\vspace{3.5em}
\noindent \begin{minipage}{\textwidth}
\stepcounter{mallinumber}
\noindent \textbf{Tüüpsõnamall \arabic{mallinumber}\,\vadja{baldõhina}}\\

\begin{sideways}
\begin{tabular}{l l}
muutvormimall & tunnused \\
\hline
\underline{baldõhin}\,$\oplus$\,a & \textsc{ sg nom } \\
\underline{baldõhin}\,$\oplus$\,a & \textsc{ sg gen } \\
\underline{baldõhin}\,$\oplus$\,a & \textsc{ sg par } \\
\underline{baldõhin}\,$\oplus$\,asõ & \textsc{ sg ill } \\
\underline{baldõhin}\,$\oplus$\,az & \textsc{ sg ine } \\
\underline{baldõhin}\,$\oplus$\,ass & \textsc{ sg ela } \\
\underline{baldõhin}\,$\oplus$\,allõ & \textsc{ sg all } \\
\underline{baldõhin}\,$\oplus$\,all & \textsc{ sg ade } \\
\underline{baldõhin}\,$\oplus$\,alt & \textsc{ sg abl } \\
\underline{baldõhin}\,$\oplus$\,assi & \textsc{ sg tra } \\
\underline{baldõhin}\,$\oplus$\,assaa & \textsc{ sg ter } \\
\underline{baldõhin}\,$\oplus$\,aka & \textsc{ sg com } \\
\underline{baldõhin}\,$\oplus$\,ad & \textsc{ pl nom } \\
\underline{baldõhin}\,$\oplus$\,ojõ & \textsc{ pl gen } \\
\underline{baldõhin}\,$\oplus$\,oit & \textsc{ pl par } \\
\underline{baldõhin}\,$\oplus$\,oisõ & \textsc{ pl ill } \\
\underline{baldõhin}\,$\oplus$\,oiz & \textsc{ pl ine } \\
\underline{baldõhin}\,$\oplus$\,oiss & \textsc{ pl ela } \\
\underline{baldõhin}\,$\oplus$\,oillõ & \textsc{ pl all } \\
\underline{baldõhin}\,$\oplus$\,oill & \textsc{ pl ade } \\
\underline{baldõhin}\,$\oplus$\,oilt & \textsc{ pl abl } \\
\underline{baldõhin}\,$\oplus$\,oissi & \textsc{ pl tra } \\
\underline{baldõhin}\,$\oplus$\,oissaa & \textsc{ pl ter } \\
\underline{baldõhin}\,$\oplus$\,oika & \textsc{ pl com } \\
\end{tabular}
\end{sideways}
\captionof{table}{Tüüpsõnamall \arabic{mallinumber}\,\vadja{baldõhina} ekstraheeritud muutvormimallid.}
\label{tab:tüüpsõnamall-baldõhina}

\end{minipage}

 
\vspace{1em}
\noindent Tüüpsõnamall \vadja{baldõhina} hõlmab vormisõnastikus 20 lekseemi: \vadja{\underline{baldõhin}a, \underline{baraban}a, \underline{fotokartočk}a, \underline{gran}a, \underline{grib}a, \underline{kal}a, \underline{kan}a, \underline{lih}a, \underline{lin}a, \underline{litr}a, \underline{maj}a, \underline{rah}a, \underline{sum}a, \underline{sõn}a, \underline{tar}a, \underline{telefon}a, \underline{televizor}a, \underline{til}a, \underline{van}a} ja \vadja{\underline{asti}a}.

Tüüp\-sõna\-mall kirjeldab tagapoolseid tüvemuutuseta sõnu, mille lõpuvokaal on \textit{a}.


\vspace{3.5em}
\noindent \begin{minipage}{\textwidth}
\stepcounter{mallinumber}
\noindent \textbf{Tüüpsõnamall \arabic{mallinumber}\,\vadja{az̕z̕õ}}\\

\begin{sideways}
\begin{tabular}{l l}
muutvormimall & tunnused \\
\hline
\underline{az̕z̕}\,$\oplus$\,õ & \textsc{ sg nom } \\
\underline{az̕z̕}\,$\oplus$\,a & \textsc{ sg gen } \\
\underline{az̕z̕}\,$\oplus$\,a & \textsc{ sg par } \\
\underline{az̕z̕}\,$\oplus$\,asõ & \textsc{ sg ill } \\
\underline{az̕z̕}\,$\oplus$\,õz & \textsc{ sg ine } \\
\underline{az̕z̕}\,$\oplus$\,õss & \textsc{ sg ela } \\
\underline{az̕z̕}\,$\oplus$\,õllõ & \textsc{ sg all } \\
\underline{az̕z̕}\,$\oplus$\,õll & \textsc{ sg ade } \\
\underline{az̕z̕}\,$\oplus$\,õlt & \textsc{ sg abl } \\
\underline{az̕z̕}\,$\oplus$\,õssi & \textsc{ sg tra } \\
\underline{az̕z̕}\,$\oplus$\,õssaa & \textsc{ sg ter } \\
\underline{az̕z̕}\,$\oplus$\,õka & \textsc{ sg com } \\
\underline{az̕z̕}\,$\oplus$\,õd & \textsc{ pl nom } \\
\underline{az̕z̕}\,$\oplus$\,ojõ & \textsc{ pl gen } \\
\underline{az̕z̕}\,$\oplus$\,oit & \textsc{ pl par } \\
\underline{az̕z̕}\,$\oplus$\,oisõ & \textsc{ pl ill } \\
\underline{az̕z̕}\,$\oplus$\,oiz & \textsc{ pl ine } \\
\underline{az̕z̕}\,$\oplus$\,oiss & \textsc{ pl ela } \\
\underline{az̕z̕}\,$\oplus$\,oillõ & \textsc{ pl all } \\
\underline{az̕z̕}\,$\oplus$\,oill & \textsc{ pl ade } \\
\underline{az̕z̕}\,$\oplus$\,oilt & \textsc{ pl abl } \\
\underline{az̕z̕}\,$\oplus$\,oissi & \textsc{ pl tra } \\
\underline{az̕z̕}\,$\oplus$\,oissaa & \textsc{ pl ter } \\
\underline{az̕z̕}\,$\oplus$\,oika & \textsc{ pl com } \\
\end{tabular}
\end{sideways}
\captionof{table}{Tüüpsõnamall \arabic{mallinumber}\,\vadja{az̕z̕õ} ekstraheeritud muutvormimallid.}
\label{tab:tüüpsõnamall-az̕z̕õ}

\end{minipage}

 
\vspace{1em}
\noindent Tüüpsõnamall \vadja{az̕z̕õ} hõlmab vormisõnastikus 76 lekseemi: \vadja{\underline{az̕z̕}õ, \underline{bad̕d̕}õ, \underline{bahvõl}õ, \underline{bl̕aah}õ, \underline{bobr}õ, \underline{borkkan}õ, \underline{bruud}õ, \underline{čirjav}õ, \underline{čirj}õ, \underline{d̕eel}õ, \underline{dobr}õ, \underline{film}õ, \underline{glaiz}õ, \underline{gramm}õ, \underline{gribavihm}õ, \underline{iiv}õ, \underline{jumal}õ, \underline{jurm}õ, \underline{kabj}õ, \underline{kagl}õ, \underline{kagr}õ, \underline{kajag}õ, \underline{kambõl}õ, \underline{kanav}õ, \underline{karj}õ, \underline{kass}õ, \underline{katag}õ, \underline{kaval}õ, \underline{kvartir}õ, \underline{lad̕d̕}õ, \underline{ladv}õ, \underline{lahj}õ, \underline{lahn}õ, \underline{lain}õ, \underline{laiv}õ, \underline{liiv}õ, \underline{linn}õ, \underline{l̕ist}õ, \underline{maailm}õ, \underline{maam}õ, \underline{mahl}õ, \underline{mann}õ, \underline{marj}õ, \underline{matal}õ, \underline{metl}õ, \underline{murag}õ, \underline{mussõmarj}õ, \underline{nagl}õ, \underline{n̕egr}õ, \underline{nigl}õ, \underline{ostanofk}õ, \underline{pagl}õ, \underline{progon}õ, \underline{pudr}õ, \underline{pupušk}õ, \underline{rauh}õ, \underline{saappõg}õ, \underline{sarj}õ, \underline{saun}õ, \underline{sigl}õ, \underline{sisav}õ, \underline{sl̕ifk}õ, \underline{summ}õ, \underline{surm}õ, \underline{suukkurliiv}õ, \underline{sõbr}õ, \underline{šuub}õ, \underline{ženih}õ, \underline{taičin}õ, \underline{trub}õ, \underline{vihm}õ, \underline{vikaht}õ, \underline{vill}õ, \underline{õrav}õ, \underline{õzr}õ} ja \vadja{\underline{akkun}õ}.

Tüüp\-sõna\-mall kirjeldab tagapoolseid tüvemuutuseta sõnu, mille lõpuvokaal on \textit{õ}.


\vspace{3.5em}
\noindent \begin{minipage}{\textwidth}
\stepcounter{mallinumber}
\noindent \textbf{Tüüpsõnamall \arabic{mallinumber}\,\vadja{bank}}\\

\begin{sideways}
\begin{tabular}{l l}
muutvormimall & tunnused \\
\hline
\underline{bank} & \textsc{ sg nom } \\
\underline{bank}\,$\oplus$\,a & \textsc{ sg gen } \\
\underline{bank}\,$\oplus$\,a & \textsc{ sg par } \\
\underline{bank}\,$\oplus$\,asõ & \textsc{ sg ill } \\
\underline{bank}\,$\oplus$\,õz & \textsc{ sg ine } \\
\underline{bank}\,$\oplus$\,õss & \textsc{ sg ela } \\
\underline{bank}\,$\oplus$\,õllõ & \textsc{ sg all } \\
\underline{bank}\,$\oplus$\,õll & \textsc{ sg ade } \\
\underline{bank}\,$\oplus$\,õlt & \textsc{ sg abl } \\
\underline{bank}\,$\oplus$\,õssi & \textsc{ sg tra } \\
\underline{bank}\,$\oplus$\,õssaa & \textsc{ sg ter } \\
\underline{bank}\,$\oplus$\,õka & \textsc{ sg com } \\
\underline{bank}\,$\oplus$\,õd & \textsc{ pl nom } \\
\underline{bank}\,$\oplus$\,ojõ & \textsc{ pl gen } \\
\underline{bank}\,$\oplus$\,oit & \textsc{ pl par } \\
\underline{bank}\,$\oplus$\,oisõ & \textsc{ pl ill } \\
\underline{bank}\,$\oplus$\,oiz & \textsc{ pl ine } \\
\underline{bank}\,$\oplus$\,oiss & \textsc{ pl ela } \\
\underline{bank}\,$\oplus$\,oillõ & \textsc{ pl all } \\
\underline{bank}\,$\oplus$\,oill & \textsc{ pl ade } \\
\underline{bank}\,$\oplus$\,oilt & \textsc{ pl abl } \\
\underline{bank}\,$\oplus$\,oissi & \textsc{ pl tra } \\
\underline{bank}\,$\oplus$\,oissaa & \textsc{ pl ter } \\
\underline{bank}\,$\oplus$\,oika & \textsc{ pl com } \\
\end{tabular}
\end{sideways}
\captionof{table}{Tüüpsõnamall \arabic{mallinumber}\,\vadja{bank} ekstraheeritud muutvormimallid.}
\label{tab:tüüpsõnamall-bank}

\end{minipage}

 
\vspace{1em}
\noindent Tüüpsõnamall \vadja{bank} hõlmab vormisõnastikus 34 lekseemi: \vadja{\underline{bank}, \underline{bl̕uud}, \underline{bl̕uudõčk}, \underline{boran}, \underline{fartõl}, \underline{fialk}, \underline{figur}, \underline{fortočk}, \underline{frikad̕el̕k}, \underline{golod}, \underline{greettsin}, \underline{gupk}, \underline{invaliid}, \underline{kaban}, \underline{kamal}, \underline{kamin}, \underline{kanal}, \underline{kipun}, \underline{kluub}, \underline{kohin}, \underline{l̕ihoratk}, \underline{mašin}, \underline{mašinist}, \underline{muudõr}, \underline{omõn}, \underline{pagan}, \underline{pen̕sioner}, \underline{sammõl}, \underline{zanavesk}, \underline{žurnalist}, \underline{tarelk}, \underline{vaahtõr}, \underline{viks}} ja \vadja{\underline{ahvõn}}.

Tüüp\-sõna\-mall kirjeldab tagapoolseid tüvemuutuseta sõnu, millel puudub lõpuvokaal.


\vspace{3.5em}
\noindent \begin{minipage}{\textwidth}
\stepcounter{mallinumber}
\noindent \textbf{Tüüpsõnamall \arabic{mallinumber}\,\vadja{pintõ}}\\

\begin{sideways}
\begin{tabular}{l l}
muutvormimall & tunnused \\
\hline
\underline{pin}\,$\oplus$\,tõ & \textsc{ sg nom } \\
\underline{pin}\,$\oplus$\,na & \textsc{ sg gen } \\
\underline{pin}\,$\oplus$\,ta & \textsc{ sg par } \\
\underline{pin}\,$\oplus$\,tasõ & \textsc{ sg ill } \\
\underline{pin}\,$\oplus$\,nõz & \textsc{ sg ine } \\
\underline{pin}\,$\oplus$\,nõss & \textsc{ sg ela } \\
\underline{pin}\,$\oplus$\,nõllõ & \textsc{ sg all } \\
\underline{pin}\,$\oplus$\,nõll & \textsc{ sg ade } \\
\underline{pin}\,$\oplus$\,nõlt & \textsc{ sg abl } \\
\underline{pin}\,$\oplus$\,nõssi & \textsc{ sg tra } \\
\underline{pin}\,$\oplus$\,nõssaa & \textsc{ sg ter } \\
\underline{pin}\,$\oplus$\,nõka & \textsc{ sg com } \\
\underline{pin}\,$\oplus$\,nõd & \textsc{ pl nom } \\
\underline{pin}\,$\oplus$\,tojõ & \textsc{ pl gen } \\
\underline{pin}\,$\oplus$\,toit & \textsc{ pl par } \\
\underline{pin}\,$\oplus$\,toisõ & \textsc{ pl ill } \\
\underline{pin}\,$\oplus$\,toiz & \textsc{ pl ine } \\
\underline{pin}\,$\oplus$\,toiss & \textsc{ pl ela } \\
\underline{pin}\,$\oplus$\,toillõ & \textsc{ pl all } \\
\underline{pin}\,$\oplus$\,toill & \textsc{ pl ade } \\
\underline{pin}\,$\oplus$\,toilt & \textsc{ pl abl } \\
\underline{pin}\,$\oplus$\,toissi & \textsc{ pl tra } \\
\underline{pin}\,$\oplus$\,toissaa & \textsc{ pl ter } \\
\underline{pin}\,$\oplus$\,toika & \textsc{ pl com } \\
\end{tabular}
\end{sideways}
\captionof{table}{Tüüpsõnamall \arabic{mallinumber}\,\vadja{pintõ} ekstraheeritud muutvormimallid.}
\label{tab:tüüpsõnamall-pintõ}

\end{minipage}

 
\vspace{1em}
\noindent Tüüpsõnamall \vadja{pintõ} hõlmab vormisõnastikus 4 lekseemi: \vadja{\underline{pin}tõ, \underline{ran}tõ, \underline{rin}tõ} ja \vadja{\underline{kan}tõ}.

Tüüp\-sõna\-mall kirjeldab tagapoolseid sõnu tüvemuutusega nt:nn.


\vspace{3.5em}
\noindent \begin{minipage}{\textwidth}
\stepcounter{mallinumber}
\noindent \textbf{Tüüpsõnamall \arabic{mallinumber}\,\vadja{rapa}}\\

\begin{sideways}
\begin{tabular}{l l}
muutvormimall & tunnused \\
\hline
\underline{ra}\,$\oplus$\,pa & \textsc{ sg nom } \\
\underline{ra}\,$\oplus$\,va & \textsc{ sg gen } \\
\underline{ra}\,$\oplus$\,ppa & \textsc{ sg par } \\
\underline{ra}\,$\oplus$\,ppasõ & \textsc{ sg ill } \\
\underline{ra}\,$\oplus$\,vaz & \textsc{ sg ine } \\
\underline{ra}\,$\oplus$\,vass & \textsc{ sg ela } \\
\underline{ra}\,$\oplus$\,vallõ & \textsc{ sg all } \\
\underline{ra}\,$\oplus$\,vall & \textsc{ sg ade } \\
\underline{ra}\,$\oplus$\,valt & \textsc{ sg abl } \\
\underline{ra}\,$\oplus$\,vassi & \textsc{ sg tra } \\
\underline{ra}\,$\oplus$\,vassaa & \textsc{ sg ter } \\
\underline{ra}\,$\oplus$\,vaka & \textsc{ sg com } \\
\underline{ra}\,$\oplus$\,vad & \textsc{ pl nom } \\
\underline{ra}\,$\oplus$\,pojõ & \textsc{ pl gen } \\
\underline{ra}\,$\oplus$\,poit & \textsc{ pl par } \\
\underline{ra}\,$\oplus$\,poisõ & \textsc{ pl ill } \\
\underline{ra}\,$\oplus$\,poiz & \textsc{ pl ine } \\
\underline{ra}\,$\oplus$\,poiss & \textsc{ pl ela } \\
\underline{ra}\,$\oplus$\,poillõ & \textsc{ pl all } \\
\underline{ra}\,$\oplus$\,poill & \textsc{ pl ade } \\
\underline{ra}\,$\oplus$\,poilt & \textsc{ pl abl } \\
\underline{ra}\,$\oplus$\,poissi & \textsc{ pl tra } \\
\underline{ra}\,$\oplus$\,poissaa & \textsc{ pl ter } \\
\underline{ra}\,$\oplus$\,poika & \textsc{ pl com } \\
\end{tabular}
\end{sideways}
\captionof{table}{Tüüpsõnamall \arabic{mallinumber}\,\vadja{rapa} ekstraheeritud muutvormimallid.}
\label{tab:tüüpsõnamall-rapa}

\end{minipage}

 
\vspace{1em}
\noindent Tüüpsõnamall \vadja{rapa} hõlmab vormisõnastikus 3 lekseemi: \vadja{\underline{ra}pa, \underline{sõ}pa} ja \vadja{\underline{na}pa}.

Tüüp\-sõna\-mall kirjeldab tagapoolseid sõnu tüvemuutusega p:v, mille lõpuvokaal on \textit{a}.
\\

\vspace{3.5em}
\noindent \begin{minipage}{\textwidth}
\stepcounter{mallinumber}
\noindent \textbf{Tüüpsõnamall \arabic{mallinumber}\,\vadja{aapõ}}\\

\begin{sideways}
\begin{tabular}{l l}
muutvormimall & tunnused \\
\hline
\underline{aa}\,$\oplus$\,põ & \textsc{ sg nom } \\
\underline{aa}\,$\oplus$\,va & \textsc{ sg gen } \\
\underline{aa}\,$\oplus$\,pa & \textsc{ sg par } \\
\underline{aa}\,$\oplus$\,pasõ & \textsc{ sg ill } \\
\underline{aa}\,$\oplus$\,võz & \textsc{ sg ine } \\
\underline{aa}\,$\oplus$\,võss & \textsc{ sg ela } \\
\underline{aa}\,$\oplus$\,võllõ & \textsc{ sg all } \\
\underline{aa}\,$\oplus$\,võll & \textsc{ sg ade } \\
\underline{aa}\,$\oplus$\,võlt & \textsc{ sg abl } \\
\underline{aa}\,$\oplus$\,võssi & \textsc{ sg tra } \\
\underline{aa}\,$\oplus$\,võssaa & \textsc{ sg ter } \\
\underline{aa}\,$\oplus$\,võka & \textsc{ sg com } \\
\underline{aa}\,$\oplus$\,võd & \textsc{ pl nom } \\
\underline{aa}\,$\oplus$\,pojõ & \textsc{ pl gen } \\
\underline{aa}\,$\oplus$\,poit & \textsc{ pl par } \\
\underline{aa}\,$\oplus$\,poisõ & \textsc{ pl ill } \\
\underline{aa}\,$\oplus$\,poiz & \textsc{ pl ine } \\
\underline{aa}\,$\oplus$\,poiss & \textsc{ pl ela } \\
\underline{aa}\,$\oplus$\,poillõ & \textsc{ pl all } \\
\underline{aa}\,$\oplus$\,poill & \textsc{ pl ade } \\
\underline{aa}\,$\oplus$\,poilt & \textsc{ pl abl } \\
\underline{aa}\,$\oplus$\,poissi & \textsc{ pl tra } \\
\underline{aa}\,$\oplus$\,poissaa & \textsc{ pl ter } \\
\underline{aa}\,$\oplus$\,poika & \textsc{ pl com } \\
\end{tabular}
\end{sideways}
\captionof{table}{Tüüpsõnamall \arabic{mallinumber}\,\vadja{aapõ} ekstraheeritud muutvormimallid.}
\label{tab:tüüpsõnamall-aapõ}

\end{minipage}

 
\vspace{1em}
\noindent Tüüpsõnamall \vadja{aapõ} ei hõlma teisi lekseeme vormi\-sõnastikus.

Tüüp\-sõna\-mall kirjeldab tagapoolseid sõnu tüvemuutusega p:v, mille lõpuvokaal on \textit{õ}.


\vspace{3.5em}
\noindent \begin{minipage}{\textwidth}
\stepcounter{mallinumber}
\noindent \textbf{Tüüpsõnamall \arabic{mallinumber}\,\vadja{liippõ}}\\

\begin{sideways}
\begin{tabular}{l l}
muutvormimall & tunnused \\
\hline
\underline{liip}\,$\oplus$\,põ & \textsc{ sg nom } \\
\underline{liip}\,$\oplus$\,a & \textsc{ sg gen } \\
\underline{liip}\,$\oplus$\,pa & \textsc{ sg par } \\
\underline{liip}\,$\oplus$\,pasõ & \textsc{ sg ill } \\
\underline{liip}\,$\oplus$\,põz & \textsc{ sg ine } \\
\underline{liip}\,$\oplus$\,õss & \textsc{ sg ela } \\
\underline{liip}\,$\oplus$\,õllõ & \textsc{ sg all } \\
\underline{liip}\,$\oplus$\,õll & \textsc{ sg ade } \\
\underline{liip}\,$\oplus$\,õlt & \textsc{ sg abl } \\
\underline{liip}\,$\oplus$\,õssi & \textsc{ sg tra } \\
\underline{liip}\,$\oplus$\,põssaa & \textsc{ sg ter } \\
\underline{liip}\,$\oplus$\,õka & \textsc{ sg com } \\
\underline{liip}\,$\oplus$\,õd & \textsc{ pl nom } \\
\underline{liip}\,$\oplus$\,pojõ & \textsc{ pl gen } \\
\underline{liip}\,$\oplus$\,poit & \textsc{ pl par } \\
\underline{liip}\,$\oplus$\,poisõ & \textsc{ pl ill } \\
\underline{liip}\,$\oplus$\,poiz & \textsc{ pl ine } \\
\underline{liip}\,$\oplus$\,poiss & \textsc{ pl ela } \\
\underline{liip}\,$\oplus$\,poillõ & \textsc{ pl all } \\
\underline{liip}\,$\oplus$\,poill & \textsc{ pl ade } \\
\underline{liip}\,$\oplus$\,poilt & \textsc{ pl abl } \\
\underline{liip}\,$\oplus$\,poissi & \textsc{ pl tra } \\
\underline{liip}\,$\oplus$\,poissaa & \textsc{ pl ter } \\
\underline{liip}\,$\oplus$\,poika & \textsc{ pl com } \\
\end{tabular}
\end{sideways}
\captionof{table}{Tüüpsõnamall \arabic{mallinumber}\,\vadja{liippõ} ekstraheeritud muutvormimallid.}
\label{tab:tüüpsõnamall-liippõ}

\end{minipage}

 
\vspace{1em}
\noindent Tüüpsõnamall \vadja{liippõ} hõlmab vormisõnastikus 2 lekseemi: \vadja{\underline{liip}põ} ja \vadja{\underline{kaup}põ}.

Tüüp\-sõna\-mall kirjeldab tagapoolseid sõnu tüvemuutusega pp:p.


\vspace{3.5em}
\noindent \begin{minipage}{\textwidth}
\stepcounter{mallinumber}
\noindent \textbf{Tüüpsõnamall \arabic{mallinumber}\,\vadja{partõ}}\\

\begin{sideways}
\begin{tabular}{l l}
muutvormimall & tunnused \\
\hline
\underline{par}\,$\oplus$\,tõ & \textsc{ sg nom } \\
\underline{par}\,$\oplus$\,ra & \textsc{ sg gen } \\
\underline{par}\,$\oplus$\,ta & \textsc{ sg par } \\
\underline{par}\,$\oplus$\,tasõ & \textsc{ sg ill } \\
\underline{par}\,$\oplus$\,rõz & \textsc{ sg ine } \\
\underline{par}\,$\oplus$\,rõss & \textsc{ sg ela } \\
\underline{par}\,$\oplus$\,rõllõ & \textsc{ sg all } \\
\underline{par}\,$\oplus$\,rõll & \textsc{ sg ade } \\
\underline{par}\,$\oplus$\,rõlt & \textsc{ sg abl } \\
\underline{par}\,$\oplus$\,rõssi & \textsc{ sg tra } \\
\underline{par}\,$\oplus$\,rõssaa & \textsc{ sg ter } \\
\underline{par}\,$\oplus$\,rõka & \textsc{ sg com } \\
\underline{par}\,$\oplus$\,rõd & \textsc{ pl nom } \\
\underline{par}\,$\oplus$\,tojõ & \textsc{ pl gen } \\
\underline{par}\,$\oplus$\,toit & \textsc{ pl par } \\
\underline{par}\,$\oplus$\,toisõ & \textsc{ pl ill } \\
\underline{par}\,$\oplus$\,toiz & \textsc{ pl ine } \\
\underline{par}\,$\oplus$\,toiss & \textsc{ pl ela } \\
\underline{par}\,$\oplus$\,toillõ & \textsc{ pl all } \\
\underline{par}\,$\oplus$\,toill & \textsc{ pl ade } \\
\underline{par}\,$\oplus$\,toilt & \textsc{ pl abl } \\
\underline{par}\,$\oplus$\,toissi & \textsc{ pl tra } \\
\underline{par}\,$\oplus$\,toissaa & \textsc{ pl ter } \\
\underline{par}\,$\oplus$\,toika & \textsc{ pl com } \\
\end{tabular}
\end{sideways}
\captionof{table}{Tüüpsõnamall \arabic{mallinumber}\,\vadja{partõ} ekstraheeritud muutvormimallid.}
\label{tab:tüüpsõnamall-partõ}

\end{minipage}

 
\vspace{1em}
\noindent Tüüpsõnamall \vadja{partõ} hõlmab vormisõnastikus 2 lekseemi: \vadja{\underline{par}tõ} ja \vadja{\underline{kõr}tõ}.

Tüüp\-sõna\-mall kirjeldab tagapoolseid sõnu tüvemuutusega rt:rr.


\vspace{3.5em}
\noindent \begin{minipage}{\textwidth}
\stepcounter{mallinumber}
\noindent \textbf{Tüüpsõnamall \arabic{mallinumber}\,\vadja{kraaskõ}}\\

\begin{sideways}
\begin{tabular}{l l}
muutvormimall & tunnused \\
\hline
\underline{kraa}\,$\oplus$\,skõ & \textsc{ sg nom } \\
\underline{kraa}\,$\oplus$\,zga & \textsc{ sg gen } \\
\underline{kraa}\,$\oplus$\,ska & \textsc{ sg par } \\
\underline{kraa}\,$\oplus$\,skasõ & \textsc{ sg ill } \\
\underline{kraa}\,$\oplus$\,zgõz & \textsc{ sg ine } \\
\underline{kraa}\,$\oplus$\,zgõss & \textsc{ sg ela } \\
\underline{kraa}\,$\oplus$\,zgõllõ & \textsc{ sg all } \\
\underline{kraa}\,$\oplus$\,zgõll & \textsc{ sg ade } \\
\underline{kraa}\,$\oplus$\,zgõlt & \textsc{ sg abl } \\
\underline{kraa}\,$\oplus$\,zgõssi & \textsc{ sg tra } \\
\underline{kraa}\,$\oplus$\,zgõssaa & \textsc{ sg ter } \\
\underline{kraa}\,$\oplus$\,zgõka & \textsc{ sg com } \\
\underline{kraa}\,$\oplus$\,zgõd & \textsc{ pl nom } \\
\underline{kraa}\,$\oplus$\,skojõ & \textsc{ pl gen } \\
\underline{kraa}\,$\oplus$\,skoit & \textsc{ pl par } \\
\underline{kraa}\,$\oplus$\,skoisõ & \textsc{ pl ill } \\
\underline{kraa}\,$\oplus$\,skoiz & \textsc{ pl ine } \\
\underline{kraa}\,$\oplus$\,skoiss & \textsc{ pl ela } \\
\underline{kraa}\,$\oplus$\,skoillõ & \textsc{ pl all } \\
\underline{kraa}\,$\oplus$\,skoill & \textsc{ pl ade } \\
\underline{kraa}\,$\oplus$\,skoilt & \textsc{ pl abl } \\
\underline{kraa}\,$\oplus$\,skoissi & \textsc{ pl tra } \\
\underline{kraa}\,$\oplus$\,skoissaa & \textsc{ pl ter } \\
\underline{kraa}\,$\oplus$\,skoika & \textsc{ pl com } \\
\end{tabular}
\end{sideways}
\captionof{table}{Tüüpsõnamall \arabic{mallinumber}\,\vadja{kraaskõ} ekstraheeritud muutvormimallid.}
\label{tab:tüüpsõnamall-kraaskõ}

\end{minipage}

 
\vspace{1em}
\noindent Tüüpsõnamall \vadja{kraaskõ} hõlmab vormisõnastikus 6 lekseemi: \vadja{\underline{kraa}skõ, \underline{lai}skõ, \underline{nagriskaa}skõ, \underline{ni}skõ, \underline{pa}skõ} ja \vadja{\underline{kaa}skõ}.

Tüüp\-sõna\-mall kirjeldab tagapoolseid sõnu tüvemuutusega sk:zg.


\vspace{3.5em}
\noindent \begin{minipage}{\textwidth}
\stepcounter{mallinumber}
\noindent \textbf{Tüüpsõnamall \arabic{mallinumber}\,\vadja{klaassõ}}\\

\begin{sideways}
\begin{tabular}{l l}
muutvormimall & tunnused \\
\hline
\underline{klaas}\,$\oplus$\,sõ & \textsc{ sg nom } \\
\underline{klaas}\,$\oplus$\,a & \textsc{ sg gen } \\
\underline{klaas}\,$\oplus$\,sa & \textsc{ sg par } \\
\underline{klaas}\,$\oplus$\,sasõ & \textsc{ sg ill } \\
\underline{klaas}\,$\oplus$\,sõz & \textsc{ sg ine } \\
\underline{klaas}\,$\oplus$\,õss & \textsc{ sg ela } \\
\underline{klaas}\,$\oplus$\,õllõ & \textsc{ sg all } \\
\underline{klaas}\,$\oplus$\,õll & \textsc{ sg ade } \\
\underline{klaas}\,$\oplus$\,õlt & \textsc{ sg abl } \\
\underline{klaas}\,$\oplus$\,õssi & \textsc{ sg tra } \\
\underline{klaas}\,$\oplus$\,sõssaa & \textsc{ sg ter } \\
\underline{klaas}\,$\oplus$\,õka & \textsc{ sg com } \\
\underline{klaas}\,$\oplus$\,õd & \textsc{ pl nom } \\
\underline{klaas}\,$\oplus$\,sojõ & \textsc{ pl gen } \\
\underline{klaas}\,$\oplus$\,soit & \textsc{ pl par } \\
\underline{klaas}\,$\oplus$\,soisõ & \textsc{ pl ill } \\
\underline{klaas}\,$\oplus$\,soiz & \textsc{ pl ine } \\
\underline{klaas}\,$\oplus$\,soiss & \textsc{ pl ela } \\
\underline{klaas}\,$\oplus$\,soillõ & \textsc{ pl all } \\
\underline{klaas}\,$\oplus$\,soill & \textsc{ pl ade } \\
\underline{klaas}\,$\oplus$\,soilt & \textsc{ pl abl } \\
\underline{klaas}\,$\oplus$\,soissi & \textsc{ pl tra } \\
\underline{klaas}\,$\oplus$\,soissaa & \textsc{ pl ter } \\
\underline{klaas}\,$\oplus$\,soika & \textsc{ pl com } \\
\end{tabular}
\end{sideways}
\captionof{table}{Tüüpsõnamall \arabic{mallinumber}\,\vadja{klaassõ} ekstraheeritud muutvormimallid.}
\label{tab:tüüpsõnamall-klaassõ}

\end{minipage}

 
\vspace{1em}
\noindent Tüüpsõnamall \vadja{klaassõ} hõlmab vormisõnastikus 2 lekseemi: \vadja{\underline{klaas}sõ} ja \vadja{\underline{bruus}sõ}.

Tüüp\-sõna\-mall kirjeldab tagapoolseid sõnu tüvemuutusega ss:s.


\vspace{3.5em}
\noindent \begin{minipage}{\textwidth}
\stepcounter{mallinumber}
\noindent \textbf{Tüüpsõnamall \arabic{mallinumber}\,\vadja{poduškõ}}\\

\begin{sideways}
\begin{tabular}{l l}
muutvormimall & tunnused \\
\hline
\underline{podu}\,$\oplus$\,škõ & \textsc{ sg nom } \\
\underline{podu}\,$\oplus$\,žga & \textsc{ sg gen } \\
\underline{podu}\,$\oplus$\,ška & \textsc{ sg par } \\
\underline{podu}\,$\oplus$\,škasõ & \textsc{ sg ill } \\
\underline{podu}\,$\oplus$\,žgõz & \textsc{ sg ine } \\
\underline{podu}\,$\oplus$\,žgõss & \textsc{ sg ela } \\
\underline{podu}\,$\oplus$\,žgõllõ & \textsc{ sg all } \\
\underline{podu}\,$\oplus$\,žgõll & \textsc{ sg ade } \\
\underline{podu}\,$\oplus$\,žgõlt & \textsc{ sg abl } \\
\underline{podu}\,$\oplus$\,žgõssi & \textsc{ sg tra } \\
\underline{podu}\,$\oplus$\,žgõssaa & \textsc{ sg ter } \\
\underline{podu}\,$\oplus$\,žgõka & \textsc{ sg com } \\
\underline{podu}\,$\oplus$\,žgõd & \textsc{ pl nom } \\
\underline{podu}\,$\oplus$\,škojõ & \textsc{ pl gen } \\
\underline{podu}\,$\oplus$\,škoit & \textsc{ pl par } \\
\underline{podu}\,$\oplus$\,škoisõ & \textsc{ pl ill } \\
\underline{podu}\,$\oplus$\,škoiz & \textsc{ pl ine } \\
\underline{podu}\,$\oplus$\,škoiss & \textsc{ pl ela } \\
\underline{podu}\,$\oplus$\,škoillõ & \textsc{ pl all } \\
\underline{podu}\,$\oplus$\,škoill & \textsc{ pl ade } \\
\underline{podu}\,$\oplus$\,škoilt & \textsc{ pl abl } \\
\underline{podu}\,$\oplus$\,škoissi & \textsc{ pl tra } \\
\underline{podu}\,$\oplus$\,škoissaa & \textsc{ pl ter } \\
\underline{podu}\,$\oplus$\,škoika & \textsc{ pl com } \\
\end{tabular}
\end{sideways}
\captionof{table}{Tüüpsõnamall \arabic{mallinumber}\,\vadja{poduškõ} ekstraheeritud muutvormimallid.}
\label{tab:tüüpsõnamall-poduškõ}

\end{minipage}

 
\vspace{1em}
\noindent Tüüpsõnamall \vadja{poduškõ} ei hõlma teisi lekseeme vormi\-sõnastikus.

Tüüp\-sõna\-mall kirjeldab tagapoolseid sõnu tüvemuutusega šk:žg.


\vspace{3.5em}
\noindent \begin{minipage}{\textwidth}
\stepcounter{mallinumber}
\noindent \textbf{Tüüpsõnamall \arabic{mallinumber}\,\vadja{dovariššõ}}\\

\begin{sideways}
\begin{tabular}{l l}
muutvormimall & tunnused \\
\hline
\underline{dovariš}\,$\oplus$\,šõ & \textsc{ sg nom } \\
\underline{dovariš}\,$\oplus$\,a & \textsc{ sg gen } \\
\underline{dovariš}\,$\oplus$\,ša & \textsc{ sg par } \\
\underline{dovariš}\,$\oplus$\,šasõ & \textsc{ sg ill } \\
\underline{dovariš}\,$\oplus$\,az & \textsc{ sg ine } \\
\underline{dovariš}\,$\oplus$\,ass & \textsc{ sg ela } \\
\underline{dovariš}\,$\oplus$\,allõ & \textsc{ sg all } \\
\underline{dovariš}\,$\oplus$\,all & \textsc{ sg ade } \\
\underline{dovariš}\,$\oplus$\,alt & \textsc{ sg abl } \\
\underline{dovariš}\,$\oplus$\,assi & \textsc{ sg tra } \\
\underline{dovariš}\,$\oplus$\,assaa & \textsc{ sg ter } \\
\underline{dovariš}\,$\oplus$\,aka & \textsc{ sg com } \\
\underline{dovariš}\,$\oplus$\,ad & \textsc{ pl nom } \\
\underline{dovariš}\,$\oplus$\,šojõ & \textsc{ pl gen } \\
\underline{dovariš}\,$\oplus$\,šoit & \textsc{ pl par } \\
\underline{dovariš}\,$\oplus$\,šoisõ & \textsc{ pl ill } \\
\underline{dovariš}\,$\oplus$\,šoiz & \textsc{ pl ine } \\
\underline{dovariš}\,$\oplus$\,šoiss & \textsc{ pl ela } \\
\underline{dovariš}\,$\oplus$\,šoillõ & \textsc{ pl all } \\
\underline{dovariš}\,$\oplus$\,šoill & \textsc{ pl ade } \\
\underline{dovariš}\,$\oplus$\,šoilt & \textsc{ pl abl } \\
\underline{dovariš}\,$\oplus$\,šoissi & \textsc{ pl tra } \\
\underline{dovariš}\,$\oplus$\,šoissaa & \textsc{ pl ter } \\
\underline{dovariš}\,$\oplus$\,šoika & \textsc{ pl com } \\
\end{tabular}
\end{sideways}
\captionof{table}{Tüüpsõnamall \arabic{mallinumber}\,\vadja{dovariššõ} ekstraheeritud muutvormimallid.}
\label{tab:tüüpsõnamall-dovariššõ}

\end{minipage}

 
\vspace{1em}
\noindent Tüüpsõnamall \vadja{dovariššõ} ei hõlma teisi lekseeme vormi\-sõnastikus.

Tüüp\-sõna\-mall kirjeldab tagapoolseid sõnu tüvemuutusega šš:š.


\vspace{3.5em}
\noindent \begin{minipage}{\textwidth}
\stepcounter{mallinumber}
\noindent \textbf{Tüüpsõnamall \arabic{mallinumber}\,\vadja{sata}}\\

\begin{sideways}
\begin{tabular}{l l}
muutvormimall & tunnused \\
\hline
\underline{sa}\,$\oplus$\,ta & \textsc{ sg nom } \\
\underline{sa}\,$\oplus$\,a & \textsc{ sg gen } \\
\underline{sa}\,$\oplus$\,tta & \textsc{ sg par } \\
\underline{sa}\,$\oplus$\,ttasõ & \textsc{ sg ill } \\
\underline{sa}\,$\oplus$\,az & \textsc{ sg ine } \\
\underline{sa}\,$\oplus$\,ass & \textsc{ sg ela } \\
\underline{sa}\,$\oplus$\,allõ & \textsc{ sg all } \\
\underline{sa}\,$\oplus$\,all & \textsc{ sg ade } \\
\underline{sa}\,$\oplus$\,alt & \textsc{ sg abl } \\
\underline{sa}\,$\oplus$\,assi & \textsc{ sg tra } \\
\underline{sa}\,$\oplus$\,assaa & \textsc{ sg ter } \\
\underline{sa}\,$\oplus$\,aka & \textsc{ sg com } \\
\underline{sa}\,$\oplus$\,ad & \textsc{ pl nom } \\
\underline{sa}\,$\oplus$\,tojõ & \textsc{ pl gen } \\
\underline{sa}\,$\oplus$\,toit & \textsc{ pl par } \\
\underline{sa}\,$\oplus$\,toisõ & \textsc{ pl ill } \\
\underline{sa}\,$\oplus$\,toiz & \textsc{ pl ine } \\
\underline{sa}\,$\oplus$\,toiss & \textsc{ pl ela } \\
\underline{sa}\,$\oplus$\,toillõ & \textsc{ pl all } \\
\underline{sa}\,$\oplus$\,toill & \textsc{ pl ade } \\
\underline{sa}\,$\oplus$\,toilt & \textsc{ pl abl } \\
\underline{sa}\,$\oplus$\,toissi & \textsc{ pl tra } \\
\underline{sa}\,$\oplus$\,toissaa & \textsc{ pl ter } \\
\underline{sa}\,$\oplus$\,toika & \textsc{ pl com } \\
\end{tabular}
\end{sideways}
\captionof{table}{Tüüpsõnamall \arabic{mallinumber}\,\vadja{sata} ekstraheeritud muutvormimallid.}
\label{tab:tüüpsõnamall-sata}

\end{minipage}

 
\vspace{1em}
\noindent Tüüpsõnamall \vadja{sata} hõlmab vormisõnastikus 3 lekseemi: \vadja{\underline{sa}ta, \underline{sõ}ta} ja \vadja{\underline{mu}ta}.

Tüüp\-sõna\-mall kirjeldab tagapoolseid sõnu tüvemuutusega t:∅.


\vspace{3.5em}
\noindent \begin{minipage}{\textwidth}
\stepcounter{mallinumber}
\noindent \textbf{Tüüpsõnamall \arabic{mallinumber}\,\vadja{l̕iitkõ}}\\

\begin{sideways}
\begin{tabular}{l l}
muutvormimall & tunnused \\
\hline
\underline{l̕ii}\,$\oplus$\,tkõ & \textsc{ sg nom } \\
\underline{l̕ii}\,$\oplus$\,dga & \textsc{ sg gen } \\
\underline{l̕ii}\,$\oplus$\,tka & \textsc{ sg par } \\
\underline{l̕ii}\,$\oplus$\,tkasõ & \textsc{ sg ill } \\
\underline{l̕ii}\,$\oplus$\,dgõz & \textsc{ sg ine } \\
\underline{l̕ii}\,$\oplus$\,dgõss & \textsc{ sg ela } \\
\underline{l̕ii}\,$\oplus$\,dgõllõ & \textsc{ sg all } \\
\underline{l̕ii}\,$\oplus$\,dgõll & \textsc{ sg ade } \\
\underline{l̕ii}\,$\oplus$\,dgõlt & \textsc{ sg abl } \\
\underline{l̕ii}\,$\oplus$\,dgõssi & \textsc{ sg tra } \\
\underline{l̕ii}\,$\oplus$\,dgõssaa & \textsc{ sg ter } \\
\underline{l̕ii}\,$\oplus$\,dgõka & \textsc{ sg com } \\
\underline{l̕ii}\,$\oplus$\,dgõd & \textsc{ pl nom } \\
\underline{l̕ii}\,$\oplus$\,tkojõ & \textsc{ pl gen } \\
\underline{l̕ii}\,$\oplus$\,tkoit & \textsc{ pl par } \\
\underline{l̕ii}\,$\oplus$\,tkoisõ & \textsc{ pl ill } \\
\underline{l̕ii}\,$\oplus$\,tkoiz & \textsc{ pl ine } \\
\underline{l̕ii}\,$\oplus$\,tkoiss & \textsc{ pl ela } \\
\underline{l̕ii}\,$\oplus$\,tkoillõ & \textsc{ pl all } \\
\underline{l̕ii}\,$\oplus$\,tkoill & \textsc{ pl ade } \\
\underline{l̕ii}\,$\oplus$\,tkoilt & \textsc{ pl abl } \\
\underline{l̕ii}\,$\oplus$\,tkoissi & \textsc{ pl tra } \\
\underline{l̕ii}\,$\oplus$\,tkoissaa & \textsc{ pl ter } \\
\underline{l̕ii}\,$\oplus$\,tkoika & \textsc{ pl com } \\
\end{tabular}
\end{sideways}
\captionof{table}{Tüüpsõnamall \arabic{mallinumber}\,\vadja{l̕iitkõ} ekstraheeritud muutvormimallid.}
\label{tab:tüüpsõnamall-l̕iitkõ}

\end{minipage}

 
\vspace{1em}
\noindent Tüüpsõnamall \vadja{l̕iitkõ} hõlmab vormisõnastikus 2 lekseemi: \vadja{\underline{l̕ii}tkõ} ja \vadja{\underline{bu}tkõ}.

Tüüp\-sõna\-mall kirjeldab tagapoolseid sõnu tüvemuutusega tk:dg.


\vspace{3.5em}
\noindent \begin{minipage}{\textwidth}
\stepcounter{mallinumber}
\noindent \textbf{Tüüpsõnamall \arabic{mallinumber}\,\vadja{kand̕idaattõ}}\\

\begin{sideways}
\begin{tabular}{l l}
muutvormimall & tunnused \\
\hline
\underline{kand̕idaat}\,$\oplus$\,tõ & \textsc{ sg nom } \\
\underline{kand̕idaat}\,$\oplus$\,a & \textsc{ sg gen } \\
\underline{kand̕idaat}\,$\oplus$\,ta & \textsc{ sg par } \\
\underline{kand̕idaat}\,$\oplus$\,tasõ & \textsc{ sg ill } \\
\underline{kand̕idaat}\,$\oplus$\,tõz & \textsc{ sg ine } \\
\underline{kand̕idaat}\,$\oplus$\,õss & \textsc{ sg ela } \\
\underline{kand̕idaat}\,$\oplus$\,õllõ & \textsc{ sg all } \\
\underline{kand̕idaat}\,$\oplus$\,õll & \textsc{ sg ade } \\
\underline{kand̕idaat}\,$\oplus$\,õlt & \textsc{ sg abl } \\
\underline{kand̕idaat}\,$\oplus$\,õssi & \textsc{ sg tra } \\
\underline{kand̕idaat}\,$\oplus$\,tõssaa & \textsc{ sg ter } \\
\underline{kand̕idaat}\,$\oplus$\,õka & \textsc{ sg com } \\
\underline{kand̕idaat}\,$\oplus$\,õd & \textsc{ pl nom } \\
\underline{kand̕idaat}\,$\oplus$\,tojõ & \textsc{ pl gen } \\
\underline{kand̕idaat}\,$\oplus$\,toit & \textsc{ pl par } \\
\underline{kand̕idaat}\,$\oplus$\,toisõ & \textsc{ pl ill } \\
\underline{kand̕idaat}\,$\oplus$\,toiz & \textsc{ pl ine } \\
\underline{kand̕idaat}\,$\oplus$\,toiss & \textsc{ pl ela } \\
\underline{kand̕idaat}\,$\oplus$\,toillõ & \textsc{ pl all } \\
\underline{kand̕idaat}\,$\oplus$\,toill & \textsc{ pl ade } \\
\underline{kand̕idaat}\,$\oplus$\,toilt & \textsc{ pl abl } \\
\underline{kand̕idaat}\,$\oplus$\,toissi & \textsc{ pl tra } \\
\underline{kand̕idaat}\,$\oplus$\,toissaa & \textsc{ pl ter } \\
\underline{kand̕idaat}\,$\oplus$\,toika & \textsc{ pl com } \\
\end{tabular}
\end{sideways}
\captionof{table}{Tüüpsõnamall \arabic{mallinumber}\,\vadja{kand̕idaattõ} ekstraheeritud muutvormimallid.}
\label{tab:tüüpsõnamall-kand̕idaattõ}

\end{minipage}

 
\vspace{1em}
\noindent Tüüpsõnamall \vadja{kand̕idaattõ} hõlmab vormisõnastikus 7 lekseemi: \vadja{\underline{kand̕idaat}tõ, \underline{laut}tõ, \underline{pliit}tõ, \underline{riit}tõ, \underline{žiivõt}tõ, \underline{taat}tõ} ja \vadja{\underline{gaazapliit}tõ}.

Tüüp\-sõna\-mall kirjeldab tagapoolseid sõnu tüvemuutusega tt:t.


\vspace{3.5em}
\noindent \begin{minipage}{\textwidth}
\stepcounter{mallinumber}
\noindent \textbf{Tüüpsõnamall \arabic{mallinumber}\,\vadja{inostranttsõ}}\\

\begin{sideways}
\begin{tabular}{l l}
muutvormimall & tunnused \\
\hline
\underline{inostrant}\,$\oplus$\,t\,$\oplus$\,\underline{s}\,$\oplus$\,õ & \textsc{ sg nom } \\
\underline{inostrant}\,$\oplus$\,\underline{s}\,$\oplus$\,a & \textsc{ sg gen } \\
\underline{inostrant}\,$\oplus$\,t\,$\oplus$\,\underline{s}\,$\oplus$\,a & \textsc{ sg par } \\
\underline{inostrant}\,$\oplus$\,t\,$\oplus$\,\underline{s}\,$\oplus$\,asõ & \textsc{ sg ill } \\
\underline{inostrant}\,$\oplus$\,t\,$\oplus$\,\underline{s}\,$\oplus$\,õz & \textsc{ sg ine } \\
\underline{inostrant}\,$\oplus$\,\underline{s}\,$\oplus$\,õss & \textsc{ sg ela } \\
\underline{inostrant}\,$\oplus$\,\underline{s}\,$\oplus$\,õllõ & \textsc{ sg all } \\
\underline{inostrant}\,$\oplus$\,\underline{s}\,$\oplus$\,õll & \textsc{ sg ade } \\
\underline{inostrant}\,$\oplus$\,\underline{s}\,$\oplus$\,õlt & \textsc{ sg abl } \\
\underline{inostrant}\,$\oplus$\,\underline{s}\,$\oplus$\,õssi & \textsc{ sg tra } \\
\underline{inostrant}\,$\oplus$\,t\,$\oplus$\,\underline{s}\,$\oplus$\,õssaa & \textsc{ sg ter } \\
\underline{inostrant}\,$\oplus$\,\underline{s}\,$\oplus$\,õka & \textsc{ sg com } \\
\underline{inostrant}\,$\oplus$\,\underline{s}\,$\oplus$\,õd & \textsc{ pl nom } \\
\underline{inostrant}\,$\oplus$\,t\,$\oplus$\,\underline{s}\,$\oplus$\,ojõ & \textsc{ pl gen } \\
\underline{inostrant}\,$\oplus$\,t\,$\oplus$\,\underline{s}\,$\oplus$\,oit & \textsc{ pl par } \\
\underline{inostrant}\,$\oplus$\,t\,$\oplus$\,\underline{s}\,$\oplus$\,oisõ & \textsc{ pl ill } \\
\underline{inostrant}\,$\oplus$\,t\,$\oplus$\,\underline{s}\,$\oplus$\,oiz & \textsc{ pl ine } \\
\underline{inostrant}\,$\oplus$\,t\,$\oplus$\,\underline{s}\,$\oplus$\,oiss & \textsc{ pl ela } \\
\underline{inostrant}\,$\oplus$\,t\,$\oplus$\,\underline{s}\,$\oplus$\,oillõ & \textsc{ pl all } \\
\underline{inostrant}\,$\oplus$\,t\,$\oplus$\,\underline{s}\,$\oplus$\,oill & \textsc{ pl ade } \\
\underline{inostrant}\,$\oplus$\,t\,$\oplus$\,\underline{s}\,$\oplus$\,oilt & \textsc{ pl abl } \\
\underline{inostrant}\,$\oplus$\,t\,$\oplus$\,\underline{s}\,$\oplus$\,oissi & \textsc{ pl tra } \\
\underline{inostrant}\,$\oplus$\,t\,$\oplus$\,\underline{s}\,$\oplus$\,oissaa & \textsc{ pl ter } \\
\underline{inostrant}\,$\oplus$\,t\,$\oplus$\,\underline{s}\,$\oplus$\,oika & \textsc{ pl com } \\
\end{tabular}
\end{sideways}
\captionof{table}{Tüüpsõnamall \arabic{mallinumber}\,\vadja{inostranttsõ} ekstraheeritud muutvormimallid.}
\label{tab:tüüpsõnamall-inostranttsõ}

\end{minipage}

 
\vspace{1em}
\noindent Tüüpsõnamall \vadja{inostranttsõ} hõlmab vormisõnastikus 6 lekseemi: \vadja{\underline{inostrant}t\underline{s}õ, \underline{liit}t\underline{s}õ, \underline{tablit}t\underline{s}õ, \underline{vat}t\underline{s}õ, \underline{õt}t\underline{s}õ} ja \vadja{\underline{bol̕nit}t\underline{s}õ}.

Tüüp\-sõna\-mall kirjeldab tagapoolseid sõnu tüvemuutusega tts:ts.


\vspace{3.5em}
\noindent \begin{minipage}{\textwidth}
\stepcounter{mallinumber}
\noindent \textbf{Tüüpsõnamall \arabic{mallinumber}\,\vadja{jõutu}}\\

\begin{sideways}
\begin{tabular}{l l}
muutvormimall & tunnused \\
\hline
\underline{jõ}\,$\oplus$\,\underline{u}\,$\oplus$\,tu & \textsc{ sg nom } \\
\underline{jõ}\,$\oplus$\,vv\,$\oplus$\,\underline{u} & \textsc{ sg gen } \\
\underline{jõ}\,$\oplus$\,\underline{u}\,$\oplus$\,tua & \textsc{ sg par } \\
\underline{jõ}\,$\oplus$\,\underline{u}\,$\oplus$\,tusõ & \textsc{ sg ill } \\
\underline{jõ}\,$\oplus$\,vv\,$\oplus$\,\underline{u}\,$\oplus$\,z & \textsc{ sg ine } \\
\underline{jõ}\,$\oplus$\,vv\,$\oplus$\,\underline{u}\,$\oplus$\,ss & \textsc{ sg ela } \\
\underline{jõ}\,$\oplus$\,vv\,$\oplus$\,\underline{u}\,$\oplus$\,llõ & \textsc{ sg all } \\
\underline{jõ}\,$\oplus$\,vv\,$\oplus$\,\underline{u}\,$\oplus$\,ll & \textsc{ sg ade } \\
\underline{jõ}\,$\oplus$\,vv\,$\oplus$\,\underline{u}\,$\oplus$\,lt & \textsc{ sg abl } \\
\underline{jõ}\,$\oplus$\,vv\,$\oplus$\,\underline{u}\,$\oplus$\,ssi & \textsc{ sg tra } \\
\underline{jõ}\,$\oplus$\,vv\,$\oplus$\,\underline{u}\,$\oplus$\,ssaa & \textsc{ sg ter } \\
\underline{jõ}\,$\oplus$\,vv\,$\oplus$\,\underline{u}\,$\oplus$\,ka & \textsc{ sg com } \\
\underline{jõ}\,$\oplus$\,vv\,$\oplus$\,\underline{u}\,$\oplus$\,d & \textsc{ pl nom } \\
\underline{jõ}\,$\oplus$\,\underline{u}\,$\oplus$\,tujõ & \textsc{ pl gen } \\
\underline{jõ}\,$\oplus$\,\underline{u}\,$\oplus$\,tuit & \textsc{ pl par } \\
\underline{jõ}\,$\oplus$\,\underline{u}\,$\oplus$\,tuisõ & \textsc{ pl ill } \\
\underline{jõ}\,$\oplus$\,\underline{u}\,$\oplus$\,tuiz & \textsc{ pl ine } \\
\underline{jõ}\,$\oplus$\,\underline{u}\,$\oplus$\,tuiss & \textsc{ pl ela } \\
\underline{jõ}\,$\oplus$\,\underline{u}\,$\oplus$\,tuillõ & \textsc{ pl all } \\
\underline{jõ}\,$\oplus$\,\underline{u}\,$\oplus$\,tuill & \textsc{ pl ade } \\
\underline{jõ}\,$\oplus$\,\underline{u}\,$\oplus$\,tuilt & \textsc{ pl abl } \\
\underline{jõ}\,$\oplus$\,\underline{u}\,$\oplus$\,tuissi & \textsc{ pl tra } \\
\underline{jõ}\,$\oplus$\,\underline{u}\,$\oplus$\,tuissaa & \textsc{ pl ter } \\
\underline{jõ}\,$\oplus$\,\underline{u}\,$\oplus$\,tuika & \textsc{ pl com } \\
\end{tabular}
\end{sideways}
\captionof{table}{Tüüpsõnamall \arabic{mallinumber}\,\vadja{jõutu} ekstraheeritud muutvormimallid.}
\label{tab:tüüpsõnamall-jõutu}

\end{minipage}

 
\vspace{1em}
\noindent Tüüpsõnamall \vadja{jõutu} ei hõlma teisi lekseeme vormi\-sõnastikus.

Tüüp\-sõna\-mall kirjeldab tagapoolseid sõnu tüvemuutusega ut:vv, mille lõpuvokaal on \textit{u}.


\vspace{3.5em}
\noindent \begin{minipage}{\textwidth}
\stepcounter{mallinumber}
\noindent \textbf{Tüüpsõnamall \arabic{mallinumber}\,\vadja{poutõ}}\\

\begin{sideways}
\begin{tabular}{l l}
muutvormimall & tunnused \\
\hline
\underline{po}\,$\oplus$\,utõ & \textsc{ sg nom } \\
\underline{po}\,$\oplus$\,vva & \textsc{ sg gen } \\
\underline{po}\,$\oplus$\,uta & \textsc{ sg par } \\
\underline{po}\,$\oplus$\,utasõ & \textsc{ sg ill } \\
\underline{po}\,$\oplus$\,vvõz & \textsc{ sg ine } \\
\underline{po}\,$\oplus$\,vvõss & \textsc{ sg ela } \\
\underline{po}\,$\oplus$\,vvõllõ & \textsc{ sg all } \\
\underline{po}\,$\oplus$\,vvõll & \textsc{ sg ade } \\
\underline{po}\,$\oplus$\,vvõlt & \textsc{ sg abl } \\
\underline{po}\,$\oplus$\,vvõssi & \textsc{ sg tra } \\
\underline{po}\,$\oplus$\,vvõssaa & \textsc{ sg ter } \\
\underline{po}\,$\oplus$\,vvõka & \textsc{ sg com } \\
\underline{po}\,$\oplus$\,vvõd & \textsc{ pl nom } \\
\underline{po}\,$\oplus$\,utojõ & \textsc{ pl gen } \\
\underline{po}\,$\oplus$\,utoit & \textsc{ pl par } \\
\underline{po}\,$\oplus$\,utoisõ & \textsc{ pl ill } \\
\underline{po}\,$\oplus$\,utoiz & \textsc{ pl ine } \\
\underline{po}\,$\oplus$\,utoiss & \textsc{ pl ela } \\
\underline{po}\,$\oplus$\,utoillõ & \textsc{ pl all } \\
\underline{po}\,$\oplus$\,utoill & \textsc{ pl ade } \\
\underline{po}\,$\oplus$\,utoilt & \textsc{ pl abl } \\
\underline{po}\,$\oplus$\,utoissi & \textsc{ pl tra } \\
\underline{po}\,$\oplus$\,utoissaa & \textsc{ pl ter } \\
\underline{po}\,$\oplus$\,utoika & \textsc{ pl com } \\
\end{tabular}
\end{sideways}
\captionof{table}{Tüüpsõnamall \arabic{mallinumber}\,\vadja{poutõ} ekstraheeritud muutvormimallid.}
\label{tab:tüüpsõnamall-poutõ}

\end{minipage}

 
\vspace{1em}
\noindent Tüüpsõnamall \vadja{poutõ} hõlmab vormisõnastikus 2 lekseemi: \vadja{\underline{po}utõ} ja \vadja{\underline{la}utõ}.

Tüüp\-sõna\-mall kirjeldab tagapoolseid sõnu tüvemuutusega ut:vv, mille lõpuvokaal on \textit{õ}.


\vspace{3.5em}
\noindent \begin{minipage}{\textwidth}
\stepcounter{mallinumber}
\noindent \textbf{Tüüpsõnamall \arabic{mallinumber}\,\vadja{štanad}}\\

\begin{sideways}
\begin{tabular}{l l}
muutvormimall & tunnused \\
\hline
\underline{štan}\,$\oplus$\,ad & \textsc{ sg nom } \\
\underline{štan}\,$\oplus$\,ojõ & \textsc{ sg gen } \\
\underline{štan}\,$\oplus$\,oit & \textsc{ sg par } \\
\underline{štan}\,$\oplus$\,oisõ & \textsc{ sg ill } \\
\underline{štan}\,$\oplus$\,az & \textsc{ sg ine } \\
\underline{štan}\,$\oplus$\,ass & \textsc{ sg ela } \\
\underline{štan}\,$\oplus$\,allõ & \textsc{ sg all } \\
\underline{štan}\,$\oplus$\,all & \textsc{ sg ade } \\
\underline{štan}\,$\oplus$\,alt & \textsc{ sg abl } \\
\underline{štan}\,$\oplus$\,assi & \textsc{ sg tra } \\
\underline{štan}\,$\oplus$\,assaa & \textsc{ sg ter } \\
\underline{štan}\,$\oplus$\,aka & \textsc{ sg com } \\
\underline{štan}\,$\oplus$\,ad & \textsc{ pl nom } \\
\underline{štan}\,$\oplus$\,ojõ & \textsc{ pl gen } \\
\underline{štan}\,$\oplus$\,oit & \textsc{ pl par } \\
\underline{štan}\,$\oplus$\,oisõ & \textsc{ pl ill } \\
\underline{štan}\,$\oplus$\,oiz & \textsc{ pl ine } \\
\underline{štan}\,$\oplus$\,oiss & \textsc{ pl ela } \\
\underline{štan}\,$\oplus$\,oillõ & \textsc{ pl all } \\
\underline{štan}\,$\oplus$\,oill & \textsc{ pl ade } \\
\underline{štan}\,$\oplus$\,oilt & \textsc{ pl abl } \\
\underline{štan}\,$\oplus$\,oissi & \textsc{ pl tra } \\
\underline{štan}\,$\oplus$\,oissaa & \textsc{ pl ter } \\
\underline{štan}\,$\oplus$\,oika & \textsc{ pl com } \\
\end{tabular}
\end{sideways}
\captionof{table}{Tüüpsõnamall \arabic{mallinumber}\,\vadja{štanad} ekstraheeritud muutvormimallid.}
\label{tab:tüüpsõnamall-štanad}

\end{minipage}

 
\vspace{1em}
\noindent Tüüpsõnamall \vadja{štanad} ei hõlma teisi lekseeme vormi\-sõnastikus.

Tüüp\-sõna\-mall kirjeldab tagapoolset tüvemuutuseta \textit{plurale tantum} sõna.
\spacing{1.5}


\subsection{\RN{4} käändkond}

Neljandasse käändkonda kuuluvad mitmed sõnad, mis on ainsuses eespoolse vokalismiga, ent mitmuses on tagapoolsed (\cite[43]{ariste_grammar_1968}). Selliseid sõnu Heinsoo loodavas kirjakeeles ei esine (isiklik kommunikatsioon). % TODO: küsi ja kuidas viidata?


\subsection{\RN{5} käändkond}

Viiendasse käändkonda kuuluvad Ariste (\citeyear[44]{ariste_grammar_1968}) järgi kahe\-silbilised sõnad, mille tüve\-vokaal on \vadja{-a} ja millel esineb esimeses silbis \vadja{-o-}, \vadja{-u-} või \vadja{-õ-}.

Ariste (\citeyear[44]{ariste_grammar_1968}) järgi kuuluvad siia käändkonda enamik nendest sõnadest, mille esimene silp sisaldab \textit{-õ-}d, kuigi need võiksid kuuluda ka \RN{3} käändkonda.

%% Avatuid küsimusi-tähelepanekuid:
%% \begin{itemize}
%% \item 5 käändkonna liikmed Aristel -õi- on suuresti muudetud -ii-
%% \item 'mussõ' leiti mitu pl 'mussii' VKSi näitelausete hulgast
%% \end{itemize}


\subsubsection*{Ekstraktmorfoloogia tüüpsõnad}
\spacing{1.1}
\vspace{-3.5em}


\vspace{3.5em}
\noindent \begin{minipage}{\textwidth}
\stepcounter{mallinumber}
\noindent \textbf{Tüüpsõnamall \arabic{mallinumber}\,\vadja{kuha}}\\

\begin{sideways}
\begin{tabular}{l l}
muutvormimall & tunnused \\
\hline
\underline{kuh}\,$\oplus$\,a & \textsc{ sg nom } \\
\underline{kuh}\,$\oplus$\,a & \textsc{ sg gen } \\
\underline{kuh}\,$\oplus$\,ha & \textsc{ sg par } \\
\underline{kuh}\,$\oplus$\,hasõ & \textsc{ sg ill } \\
\underline{kuh}\,$\oplus$\,õz & \textsc{ sg ine } \\
\underline{kuh}\,$\oplus$\,õss & \textsc{ sg ela } \\
\underline{kuh}\,$\oplus$\,õllõ & \textsc{ sg all } \\
\underline{kuh}\,$\oplus$\,õll & \textsc{ sg ade } \\
\underline{kuh}\,$\oplus$\,õlt & \textsc{ sg abl } \\
\underline{kuh}\,$\oplus$\,õssi & \textsc{ sg tra } \\
\underline{kuh}\,$\oplus$\,õssaa & \textsc{ sg ter } \\
\underline{kuh}\,$\oplus$\,õka & \textsc{ sg com } \\
\underline{kuh}\,$\oplus$\,õd & \textsc{ pl nom } \\
\underline{kuh}\,$\oplus$\,hijõ & \textsc{ pl gen } \\
\underline{kuh}\,$\oplus$\,hiit & \textsc{ pl par } \\
\underline{kuh}\,$\oplus$\,hiisõ & \textsc{ pl ill } \\
\underline{kuh}\,$\oplus$\,hiiz & \textsc{ pl ine } \\
\underline{kuh}\,$\oplus$\,hiiss & \textsc{ pl ela } \\
\underline{kuh}\,$\oplus$\,hiillõ & \textsc{ pl all } \\
\underline{kuh}\,$\oplus$\,hiill & \textsc{ pl ade } \\
\underline{kuh}\,$\oplus$\,hiilt & \textsc{ pl abl } \\
\underline{kuh}\,$\oplus$\,hiissi & \textsc{ pl tra } \\
\underline{kuh}\,$\oplus$\,hiissaa & \textsc{ pl ter } \\
\underline{kuh}\,$\oplus$\,hijka & \textsc{ pl com } \\
\end{tabular}
\end{sideways}
\captionof{table}{Tüüpsõnamall \arabic{mallinumber}\,\vadja{kuha} ekstraheeritud muutvormimallid.}
\label{tab:tüüpsõnamall-kuha}

\end{minipage}

 
\vspace{1em}
\noindent Tüüpsõnamall \vadja{kuha} ei hõlma teisi lekseeme vormi\-sõnastikus.

Tüüp\-sõna\-mall kirjeldab tagapoolset tüvemuutuseta sõna.


\vspace{3.5em}
\noindent \begin{minipage}{\textwidth}
\stepcounter{mallinumber}
\noindent \textbf{Tüüpsõnamall \arabic{mallinumber}\,\vadja{poikõ}}\\

\begin{sideways}
\begin{tabular}{l l}
muutvormimall & tunnused \\
\hline
\underline{poi}\,$\oplus$\,kõ & \textsc{ sg nom } \\
\underline{poi}\,$\oplus$\,ga & \textsc{ sg gen } \\
\underline{poi}\,$\oplus$\,ka & \textsc{ sg par } \\
\underline{poi}\,$\oplus$\,kasõ & \textsc{ sg ill } \\
\underline{poi}\,$\oplus$\,gõz & \textsc{ sg ine } \\
\underline{poi}\,$\oplus$\,gõss & \textsc{ sg ela } \\
\underline{poi}\,$\oplus$\,gõllõ & \textsc{ sg all } \\
\underline{poi}\,$\oplus$\,gõll & \textsc{ sg ade } \\
\underline{poi}\,$\oplus$\,gõlt & \textsc{ sg abl } \\
\underline{poi}\,$\oplus$\,gõssi & \textsc{ sg tra } \\
\underline{poi}\,$\oplus$\,gõssaa & \textsc{ sg ter } \\
\underline{poi}\,$\oplus$\,gõka & \textsc{ sg com } \\
\underline{poi}\,$\oplus$\,gõd & \textsc{ pl nom } \\
\underline{poi}\,$\oplus$\,kijõ & \textsc{ pl gen } \\
\underline{poi}\,$\oplus$\,kiit & \textsc{ pl par } \\
\underline{poi}\,$\oplus$\,kiisõ & \textsc{ pl ill } \\
\underline{poi}\,$\oplus$\,kiiz & \textsc{ pl ine } \\
\underline{poi}\,$\oplus$\,kiiss & \textsc{ pl ela } \\
\underline{poi}\,$\oplus$\,kiillõ & \textsc{ pl all } \\
\underline{poi}\,$\oplus$\,kiill & \textsc{ pl ade } \\
\underline{poi}\,$\oplus$\,kiilt & \textsc{ pl abl } \\
\underline{poi}\,$\oplus$\,kiissi & \textsc{ pl tra } \\
\underline{poi}\,$\oplus$\,kiissaa & \textsc{ pl ter } \\
\underline{poi}\,$\oplus$\,kijka & \textsc{ pl com } \\
\end{tabular}
\end{sideways}
\captionof{table}{Tüüpsõnamall \arabic{mallinumber}\,\vadja{poikõ} ekstraheeritud muutvormimallid.}
\label{tab:tüüpsõnamall-poikõ}

\end{minipage}

 
\vspace{1em}
\noindent Tüüpsõnamall \vadja{poikõ} hõlmab vormisõnastikus 3 lekseemi: \vadja{\underline{poi}kõ, \underline{roo}kõ} ja \vadja{\underline{lõu}kõ}.

Tüüp\-sõna\-mall kirjeldab tagapoolseid sõnu tüvemuutusega k:g.


\vspace{3.5em}
\noindent \begin{minipage}{\textwidth}
\stepcounter{mallinumber}
\noindent \textbf{Tüüpsõnamall \arabic{mallinumber}\,\vadja{biblioteekkõ}}\\

\begin{sideways}
\begin{tabular}{l l}
muutvormimall & tunnused \\
\hline
\underline{biblioteek}\,$\oplus$\,kõ & \textsc{ sg nom } \\
\underline{biblioteek}\,$\oplus$\,a & \textsc{ sg gen } \\
\underline{biblioteek}\,$\oplus$\,ka & \textsc{ sg par } \\
\underline{biblioteek}\,$\oplus$\,kasõ & \textsc{ sg ill } \\
\underline{biblioteek}\,$\oplus$\,kõz & \textsc{ sg ine } \\
\underline{biblioteek}\,$\oplus$\,õss & \textsc{ sg ela } \\
\underline{biblioteek}\,$\oplus$\,õllõ & \textsc{ sg all } \\
\underline{biblioteek}\,$\oplus$\,õll & \textsc{ sg ade } \\
\underline{biblioteek}\,$\oplus$\,õlt & \textsc{ sg abl } \\
\underline{biblioteek}\,$\oplus$\,õssi & \textsc{ sg tra } \\
\underline{biblioteek}\,$\oplus$\,kõssaa & \textsc{ sg ter } \\
\underline{biblioteek}\,$\oplus$\,õka & \textsc{ sg com } \\
\underline{biblioteek}\,$\oplus$\,õd & \textsc{ pl nom } \\
\underline{biblioteek}\,$\oplus$\,kijõ & \textsc{ pl gen } \\
\underline{biblioteek}\,$\oplus$\,kiit & \textsc{ pl par } \\
\underline{biblioteek}\,$\oplus$\,kiisõ & \textsc{ pl ill } \\
\underline{biblioteek}\,$\oplus$\,kiiz & \textsc{ pl ine } \\
\underline{biblioteek}\,$\oplus$\,kiiss & \textsc{ pl ela } \\
\underline{biblioteek}\,$\oplus$\,kiillõ & \textsc{ pl all } \\
\underline{biblioteek}\,$\oplus$\,kiill & \textsc{ pl ade } \\
\underline{biblioteek}\,$\oplus$\,kiilt & \textsc{ pl abl } \\
\underline{biblioteek}\,$\oplus$\,kiissi & \textsc{ pl tra } \\
\underline{biblioteek}\,$\oplus$\,kiissaa & \textsc{ pl ter } \\
\underline{biblioteek}\,$\oplus$\,kijka & \textsc{ pl com } \\
\end{tabular}
\end{sideways}
\captionof{table}{Tüüpsõnamall \arabic{mallinumber}\,\vadja{biblioteekkõ} ekstraheeritud muutvormimallid.}
\label{tab:tüüpsõnamall-biblioteekkõ}

\end{minipage}

 
\vspace{1em}
\noindent Tüüpsõnamall \vadja{biblioteekkõ} hõlmab vormisõnastikus 10 lekseemi: \vadja{\underline{biblioteek}kõ, \underline{hoik}kõ, \underline{ikolook}kõ, \underline{jaanikuk}kõ, \underline{kolk}kõ, \underline{kon̕jõk}kõ, \underline{kuk}kõ, \underline{rok}kõ, \underline{suk}kõ} ja \vadja{\underline{bambuk}kõ}.

Tüüp\-sõna\-mall kirjeldab tagapoolseid sõnu tüvemuutusega kk:k.


\vspace{3.5em}
\noindent \begin{minipage}{\textwidth}
\stepcounter{mallinumber}
\noindent \textbf{Tüüpsõnamall \arabic{mallinumber}\,\vadja{kõik}}\\

\begin{sideways}
\begin{tabular}{l l}
muutvormimall & tunnused \\
\hline
\underline{kõik} & \textsc{ sg nom } \\
\underline{kõik}\,$\oplus$\,a & \textsc{ sg gen } \\
\underline{kõik}\,$\oplus$\,ka & \textsc{ sg par } \\
\underline{kõik}\,$\oplus$\,kasõ & \textsc{ sg ill } \\
\underline{kõik}\,$\oplus$\,õz & \textsc{ sg ine } \\
\underline{kõik}\,$\oplus$\,õss & \textsc{ sg ela } \\
\underline{kõik}\,$\oplus$\,õllõ & \textsc{ sg all } \\
\underline{kõik}\,$\oplus$\,õll & \textsc{ sg ade } \\
\underline{kõik}\,$\oplus$\,õlt & \textsc{ sg abl } \\
\underline{kõik}\,$\oplus$\,õssi & \textsc{ sg tra } \\
\underline{kõik}\,$\oplus$\,õssaa & \textsc{ sg ter } \\
\underline{kõik}\,$\oplus$\,õka & \textsc{ sg com } \\
\underline{kõik}\,$\oplus$\,õd & \textsc{ pl nom } \\
\underline{kõik}\,$\oplus$\,kijõ & \textsc{ pl gen } \\
\underline{kõik}\,$\oplus$\,kiit & \textsc{ pl par } \\
\underline{kõik}\,$\oplus$\,kiisõ & \textsc{ pl ill } \\
\underline{kõik}\,$\oplus$\,kiiz & \textsc{ pl ine } \\
\underline{kõik}\,$\oplus$\,kiiss & \textsc{ pl ela } \\
\underline{kõik}\,$\oplus$\,kiillõ & \textsc{ pl all } \\
\underline{kõik}\,$\oplus$\,kiill & \textsc{ pl ade } \\
\underline{kõik}\,$\oplus$\,kiilt & \textsc{ pl abl } \\
\underline{kõik}\,$\oplus$\,kiissi & \textsc{ pl tra } \\
\underline{kõik}\,$\oplus$\,kiissaa & \textsc{ pl ter } \\
\underline{kõik}\,$\oplus$\,kijka & \textsc{ pl com } \\
\end{tabular}
\end{sideways}
\captionof{table}{Tüüpsõnamall \arabic{mallinumber}\,\vadja{kõik} ekstraheeritud muutvormimallid.}
\label{tab:tüüpsõnamall-kõik}

\end{minipage}

 
\vspace{1em}
\noindent Tüüpsõnamall \vadja{kõik} ei hõlma teisi lekseeme vormi\-sõnastikus.

Tüüp\-sõna\-mall kirjeldab tagapoolseid sõnu tüvemuutusega kk:k, millel puudub lõpuvokaal.


\vspace{3.5em}
\noindent \begin{minipage}{\textwidth}
\stepcounter{mallinumber}
\noindent \textbf{Tüüpsõnamall \arabic{mallinumber}\,\vadja{kultõ}}\\

\begin{sideways}
\begin{tabular}{l l}
muutvormimall & tunnused \\
\hline
\underline{kul}\,$\oplus$\,tõ & \textsc{ sg nom } \\
\underline{kul}\,$\oplus$\,la & \textsc{ sg gen } \\
\underline{kul}\,$\oplus$\,ta & \textsc{ sg par } \\
\underline{kul}\,$\oplus$\,tasõ & \textsc{ sg ill } \\
\underline{kul}\,$\oplus$\,lõz & \textsc{ sg ine } \\
\underline{kul}\,$\oplus$\,lõss & \textsc{ sg ela } \\
\underline{kul}\,$\oplus$\,lõllõ & \textsc{ sg all } \\
\underline{kul}\,$\oplus$\,lõll & \textsc{ sg ade } \\
\underline{kul}\,$\oplus$\,lõlt & \textsc{ sg abl } \\
\underline{kul}\,$\oplus$\,lõssi & \textsc{ sg tra } \\
\underline{kul}\,$\oplus$\,lõssaa & \textsc{ sg ter } \\
\underline{kul}\,$\oplus$\,lõka & \textsc{ sg com } \\
\underline{kul}\,$\oplus$\,lõd & \textsc{ pl nom } \\
\underline{kul}\,$\oplus$\,tõjõ & \textsc{ pl gen } \\
\underline{kul}\,$\oplus$\,tõit & \textsc{ pl par } \\
\underline{kul}\,$\oplus$\,tõisõ & \textsc{ pl ill } \\
\underline{kul}\,$\oplus$\,tõiz & \textsc{ pl ine } \\
\underline{kul}\,$\oplus$\,tõiss & \textsc{ pl ela } \\
\underline{kul}\,$\oplus$\,tõillõ & \textsc{ pl all } \\
\underline{kul}\,$\oplus$\,tõill & \textsc{ pl ade } \\
\underline{kul}\,$\oplus$\,tõilt & \textsc{ pl abl } \\
\underline{kul}\,$\oplus$\,tõissi & \textsc{ pl tra } \\
\underline{kul}\,$\oplus$\,tõissaa & \textsc{ pl ter } \\
\underline{kul}\,$\oplus$\,tõika & \textsc{ pl com } \\
\end{tabular}
\end{sideways}
\captionof{table}{Tüüpsõnamall \arabic{mallinumber}\,\vadja{kultõ} ekstraheeritud muutvormimallid.}
\label{tab:tüüpsõnamall-kultõ}

\end{minipage}

 
\vspace{1em}
\noindent Tüüpsõnamall \vadja{kultõ} ei hõlma teisi lekseeme vormi\-sõnastikus.

Tüüp\-sõna\-mall kirjeldab tagapoolseid sõnu tüvemuutusega lt:ll.


\vspace{3.5em}
\noindent \begin{minipage}{\textwidth}
\stepcounter{mallinumber}
\noindent \textbf{Tüüpsõnamall \arabic{mallinumber}\,\vadja{vimpõ}}\\

\begin{sideways}
\begin{tabular}{l l}
muutvormimall & tunnused \\
\hline
\underline{vim}\,$\oplus$\,põ & \textsc{ sg nom } \\
\underline{vim}\,$\oplus$\,ma & \textsc{ sg gen } \\
\underline{vim}\,$\oplus$\,pa & \textsc{ sg par } \\
\underline{vim}\,$\oplus$\,pasõ & \textsc{ sg ill } \\
\underline{vim}\,$\oplus$\,mõz & \textsc{ sg ine } \\
\underline{vim}\,$\oplus$\,mõss & \textsc{ sg ela } \\
\underline{vim}\,$\oplus$\,mõllõ & \textsc{ sg all } \\
\underline{vim}\,$\oplus$\,mõll & \textsc{ sg ade } \\
\underline{vim}\,$\oplus$\,mõlt & \textsc{ sg abl } \\
\underline{vim}\,$\oplus$\,mõssi & \textsc{ sg tra } \\
\underline{vim}\,$\oplus$\,mõssaa & \textsc{ sg ter } \\
\underline{vim}\,$\oplus$\,mõka & \textsc{ sg com } \\
\underline{vim}\,$\oplus$\,mõd & \textsc{ pl nom } \\
\underline{vim}\,$\oplus$\,pijõ & \textsc{ pl gen } \\
\underline{vim}\,$\oplus$\,piit & \textsc{ pl par } \\
\underline{vim}\,$\oplus$\,piisõ & \textsc{ pl ill } \\
\underline{vim}\,$\oplus$\,piiz & \textsc{ pl ine } \\
\underline{vim}\,$\oplus$\,piiss & \textsc{ pl ela } \\
\underline{vim}\,$\oplus$\,piillõ & \textsc{ pl all } \\
\underline{vim}\,$\oplus$\,piill & \textsc{ pl ade } \\
\underline{vim}\,$\oplus$\,piilt & \textsc{ pl abl } \\
\underline{vim}\,$\oplus$\,piissi & \textsc{ pl tra } \\
\underline{vim}\,$\oplus$\,piissaa & \textsc{ pl ter } \\
\underline{vim}\,$\oplus$\,pijka & \textsc{ pl com } \\
\end{tabular}
\end{sideways}
\captionof{table}{Tüüpsõnamall \arabic{mallinumber}\,\vadja{vimpõ} ekstraheeritud muutvormimallid.}
\label{tab:tüüpsõnamall-vimpõ}

\end{minipage}

 
\vspace{1em}
\noindent Tüüpsõnamall \vadja{vimpõ} hõlmab vormisõnastikus 2 lekseemi: \vadja{\underline{vim}põ} ja \vadja{\underline{kum}põ}.

Tüüp\-sõna\-mall kirjeldab tagapoolseid sõnu tüvemuutusega mp:mm.


\vspace{3.5em}
\noindent \begin{minipage}{\textwidth}
\stepcounter{mallinumber}
\noindent \textbf{Tüüpsõnamall \arabic{mallinumber}\,\vadja{kompjutera}}\\

\begin{sideways}
\begin{tabular}{l l}
muutvormimall & tunnused \\
\hline
\underline{kompjuter}\,$\oplus$\,a & \textsc{ sg nom } \\
\underline{kompjuter}\,$\oplus$\,a & \textsc{ sg gen } \\
\underline{kompjuter}\,$\oplus$\,a & \textsc{ sg par } \\
\underline{kompjuter}\,$\oplus$\,asõ & \textsc{ sg ill } \\
\underline{kompjuter}\,$\oplus$\,az & \textsc{ sg ine } \\
\underline{kompjuter}\,$\oplus$\,ass & \textsc{ sg ela } \\
\underline{kompjuter}\,$\oplus$\,allõ & \textsc{ sg all } \\
\underline{kompjuter}\,$\oplus$\,all & \textsc{ sg ade } \\
\underline{kompjuter}\,$\oplus$\,alt & \textsc{ sg abl } \\
\underline{kompjuter}\,$\oplus$\,assi & \textsc{ sg tra } \\
\underline{kompjuter}\,$\oplus$\,assaa & \textsc{ sg ter } \\
\underline{kompjuter}\,$\oplus$\,aka & \textsc{ sg com } \\
\underline{kompjuter}\,$\oplus$\,ad & \textsc{ pl nom } \\
\underline{kompjuter}\,$\oplus$\,ijõ & \textsc{ pl gen } \\
\underline{kompjuter}\,$\oplus$\,iit & \textsc{ pl par } \\
\underline{kompjuter}\,$\oplus$\,iisõ & \textsc{ pl ill } \\
\underline{kompjuter}\,$\oplus$\,iiz & \textsc{ pl ine } \\
\underline{kompjuter}\,$\oplus$\,iiss & \textsc{ pl ela } \\
\underline{kompjuter}\,$\oplus$\,iillõ & \textsc{ pl all } \\
\underline{kompjuter}\,$\oplus$\,iill & \textsc{ pl ade } \\
\underline{kompjuter}\,$\oplus$\,iilt & \textsc{ pl abl } \\
\underline{kompjuter}\,$\oplus$\,iissi & \textsc{ pl tra } \\
\underline{kompjuter}\,$\oplus$\,iissaa & \textsc{ pl ter } \\
\underline{kompjuter}\,$\oplus$\,ijka & \textsc{ pl com } \\
\end{tabular}
\end{sideways}
\captionof{table}{Tüüpsõnamall \arabic{mallinumber}\,\vadja{kompjutera} ekstraheeritud muutvormimallid.}
\label{tab:tüüpsõnamall-kompjutera}

\end{minipage}

 
\vspace{1em}
\noindent Tüüpsõnamall \vadja{kompjutera} hõlmab vormisõnastikus 6 lekseemi: \vadja{\underline{kompjuter}a, \underline{kuj}a, \underline{lob}a, \underline{õmpõj}a, \underline{õpõttõj}a} ja \vadja{\underline{ižor}a}.

Tüüp\-sõna\-mall kirjeldab tagapoolseid tüvemuutuseta sõnu, mille lõpuvokaal on \textit{a}.


\vspace{3.5em}
\noindent \begin{minipage}{\textwidth}
\stepcounter{mallinumber}
\noindent \textbf{Tüüpsõnamall \arabic{mallinumber}\,\vadja{maamuna}}\\

\begin{sideways}
\begin{tabular}{l l}
muutvormimall & tunnused \\
\hline
\underline{maamun}\,$\oplus$\,a & \textsc{ sg nom } \\
\underline{maamun}\,$\oplus$\,a & \textsc{ sg gen } \\
\underline{maamun}\,$\oplus$\,a & \textsc{ sg par } \\
\underline{maamun}\,$\oplus$\,asõ & \textsc{ sg ill } \\
\underline{maamun}\,$\oplus$\,az & \textsc{ sg ine } \\
\underline{maamun}\,$\oplus$\,ass & \textsc{ sg ela } \\
\underline{maamun}\,$\oplus$\,allõ & \textsc{ sg all } \\
\underline{maamun}\,$\oplus$\,all & \textsc{ sg ade } \\
\underline{maamun}\,$\oplus$\,alt & \textsc{ sg abl } \\
\underline{maamun}\,$\oplus$\,assi & \textsc{ sg tra } \\
\underline{maamun}\,$\oplus$\,assaa & \textsc{ sg ter } \\
\underline{maamun}\,$\oplus$\,aka & \textsc{ sg com } \\
\underline{maamun}\,$\oplus$\,ad & \textsc{ pl nom } \\
\underline{maamun}\,$\oplus$\,õjõ & \textsc{ pl gen } \\
\underline{maamun}\,$\oplus$\,õit & \textsc{ pl par } \\
\underline{maamun}\,$\oplus$\,õisõ & \textsc{ pl ill } \\
\underline{maamun}\,$\oplus$\,õiz & \textsc{ pl ine } \\
\underline{maamun}\,$\oplus$\,õiss & \textsc{ pl ela } \\
\underline{maamun}\,$\oplus$\,õillõ & \textsc{ pl all } \\
\underline{maamun}\,$\oplus$\,õill & \textsc{ pl ade } \\
\underline{maamun}\,$\oplus$\,õilt & \textsc{ pl abl } \\
\underline{maamun}\,$\oplus$\,õissi & \textsc{ pl tra } \\
\underline{maamun}\,$\oplus$\,õissaa & \textsc{ pl ter } \\
\underline{maamun}\,$\oplus$\,õika & \textsc{ pl com } \\
\end{tabular}
\end{sideways}
\captionof{table}{Tüüpsõnamall \arabic{mallinumber}\,\vadja{maamuna} ekstraheeritud muutvormimallid.}
\label{tab:tüüpsõnamall-maamuna}

\end{minipage}

 
\vspace{1em}
\noindent Tüüpsõnamall \vadja{maamuna} hõlmab vormisõnastikus 3 lekseemi: \vadja{\underline{maamun}a, \underline{mun}a} ja \vadja{\underline{kal̕indor}a}.

Tüüp\-sõna\-mall kirjeldab tagapoolseid tüvemuutuseta sõnu, mille mitmuse tüvi on -\textit{õi}-.


\vspace{3.5em}
\noindent \begin{minipage}{\textwidth}
\stepcounter{mallinumber}
\noindent \textbf{Tüüpsõnamall \arabic{mallinumber}\,\vadja{mokom}}\\

\begin{sideways}
\begin{tabular}{l l}
muutvormimall & tunnused \\
\hline
\underline{mokom} & \textsc{ sg nom } \\
\underline{mokom}\,$\oplus$\,a & \textsc{ sg gen } \\
\underline{mokom}\,$\oplus$\,a & \textsc{ sg par } \\
\underline{mokom}\,$\oplus$\,asõ & \textsc{ sg ill } \\
\underline{mokom}\,$\oplus$\,õz & \textsc{ sg ine } \\
\underline{mokom}\,$\oplus$\,õss & \textsc{ sg ela } \\
\underline{mokom}\,$\oplus$\,õllõ & \textsc{ sg all } \\
\underline{mokom}\,$\oplus$\,õll & \textsc{ sg ade } \\
\underline{mokom}\,$\oplus$\,õlt & \textsc{ sg abl } \\
\underline{mokom}\,$\oplus$\,õssi & \textsc{ sg tra } \\
\underline{mokom}\,$\oplus$\,õssaa & \textsc{ sg ter } \\
\underline{mokom}\,$\oplus$\,õka & \textsc{ sg com } \\
\underline{mokom}\,$\oplus$\,õd & \textsc{ pl nom } \\
\underline{mokom}\,$\oplus$\,ijõ & \textsc{ pl gen } \\
\underline{mokom}\,$\oplus$\,iit & \textsc{ pl par } \\
\underline{mokom}\,$\oplus$\,iisõ & \textsc{ pl ill } \\
\underline{mokom}\,$\oplus$\,iiz & \textsc{ pl ine } \\
\underline{mokom}\,$\oplus$\,iiss & \textsc{ pl ela } \\
\underline{mokom}\,$\oplus$\,iillõ & \textsc{ pl all } \\
\underline{mokom}\,$\oplus$\,iill & \textsc{ pl ade } \\
\underline{mokom}\,$\oplus$\,iilt & \textsc{ pl abl } \\
\underline{mokom}\,$\oplus$\,iissi & \textsc{ pl tra } \\
\underline{mokom}\,$\oplus$\,iissaa & \textsc{ pl ter } \\
\underline{mokom}\,$\oplus$\,ijka & \textsc{ pl com } \\
\end{tabular}
\end{sideways}
\captionof{table}{Tüüpsõnamall \arabic{mallinumber}\,\vadja{mokom} ekstraheeritud muutvormimallid.}
\label{tab:tüüpsõnamall-mokom}

\end{minipage}

 
\vspace{1em}
\noindent Tüüpsõnamall \vadja{mokom} ei hõlma teisi lekseeme vormi\-sõnastikus.

Tüüp\-sõna\-mall kirjeldab tagapoolseid tüvemuutuseta sõnu, millel puudub lõpuvokaal.


\vspace{3.5em}
\noindent \begin{minipage}{\textwidth}
\stepcounter{mallinumber}
\noindent \textbf{Tüüpsõnamall \arabic{mallinumber}\,\vadja{bukvõ}}\\

\begin{sideways}
\begin{tabular}{l l}
muutvormimall & tunnused \\
\hline
\underline{bukv}\,$\oplus$\,õ & \textsc{ sg nom } \\
\underline{bukv}\,$\oplus$\,a & \textsc{ sg gen } \\
\underline{bukv}\,$\oplus$\,a & \textsc{ sg par } \\
\underline{bukv}\,$\oplus$\,asõ & \textsc{ sg ill } \\
\underline{bukv}\,$\oplus$\,õz & \textsc{ sg ine } \\
\underline{bukv}\,$\oplus$\,õss & \textsc{ sg ela } \\
\underline{bukv}\,$\oplus$\,õllõ & \textsc{ sg all } \\
\underline{bukv}\,$\oplus$\,õll & \textsc{ sg ade } \\
\underline{bukv}\,$\oplus$\,õlt & \textsc{ sg abl } \\
\underline{bukv}\,$\oplus$\,õssi & \textsc{ sg tra } \\
\underline{bukv}\,$\oplus$\,õssaa & \textsc{ sg ter } \\
\underline{bukv}\,$\oplus$\,õka & \textsc{ sg com } \\
\underline{bukv}\,$\oplus$\,õd & \textsc{ pl nom } \\
\underline{bukv}\,$\oplus$\,ijõ & \textsc{ pl gen } \\
\underline{bukv}\,$\oplus$\,iit & \textsc{ pl par } \\
\underline{bukv}\,$\oplus$\,iisõ & \textsc{ pl ill } \\
\underline{bukv}\,$\oplus$\,iiz & \textsc{ pl ine } \\
\underline{bukv}\,$\oplus$\,iiss & \textsc{ pl ela } \\
\underline{bukv}\,$\oplus$\,iillõ & \textsc{ pl all } \\
\underline{bukv}\,$\oplus$\,iill & \textsc{ pl ade } \\
\underline{bukv}\,$\oplus$\,iilt & \textsc{ pl abl } \\
\underline{bukv}\,$\oplus$\,iissi & \textsc{ pl tra } \\
\underline{bukv}\,$\oplus$\,iissaa & \textsc{ pl ter } \\
\underline{bukv}\,$\oplus$\,ijka & \textsc{ pl com } \\
\end{tabular}
\end{sideways}
\captionof{table}{Tüüpsõnamall \arabic{mallinumber}\,\vadja{bukvõ} ekstraheeritud muutvormimallid.}
\label{tab:tüüpsõnamall-bukvõ}

\end{minipage}

 
\vspace{1em}
\noindent Tüüpsõnamall \vadja{bukvõ} hõlmab vormisõnastikus 28 lekseemi: \vadja{\underline{bukv}õ, \underline{duum}õ, \underline{form}õ, \underline{ilm}õ, \underline{koir}õ, \underline{konn}õ, \underline{kuhj}õ, \underline{kuiv}õ, \underline{kumm}õ, \underline{kuuluv}õ, \underline{kõrv}õ, \underline{mood}õ, \underline{muudr}õ, \underline{mõiz}õ, \underline{obraaz}õ, \underline{post}õ, \underline{programm}õ, \underline{rooj}õ, \underline{sool}õ, \underline{sveež}õ, \underline{trubačist}õ, \underline{tuim}õ, \underline{tuttav}õ, \underline{velosiped}õ, \underline{vohm}õ, \underline{vool}õ, \underline{võim}õ} ja \vadja{\underline{bomb}õ}.

Tüüp\-sõna\-mall kirjeldab tagapoolseid tüvemuutuseta sõnu, mille lõpuvokaal on \textit{õ}.


\vspace{3.5em}
\noindent \begin{minipage}{\textwidth}
\stepcounter{mallinumber}
\noindent \textbf{Tüüpsõnamall \arabic{mallinumber}\,\vadja{propkõ}}\\

\begin{sideways}
\begin{tabular}{l l}
muutvormimall & tunnused \\
\hline
\underline{pro}\,$\oplus$\,pkõ & \textsc{ sg nom } \\
\underline{pro}\,$\oplus$\,bga & \textsc{ sg gen } \\
\underline{pro}\,$\oplus$\,pka & \textsc{ sg par } \\
\underline{pro}\,$\oplus$\,pkasõ & \textsc{ sg ill } \\
\underline{pro}\,$\oplus$\,bgõz & \textsc{ sg ine } \\
\underline{pro}\,$\oplus$\,bgõss & \textsc{ sg ela } \\
\underline{pro}\,$\oplus$\,bgõllõ & \textsc{ sg all } \\
\underline{pro}\,$\oplus$\,bgõll & \textsc{ sg ade } \\
\underline{pro}\,$\oplus$\,bgõlt & \textsc{ sg abl } \\
\underline{pro}\,$\oplus$\,bgõssi & \textsc{ sg tra } \\
\underline{pro}\,$\oplus$\,bgõssaa & \textsc{ sg ter } \\
\underline{pro}\,$\oplus$\,bgõka & \textsc{ sg com } \\
\underline{pro}\,$\oplus$\,bgõd & \textsc{ pl nom } \\
\underline{pro}\,$\oplus$\,pkijõ & \textsc{ pl gen } \\
\underline{pro}\,$\oplus$\,pkiit & \textsc{ pl par } \\
\underline{pro}\,$\oplus$\,pkiisõ & \textsc{ pl ill } \\
\underline{pro}\,$\oplus$\,pkiiz & \textsc{ pl ine } \\
\underline{pro}\,$\oplus$\,pkiiss & \textsc{ pl ela } \\
\underline{pro}\,$\oplus$\,pkiillõ & \textsc{ pl all } \\
\underline{pro}\,$\oplus$\,pkiill & \textsc{ pl ade } \\
\underline{pro}\,$\oplus$\,pkiilt & \textsc{ pl abl } \\
\underline{pro}\,$\oplus$\,pkiissi & \textsc{ pl tra } \\
\underline{pro}\,$\oplus$\,pkiissaa & \textsc{ pl ter } \\
\underline{pro}\,$\oplus$\,pkijka & \textsc{ pl com } \\
\end{tabular}
\end{sideways}
\captionof{table}{Tüüpsõnamall \arabic{mallinumber}\,\vadja{propkõ} ekstraheeritud muutvormimallid.}
\label{tab:tüüpsõnamall-propkõ}

\end{minipage}

 
\vspace{1em}
\noindent Tüüpsõnamall \vadja{propkõ} hõlmab vormisõnastikus 3 lekseemi: \vadja{\underline{pro}pkõ, \underline{sko}pkõ} ja \vadja{\underline{ju}pkõ}.

Tüüp\-sõna\-mall kirjeldab tagapoolseid sõnu tüvemuutusega pk:bg.


\vspace{3.5em}
\noindent \begin{minipage}{\textwidth}
\stepcounter{mallinumber}
\noindent \textbf{Tüüpsõnamall \arabic{mallinumber}\,\vadja{rooppõ}}\\

\begin{sideways}
\begin{tabular}{l l}
muutvormimall & tunnused \\
\hline
\underline{roop}\,$\oplus$\,põ & \textsc{ sg nom } \\
\underline{roop}\,$\oplus$\,a & \textsc{ sg gen } \\
\underline{roop}\,$\oplus$\,pa & \textsc{ sg par } \\
\underline{roop}\,$\oplus$\,pasõ & \textsc{ sg ill } \\
\underline{roop}\,$\oplus$\,põz & \textsc{ sg ine } \\
\underline{roop}\,$\oplus$\,õss & \textsc{ sg ela } \\
\underline{roop}\,$\oplus$\,õllõ & \textsc{ sg all } \\
\underline{roop}\,$\oplus$\,õll & \textsc{ sg ade } \\
\underline{roop}\,$\oplus$\,õlt & \textsc{ sg abl } \\
\underline{roop}\,$\oplus$\,õssi & \textsc{ sg tra } \\
\underline{roop}\,$\oplus$\,põssaa & \textsc{ sg ter } \\
\underline{roop}\,$\oplus$\,õka & \textsc{ sg com } \\
\underline{roop}\,$\oplus$\,õd & \textsc{ pl nom } \\
\underline{roop}\,$\oplus$\,põjõ & \textsc{ pl gen } \\
\underline{roop}\,$\oplus$\,õit & \textsc{ pl par } \\
\underline{roop}\,$\oplus$\,õisõ & \textsc{ pl ill } \\
\underline{roop}\,$\oplus$\,õiz & \textsc{ pl ine } \\
\underline{roop}\,$\oplus$\,õiss & \textsc{ pl ela } \\
\underline{roop}\,$\oplus$\,õillõ & \textsc{ pl all } \\
\underline{roop}\,$\oplus$\,õill & \textsc{ pl ade } \\
\underline{roop}\,$\oplus$\,õilt & \textsc{ pl abl } \\
\underline{roop}\,$\oplus$\,õissi & \textsc{ pl tra } \\
\underline{roop}\,$\oplus$\,õissaa & \textsc{ pl ter } \\
\underline{roop}\,$\oplus$\,õika & \textsc{ pl com } \\
\end{tabular}
\end{sideways}
\captionof{table}{Tüüpsõnamall \arabic{mallinumber}\,\vadja{rooppõ} ekstraheeritud muutvormimallid.}
\label{tab:tüüpsõnamall-rooppõ}

\end{minipage}

 
\vspace{1em}
\noindent Tüüpsõnamall \vadja{rooppõ} ei hõlma teisi lekseeme vormi\-sõnastikus.

Tüüp\-sõna\-mall kirjeldab tagapoolseid sõnu tüvemuutusega pp:p.


\vspace{3.5em}
\noindent \begin{minipage}{\textwidth}
\stepcounter{mallinumber}
\noindent \textbf{Tüüpsõnamall \arabic{mallinumber}\,\vadja{kurp}}\\

\begin{sideways}
\begin{tabular}{l l}
muutvormimall & tunnused \\
\hline
\underline{kur}\,$\oplus$\,p & \textsc{ sg nom } \\
\underline{kur}\,$\oplus$\,va & \textsc{ sg gen } \\
\underline{kur}\,$\oplus$\,pa & \textsc{ sg par } \\
\underline{kur}\,$\oplus$\,pasõ & \textsc{ sg ill } \\
\underline{kur}\,$\oplus$\,võz & \textsc{ sg ine } \\
\underline{kur}\,$\oplus$\,võss & \textsc{ sg ela } \\
\underline{kur}\,$\oplus$\,võllõ & \textsc{ sg all } \\
\underline{kur}\,$\oplus$\,võll & \textsc{ sg ade } \\
\underline{kur}\,$\oplus$\,võlt & \textsc{ sg abl } \\
\underline{kur}\,$\oplus$\,võssi & \textsc{ sg tra } \\
\underline{kur}\,$\oplus$\,võssaa & \textsc{ sg ter } \\
\underline{kur}\,$\oplus$\,võka & \textsc{ sg com } \\
\underline{kur}\,$\oplus$\,võd & \textsc{ pl nom } \\
\underline{kur}\,$\oplus$\,pijõ & \textsc{ pl gen } \\
\underline{kur}\,$\oplus$\,piit & \textsc{ pl par } \\
\underline{kur}\,$\oplus$\,piisõ & \textsc{ pl ill } \\
\underline{kur}\,$\oplus$\,piiz & \textsc{ pl ine } \\
\underline{kur}\,$\oplus$\,piiss & \textsc{ pl ela } \\
\underline{kur}\,$\oplus$\,piillõ & \textsc{ pl all } \\
\underline{kur}\,$\oplus$\,piill & \textsc{ pl ade } \\
\underline{kur}\,$\oplus$\,piilt & \textsc{ pl abl } \\
\underline{kur}\,$\oplus$\,piissi & \textsc{ pl tra } \\
\underline{kur}\,$\oplus$\,piissaa & \textsc{ pl ter } \\
\underline{kur}\,$\oplus$\,pijka & \textsc{ pl com } \\
\end{tabular}
\end{sideways}
\captionof{table}{Tüüpsõnamall \arabic{mallinumber}\,\vadja{kurp} ekstraheeritud muutvormimallid.}
\label{tab:tüüpsõnamall-kurp}

\end{minipage}

 
\vspace{1em}
\noindent Tüüpsõnamall \vadja{kurp} ei hõlma teisi lekseeme vormi\-sõnastikus.

Tüüp\-sõna\-mall kirjeldab tagapoolseid sõnu tüvemuutusega rp:rv.


\vspace{3.5em}
\noindent \begin{minipage}{\textwidth}
\stepcounter{mallinumber}
\noindent \textbf{Tüüpsõnamall \arabic{mallinumber}\,\vadja{usa}}\\

\begin{sideways}
\begin{tabular}{l l}
muutvormimall & tunnused \\
\hline
\underline{u}\,$\oplus$\,sa & \textsc{ sg nom } \\
\underline{u}\,$\oplus$\,za & \textsc{ sg gen } \\
\underline{u}\,$\oplus$\,ssa & \textsc{ sg par } \\
\underline{u}\,$\oplus$\,ssasõ & \textsc{ sg ill } \\
\underline{u}\,$\oplus$\,zaz & \textsc{ sg ine } \\
\underline{u}\,$\oplus$\,zass & \textsc{ sg ela } \\
\underline{u}\,$\oplus$\,zallõ & \textsc{ sg all } \\
\underline{u}\,$\oplus$\,zall & \textsc{ sg ade } \\
\underline{u}\,$\oplus$\,zalt & \textsc{ sg abl } \\
\underline{u}\,$\oplus$\,zassi & \textsc{ sg tra } \\
\underline{u}\,$\oplus$\,zassaa & \textsc{ sg ter } \\
\underline{u}\,$\oplus$\,zaka & \textsc{ sg com } \\
\underline{u}\,$\oplus$\,zad & \textsc{ pl nom } \\
\underline{u}\,$\oplus$\,sijõ & \textsc{ pl gen } \\
\underline{u}\,$\oplus$\,siit & \textsc{ pl par } \\
\underline{u}\,$\oplus$\,siisõ & \textsc{ pl ill } \\
\underline{u}\,$\oplus$\,siiz & \textsc{ pl ine } \\
\underline{u}\,$\oplus$\,siiss & \textsc{ pl ela } \\
\underline{u}\,$\oplus$\,siillõ & \textsc{ pl all } \\
\underline{u}\,$\oplus$\,siill & \textsc{ pl ade } \\
\underline{u}\,$\oplus$\,siilt & \textsc{ pl abl } \\
\underline{u}\,$\oplus$\,siissi & \textsc{ pl tra } \\
\underline{u}\,$\oplus$\,siissaa & \textsc{ pl ter } \\
\underline{u}\,$\oplus$\,sijka & \textsc{ pl com } \\
\end{tabular}
\end{sideways}
\captionof{table}{Tüüpsõnamall \arabic{mallinumber}\,\vadja{usa} ekstraheeritud muutvormimallid.}
\label{tab:tüüpsõnamall-usa}

\end{minipage}

 
\vspace{1em}
\noindent Tüüpsõnamall \vadja{usa} ei hõlma teisi lekseeme vormi\-sõnastikus.

Tüüp\-sõna\-mall kirjeldab tagapoolseid sõnu tüvemuutusega s:z, mille lõpuvokaal on \textit{a}.


\vspace{3.5em}
\noindent \begin{minipage}{\textwidth}
\stepcounter{mallinumber}
\noindent \textbf{Tüüpsõnamall \arabic{mallinumber}\,\vadja{õhsõ}}\\

\begin{sideways}
\begin{tabular}{l l}
muutvormimall & tunnused \\
\hline
\underline{õh}\,$\oplus$\,sõ & \textsc{ sg nom } \\
\underline{õh}\,$\oplus$\,za & \textsc{ sg gen } \\
\underline{õh}\,$\oplus$\,sa & \textsc{ sg par } \\
\underline{õh}\,$\oplus$\,sasõ & \textsc{ sg ill } \\
\underline{õh}\,$\oplus$\,zõz & \textsc{ sg ine } \\
\underline{õh}\,$\oplus$\,zõss & \textsc{ sg ela } \\
\underline{õh}\,$\oplus$\,zõllõ & \textsc{ sg all } \\
\underline{õh}\,$\oplus$\,zõll & \textsc{ sg ade } \\
\underline{õh}\,$\oplus$\,zõlt & \textsc{ sg abl } \\
\underline{õh}\,$\oplus$\,zõssi & \textsc{ sg tra } \\
\underline{õh}\,$\oplus$\,sõssaa & \textsc{ sg ter } \\
\underline{õh}\,$\oplus$\,zõka & \textsc{ sg com } \\
\underline{õh}\,$\oplus$\,zõd & \textsc{ pl nom } \\
\underline{õh}\,$\oplus$\,sijõ & \textsc{ pl gen } \\
\underline{õh}\,$\oplus$\,siit & \textsc{ pl par } \\
\underline{õh}\,$\oplus$\,siisõ & \textsc{ pl ill } \\
\underline{õh}\,$\oplus$\,siiz & \textsc{ pl ine } \\
\underline{õh}\,$\oplus$\,siiss & \textsc{ pl ela } \\
\underline{õh}\,$\oplus$\,siillõ & \textsc{ pl all } \\
\underline{õh}\,$\oplus$\,siill & \textsc{ pl ade } \\
\underline{õh}\,$\oplus$\,siilt & \textsc{ pl abl } \\
\underline{õh}\,$\oplus$\,siissi & \textsc{ pl tra } \\
\underline{õh}\,$\oplus$\,siissaa & \textsc{ pl ter } \\
\underline{õh}\,$\oplus$\,sijka & \textsc{ pl com } \\
\end{tabular}
\end{sideways}
\captionof{table}{Tüüpsõnamall \arabic{mallinumber}\,\vadja{õhsõ} ekstraheeritud muutvormimallid.}
\label{tab:tüüpsõnamall-õhsõ}

\end{minipage}

 
\vspace{1em}
\noindent Tüüpsõnamall \vadja{õhsõ} hõlmab vormisõnastikus 2 lekseemi: \vadja{\underline{õh}sõ} ja \vadja{\underline{sor}sõ}.

Tüüp\-sõna\-mall kirjeldab tagapoolseid sõnu tüvemuutusega s:z, mille lõpuvokaal on \textit{õ}.


\vspace{3.5em}
\noindent \begin{minipage}{\textwidth}
\stepcounter{mallinumber}
\noindent \textbf{Tüüpsõnamall \arabic{mallinumber}\,\vadja{luiskõ}}\\

\begin{sideways}
\begin{tabular}{l l}
muutvormimall & tunnused \\
\hline
\underline{lui}\,$\oplus$\,skõ & \textsc{ sg nom } \\
\underline{lui}\,$\oplus$\,zga & \textsc{ sg gen } \\
\underline{lui}\,$\oplus$\,ska & \textsc{ sg par } \\
\underline{lui}\,$\oplus$\,skasõ & \textsc{ sg ill } \\
\underline{lui}\,$\oplus$\,zgõz & \textsc{ sg ine } \\
\underline{lui}\,$\oplus$\,zgõss & \textsc{ sg ela } \\
\underline{lui}\,$\oplus$\,zgõllõ & \textsc{ sg all } \\
\underline{lui}\,$\oplus$\,zgõll & \textsc{ sg ade } \\
\underline{lui}\,$\oplus$\,zgõlt & \textsc{ sg abl } \\
\underline{lui}\,$\oplus$\,zgõssi & \textsc{ sg tra } \\
\underline{lui}\,$\oplus$\,zgõssaa & \textsc{ sg ter } \\
\underline{lui}\,$\oplus$\,zgõka & \textsc{ sg com } \\
\underline{lui}\,$\oplus$\,zgõd & \textsc{ pl nom } \\
\underline{lui}\,$\oplus$\,skijõ & \textsc{ pl gen } \\
\underline{lui}\,$\oplus$\,skiit & \textsc{ pl par } \\
\underline{lui}\,$\oplus$\,skiisõ & \textsc{ pl ill } \\
\underline{lui}\,$\oplus$\,skiiz & \textsc{ pl ine } \\
\underline{lui}\,$\oplus$\,skiiss & \textsc{ pl ela } \\
\underline{lui}\,$\oplus$\,skiillõ & \textsc{ pl all } \\
\underline{lui}\,$\oplus$\,skiill & \textsc{ pl ade } \\
\underline{lui}\,$\oplus$\,skiilt & \textsc{ pl abl } \\
\underline{lui}\,$\oplus$\,skiissi & \textsc{ pl tra } \\
\underline{lui}\,$\oplus$\,skiissaa & \textsc{ pl ter } \\
\underline{lui}\,$\oplus$\,skijka & \textsc{ pl com } \\
\end{tabular}
\end{sideways}
\captionof{table}{Tüüpsõnamall \arabic{mallinumber}\,\vadja{luiskõ} ekstraheeritud muutvormimallid.}
\label{tab:tüüpsõnamall-luiskõ}

\end{minipage}

 
\vspace{1em}
\noindent Tüüpsõnamall \vadja{luiskõ} ei hõlma teisi lekseeme vormi\-sõnastikus.

Tüüp\-sõna\-mall kirjeldab tagapoolseid sõnu tüvemuutusega sk:zg.


\vspace{3.5em}
\noindent \begin{minipage}{\textwidth}
\stepcounter{mallinumber}
\noindent \textbf{Tüüpsõnamall \arabic{mallinumber}\,\vadja{mussõ}}\\

\begin{sideways}
\begin{tabular}{l l}
muutvormimall & tunnused \\
\hline
\underline{mus}\,$\oplus$\,sõ & \textsc{ sg nom } \\
\underline{mus}\,$\oplus$\,a & \textsc{ sg gen } \\
\underline{mus}\,$\oplus$\,sa & \textsc{ sg par } \\
\underline{mus}\,$\oplus$\,sasõ & \textsc{ sg ill } \\
\underline{mus}\,$\oplus$\,saz & \textsc{ sg ine } \\
\underline{mus}\,$\oplus$\,ass & \textsc{ sg ela } \\
\underline{mus}\,$\oplus$\,allõ & \textsc{ sg all } \\
\underline{mus}\,$\oplus$\,all & \textsc{ sg ade } \\
\underline{mus}\,$\oplus$\,alt & \textsc{ sg abl } \\
\underline{mus}\,$\oplus$\,assi & \textsc{ sg tra } \\
\underline{mus}\,$\oplus$\,sassaa & \textsc{ sg ter } \\
\underline{mus}\,$\oplus$\,aka & \textsc{ sg com } \\
\underline{mus}\,$\oplus$\,ad & \textsc{ pl nom } \\
\underline{mus}\,$\oplus$\,sõjõ & \textsc{ pl gen } \\
\underline{mus}\,$\oplus$\,sõit & \textsc{ pl par } \\
\underline{mus}\,$\oplus$\,sõisõ & \textsc{ pl ill } \\
\underline{mus}\,$\oplus$\,sõiz & \textsc{ pl ine } \\
\underline{mus}\,$\oplus$\,sõiss & \textsc{ pl ela } \\
\underline{mus}\,$\oplus$\,sõillõ & \textsc{ pl all } \\
\underline{mus}\,$\oplus$\,sõill & \textsc{ pl ade } \\
\underline{mus}\,$\oplus$\,sõilt & \textsc{ pl abl } \\
\underline{mus}\,$\oplus$\,sõissi & \textsc{ pl tra } \\
\underline{mus}\,$\oplus$\,sõissaa & \textsc{ pl ter } \\
\underline{mus}\,$\oplus$\,sõika & \textsc{ pl com } \\
\end{tabular}
\end{sideways}
\captionof{table}{Tüüpsõnamall \arabic{mallinumber}\,\vadja{mussõ} ekstraheeritud muutvormimallid.}
\label{tab:tüüpsõnamall-mussõ}

\end{minipage}

 
\vspace{1em}
\noindent Tüüpsõnamall \vadja{mussõ} hõlmab vormisõnastikus 2 lekseemi: \vadja{\underline{mus}sõ} ja \vadja{\underline{kapus}sõ}.

Tüüp\-sõna\-mall kirjeldab tagapoolseid sõnu tüvemuutusega ss:s.


\vspace{3.5em}
\noindent \begin{minipage}{\textwidth}
\stepcounter{mallinumber}
\noindent \textbf{Tüüpsõnamall \arabic{mallinumber}\,\vadja{moškõ}}\\

\begin{sideways}
\begin{tabular}{l l}
muutvormimall & tunnused \\
\hline
\underline{mo}\,$\oplus$\,škõ & \textsc{ sg nom } \\
\underline{mo}\,$\oplus$\,žga & \textsc{ sg gen } \\
\underline{mo}\,$\oplus$\,ška & \textsc{ sg par } \\
\underline{mo}\,$\oplus$\,škasõ & \textsc{ sg ill } \\
\underline{mo}\,$\oplus$\,žgõz & \textsc{ sg ine } \\
\underline{mo}\,$\oplus$\,žgõss & \textsc{ sg ela } \\
\underline{mo}\,$\oplus$\,žgõllõ & \textsc{ sg all } \\
\underline{mo}\,$\oplus$\,žgõll & \textsc{ sg ade } \\
\underline{mo}\,$\oplus$\,žgõlt & \textsc{ sg abl } \\
\underline{mo}\,$\oplus$\,žgõssi & \textsc{ sg tra } \\
\underline{mo}\,$\oplus$\,žgõssaa & \textsc{ sg ter } \\
\underline{mo}\,$\oplus$\,žgõka & \textsc{ sg com } \\
\underline{mo}\,$\oplus$\,žgõd & \textsc{ pl nom } \\
\underline{mo}\,$\oplus$\,škijõ & \textsc{ pl gen } \\
\underline{mo}\,$\oplus$\,škiit & \textsc{ pl par } \\
\underline{mo}\,$\oplus$\,škiisõ & \textsc{ pl ill } \\
\underline{mo}\,$\oplus$\,škiiz & \textsc{ pl ine } \\
\underline{mo}\,$\oplus$\,škiiss & \textsc{ pl ela } \\
\underline{mo}\,$\oplus$\,škiillõ & \textsc{ pl all } \\
\underline{mo}\,$\oplus$\,škiill & \textsc{ pl ade } \\
\underline{mo}\,$\oplus$\,škiilt & \textsc{ pl abl } \\
\underline{mo}\,$\oplus$\,škiissi & \textsc{ pl tra } \\
\underline{mo}\,$\oplus$\,škiissaa & \textsc{ pl ter } \\
\underline{mo}\,$\oplus$\,škijka & \textsc{ pl com } \\
\end{tabular}
\end{sideways}
\captionof{table}{Tüüpsõnamall \arabic{mallinumber}\,\vadja{moškõ} ekstraheeritud muutvormimallid.}
\label{tab:tüüpsõnamall-moškõ}

\end{minipage}

 
\vspace{1em}
\noindent Tüüpsõnamall \vadja{moškõ} hõlmab vormisõnastikus 2 lekseemi: \vadja{\underline{mo}škõ} ja \vadja{\underline{krõ}škõ}.

Tüüp\-sõna\-mall kirjeldab tagapoolseid sõnu tüvemuutusega šk:žg.


\vspace{3.5em}
\noindent \begin{minipage}{\textwidth}
\stepcounter{mallinumber}
\noindent \textbf{Tüüpsõnamall \arabic{mallinumber}\,\vadja{lootõ}}\\

\begin{sideways}
\begin{tabular}{l l}
muutvormimall & tunnused \\
\hline
\underline{loo}\,$\oplus$\,tõ & \textsc{ sg nom } \\
\underline{loo}\,$\oplus$\,vva & \textsc{ sg gen } \\
\underline{loo}\,$\oplus$\,ta & \textsc{ sg par } \\
\underline{loo}\,$\oplus$\,tasõ & \textsc{ sg ill } \\
\underline{loo}\,$\oplus$\,vvõz & \textsc{ sg ine } \\
\underline{loo}\,$\oplus$\,vvõss & \textsc{ sg ela } \\
\underline{loo}\,$\oplus$\,vvõllõ & \textsc{ sg all } \\
\underline{loo}\,$\oplus$\,vvõll & \textsc{ sg ade } \\
\underline{loo}\,$\oplus$\,vvõlt & \textsc{ sg abl } \\
\underline{loo}\,$\oplus$\,vvõssi & \textsc{ sg tra } \\
\underline{loo}\,$\oplus$\,vvõssaa & \textsc{ sg ter } \\
\underline{loo}\,$\oplus$\,vvõka & \textsc{ sg com } \\
\underline{loo}\,$\oplus$\,vvõd & \textsc{ pl nom } \\
\underline{loo}\,$\oplus$\,tijõ & \textsc{ pl gen } \\
\underline{loo}\,$\oplus$\,tiit & \textsc{ pl par } \\
\underline{loo}\,$\oplus$\,tiisõ & \textsc{ pl ill } \\
\underline{loo}\,$\oplus$\,tiiz & \textsc{ pl ine } \\
\underline{loo}\,$\oplus$\,tiiss & \textsc{ pl ela } \\
\underline{loo}\,$\oplus$\,tiillõ & \textsc{ pl all } \\
\underline{loo}\,$\oplus$\,tiill & \textsc{ pl ade } \\
\underline{loo}\,$\oplus$\,tiilt & \textsc{ pl abl } \\
\underline{loo}\,$\oplus$\,tiissi & \textsc{ pl tra } \\
\underline{loo}\,$\oplus$\,tiissaa & \textsc{ pl ter } \\
\underline{loo}\,$\oplus$\,tijka & \textsc{ pl com } \\
\end{tabular}
\end{sideways}
\captionof{table}{Tüüpsõnamall \arabic{mallinumber}\,\vadja{lootõ} ekstraheeritud muutvormimallid.}
\label{tab:tüüpsõnamall-lootõ}

\end{minipage}

 
\vspace{1em}
\noindent Tüüpsõnamall \vadja{lootõ} ei hõlma teisi lekseeme vormi\-sõnastikus.

Tüüp\-sõna\-mall kirjeldab tagapoolseid sõnu tüvemuutusega t:vv.


\vspace{3.5em}
\noindent \begin{minipage}{\textwidth}
\stepcounter{mallinumber}
\noindent \textbf{Tüüpsõnamall \arabic{mallinumber}\,\vadja{biskvittõ}}\\

\begin{sideways}
\begin{tabular}{l l}
muutvormimall & tunnused \\
\hline
\underline{biskvit}\,$\oplus$\,tõ & \textsc{ sg nom } \\
\underline{biskvit}\,$\oplus$\,a & \textsc{ sg gen } \\
\underline{biskvit}\,$\oplus$\,ta & \textsc{ sg par } \\
\underline{biskvit}\,$\oplus$\,tasõ & \textsc{ sg ill } \\
\underline{biskvit}\,$\oplus$\,tõz & \textsc{ sg ine } \\
\underline{biskvit}\,$\oplus$\,õss & \textsc{ sg ela } \\
\underline{biskvit}\,$\oplus$\,õllõ & \textsc{ sg all } \\
\underline{biskvit}\,$\oplus$\,õll & \textsc{ sg ade } \\
\underline{biskvit}\,$\oplus$\,õlt & \textsc{ sg abl } \\
\underline{biskvit}\,$\oplus$\,õssi & \textsc{ sg tra } \\
\underline{biskvit}\,$\oplus$\,tõssaa & \textsc{ sg ter } \\
\underline{biskvit}\,$\oplus$\,õka & \textsc{ sg com } \\
\underline{biskvit}\,$\oplus$\,õd & \textsc{ pl nom } \\
\underline{biskvit}\,$\oplus$\,tajõ & \textsc{ pl gen } \\
\underline{biskvit}\,$\oplus$\,tiit & \textsc{ pl par } \\
\underline{biskvit}\,$\oplus$\,tiisõ & \textsc{ pl ill } \\
\underline{biskvit}\,$\oplus$\,tiiz & \textsc{ pl ine } \\
\underline{biskvit}\,$\oplus$\,tiiss & \textsc{ pl ela } \\
\underline{biskvit}\,$\oplus$\,tiillõ & \textsc{ pl all } \\
\underline{biskvit}\,$\oplus$\,tiill & \textsc{ pl ade } \\
\underline{biskvit}\,$\oplus$\,tiilt & \textsc{ pl abl } \\
\underline{biskvit}\,$\oplus$\,tiissi & \textsc{ pl tra } \\
\underline{biskvit}\,$\oplus$\,tiissaa & \textsc{ pl ter } \\
\underline{biskvit}\,$\oplus$\,tijka & \textsc{ pl com } \\
\end{tabular}
\end{sideways}
\captionof{table}{Tüüpsõnamall \arabic{mallinumber}\,\vadja{biskvittõ} ekstraheeritud muutvormimallid.}
\label{tab:tüüpsõnamall-biskvittõ}

\end{minipage}

 
\vspace{1em}
\noindent Tüüpsõnamall \vadja{biskvittõ} ei hõlma teisi lekseeme vormi\-sõnastikus.

Tüüp\-sõna\-mall kirjeldab tagapoolseid sõnu tüvemuutusega tt:t.
\spacing{1.5}
  
\subsection{\RN{6} käändkond}

Kuuendasse käändkonda kuuluvad Ariste sõnul need sõnad, mis lõpevad \vadja{-õa/-eä/-iä}. Jõgõperä murret iseloomustab ta diftongi puudumisega sõna lõpus. (\cite[44]{ariste_grammar_1968})

Tsvetkovi sõnaraamatus esineb suur variatiivsus mitmuse tüves. Variatiivsust on vormi\-sõnastikus vähendatud ja ühtlustatud kolmele tüvele: tüüp\-sõna \textit{kerkiä} tüvega \textit{-ii-}, tüüpsõna \textit{terve} tüvega \textit{-ei-} ja tüüpsõna \textit{kanka} tüvega \textit{-õi-}.

%% Avatuid küsimusi-tähelepanekuid:
%% \begin{itemize}
%% \item käändkonna liikmete pluurali tüved on ühtlustatud -- kas jätta nii või taastada Tsvetkovi variatiivsus?
%% \end{itemize}

\subsubsection*{Ekstraktmorfoloogia tüüpsõnad}
\spacing{1.1}
\vspace{-3.5em}


\vspace{3.5em}
\noindent \begin{minipage}{\textwidth}
\stepcounter{mallinumber}
\noindent \textbf{Tüüpsõnamall \arabic{mallinumber}\,\vadja{kerkiä}}\\

\begin{sideways}
\begin{tabular}{l l}
muutvormimall & tunnused \\
\hline
\underline{kerki}\,$\oplus$\,ä & \textsc{ sg nom } \\
\underline{kerki}\,$\oplus$\,ä & \textsc{ sg gen } \\
\underline{kerki}\,$\oplus$\,ätä & \textsc{ sg par } \\
\underline{kerki}\,$\oplus$\,äse & \textsc{ sg ill } \\
\underline{kerki}\,$\oplus$\,äz & \textsc{ sg ine } \\
\underline{kerki}\,$\oplus$\,äss & \textsc{ sg ela } \\
\underline{kerki}\,$\oplus$\,älle & \textsc{ sg all } \\
\underline{kerki}\,$\oplus$\,äll & \textsc{ sg ade } \\
\underline{kerki}\,$\oplus$\,ält & \textsc{ sg abl } \\
\underline{kerki}\,$\oplus$\,ässi & \textsc{ sg tra } \\
\underline{kerki}\,$\oplus$\,ässaa & \textsc{ sg ter } \\
\underline{kerki}\,$\oplus$\,äka & \textsc{ sg com } \\
\underline{kerki}\,$\oplus$\,äd & \textsc{ pl nom } \\
\underline{kerki}\,$\oplus$\,je & \textsc{ pl gen } \\
\underline{kerki}\,$\oplus$\,it & \textsc{ pl par } \\
\underline{kerki}\,$\oplus$\,ise & \textsc{ pl ill } \\
\underline{kerki}\,$\oplus$\,iz & \textsc{ pl ine } \\
\underline{kerki}\,$\oplus$\,iss & \textsc{ pl ela } \\
\underline{kerki}\,$\oplus$\,ille & \textsc{ pl all } \\
\underline{kerki}\,$\oplus$\,ill & \textsc{ pl ade } \\
\underline{kerki}\,$\oplus$\,ilt & \textsc{ pl abl } \\
\underline{kerki}\,$\oplus$\,issi & \textsc{ pl tra } \\
\underline{kerki}\,$\oplus$\,issaa & \textsc{ pl ter } \\
\underline{kerki}\,$\oplus$\,jka & \textsc{ pl com } \\
\end{tabular}
\end{sideways}
\captionof{table}{Tüüpsõnamall \arabic{mallinumber}\,\vadja{kerkiä} ekstraheeritud muutvormimallid.}
\label{tab:tüüpsõnamall-kerkiä}

\end{minipage}

 
\vspace{1em}
\noindent Tüüpsõnamall \vadja{kerkiä} hõlmab vormisõnastikus 5 lekseemi: \vadja{\underline{kerki}ä, \underline{peh̕mi}ä, \underline{pimmi}ä, \underline{siiti}ä} ja \vadja{\underline{jämi}ä}.

Tüüp\-sõna\-mall kirjeldab eespoolseid tüvemuutuseta sõnu.


\vspace{3.5em}
\noindent \begin{minipage}{\textwidth}
\stepcounter{mallinumber}
\noindent \textbf{Tüüpsõnamall \arabic{mallinumber}\,\vadja{terve}}\\

\begin{sideways}
\begin{tabular}{l l}
muutvormimall & tunnused \\
\hline
\underline{terv}\,$\oplus$\,\underline{e} & \textsc{ sg nom } \\
\underline{terv}\,$\oplus$\,\underline{e} & \textsc{ sg gen } \\
\underline{terv}\,$\oplus$\,\underline{e}\,$\oplus$\,ttä & \textsc{ sg par } \\
\underline{terv}\,$\oplus$\,\underline{e}\,$\oplus$\,se & \textsc{ sg ill } \\
\underline{terv}\,$\oplus$\,\underline{e}\,$\oplus$\,z & \textsc{ sg ine } \\
\underline{terv}\,$\oplus$\,\underline{e}\,$\oplus$\,ss & \textsc{ sg ela } \\
\underline{terv}\,$\oplus$\,\underline{e}\,$\oplus$\,lle & \textsc{ sg all } \\
\underline{terv}\,$\oplus$\,\underline{e}\,$\oplus$\,ll & \textsc{ sg ade } \\
\underline{terv}\,$\oplus$\,\underline{e}\,$\oplus$\,lt & \textsc{ sg abl } \\
\underline{terv}\,$\oplus$\,\underline{e}\,$\oplus$\,ssi & \textsc{ sg tra } \\
\underline{terv}\,$\oplus$\,\underline{e}\,$\oplus$\,ssaa & \textsc{ sg ter } \\
\underline{terv}\,$\oplus$\,\underline{e}\,$\oplus$\,ka & \textsc{ sg com } \\
\underline{terv}\,$\oplus$\,\underline{e}\,$\oplus$\,d & \textsc{ pl nom } \\
\underline{terv}\,$\oplus$\,ij\,$\oplus$\,\underline{e} & \textsc{ pl gen } \\
\underline{terv}\,$\oplus$\,\underline{e}\,$\oplus$\,it & \textsc{ pl par } \\
\underline{terv}\,$\oplus$\,\underline{e}\,$\oplus$\,ise & \textsc{ pl ill } \\
\underline{terv}\,$\oplus$\,\underline{e}\,$\oplus$\,iz & \textsc{ pl ine } \\
\underline{terv}\,$\oplus$\,\underline{e}\,$\oplus$\,iss & \textsc{ pl ela } \\
\underline{terv}\,$\oplus$\,\underline{e}\,$\oplus$\,ille & \textsc{ pl all } \\
\underline{terv}\,$\oplus$\,\underline{e}\,$\oplus$\,ill & \textsc{ pl ade } \\
\underline{terv}\,$\oplus$\,\underline{e}\,$\oplus$\,ilt & \textsc{ pl abl } \\
\underline{terv}\,$\oplus$\,\underline{e}\,$\oplus$\,issi & \textsc{ pl tra } \\
\underline{terv}\,$\oplus$\,\underline{e}\,$\oplus$\,issaa & \textsc{ pl ter } \\
\underline{terv}\,$\oplus$\,\underline{e}\,$\oplus$\,ika & \textsc{ pl com } \\
\end{tabular}
\end{sideways}
\captionof{table}{Tüüpsõnamall \arabic{mallinumber}\,\vadja{terve} ekstraheeritud muutvormimallid.}
\label{tab:tüüpsõnamall-terve}

\end{minipage}

 
\vspace{1em}
\noindent Tüüpsõnamall \vadja{terve} ei hõlma teisi lekseeme vormi\-sõnastikus.

Tüüp\-sõna\-mall kirjeldab eespoolseid tüvemuutuseta sõnu.


\vspace{3.5em}
\noindent \begin{minipage}{\textwidth}
\stepcounter{mallinumber}
\noindent \textbf{Tüüpsõnamall \arabic{mallinumber}\,\vadja{kanka}}\\

\begin{sideways}
\begin{tabular}{l l}
muutvormimall & tunnused \\
\hline
\underline{kank}\,$\oplus$\,a & \textsc{ sg nom } \\
\underline{kank}\,$\oplus$\,a & \textsc{ sg gen } \\
\underline{kank}\,$\oplus$\,atõ & \textsc{ sg par } \\
\underline{kank}\,$\oplus$\,asõ & \textsc{ sg ill } \\
\underline{kank}\,$\oplus$\,az & \textsc{ sg ine } \\
\underline{kank}\,$\oplus$\,ass & \textsc{ sg ela } \\
\underline{kank}\,$\oplus$\,allõ & \textsc{ sg all } \\
\underline{kank}\,$\oplus$\,all & \textsc{ sg ade } \\
\underline{kank}\,$\oplus$\,alt & \textsc{ sg abl } \\
\underline{kank}\,$\oplus$\,assi & \textsc{ sg tra } \\
\underline{kank}\,$\oplus$\,assaa & \textsc{ sg ter } \\
\underline{kank}\,$\oplus$\,aka & \textsc{ sg com } \\
\underline{kank}\,$\oplus$\,ad & \textsc{ pl nom } \\
\underline{kank}\,$\oplus$\,õjõ & \textsc{ pl gen } \\
\underline{kank}\,$\oplus$\,õit & \textsc{ pl par } \\
\underline{kank}\,$\oplus$\,õisõ & \textsc{ pl ill } \\
\underline{kank}\,$\oplus$\,õiz & \textsc{ pl ine } \\
\underline{kank}\,$\oplus$\,õiss & \textsc{ pl ela } \\
\underline{kank}\,$\oplus$\,õillõ & \textsc{ pl all } \\
\underline{kank}\,$\oplus$\,õill & \textsc{ pl ade } \\
\underline{kank}\,$\oplus$\,õilt & \textsc{ pl abl } \\
\underline{kank}\,$\oplus$\,õissi & \textsc{ pl tra } \\
\underline{kank}\,$\oplus$\,õissaa & \textsc{ pl ter } \\
\underline{kank}\,$\oplus$\,õika & \textsc{ pl com } \\
\end{tabular}
\end{sideways}
\captionof{table}{Tüüpsõnamall \arabic{mallinumber}\,\vadja{kanka} ekstraheeritud muutvormimallid.}
\label{tab:tüüpsõnamall-kanka}

\end{minipage}

 
\vspace{1em}
\noindent Tüüpsõnamall \vadja{kanka} hõlmab vormisõnastikus 8 lekseemi: \vadja{\underline{kank}a, \underline{kõrk}a, \underline{maikk}a, \underline{makk}a, \underline{rusk}a, \underline{valk}a, \underline{õik}a} ja \vadja{\underline{harm}a}.

Tüüp\-sõna\-mall kirjeldab tagapoolseid tüvemuutuseta sõnu.
\spacing{1.5}

\subsection{\RN{7} käändkond}

Seitsmendasse käändkonda kuuluvad kahesilbilised sõnad, mille \msd{sg nom} lõpp on \vadja{-i}, ent mille tüvevokaal on \vadja{-e/-õ} (\cite[45]{ariste_grammar_1968}).

%% Avatuid küsimusi-tähelepanekuid:
%% \begin{itemize}
%% \item 7 käändkonna kohta TODO kirjuta et isuri mõju tõttu on -i:-e:-iä levinud, aga normeerime nagu Aristel ja Tsvetkovil ka paralleelina tihti
%% \item eespoolsed on i:e:eä ja tagapoolsed on i:õ:õa
%% \item väci:väe aga mida teha lahti:lahe? -- VKSis esineb Lu lahõ
%% \end{itemize}


\subsubsection*{Ekstraktmorfoloogia tüüpsõnad}
\spacing{1.1}
\vspace{-3.5em}


\vspace{3.5em}
\noindent \begin{minipage}{\textwidth}
\stepcounter{mallinumber}
\noindent \textbf{Tüüpsõnamall \arabic{mallinumber}\,\vadja{väči}}\\

\begin{sideways}
\begin{tabular}{l l}
muutvormimall & tunnused \\
\hline
\underline{vä}\,$\oplus$\,či & \textsc{ sg nom } \\
\underline{vä}\,$\oplus$\,e & \textsc{ sg gen } \\
\underline{vä}\,$\oplus$\,ččeä & \textsc{ sg par } \\
\underline{vä}\,$\oplus$\,ččese & \textsc{ sg ill } \\
\underline{vä}\,$\oplus$\,ez & \textsc{ sg ine } \\
\underline{vä}\,$\oplus$\,ess & \textsc{ sg ela } \\
\underline{vä}\,$\oplus$\,elle & \textsc{ sg all } \\
\underline{vä}\,$\oplus$\,ell & \textsc{ sg ade } \\
\underline{vä}\,$\oplus$\,elt & \textsc{ sg abl } \\
\underline{vä}\,$\oplus$\,essi & \textsc{ sg tra } \\
\underline{vä}\,$\oplus$\,essaa & \textsc{ sg ter } \\
\underline{vä}\,$\oplus$\,eka & \textsc{ sg com } \\
\underline{vä}\,$\oplus$\,ed & \textsc{ pl nom } \\
\underline{vä}\,$\oplus$\,ččije & \textsc{ pl gen } \\
\underline{vä}\,$\oplus$\,ččiit & \textsc{ pl par } \\
\underline{vä}\,$\oplus$\,ččiise & \textsc{ pl ill } \\
\underline{vä}\,$\oplus$\,ččiiz & \textsc{ pl ine } \\
\underline{vä}\,$\oplus$\,ččiiss & \textsc{ pl ela } \\
\underline{vä}\,$\oplus$\,ččiille & \textsc{ pl all } \\
\underline{vä}\,$\oplus$\,ččiill & \textsc{ pl ade } \\
\underline{vä}\,$\oplus$\,ččiilt & \textsc{ pl abl } \\
\underline{vä}\,$\oplus$\,ččiissi & \textsc{ pl tra } \\
\underline{vä}\,$\oplus$\,ččiissaa & \textsc{ pl ter } \\
\underline{vä}\,$\oplus$\,ččijka & \textsc{ pl com } \\
\end{tabular}
\end{sideways}
\captionof{table}{Tüüpsõnamall \arabic{mallinumber}\,\vadja{väči} ekstraheeritud muutvormimallid.}
\label{tab:tüüpsõnamall-väči}

\end{minipage}

 
\vspace{1em}
\noindent Tüüpsõnamall \vadja{väči} hõlmab vormisõnastikus 2 lekseemi: \vadja{\underline{vä}či} ja \vadja{\underline{mä}či}.

Tüüp\-sõna\-mall kirjeldab eespoolseid sõnu tüvemuutusega č:∅.


\vspace{3.5em}
\noindent \begin{minipage}{\textwidth}
\stepcounter{mallinumber}
\noindent \textbf{Tüüpsõnamall \arabic{mallinumber}\,\vadja{lahti}}\\

\begin{sideways}
\begin{tabular}{l l}
muutvormimall & tunnused \\
\hline
\underline{lah}\,$\oplus$\,ti & \textsc{ sg nom } \\
\underline{lah}\,$\oplus$\,õ & \textsc{ sg gen } \\
\underline{lah}\,$\oplus$\,tõa & \textsc{ sg par } \\
\underline{lah}\,$\oplus$\,tõsõ & \textsc{ sg ill } \\
\underline{lah}\,$\oplus$\,õz & \textsc{ sg ine } \\
\underline{lah}\,$\oplus$\,õss & \textsc{ sg ela } \\
\underline{lah}\,$\oplus$\,õllõ & \textsc{ sg all } \\
\underline{lah}\,$\oplus$\,õll & \textsc{ sg ade } \\
\underline{lah}\,$\oplus$\,õlt & \textsc{ sg abl } \\
\underline{lah}\,$\oplus$\,õssi & \textsc{ sg tra } \\
\underline{lah}\,$\oplus$\,õssaa & \textsc{ sg ter } \\
\underline{lah}\,$\oplus$\,õka & \textsc{ sg com } \\
\underline{lah}\,$\oplus$\,õd & \textsc{ pl nom } \\
\underline{lah}\,$\oplus$\,tijõ & \textsc{ pl gen } \\
\underline{lah}\,$\oplus$\,tiit & \textsc{ pl par } \\
\underline{lah}\,$\oplus$\,tiisõ & \textsc{ pl ill } \\
\underline{lah}\,$\oplus$\,tiiz & \textsc{ pl ine } \\
\underline{lah}\,$\oplus$\,tiiss & \textsc{ pl ela } \\
\underline{lah}\,$\oplus$\,tiillõ & \textsc{ pl all } \\
\underline{lah}\,$\oplus$\,tiill & \textsc{ pl ade } \\
\underline{lah}\,$\oplus$\,tiilt & \textsc{ pl abl } \\
\underline{lah}\,$\oplus$\,tiissi & \textsc{ pl tra } \\
\underline{lah}\,$\oplus$\,tiissaa & \textsc{ pl ter } \\
\underline{lah}\,$\oplus$\,tijka & \textsc{ pl com } \\
\end{tabular}
\end{sideways}
\captionof{table}{Tüüpsõnamall \arabic{mallinumber}\,\vadja{lahti} ekstraheeritud muutvormimallid.}
\label{tab:tüüpsõnamall-lahti}

\end{minipage}

 
\vspace{1em}
\noindent Tüüpsõnamall \vadja{lahti} ei hõlma teisi lekseeme vormi\-sõnastikus.

Tüüp\-sõna\-mall kirjeldab tagapoolseid sõnu tüvemuutusega ht:h.


\vspace{3.5em}
\noindent \begin{minipage}{\textwidth}
\stepcounter{mallinumber}
\noindent \textbf{Tüüpsõnamall \arabic{mallinumber}\,\vadja{laki}}\\

\begin{sideways}
\begin{tabular}{l l}
muutvormimall & tunnused \\
\hline
\underline{la}\,$\oplus$\,ki & \textsc{ sg nom } \\
\underline{la}\,$\oplus$\,gõ & \textsc{ sg gen } \\
\underline{la}\,$\oplus$\,kkõa & \textsc{ sg par } \\
\underline{la}\,$\oplus$\,kkõsõ & \textsc{ sg ill } \\
\underline{la}\,$\oplus$\,gõz & \textsc{ sg ine } \\
\underline{la}\,$\oplus$\,gõss & \textsc{ sg ela } \\
\underline{la}\,$\oplus$\,gõllõ & \textsc{ sg all } \\
\underline{la}\,$\oplus$\,gõll & \textsc{ sg ade } \\
\underline{la}\,$\oplus$\,gõlt & \textsc{ sg abl } \\
\underline{la}\,$\oplus$\,gõssi & \textsc{ sg tra } \\
\underline{la}\,$\oplus$\,gõssaa & \textsc{ sg ter } \\
\underline{la}\,$\oplus$\,gõka & \textsc{ sg com } \\
\underline{la}\,$\oplus$\,gõd & \textsc{ pl nom } \\
\underline{la}\,$\oplus$\,kijõ & \textsc{ pl gen } \\
\underline{la}\,$\oplus$\,kiit & \textsc{ pl par } \\
\underline{la}\,$\oplus$\,kiisõ & \textsc{ pl ill } \\
\underline{la}\,$\oplus$\,kiiz & \textsc{ pl ine } \\
\underline{la}\,$\oplus$\,kiiss & \textsc{ pl ela } \\
\underline{la}\,$\oplus$\,kiillõ & \textsc{ pl all } \\
\underline{la}\,$\oplus$\,kiill & \textsc{ pl ade } \\
\underline{la}\,$\oplus$\,kiilt & \textsc{ pl abl } \\
\underline{la}\,$\oplus$\,kiissi & \textsc{ pl tra } \\
\underline{la}\,$\oplus$\,kiissaa & \textsc{ pl ter } \\
\underline{la}\,$\oplus$\,kijka & \textsc{ pl com } \\
\end{tabular}
\end{sideways}
\captionof{table}{Tüüpsõnamall \arabic{mallinumber}\,\vadja{laki} ekstraheeritud muutvormimallid.}
\label{tab:tüüpsõnamall-laki}

\end{minipage}

 
\vspace{1em}
\noindent Tüüpsõnamall \vadja{laki} hõlmab vormisõnastikus 3 lekseemi: \vadja{\underline{la}ki, \underline{nõ}ki} ja \vadja{\underline{jõ}ki}.

Tüüp\-sõna\-mall kirjeldab tagapoolseid sõnu tüvemuutusega k:g.


\vspace{3.5em}
\noindent \begin{minipage}{\textwidth}
\stepcounter{mallinumber}
\noindent \textbf{Tüüpsõnamall \arabic{mallinumber}\,\vadja{kurki}}\\

\begin{sideways}
\begin{tabular}{l l}
muutvormimall & tunnused \\
\hline
\underline{kur}\,$\oplus$\,ki & \textsc{ sg nom } \\
\underline{kur}\,$\oplus$\,gõ & \textsc{ sg gen } \\
\underline{kur}\,$\oplus$\,kõa & \textsc{ sg par } \\
\underline{kur}\,$\oplus$\,kõsõ & \textsc{ sg ill } \\
\underline{kur}\,$\oplus$\,gõz & \textsc{ sg ine } \\
\underline{kur}\,$\oplus$\,gõss & \textsc{ sg ela } \\
\underline{kur}\,$\oplus$\,gõllõ & \textsc{ sg all } \\
\underline{kur}\,$\oplus$\,gõll & \textsc{ sg ade } \\
\underline{kur}\,$\oplus$\,gõlt & \textsc{ sg abl } \\
\underline{kur}\,$\oplus$\,gõssi & \textsc{ sg tra } \\
\underline{kur}\,$\oplus$\,gõssaa & \textsc{ sg ter } \\
\underline{kur}\,$\oplus$\,gõka & \textsc{ sg com } \\
\underline{kur}\,$\oplus$\,gõd & \textsc{ pl nom } \\
\underline{kur}\,$\oplus$\,kijõ & \textsc{ pl gen } \\
\underline{kur}\,$\oplus$\,kiit & \textsc{ pl par } \\
\underline{kur}\,$\oplus$\,kiisõ & \textsc{ pl ill } \\
\underline{kur}\,$\oplus$\,kiiz & \textsc{ pl ine } \\
\underline{kur}\,$\oplus$\,kiiss & \textsc{ pl ela } \\
\underline{kur}\,$\oplus$\,kiillõ & \textsc{ pl all } \\
\underline{kur}\,$\oplus$\,kiill & \textsc{ pl ade } \\
\underline{kur}\,$\oplus$\,kiilt & \textsc{ pl abl } \\
\underline{kur}\,$\oplus$\,kiissi & \textsc{ pl tra } \\
\underline{kur}\,$\oplus$\,kiissaa & \textsc{ pl ter } \\
\underline{kur}\,$\oplus$\,kijka & \textsc{ pl com } \\
\end{tabular}
\end{sideways}
\captionof{table}{Tüüpsõnamall \arabic{mallinumber}\,\vadja{kurki} ekstraheeritud muutvormimallid.}
\label{tab:tüüpsõnamall-kurki}

\end{minipage}

 
\vspace{1em}
\noindent Tüüpsõnamall \vadja{kurki} hõlmab vormisõnastikus 3 lekseemi: \vadja{\underline{kur}ki, \underline{õn}ki} ja \vadja{\underline{kan}ki}.

Tüüp\-sõna\-mall kirjeldab tagapoolseid sõnu tüvemuutusega k:g, kusjuures tüvemuutus esineb konsont\-klustris, mistõttu gemineerumist ei toimu \msd{sg par} ja \msd{sg ill} tüvedes.


\vspace{3.5em}
\noindent \begin{minipage}{\textwidth}
\stepcounter{mallinumber}
\noindent \textbf{Tüüpsõnamall \arabic{mallinumber}\,\vadja{põski}}\\

\begin{sideways}
\begin{tabular}{l l}
muutvormimall & tunnused \\
\hline
\underline{põ}\,$\oplus$\,ski & \textsc{ sg nom } \\
\underline{põ}\,$\oplus$\,zgõ & \textsc{ sg gen } \\
\underline{põ}\,$\oplus$\,skõa & \textsc{ sg par } \\
\underline{põ}\,$\oplus$\,skõsõ & \textsc{ sg ill } \\
\underline{põ}\,$\oplus$\,zgõz & \textsc{ sg ine } \\
\underline{põ}\,$\oplus$\,zgõss & \textsc{ sg ela } \\
\underline{põ}\,$\oplus$\,zgõllõ & \textsc{ sg all } \\
\underline{põ}\,$\oplus$\,zgõll & \textsc{ sg ade } \\
\underline{põ}\,$\oplus$\,zgõlt & \textsc{ sg abl } \\
\underline{põ}\,$\oplus$\,zgõssi & \textsc{ sg tra } \\
\underline{põ}\,$\oplus$\,zgõssaa & \textsc{ sg ter } \\
\underline{põ}\,$\oplus$\,zgõka & \textsc{ sg com } \\
\underline{põ}\,$\oplus$\,zgõd & \textsc{ pl nom } \\
\underline{põ}\,$\oplus$\,skijõ & \textsc{ pl gen } \\
\underline{põ}\,$\oplus$\,skiit & \textsc{ pl par } \\
\underline{põ}\,$\oplus$\,skiisõ & \textsc{ pl ill } \\
\underline{põ}\,$\oplus$\,skiiz & \textsc{ pl ine } \\
\underline{põ}\,$\oplus$\,skiiss & \textsc{ pl ela } \\
\underline{põ}\,$\oplus$\,skiillõ & \textsc{ pl all } \\
\underline{põ}\,$\oplus$\,skiill & \textsc{ pl ade } \\
\underline{põ}\,$\oplus$\,skiilt & \textsc{ pl abl } \\
\underline{põ}\,$\oplus$\,skiissi & \textsc{ pl tra } \\
\underline{põ}\,$\oplus$\,skiissaa & \textsc{ pl ter } \\
\underline{põ}\,$\oplus$\,skijka & \textsc{ pl com } \\
\end{tabular}
\end{sideways}
\captionof{table}{Tüüpsõnamall \arabic{mallinumber}\,\vadja{põski} ekstraheeritud muutvormimallid.}
\label{tab:tüüpsõnamall-põski}

\end{minipage}

 
\vspace{1em}
\noindent Tüüpsõnamall \vadja{põski} ei hõlma teisi lekseeme vormi\-sõnastikus.

Tüüp\-sõna\-mall kirjeldab tagapoolseid sõnu tüvemuutusega sk:zg.


\vspace{3.5em}
\noindent \begin{minipage}{\textwidth}
\stepcounter{mallinumber}
\noindent \textbf{Tüüpsõnamall \arabic{mallinumber}\,\vadja{irvi}}\\

\begin{sideways}
\begin{tabular}{l l}
muutvormimall & tunnused \\
\hline
\underline{irv}\,$\oplus$\,i & \textsc{ sg nom } \\
\underline{irv}\,$\oplus$\,e & \textsc{ sg gen } \\
\underline{irv}\,$\oplus$\,eä & \textsc{ sg par } \\
\underline{irv}\,$\oplus$\,ese & \textsc{ sg ill } \\
\underline{irv}\,$\oplus$\,ez & \textsc{ sg ine } \\
\underline{irv}\,$\oplus$\,ess & \textsc{ sg ela } \\
\underline{irv}\,$\oplus$\,elle & \textsc{ sg all } \\
\underline{irv}\,$\oplus$\,ell & \textsc{ sg ade } \\
\underline{irv}\,$\oplus$\,elt & \textsc{ sg abl } \\
\underline{irv}\,$\oplus$\,essi & \textsc{ sg tra } \\
\underline{irv}\,$\oplus$\,essaa & \textsc{ sg ter } \\
\underline{irv}\,$\oplus$\,eka & \textsc{ sg com } \\
\underline{irv}\,$\oplus$\,ed & \textsc{ pl nom } \\
\underline{irv}\,$\oplus$\,ije & \textsc{ pl gen } \\
\underline{irv}\,$\oplus$\,iit & \textsc{ pl par } \\
\underline{irv}\,$\oplus$\,iise & \textsc{ pl ill } \\
\underline{irv}\,$\oplus$\,iiz & \textsc{ pl ine } \\
\underline{irv}\,$\oplus$\,iiss & \textsc{ pl ela } \\
\underline{irv}\,$\oplus$\,iille & \textsc{ pl all } \\
\underline{irv}\,$\oplus$\,iill & \textsc{ pl ade } \\
\underline{irv}\,$\oplus$\,iilt & \textsc{ pl abl } \\
\underline{irv}\,$\oplus$\,iissi & \textsc{ pl tra } \\
\underline{irv}\,$\oplus$\,iissaa & \textsc{ pl ter } \\
\underline{irv}\,$\oplus$\,ijka & \textsc{ pl com } \\
\end{tabular}
\end{sideways}
\captionof{table}{Tüüpsõnamall \arabic{mallinumber}\,\vadja{irvi} ekstraheeritud muutvormimallid.}
\label{tab:tüüpsõnamall-irvi}

\end{minipage}

 
\vspace{1em}
\noindent Tüüpsõnamall \vadja{irvi} hõlmab vormisõnastikus 6 lekseemi: \vadja{\underline{irv}i, \underline{järv}i, \underline{leem}i, \underline{nim}i, \underline{pilv}i} ja \vadja{\underline{čiv}i}.

Tüüp\-sõna\-mall kirjeldab eespoolseid tüvemuutuseta sõnu.


\vspace{3.5em}
\noindent \begin{minipage}{\textwidth}
\stepcounter{mallinumber}
\noindent \textbf{Tüüpsõnamall \arabic{mallinumber}\,\vadja{sarvi}}\\

\begin{sideways}
\begin{tabular}{l l}
muutvormimall & tunnused \\
\hline
\underline{sarv}\,$\oplus$\,i & \textsc{ sg nom } \\
\underline{sarv}\,$\oplus$\,õ & \textsc{ sg gen } \\
\underline{sarv}\,$\oplus$\,õa & \textsc{ sg par } \\
\underline{sarv}\,$\oplus$\,õsõ & \textsc{ sg ill } \\
\underline{sarv}\,$\oplus$\,õz & \textsc{ sg ine } \\
\underline{sarv}\,$\oplus$\,õss & \textsc{ sg ela } \\
\underline{sarv}\,$\oplus$\,õllõ & \textsc{ sg all } \\
\underline{sarv}\,$\oplus$\,õll & \textsc{ sg ade } \\
\underline{sarv}\,$\oplus$\,õlt & \textsc{ sg abl } \\
\underline{sarv}\,$\oplus$\,õssi & \textsc{ sg tra } \\
\underline{sarv}\,$\oplus$\,õssaa & \textsc{ sg ter } \\
\underline{sarv}\,$\oplus$\,õka & \textsc{ sg com } \\
\underline{sarv}\,$\oplus$\,õd & \textsc{ pl nom } \\
\underline{sarv}\,$\oplus$\,ijõ & \textsc{ pl gen } \\
\underline{sarv}\,$\oplus$\,iit & \textsc{ pl par } \\
\underline{sarv}\,$\oplus$\,iisõ & \textsc{ pl ill } \\
\underline{sarv}\,$\oplus$\,iiz & \textsc{ pl ine } \\
\underline{sarv}\,$\oplus$\,iiss & \textsc{ pl ela } \\
\underline{sarv}\,$\oplus$\,iillõ & \textsc{ pl all } \\
\underline{sarv}\,$\oplus$\,iill & \textsc{ pl ade } \\
\underline{sarv}\,$\oplus$\,iilt & \textsc{ pl abl } \\
\underline{sarv}\,$\oplus$\,iissi & \textsc{ pl tra } \\
\underline{sarv}\,$\oplus$\,iissaa & \textsc{ pl ter } \\
\underline{sarv}\,$\oplus$\,ijka & \textsc{ pl com } \\
\end{tabular}
\end{sideways}
\captionof{table}{Tüüpsõnamall \arabic{mallinumber}\,\vadja{sarvi} ekstraheeritud muutvormimallid.}
\label{tab:tüüpsõnamall-sarvi}

\end{minipage}

 
\vspace{1em}
\noindent Tüüpsõnamall \vadja{sarvi} hõlmab vormisõnastikus 10 lekseemi: \vadja{\underline{sarv}i, \underline{sav}i, \underline{Soom}i, \underline{sõrm}i, \underline{taim}i, \underline{talv}i, \underline{tamm}i, \underline{õnn}i, \underline{õvv}i} ja \vadja{\underline{põlv}i}.

Tüüp\-sõna\-mall kirjeldab tagapoolseid tüvemuutuseta sõnu.


\vspace{3.5em}
\noindent \begin{minipage}{\textwidth}
\stepcounter{mallinumber}
\noindent \textbf{Tüüpsõnamall \arabic{mallinumber}\,\vadja{enči}}\\

\begin{sideways}
\begin{tabular}{l l}
muutvormimall & tunnused \\
\hline
\underline{en}\,$\oplus$\,či & \textsc{ sg nom } \\
\underline{en}\,$\oplus$\,̕n̕e & \textsc{ sg gen } \\
\underline{en}\,$\oplus$\,čeä & \textsc{ sg par } \\
\underline{en}\,$\oplus$\,čese & \textsc{ sg ill } \\
\underline{en}\,$\oplus$\,̕n̕ez & \textsc{ sg ine } \\
\underline{en}\,$\oplus$\,̕n̕ess & \textsc{ sg ela } \\
\underline{en}\,$\oplus$\,̕n̕elle & \textsc{ sg all } \\
\underline{en}\,$\oplus$\,̕n̕ell & \textsc{ sg ade } \\
\underline{en}\,$\oplus$\,̕n̕elt & \textsc{ sg abl } \\
\underline{en}\,$\oplus$\,̕n̕essi & \textsc{ sg tra } \\
\underline{en}\,$\oplus$\,̕n̕essaa & \textsc{ sg ter } \\
\underline{en}\,$\oplus$\,̕n̕eka & \textsc{ sg com } \\
\underline{en}\,$\oplus$\,̕n̕ed & \textsc{ pl nom } \\
\underline{en}\,$\oplus$\,čije & \textsc{ pl gen } \\
\underline{en}\,$\oplus$\,čiit & \textsc{ pl par } \\
\underline{en}\,$\oplus$\,čiise & \textsc{ pl ill } \\
\underline{en}\,$\oplus$\,čiiz & \textsc{ pl ine } \\
\underline{en}\,$\oplus$\,čiiss & \textsc{ pl ela } \\
\underline{en}\,$\oplus$\,čiille & \textsc{ pl all } \\
\underline{en}\,$\oplus$\,čiill & \textsc{ pl ade } \\
\underline{en}\,$\oplus$\,čiilt & \textsc{ pl abl } \\
\underline{en}\,$\oplus$\,čiissi & \textsc{ pl tra } \\
\underline{en}\,$\oplus$\,čiissaa & \textsc{ pl ter } \\
\underline{en}\,$\oplus$\,čijka & \textsc{ pl com } \\
\end{tabular}
\end{sideways}
\captionof{table}{Tüüpsõnamall \arabic{mallinumber}\,\vadja{enči} ekstraheeritud muutvormimallid.}
\label{tab:tüüpsõnamall-enči}

\end{minipage}

 
\vspace{1em}
\noindent Tüüpsõnamall \vadja{enči} ei hõlma teisi lekseeme vormi\-sõnastikus.

Tüüp\-sõna\-mall kirjeldab eespoolseid sõnu tüvemuutusega nč:n̕n̕.


\vspace{3.5em}
\noindent \begin{minipage}{\textwidth}
\stepcounter{mallinumber}
\noindent \textbf{Tüüpsõnamall \arabic{mallinumber}\,\vadja{siipi}}\\

\begin{sideways}
\begin{tabular}{l l}
muutvormimall & tunnused \\
\hline
\underline{sii}\,$\oplus$\,pi & \textsc{ sg nom } \\
\underline{sii}\,$\oplus$\,ve & \textsc{ sg gen } \\
\underline{sii}\,$\oplus$\,peä & \textsc{ sg par } \\
\underline{sii}\,$\oplus$\,pese & \textsc{ sg ill } \\
\underline{sii}\,$\oplus$\,vez & \textsc{ sg ine } \\
\underline{sii}\,$\oplus$\,vess & \textsc{ sg ela } \\
\underline{sii}\,$\oplus$\,velle & \textsc{ sg all } \\
\underline{sii}\,$\oplus$\,vell & \textsc{ sg ade } \\
\underline{sii}\,$\oplus$\,velt & \textsc{ sg abl } \\
\underline{sii}\,$\oplus$\,vessi & \textsc{ sg tra } \\
\underline{sii}\,$\oplus$\,vessaa & \textsc{ sg ter } \\
\underline{sii}\,$\oplus$\,veka & \textsc{ sg com } \\
\underline{sii}\,$\oplus$\,ved & \textsc{ pl nom } \\
\underline{sii}\,$\oplus$\,pije & \textsc{ pl gen } \\
\underline{sii}\,$\oplus$\,piit & \textsc{ pl par } \\
\underline{sii}\,$\oplus$\,piise & \textsc{ pl ill } \\
\underline{sii}\,$\oplus$\,piiz & \textsc{ pl ine } \\
\underline{sii}\,$\oplus$\,piiss & \textsc{ pl ela } \\
\underline{sii}\,$\oplus$\,piille & \textsc{ pl all } \\
\underline{sii}\,$\oplus$\,piill & \textsc{ pl ade } \\
\underline{sii}\,$\oplus$\,piilt & \textsc{ pl abl } \\
\underline{sii}\,$\oplus$\,piissi & \textsc{ pl tra } \\
\underline{sii}\,$\oplus$\,piissaa & \textsc{ pl ter } \\
\underline{sii}\,$\oplus$\,pijka & \textsc{ pl com } \\
\end{tabular}
\end{sideways}
\captionof{table}{Tüüpsõnamall \arabic{mallinumber}\,\vadja{siipi} ekstraheeritud muutvormimallid.}
\label{tab:tüüpsõnamall-siipi}

\end{minipage}

 
\vspace{1em}
\noindent Tüüpsõnamall \vadja{siipi} ei hõlma teisi lekseeme vormi\-sõnastikus.

Tüüp\-sõna\-mall kirjeldab eespoolseid sõnu tüvemuutusega p:v.


\vspace{3.5em}
\noindent \begin{minipage}{\textwidth}
\stepcounter{mallinumber}
\noindent \textbf{Tüüpsõnamall \arabic{mallinumber}\,\vadja{kuusi}}\\

\begin{sideways}
\begin{tabular}{l l}
muutvormimall & tunnused \\
\hline
\underline{kuu}\,$\oplus$\,si & \textsc{ sg nom } \\
\underline{kuu}\,$\oplus$\,zõ & \textsc{ sg gen } \\
\underline{kuu}\,$\oplus$\,ssõ & \textsc{ sg par } \\
\underline{kuu}\,$\oplus$\,ssõsõ & \textsc{ sg ill } \\
\underline{kuu}\,$\oplus$\,zõz & \textsc{ sg ine } \\
\underline{kuu}\,$\oplus$\,zõss & \textsc{ sg ela } \\
\underline{kuu}\,$\oplus$\,zõllõ & \textsc{ sg all } \\
\underline{kuu}\,$\oplus$\,zõll & \textsc{ sg ade } \\
\underline{kuu}\,$\oplus$\,zõlt & \textsc{ sg abl } \\
\underline{kuu}\,$\oplus$\,zõssi & \textsc{ sg tra } \\
\underline{kuu}\,$\oplus$\,zõssaa & \textsc{ sg ter } \\
\underline{kuu}\,$\oplus$\,zõka & \textsc{ sg com } \\
\underline{kuu}\,$\oplus$\,zõd & \textsc{ pl nom } \\
\underline{kuu}\,$\oplus$\,sijõ & \textsc{ pl gen } \\
\underline{kuu}\,$\oplus$\,siit & \textsc{ pl par } \\
\underline{kuu}\,$\oplus$\,siisõ & \textsc{ pl ill } \\
\underline{kuu}\,$\oplus$\,siiz & \textsc{ pl ine } \\
\underline{kuu}\,$\oplus$\,siiss & \textsc{ pl ela } \\
\underline{kuu}\,$\oplus$\,siillõ & \textsc{ pl all } \\
\underline{kuu}\,$\oplus$\,siill & \textsc{ pl ade } \\
\underline{kuu}\,$\oplus$\,siilt & \textsc{ pl abl } \\
\underline{kuu}\,$\oplus$\,siissi & \textsc{ pl tra } \\
\underline{kuu}\,$\oplus$\,siissaa & \textsc{ pl ter } \\
\underline{kuu}\,$\oplus$\,sijka & \textsc{ pl com } \\
\end{tabular}
\end{sideways}
\captionof{table}{Tüüpsõnamall \arabic{mallinumber}\,\vadja{kuusi} ekstraheeritud muutvormimallid.}
\label{tab:tüüpsõnamall-kuusi}

\end{minipage}

 
\vspace{1em}
\noindent Tüüpsõnamall \vadja{kuusi} ei hõlma teisi lekseeme vormi\-sõnastikus.

Tüüp\-sõna\-mall kirjeldab tagapoolseid sõnu tüvemuutusega s:z.


\vspace{3.5em}
\noindent \begin{minipage}{\textwidth}
\stepcounter{mallinumber}
\noindent \textbf{Tüüpsõnamall \arabic{mallinumber}\,\vadja{tauti}}\\

\begin{sideways}
\begin{tabular}{l l}
muutvormimall & tunnused \\
\hline
\underline{ta}\,$\oplus$\,uti & \textsc{ sg nom } \\
\underline{ta}\,$\oplus$\,vvõ & \textsc{ sg gen } \\
\underline{ta}\,$\oplus$\,utõa & \textsc{ sg par } \\
\underline{ta}\,$\oplus$\,utõsõ & \textsc{ sg ill } \\
\underline{ta}\,$\oplus$\,vvõz & \textsc{ sg ine } \\
\underline{ta}\,$\oplus$\,vvõss & \textsc{ sg ela } \\
\underline{ta}\,$\oplus$\,vvõllõ & \textsc{ sg all } \\
\underline{ta}\,$\oplus$\,vvõll & \textsc{ sg ade } \\
\underline{ta}\,$\oplus$\,vvõlt & \textsc{ sg abl } \\
\underline{ta}\,$\oplus$\,vvõssi & \textsc{ sg tra } \\
\underline{ta}\,$\oplus$\,vvõssaa & \textsc{ sg ter } \\
\underline{ta}\,$\oplus$\,vvõka & \textsc{ sg com } \\
\underline{ta}\,$\oplus$\,vvõd & \textsc{ pl nom } \\
\underline{ta}\,$\oplus$\,utijõ & \textsc{ pl gen } \\
\underline{ta}\,$\oplus$\,utiit & \textsc{ pl par } \\
\underline{ta}\,$\oplus$\,utiisõ & \textsc{ pl ill } \\
\underline{ta}\,$\oplus$\,utiiz & \textsc{ pl ine } \\
\underline{ta}\,$\oplus$\,utiiss & \textsc{ pl ela } \\
\underline{ta}\,$\oplus$\,utiillõ & \textsc{ pl all } \\
\underline{ta}\,$\oplus$\,utiill & \textsc{ pl ade } \\
\underline{ta}\,$\oplus$\,utiilt & \textsc{ pl abl } \\
\underline{ta}\,$\oplus$\,utiissi & \textsc{ pl tra } \\
\underline{ta}\,$\oplus$\,utiissaa & \textsc{ pl ter } \\
\underline{ta}\,$\oplus$\,utijka & \textsc{ pl com } \\
\end{tabular}
\end{sideways}
\captionof{table}{Tüüpsõnamall \arabic{mallinumber}\,\vadja{tauti} ekstraheeritud muutvormimallid.}
\label{tab:tüüpsõnamall-tauti}

\end{minipage}

 
\vspace{1em}
\noindent Tüüpsõnamall \vadja{tauti} ei hõlma teisi lekseeme vormi\-sõnastikus.

Tüüp\-sõna\-mall kirjeldab tagapoolseid sõnu tüvemuutusega t:d.
\spacing{1.5}


\subsection{\RN{8} käändkond}

Kaheksandasse käändkonda kuuluvad \vadja{-ä}-tüvelised sõnad (\cite[46]{ariste_grammar_1968}).

%% Avatuid küsimusi-tähelepanekuid:
%% \begin{itemize}
%% \item 8 käändkond on väga variatiivne tüvevokaali suhtes (eined, leived, čenned aga sepäd,
%% \item eine (Heinsoo, Konkova ning Rozhanskiy ja Markus) aga einä (VKS)
%% \item läikkiv on ühtlustatud läikkive
%% \item Tsvetkovil paljud geminatsioonid puudu (õjja)
%% \item tegija-liides on eespoolsete sõnade puhul ühtlustatud -jä:-jä:-jä, mitte -je:-jä:-jä, VKSis esineb rohkem -jä Lu/Li/J märgenditega (Konkoval eespoolseid sõnu ei esine)
%% \item kuigi Tsvetkovil on häälduspäraselt ülesmärgitud 'õmpõlia' ja 'müüjä', on need läbivalt ühtlustatud (lisatud -j- nii \msd{sg nom} kui ka \msd{pl} vormidele)
%% \item tegija-liides on tagapoolsete sõnade puhul ühtlustatud -ja:-ja:-ja (kuigi Konkoval esineb -jõ:-ja:-ja)
%% \end{itemize}


\subsubsection*{Ekstraktmorfoloogia tüüpsõnad}
\spacing{1}
\vspace{-3.5em}


\vspace{3.5em}
\noindent \begin{minipage}{\textwidth}
\stepcounter{mallinumber}
\noindent \textbf{Tüüpsõnamall \arabic{mallinumber}\,\vadja{ičä}}\\

\begin{sideways}
\begin{tabular}{l l}
muutvormimall & tunnused \\
\hline
\underline{i}\,$\oplus$\,čä & \textsc{ sg nom } \\
\underline{i}\,$\oplus$\,ä & \textsc{ sg gen } \\
\underline{i}\,$\oplus$\,ččä & \textsc{ sg par } \\
\underline{i}\,$\oplus$\,ččäse & \textsc{ sg ill } \\
\underline{i}\,$\oplus$\,äz & \textsc{ sg ine } \\
\underline{i}\,$\oplus$\,äss & \textsc{ sg ela } \\
\underline{i}\,$\oplus$\,älle & \textsc{ sg all } \\
\underline{i}\,$\oplus$\,äll & \textsc{ sg ade } \\
\underline{i}\,$\oplus$\,ält & \textsc{ sg abl } \\
\underline{i}\,$\oplus$\,ässi & \textsc{ sg tra } \\
\underline{i}\,$\oplus$\,ässaa & \textsc{ sg ter } \\
\underline{i}\,$\oplus$\,äka & \textsc{ sg com } \\
\underline{i}\,$\oplus$\,äd & \textsc{ pl nom } \\
\underline{i}\,$\oplus$\,čije & \textsc{ pl gen } \\
\underline{i}\,$\oplus$\,čiit & \textsc{ pl par } \\
\underline{i}\,$\oplus$\,čiise & \textsc{ pl ill } \\
\underline{i}\,$\oplus$\,čiiz & \textsc{ pl ine } \\
\underline{i}\,$\oplus$\,čiiss & \textsc{ pl ela } \\
\underline{i}\,$\oplus$\,čiille & \textsc{ pl all } \\
\underline{i}\,$\oplus$\,čiill & \textsc{ pl ade } \\
\underline{i}\,$\oplus$\,čiilt & \textsc{ pl abl } \\
\underline{i}\,$\oplus$\,čiissi & \textsc{ pl tra } \\
\underline{i}\,$\oplus$\,čiissaa & \textsc{ pl ter } \\
\underline{i}\,$\oplus$\,čijka & \textsc{ pl com } \\
\end{tabular}
\end{sideways}
\captionof{table}{Tüüpsõnamall \arabic{mallinumber}\,\vadja{ičä} ekstraheeritud muutvormimallid.}
\label{tab:tüüpsõnamall-ičä}

\end{minipage}

 
\vspace{1em}
\noindent Tüüpsõnamall \vadja{ičä} ei hõlma teisi lekseeme vormi\-sõnastikus.

Tüüp\-sõna\-mall kirjeldab eespoolseid sõnu tüvemuutusega č:∅.


\vspace{3.5em}
\noindent \begin{minipage}{\textwidth}
\stepcounter{mallinumber}
\noindent \textbf{Tüüpsõnamall \arabic{mallinumber}\,\vadja{selče}}\\

\begin{sideways}
\begin{tabular}{l l}
muutvormimall & tunnused \\
\hline
\underline{sel}\,$\oplus$\,če & \textsc{ sg nom } \\
\underline{sel}\,$\oplus$\,lä & \textsc{ sg gen } \\
\underline{sel}\,$\oplus$\,čä & \textsc{ sg par } \\
\underline{sel}\,$\oplus$\,čäse & \textsc{ sg ill } \\
\underline{sel}\,$\oplus$\,lez & \textsc{ sg ine } \\
\underline{sel}\,$\oplus$\,less & \textsc{ sg ela } \\
\underline{sel}\,$\oplus$\,lelle & \textsc{ sg all } \\
\underline{sel}\,$\oplus$\,lell & \textsc{ sg ade } \\
\underline{sel}\,$\oplus$\,lelt & \textsc{ sg abl } \\
\underline{sel}\,$\oplus$\,lessi & \textsc{ sg tra } \\
\underline{sel}\,$\oplus$\,lessaa & \textsc{ sg ter } \\
\underline{sel}\,$\oplus$\,leka & \textsc{ sg com } \\
\underline{sel}\,$\oplus$\,led & \textsc{ pl nom } \\
\underline{sel}\,$\oplus$\,čije & \textsc{ pl gen } \\
\underline{sel}\,$\oplus$\,čiit & \textsc{ pl par } \\
\underline{sel}\,$\oplus$\,čiise & \textsc{ pl ill } \\
\underline{sel}\,$\oplus$\,čiiz & \textsc{ pl ine } \\
\underline{sel}\,$\oplus$\,čiiss & \textsc{ pl ela } \\
\underline{sel}\,$\oplus$\,čiille & \textsc{ pl all } \\
\underline{sel}\,$\oplus$\,čiill & \textsc{ pl ade } \\
\underline{sel}\,$\oplus$\,čiilt & \textsc{ pl abl } \\
\underline{sel}\,$\oplus$\,čiissi & \textsc{ pl tra } \\
\underline{sel}\,$\oplus$\,čiissaa & \textsc{ pl ter } \\
\underline{sel}\,$\oplus$\,čijka & \textsc{ pl com } \\
\end{tabular}
\end{sideways}
\captionof{table}{Tüüpsõnamall \arabic{mallinumber}\,\vadja{selče} ekstraheeritud muutvormimallid.}
\label{tab:tüüpsõnamall-selče}

\end{minipage}

 
\vspace{1em}
\noindent Tüüpsõnamall \vadja{selče} ei hõlma teisi lekseeme vormi\-sõnastikus.

Tüüp\-sõna\-mall kirjeldab eespoolseid sõnu tüvemuutusega lč:ll.


\vspace{3.5em}
\noindent \begin{minipage}{\textwidth}
\stepcounter{mallinumber}
\noindent \textbf{Tüüpsõnamall \arabic{mallinumber}\,\vadja{eine}}\\

\begin{sideways}
\begin{tabular}{l l}
muutvormimall & tunnused \\
\hline
\underline{ein}\,$\oplus$\,e & \textsc{ sg nom } \\
\underline{ein}\,$\oplus$\,ä & \textsc{ sg gen } \\
\underline{ein}\,$\oplus$\,ä & \textsc{ sg par } \\
\underline{ein}\,$\oplus$\,äse & \textsc{ sg ill } \\
\underline{ein}\,$\oplus$\,ez & \textsc{ sg ine } \\
\underline{ein}\,$\oplus$\,ess & \textsc{ sg ela } \\
\underline{ein}\,$\oplus$\,elle & \textsc{ sg all } \\
\underline{ein}\,$\oplus$\,ell & \textsc{ sg ade } \\
\underline{ein}\,$\oplus$\,elt & \textsc{ sg abl } \\
\underline{ein}\,$\oplus$\,essi & \textsc{ sg tra } \\
\underline{ein}\,$\oplus$\,essaa & \textsc{ sg ter } \\
\underline{ein}\,$\oplus$\,eka & \textsc{ sg com } \\
\underline{ein}\,$\oplus$\,ed & \textsc{ pl nom } \\
\underline{ein}\,$\oplus$\,ije & \textsc{ pl gen } \\
\underline{ein}\,$\oplus$\,iit & \textsc{ pl par } \\
\underline{ein}\,$\oplus$\,iise & \textsc{ pl ill } \\
\underline{ein}\,$\oplus$\,iiz & \textsc{ pl ine } \\
\underline{ein}\,$\oplus$\,iiss & \textsc{ pl ela } \\
\underline{ein}\,$\oplus$\,iille & \textsc{ pl all } \\
\underline{ein}\,$\oplus$\,iill & \textsc{ pl ade } \\
\underline{ein}\,$\oplus$\,iilt & \textsc{ pl abl } \\
\underline{ein}\,$\oplus$\,iissi & \textsc{ pl tra } \\
\underline{ein}\,$\oplus$\,iissaa & \textsc{ pl ter } \\
\underline{ein}\,$\oplus$\,ijka & \textsc{ pl com } \\
\end{tabular}
\end{sideways}
\captionof{table}{Tüüpsõnamall \arabic{mallinumber}\,\vadja{eine} ekstraheeritud muutvormimallid.}
\label{tab:tüüpsõnamall-eine}

\end{minipage}

 
\vspace{1em}
\noindent Tüüpsõnamall \vadja{eine} hõlmab vormisõnastikus 20 lekseemi: \vadja{\underline{ein}e, \underline{esimespäiv}e, \underline{fökl}e, \underline{irviein}e, \underline{lehm}e, \underline{läikkiv}e, \underline{läsiv}e, \underline{nätil̕päiv}e, \underline{petäj}e, \underline{piim}e, \underline{pominpäiv}e, \underline{pädr}e, \underline{päiv}e, \underline{rehtel}e, \underline{sein}e, \underline{silm}e, \underline{tühj}e, \underline{äjj}e, \underline{ämm}e} ja \vadja{\underline{čülm}e}.

Tüüp\-sõna\-mall kirjeldab eespoolseid tüvemuutuseta sõnu, mille lõpuvokaal on \textit{e} ja mitmuse tüvi on -\textit{ii}-.


\vspace{3.5em}
\noindent \begin{minipage}{\textwidth}
\stepcounter{mallinumber}
\noindent \textbf{Tüüpsõnamall \arabic{mallinumber}\,\vadja{läkine}}\\

\begin{sideways}
\begin{tabular}{l l}
muutvormimall & tunnused \\
\hline
\underline{läkin}\,$\oplus$\,e & \textsc{ sg nom } \\
\underline{läkin}\,$\oplus$\,ä & \textsc{ sg gen } \\
\underline{läkin}\,$\oplus$\,ä & \textsc{ sg par } \\
\underline{läkin}\,$\oplus$\,äse & \textsc{ sg ill } \\
\underline{läkin}\,$\oplus$\,ez & \textsc{ sg ine } \\
\underline{läkin}\,$\oplus$\,ess & \textsc{ sg ela } \\
\underline{läkin}\,$\oplus$\,elle & \textsc{ sg all } \\
\underline{läkin}\,$\oplus$\,ell & \textsc{ sg ade } \\
\underline{läkin}\,$\oplus$\,elt & \textsc{ sg abl } \\
\underline{läkin}\,$\oplus$\,essi & \textsc{ sg tra } \\
\underline{läkin}\,$\oplus$\,essaa & \textsc{ sg ter } \\
\underline{läkin}\,$\oplus$\,eka & \textsc{ sg com } \\
\underline{läkin}\,$\oplus$\,ed & \textsc{ pl nom } \\
\underline{läkin}\,$\oplus$\,eje & \textsc{ pl gen } \\
\underline{läkin}\,$\oplus$\,eit & \textsc{ pl par } \\
\underline{läkin}\,$\oplus$\,eise & \textsc{ pl ill } \\
\underline{läkin}\,$\oplus$\,eiz & \textsc{ pl ine } \\
\underline{läkin}\,$\oplus$\,eiss & \textsc{ pl ela } \\
\underline{läkin}\,$\oplus$\,eille & \textsc{ pl all } \\
\underline{läkin}\,$\oplus$\,eill & \textsc{ pl ade } \\
\underline{läkin}\,$\oplus$\,eilt & \textsc{ pl abl } \\
\underline{läkin}\,$\oplus$\,eissi & \textsc{ pl tra } \\
\underline{läkin}\,$\oplus$\,eissaa & \textsc{ pl ter } \\
\underline{läkin}\,$\oplus$\,eika & \textsc{ pl com } \\
\end{tabular}
\end{sideways}
\captionof{table}{Tüüpsõnamall \arabic{mallinumber}\,\vadja{läkine} ekstraheeritud muutvormimallid.}
\label{tab:tüüpsõnamall-läkine}

\end{minipage}

 
\vspace{1em}
\noindent Tüüpsõnamall \vadja{läkine} hõlmab vormisõnastikus 2 lekseemi: \vadja{\underline{läkin}e} ja \vadja{\underline{dääd}e}.

Tüüp\-sõna\-mall kirjeldab eespoolseid tüvemuutuseta sõnu, mille lõpuvokaal on \textit{e} ja mitmuse tüvi on -\textit{ei}-.


\vspace{3.5em}
\noindent \begin{minipage}{\textwidth}
\stepcounter{mallinumber}
\noindent \textbf{Tüüpsõnamall \arabic{mallinumber}\,\vadja{räpäle}}\\

\begin{sideways}
\begin{tabular}{l l}
muutvormimall & tunnused \\
\hline
\underline{räpäl}\,$\oplus$\,e & \textsc{ sg nom } \\
\underline{räpäl}\,$\oplus$\,ä & \textsc{ sg gen } \\
\underline{räpäl}\,$\oplus$\,ä & \textsc{ sg par } \\
\underline{räpäl}\,$\oplus$\,äse & \textsc{ sg ill } \\
\underline{räpäl}\,$\oplus$\,ez & \textsc{ sg ine } \\
\underline{räpäl}\,$\oplus$\,ess & \textsc{ sg ela } \\
\underline{räpäl}\,$\oplus$\,elle & \textsc{ sg all } \\
\underline{räpäl}\,$\oplus$\,ell & \textsc{ sg ade } \\
\underline{räpäl}\,$\oplus$\,elt & \textsc{ sg abl } \\
\underline{räpäl}\,$\oplus$\,essi & \textsc{ sg tra } \\
\underline{räpäl}\,$\oplus$\,essaa & \textsc{ sg ter } \\
\underline{räpäl}\,$\oplus$\,eka & \textsc{ sg com } \\
\underline{räpäl}\,$\oplus$\,ed & \textsc{ pl nom } \\
\underline{räpäl}\,$\oplus$\,öje & \textsc{ pl gen } \\
\underline{räpäl}\,$\oplus$\,öit & \textsc{ pl par } \\
\underline{räpäl}\,$\oplus$\,öise & \textsc{ pl ill } \\
\underline{räpäl}\,$\oplus$\,öiz & \textsc{ pl ine } \\
\underline{räpäl}\,$\oplus$\,öiss & \textsc{ pl ela } \\
\underline{räpäl}\,$\oplus$\,öille & \textsc{ pl all } \\
\underline{räpäl}\,$\oplus$\,öill & \textsc{ pl ade } \\
\underline{räpäl}\,$\oplus$\,öilt & \textsc{ pl abl } \\
\underline{räpäl}\,$\oplus$\,öissi & \textsc{ pl tra } \\
\underline{räpäl}\,$\oplus$\,öissaa & \textsc{ pl ter } \\
\underline{räpäl}\,$\oplus$\,öika & \textsc{ pl com } \\
\end{tabular}
\end{sideways}
\captionof{table}{Tüüpsõnamall \arabic{mallinumber}\,\vadja{räpäle} ekstraheeritud muutvormimallid.}
\label{tab:tüüpsõnamall-räpäle}

\end{minipage}

 
\vspace{1em}
\noindent Tüüpsõnamall \vadja{räpäle} ei hõlma teisi lekseeme vormi\-sõnastikus.

Tüüp\-sõna\-mall kirjeldab eespoolseid tüvemuutuseta sõnu, mille lõpuvokaal on \textit{e} ja mitmuse tüvi on -\textit{öi}-.


\vspace{3.5em}
\noindent \begin{minipage}{\textwidth}
\stepcounter{mallinumber}
\noindent \textbf{Tüüpsõnamall \arabic{mallinumber}\,\vadja{rissimä}}\\

\begin{sideways}
\begin{tabular}{l l}
muutvormimall & tunnused \\
\hline
\underline{rissim}\,$\oplus$\,ä & \textsc{ sg nom } \\
\underline{rissim}\,$\oplus$\,ä & \textsc{ sg gen } \\
\underline{rissim}\,$\oplus$\,ä & \textsc{ sg par } \\
\underline{rissim}\,$\oplus$\,äse & \textsc{ sg ill } \\
\underline{rissim}\,$\oplus$\,äz & \textsc{ sg ine } \\
\underline{rissim}\,$\oplus$\,äss & \textsc{ sg ela } \\
\underline{rissim}\,$\oplus$\,älle & \textsc{ sg all } \\
\underline{rissim}\,$\oplus$\,äll & \textsc{ sg ade } \\
\underline{rissim}\,$\oplus$\,ält & \textsc{ sg abl } \\
\underline{rissim}\,$\oplus$\,ässi & \textsc{ sg tra } \\
\underline{rissim}\,$\oplus$\,ässaa & \textsc{ sg ter } \\
\underline{rissim}\,$\oplus$\,äka & \textsc{ sg com } \\
\underline{rissim}\,$\oplus$\,äd & \textsc{ pl nom } \\
\underline{rissim}\,$\oplus$\,ije & \textsc{ pl gen } \\
\underline{rissim}\,$\oplus$\,it & \textsc{ pl par } \\
\underline{rissim}\,$\oplus$\,ise & \textsc{ pl ill } \\
\underline{rissim}\,$\oplus$\,iz & \textsc{ pl ine } \\
\underline{rissim}\,$\oplus$\,iss & \textsc{ pl ela } \\
\underline{rissim}\,$\oplus$\,ille & \textsc{ pl all } \\
\underline{rissim}\,$\oplus$\,ill & \textsc{ pl ade } \\
\underline{rissim}\,$\oplus$\,ilt & \textsc{ pl abl } \\
\underline{rissim}\,$\oplus$\,issi & \textsc{ pl tra } \\
\underline{rissim}\,$\oplus$\,issaa & \textsc{ pl ter } \\
\underline{rissim}\,$\oplus$\,ika & \textsc{ pl com } \\
\end{tabular}
\end{sideways}
\captionof{table}{Tüüpsõnamall \arabic{mallinumber}\,\vadja{rissimä} ekstraheeritud muutvormimallid.}
\label{tab:tüüpsõnamall-rissimä}

\end{minipage}

 
\vspace{1em}
\noindent Tüüpsõnamall \vadja{rissimä} hõlmab vormisõnastikus 2 lekseemi: \vadja{\underline{rissim}ä} ja \vadja{\underline{em}ä}.

Tüüp\-sõna\-mall kirjeldab eespoolseid tüvemuutuseta sõnu, mille lõpuvokaal on \textit{ä}.


\vspace{3.5em}
\noindent \begin{minipage}{\textwidth}
\stepcounter{mallinumber}
\noindent \textbf{Tüüpsõnamall \arabic{mallinumber}\,\vadja{pähčen}}\\

\begin{sideways}
\begin{tabular}{l l}
muutvormimall & tunnused \\
\hline
\underline{pähčen} & \textsc{ sg nom } \\
\underline{pähčen}\,$\oplus$\,ä & \textsc{ sg gen } \\
\underline{pähčen}\,$\oplus$\,ä & \textsc{ sg par } \\
\underline{pähčen}\,$\oplus$\,äse & \textsc{ sg ill } \\
\underline{pähčen}\,$\oplus$\,ez & \textsc{ sg ine } \\
\underline{pähčen}\,$\oplus$\,ess & \textsc{ sg ela } \\
\underline{pähčen}\,$\oplus$\,elle & \textsc{ sg all } \\
\underline{pähčen}\,$\oplus$\,ell & \textsc{ sg ade } \\
\underline{pähčen}\,$\oplus$\,elt & \textsc{ sg abl } \\
\underline{pähčen}\,$\oplus$\,essi & \textsc{ sg tra } \\
\underline{pähčen}\,$\oplus$\,essaa & \textsc{ sg ter } \\
\underline{pähčen}\,$\oplus$\,eka & \textsc{ sg com } \\
\underline{pähčen}\,$\oplus$\,ed & \textsc{ pl nom } \\
\underline{pähčen}\,$\oplus$\,ije & \textsc{ pl gen } \\
\underline{pähčen}\,$\oplus$\,iit & \textsc{ pl par } \\
\underline{pähčen}\,$\oplus$\,iise & \textsc{ pl ill } \\
\underline{pähčen}\,$\oplus$\,iiz & \textsc{ pl ine } \\
\underline{pähčen}\,$\oplus$\,iiss & \textsc{ pl ela } \\
\underline{pähčen}\,$\oplus$\,iille & \textsc{ pl all } \\
\underline{pähčen}\,$\oplus$\,iill & \textsc{ pl ade } \\
\underline{pähčen}\,$\oplus$\,iilt & \textsc{ pl abl } \\
\underline{pähčen}\,$\oplus$\,iissi & \textsc{ pl tra } \\
\underline{pähčen}\,$\oplus$\,iissaa & \textsc{ pl ter } \\
\underline{pähčen}\,$\oplus$\,ijka & \textsc{ pl com } \\
\end{tabular}
\end{sideways}
\captionof{table}{Tüüpsõnamall \arabic{mallinumber}\,\vadja{pähčen} ekstraheeritud muutvormimallid.}
\label{tab:tüüpsõnamall-pähčen}

\end{minipage}

 
\vspace{1em}
\noindent Tüüpsõnamall \vadja{pähčen} hõlmab vormisõnastikus 2 lekseemi: \vadja{\underline{pähčen}} ja \vadja{\underline{ičäv}}.

Tüüp\-sõna\-mall kirjeldab eespoolseid tüvemuutuseta sõnu, millel puudub lõpuvokaal.


\vspace{3.5em}
\noindent \begin{minipage}{\textwidth}
\stepcounter{mallinumber}
\noindent \textbf{Tüüpsõnamall \arabic{mallinumber}\,\vadja{läsijä}}\\

\begin{sideways}
\begin{tabular}{l l}
muutvormimall & tunnused \\
\hline
\underline{läsij}\,$\oplus$\,ä & \textsc{ sg nom } \\
\underline{läsij}\,$\oplus$\,ä & \textsc{ sg gen } \\
\underline{läsij}\,$\oplus$\,ä & \textsc{ sg par } \\
\underline{läsij}\,$\oplus$\,äse & \textsc{ sg ill } \\
\underline{läsij}\,$\oplus$\,ez & \textsc{ sg ine } \\
\underline{läsij}\,$\oplus$\,ess & \textsc{ sg ela } \\
\underline{läsij}\,$\oplus$\,elle & \textsc{ sg all } \\
\underline{läsij}\,$\oplus$\,ell & \textsc{ sg ade } \\
\underline{läsij}\,$\oplus$\,elt & \textsc{ sg abl } \\
\underline{läsij}\,$\oplus$\,essi & \textsc{ sg tra } \\
\underline{läsij}\,$\oplus$\,essaa & \textsc{ sg ter } \\
\underline{läsij}\,$\oplus$\,eka & \textsc{ sg com } \\
\underline{läsij}\,$\oplus$\,ed & \textsc{ pl nom } \\
\underline{läsij}\,$\oplus$\,ije & \textsc{ pl gen } \\
\underline{läsij}\,$\oplus$\,iit & \textsc{ pl par } \\
\underline{läsij}\,$\oplus$\,iise & \textsc{ pl ill } \\
\underline{läsij}\,$\oplus$\,iiz & \textsc{ pl ine } \\
\underline{läsij}\,$\oplus$\,iiss & \textsc{ pl ela } \\
\underline{läsij}\,$\oplus$\,iille & \textsc{ pl all } \\
\underline{läsij}\,$\oplus$\,iill & \textsc{ pl ade } \\
\underline{läsij}\,$\oplus$\,iilt & \textsc{ pl abl } \\
\underline{läsij}\,$\oplus$\,iissi & \textsc{ pl tra } \\
\underline{läsij}\,$\oplus$\,iissaa & \textsc{ pl ter } \\
\underline{läsij}\,$\oplus$\,ijka & \textsc{ pl com } \\
\end{tabular}
\end{sideways}
\captionof{table}{Tüüpsõnamall \arabic{mallinumber}\,\vadja{läsijä} ekstraheeritud muutvormimallid.}
\label{tab:tüüpsõnamall-läsijä}

\end{minipage}

 
\vspace{1em}
\noindent Tüüpsõnamall \vadja{läsijä} hõlmab vormisõnastikus 4 lekseemi: \vadja{\underline{läsij}ä, \underline{müüj}ä, \underline{tečej}ä} ja \vadja{\underline{köüh}ä}.

Tüüp\-sõna\-mall kirjeldab eespoolseid tüvemuutuseta sõnu.


\vspace{3.5em}
\noindent \begin{minipage}{\textwidth}
\stepcounter{mallinumber}
\noindent \textbf{Tüüpsõnamall \arabic{mallinumber}\,\vadja{kitai}}\\

\begin{sideways}
\begin{tabular}{l l}
muutvormimall & tunnused \\
\hline
\underline{kita}\,$\oplus$\,i & \textsc{ sg nom } \\
\underline{kita}\,$\oplus$\,ja & \textsc{ sg gen } \\
\underline{kita}\,$\oplus$\,ja & \textsc{ sg par } \\
\underline{kita}\,$\oplus$\,jasõ & \textsc{ sg ill } \\
\underline{kita}\,$\oplus$\,jaz & \textsc{ sg ine } \\
\underline{kita}\,$\oplus$\,jass & \textsc{ sg ela } \\
\underline{kita}\,$\oplus$\,jallõ & \textsc{ sg all } \\
\underline{kita}\,$\oplus$\,jall & \textsc{ sg ade } \\
\underline{kita}\,$\oplus$\,jalt & \textsc{ sg abl } \\
\underline{kita}\,$\oplus$\,jassi & \textsc{ sg tra } \\
\underline{kita}\,$\oplus$\,jassaa & \textsc{ sg ter } \\
\underline{kita}\,$\oplus$\,jaka & \textsc{ sg com } \\
\underline{kita}\,$\oplus$\,jad & \textsc{ pl nom } \\
\underline{kita}\,$\oplus$\,jojõ & \textsc{ pl gen } \\
\underline{kita}\,$\oplus$\,joit & \textsc{ pl par } \\
\underline{kita}\,$\oplus$\,joisõ & \textsc{ pl ill } \\
\underline{kita}\,$\oplus$\,joiz & \textsc{ pl ine } \\
\underline{kita}\,$\oplus$\,joiss & \textsc{ pl ela } \\
\underline{kita}\,$\oplus$\,joillõ & \textsc{ pl all } \\
\underline{kita}\,$\oplus$\,joill & \textsc{ pl ade } \\
\underline{kita}\,$\oplus$\,joilt & \textsc{ pl abl } \\
\underline{kita}\,$\oplus$\,joissi & \textsc{ pl tra } \\
\underline{kita}\,$\oplus$\,joissaa & \textsc{ pl ter } \\
\underline{kita}\,$\oplus$\,joika & \textsc{ pl com } \\
\end{tabular}
\end{sideways}
\captionof{table}{Tüüpsõnamall \arabic{mallinumber}\,\vadja{kitai} ekstraheeritud muutvormimallid.}
\label{tab:tüüpsõnamall-kitai}

\end{minipage}

 
\vspace{1em}
\noindent Tüüpsõnamall \vadja{kitai} ei hõlma teisi lekseeme vormi\-sõnastikus.

Tüüp\-sõna\-mall kirjeldab tagapoolseid tüvemuutuseta sõnu.


\vspace{3.5em}
\noindent \begin{minipage}{\textwidth}
\stepcounter{mallinumber}
\noindent \textbf{Tüüpsõnamall \arabic{mallinumber}\,\vadja{slona}}\\

\begin{sideways}
\begin{tabular}{l l}
muutvormimall & tunnused \\
\hline
\underline{slon}\,$\oplus$\,a & \textsc{ sg nom } \\
\underline{slon}\,$\oplus$\,a & \textsc{ sg gen } \\
\underline{slon}\,$\oplus$\,a & \textsc{ sg par } \\
\underline{slon}\,$\oplus$\,asõ & \textsc{ sg ill } \\
\underline{slon}\,$\oplus$\,az & \textsc{ sg ine } \\
\underline{slon}\,$\oplus$\,ass & \textsc{ sg ela } \\
\underline{slon}\,$\oplus$\,allõ & \textsc{ sg all } \\
\underline{slon}\,$\oplus$\,all & \textsc{ sg ade } \\
\underline{slon}\,$\oplus$\,alt & \textsc{ sg abl } \\
\underline{slon}\,$\oplus$\,assi & \textsc{ sg tra } \\
\underline{slon}\,$\oplus$\,assaa & \textsc{ sg ter } \\
\underline{slon}\,$\oplus$\,aka & \textsc{ sg com } \\
\underline{slon}\,$\oplus$\,ad & \textsc{ pl nom } \\
\underline{slon}\,$\oplus$\,ijõ & \textsc{ pl gen } \\
\underline{slon}\,$\oplus$\,õit & \textsc{ pl par } \\
\underline{slon}\,$\oplus$\,õisõ & \textsc{ pl ill } \\
\underline{slon}\,$\oplus$\,õiz & \textsc{ pl ine } \\
\underline{slon}\,$\oplus$\,õiss & \textsc{ pl ela } \\
\underline{slon}\,$\oplus$\,õillõ & \textsc{ pl all } \\
\underline{slon}\,$\oplus$\,õill & \textsc{ pl ade } \\
\underline{slon}\,$\oplus$\,õilt & \textsc{ pl abl } \\
\underline{slon}\,$\oplus$\,õissi & \textsc{ pl tra } \\
\underline{slon}\,$\oplus$\,õissaa & \textsc{ pl ter } \\
\underline{slon}\,$\oplus$\,õika & \textsc{ pl com } \\
\end{tabular}
\end{sideways}
\captionof{table}{Tüüpsõnamall \arabic{mallinumber}\,\vadja{slona} ekstraheeritud muutvormimallid.}
\label{tab:tüüpsõnamall-slona}

\end{minipage}

 
\vspace{1em}
\noindent Tüüpsõnamall \vadja{slona} ei hõlma teisi lekseeme vormi\-sõnastikus.

Tüüp\-sõna\-mall kirjeldab tagapoolseid tüvemuutuseta sõnu.


\vspace{3.5em}
\noindent \begin{minipage}{\textwidth}
\stepcounter{mallinumber}
\noindent \textbf{Tüüpsõnamall \arabic{mallinumber}\,\vadja{õmpõlija}}\\

\begin{sideways}
\begin{tabular}{l l}
muutvormimall & tunnused \\
\hline
\underline{õmpõlij}\,$\oplus$\,a & \textsc{ sg nom } \\
\underline{õmpõlij}\,$\oplus$\,a & \textsc{ sg gen } \\
\underline{õmpõlij}\,$\oplus$\,a & \textsc{ sg par } \\
\underline{õmpõlij}\,$\oplus$\,asõ & \textsc{ sg ill } \\
\underline{õmpõlij}\,$\oplus$\,az & \textsc{ sg ine } \\
\underline{õmpõlij}\,$\oplus$\,ass & \textsc{ sg ela } \\
\underline{õmpõlij}\,$\oplus$\,allõ & \textsc{ sg all } \\
\underline{õmpõlij}\,$\oplus$\,all & \textsc{ sg ade } \\
\underline{õmpõlij}\,$\oplus$\,alt & \textsc{ sg abl } \\
\underline{õmpõlij}\,$\oplus$\,assi & \textsc{ sg tra } \\
\underline{õmpõlij}\,$\oplus$\,assaa & \textsc{ sg ter } \\
\underline{õmpõlij}\,$\oplus$\,aka & \textsc{ sg com } \\
\underline{õmpõlij}\,$\oplus$\,ad & \textsc{ pl nom } \\
\underline{õmpõlij}\,$\oplus$\,ijõ & \textsc{ pl gen } \\
\underline{õmpõlij}\,$\oplus$\,ait & \textsc{ pl par } \\
\underline{õmpõlij}\,$\oplus$\,aisõ & \textsc{ pl ill } \\
\underline{õmpõlij}\,$\oplus$\,aiz & \textsc{ pl ine } \\
\underline{õmpõlij}\,$\oplus$\,aiss & \textsc{ pl ela } \\
\underline{õmpõlij}\,$\oplus$\,aillõ & \textsc{ pl all } \\
\underline{õmpõlij}\,$\oplus$\,aill & \textsc{ pl ade } \\
\underline{õmpõlij}\,$\oplus$\,ailt & \textsc{ pl abl } \\
\underline{õmpõlij}\,$\oplus$\,aissi & \textsc{ pl tra } \\
\underline{õmpõlij}\,$\oplus$\,aissaa & \textsc{ pl ter } \\
\underline{õmpõlij}\,$\oplus$\,aika & \textsc{ pl com } \\
\end{tabular}
\end{sideways}
\captionof{table}{Tüüpsõnamall \arabic{mallinumber}\,\vadja{õmpõlija} ekstraheeritud muutvormimallid.}
\label{tab:tüüpsõnamall-õmpõlija}

\end{minipage}

 
\vspace{1em}
\noindent Tüüpsõnamall \vadja{õmpõlija} ei hõlma teisi lekseeme vormi\-sõnastikus.

Tüüp\-sõna\-mall kirjeldab tagapoolseid tüvemuutuseta sõnu.


\vspace{3.5em}
\noindent \begin{minipage}{\textwidth}
\stepcounter{mallinumber}
\noindent \textbf{Tüüpsõnamall \arabic{mallinumber}\,\vadja{čenče}}\\

\begin{sideways}
\begin{tabular}{l l}
muutvormimall & tunnused \\
\hline
\underline{čen}\,$\oplus$\,če & \textsc{ sg nom } \\
\underline{čen}\,$\oplus$\,nä & \textsc{ sg gen } \\
\underline{čen}\,$\oplus$\,čä & \textsc{ sg par } \\
\underline{čen}\,$\oplus$\,čäse & \textsc{ sg ill } \\
\underline{čen}\,$\oplus$\,nez & \textsc{ sg ine } \\
\underline{čen}\,$\oplus$\,ness & \textsc{ sg ela } \\
\underline{čen}\,$\oplus$\,nelle & \textsc{ sg all } \\
\underline{čen}\,$\oplus$\,nell & \textsc{ sg ade } \\
\underline{čen}\,$\oplus$\,nelt & \textsc{ sg abl } \\
\underline{čen}\,$\oplus$\,nessi & \textsc{ sg tra } \\
\underline{čen}\,$\oplus$\,nessaa & \textsc{ sg ter } \\
\underline{čen}\,$\oplus$\,neka & \textsc{ sg com } \\
\underline{čen}\,$\oplus$\,ned & \textsc{ pl nom } \\
\underline{čen}\,$\oplus$\,čije & \textsc{ pl gen } \\
\underline{čen}\,$\oplus$\,čiit & \textsc{ pl par } \\
\underline{čen}\,$\oplus$\,čiise & \textsc{ pl ill } \\
\underline{čen}\,$\oplus$\,čiiz & \textsc{ pl ine } \\
\underline{čen}\,$\oplus$\,čiiss & \textsc{ pl ela } \\
\underline{čen}\,$\oplus$\,čiille & \textsc{ pl all } \\
\underline{čen}\,$\oplus$\,čiill & \textsc{ pl ade } \\
\underline{čen}\,$\oplus$\,čiilt & \textsc{ pl abl } \\
\underline{čen}\,$\oplus$\,čiissi & \textsc{ pl tra } \\
\underline{čen}\,$\oplus$\,čiissaa & \textsc{ pl ter } \\
\underline{čen}\,$\oplus$\,čijka & \textsc{ pl com } \\
\end{tabular}
\end{sideways}
\captionof{table}{Tüüpsõnamall \arabic{mallinumber}\,\vadja{čenče} ekstraheeritud muutvormimallid.}
\label{tab:tüüpsõnamall-čenče}

\end{minipage}

 
\vspace{1em}
\noindent Tüüpsõnamall \vadja{čenče} ei hõlma teisi lekseeme vormi\-sõnastikus.

Tüüp\-sõna\-mall kirjeldab eespoolseid sõnu tüvemuutusega nč:nn.


\vspace{3.5em}
\noindent \begin{minipage}{\textwidth}
\stepcounter{mallinumber}
\noindent \textbf{Tüüpsõnamall \arabic{mallinumber}\,\vadja{tünke}}\\

\begin{sideways}
\begin{tabular}{l l}
muutvormimall & tunnused \\
\hline
\underline{tün}\,$\oplus$\,ke & \textsc{ sg nom } \\
\underline{tün}\,$\oplus$\,gä & \textsc{ sg gen } \\
\underline{tün}\,$\oplus$\,kä & \textsc{ sg par } \\
\underline{tün}\,$\oplus$\,käse & \textsc{ sg ill } \\
\underline{tün}\,$\oplus$\,gez & \textsc{ sg ine } \\
\underline{tün}\,$\oplus$\,gess & \textsc{ sg ela } \\
\underline{tün}\,$\oplus$\,gelle & \textsc{ sg all } \\
\underline{tün}\,$\oplus$\,gell & \textsc{ sg ade } \\
\underline{tün}\,$\oplus$\,gelt & \textsc{ sg abl } \\
\underline{tün}\,$\oplus$\,gessi & \textsc{ sg tra } \\
\underline{tün}\,$\oplus$\,gessaa & \textsc{ sg ter } \\
\underline{tün}\,$\oplus$\,geka & \textsc{ sg com } \\
\underline{tün}\,$\oplus$\,ged & \textsc{ pl nom } \\
\underline{tün}\,$\oplus$\,kije & \textsc{ pl gen } \\
\underline{tün}\,$\oplus$\,kiit & \textsc{ pl par } \\
\underline{tün}\,$\oplus$\,kiise & \textsc{ pl ill } \\
\underline{tün}\,$\oplus$\,kiiz & \textsc{ pl ine } \\
\underline{tün}\,$\oplus$\,kiiss & \textsc{ pl ela } \\
\underline{tün}\,$\oplus$\,kiille & \textsc{ pl all } \\
\underline{tün}\,$\oplus$\,kiill & \textsc{ pl ade } \\
\underline{tün}\,$\oplus$\,kiilt & \textsc{ pl abl } \\
\underline{tün}\,$\oplus$\,kiissi & \textsc{ pl tra } \\
\underline{tün}\,$\oplus$\,kiissaa & \textsc{ pl ter } \\
\underline{tün}\,$\oplus$\,kijka & \textsc{ pl com } \\
\end{tabular}
\end{sideways}
\captionof{table}{Tüüpsõnamall \arabic{mallinumber}\,\vadja{tünke} ekstraheeritud muutvormimallid.}
\label{tab:tüüpsõnamall-tünke}

\end{minipage}

 
\vspace{1em}
\noindent Tüüpsõnamall \vadja{tünke} ei hõlma teisi lekseeme vormi\-sõnastikus.

Tüüp\-sõna\-mall kirjeldab eespoolseid sõnu tüvemuutusega nk:ng.


\vspace{3.5em}
\noindent \begin{minipage}{\textwidth}
\stepcounter{mallinumber}
\noindent \textbf{Tüüpsõnamall \arabic{mallinumber}\,\vadja{änte}}\\

\begin{sideways}
\begin{tabular}{l l}
muutvormimall & tunnused \\
\hline
\underline{än}\,$\oplus$\,te & \textsc{ sg nom } \\
\underline{än}\,$\oplus$\,nä & \textsc{ sg gen } \\
\underline{än}\,$\oplus$\,tä & \textsc{ sg par } \\
\underline{än}\,$\oplus$\,täse & \textsc{ sg ill } \\
\underline{än}\,$\oplus$\,nez & \textsc{ sg ine } \\
\underline{än}\,$\oplus$\,ness & \textsc{ sg ela } \\
\underline{än}\,$\oplus$\,nelle & \textsc{ sg all } \\
\underline{än}\,$\oplus$\,nell & \textsc{ sg ade } \\
\underline{än}\,$\oplus$\,nelt & \textsc{ sg abl } \\
\underline{än}\,$\oplus$\,nessi & \textsc{ sg tra } \\
\underline{än}\,$\oplus$\,nessaa & \textsc{ sg ter } \\
\underline{än}\,$\oplus$\,neka & \textsc{ sg com } \\
\underline{än}\,$\oplus$\,ned & \textsc{ pl nom } \\
\underline{än}\,$\oplus$\,tije & \textsc{ pl gen } \\
\underline{än}\,$\oplus$\,tiit & \textsc{ pl par } \\
\underline{än}\,$\oplus$\,tiise & \textsc{ pl ill } \\
\underline{än}\,$\oplus$\,tiiz & \textsc{ pl ine } \\
\underline{än}\,$\oplus$\,tiiss & \textsc{ pl ela } \\
\underline{än}\,$\oplus$\,tiille & \textsc{ pl all } \\
\underline{än}\,$\oplus$\,tiill & \textsc{ pl ade } \\
\underline{än}\,$\oplus$\,tiilt & \textsc{ pl abl } \\
\underline{än}\,$\oplus$\,tiissi & \textsc{ pl tra } \\
\underline{än}\,$\oplus$\,tiissaa & \textsc{ pl ter } \\
\underline{än}\,$\oplus$\,tijka & \textsc{ pl com } \\
\end{tabular}
\end{sideways}
\captionof{table}{Tüüpsõnamall \arabic{mallinumber}\,\vadja{änte} ekstraheeritud muutvormimallid.}
\label{tab:tüüpsõnamall-änte}

\end{minipage}

 
\vspace{1em}
\noindent Tüüpsõnamall \vadja{änte} ei hõlma teisi lekseeme vormi\-sõnastikus.

Tüüp\-sõna\-mall kirjeldab eespoolseid sõnu tüvemuutusega nt:nn.


\vspace{3.5em}
\noindent \begin{minipage}{\textwidth}
\stepcounter{mallinumber}
\noindent \textbf{Tüüpsõnamall \arabic{mallinumber}\,\vadja{leipe}}\\

\begin{sideways}
\begin{tabular}{l l}
muutvormimall & tunnused \\
\hline
\underline{lei}\,$\oplus$\,pe & \textsc{ sg nom } \\
\underline{lei}\,$\oplus$\,vä & \textsc{ sg gen } \\
\underline{lei}\,$\oplus$\,pä & \textsc{ sg par } \\
\underline{lei}\,$\oplus$\,päse & \textsc{ sg ill } \\
\underline{lei}\,$\oplus$\,vez & \textsc{ sg ine } \\
\underline{lei}\,$\oplus$\,vess & \textsc{ sg ela } \\
\underline{lei}\,$\oplus$\,velle & \textsc{ sg all } \\
\underline{lei}\,$\oplus$\,vell & \textsc{ sg ade } \\
\underline{lei}\,$\oplus$\,velt & \textsc{ sg abl } \\
\underline{lei}\,$\oplus$\,vessi & \textsc{ sg tra } \\
\underline{lei}\,$\oplus$\,vessaa & \textsc{ sg ter } \\
\underline{lei}\,$\oplus$\,veka & \textsc{ sg com } \\
\underline{lei}\,$\oplus$\,ved & \textsc{ pl nom } \\
\underline{lei}\,$\oplus$\,pije & \textsc{ pl gen } \\
\underline{lei}\,$\oplus$\,piit & \textsc{ pl par } \\
\underline{lei}\,$\oplus$\,piise & \textsc{ pl ill } \\
\underline{lei}\,$\oplus$\,piiz & \textsc{ pl ine } \\
\underline{lei}\,$\oplus$\,piiss & \textsc{ pl ela } \\
\underline{lei}\,$\oplus$\,piille & \textsc{ pl all } \\
\underline{lei}\,$\oplus$\,piill & \textsc{ pl ade } \\
\underline{lei}\,$\oplus$\,piilt & \textsc{ pl abl } \\
\underline{lei}\,$\oplus$\,piissi & \textsc{ pl tra } \\
\underline{lei}\,$\oplus$\,piissaa & \textsc{ pl ter } \\
\underline{lei}\,$\oplus$\,pijka & \textsc{ pl com } \\
\end{tabular}
\end{sideways}
\captionof{table}{Tüüpsõnamall \arabic{mallinumber}\,\vadja{leipe} ekstraheeritud muutvormimallid.}
\label{tab:tüüpsõnamall-leipe}

\end{minipage}

 
\vspace{1em}
\noindent Tüüpsõnamall \vadja{leipe} ei hõlma teisi lekseeme vormi\-sõnastikus.

Tüüp\-sõna\-mall kirjeldab eespoolseid sõnu tüvemuutusega p:v.


\vspace{3.5em}
\noindent \begin{minipage}{\textwidth}
\stepcounter{mallinumber}
\noindent \textbf{Tüüpsõnamall \arabic{mallinumber}\,\vadja{seppe}}\\

\begin{sideways}
\begin{tabular}{l l}
muutvormimall & tunnused \\
\hline
\underline{sep}\,$\oplus$\,pe & \textsc{ sg nom } \\
\underline{sep}\,$\oplus$\,ä & \textsc{ sg gen } \\
\underline{sep}\,$\oplus$\,pä & \textsc{ sg par } \\
\underline{sep}\,$\oplus$\,päse & \textsc{ sg ill } \\
\underline{sep}\,$\oplus$\,päz & \textsc{ sg ine } \\
\underline{sep}\,$\oplus$\,äss & \textsc{ sg ela } \\
\underline{sep}\,$\oplus$\,älle & \textsc{ sg all } \\
\underline{sep}\,$\oplus$\,äll & \textsc{ sg ade } \\
\underline{sep}\,$\oplus$\,ält & \textsc{ sg abl } \\
\underline{sep}\,$\oplus$\,ässi & \textsc{ sg tra } \\
\underline{sep}\,$\oplus$\,pässaa & \textsc{ sg ter } \\
\underline{sep}\,$\oplus$\,äka & \textsc{ sg com } \\
\underline{sep}\,$\oplus$\,äd & \textsc{ pl nom } \\
\underline{sep}\,$\oplus$\,pije & \textsc{ pl gen } \\
\underline{sep}\,$\oplus$\,piit & \textsc{ pl par } \\
\underline{sep}\,$\oplus$\,piise & \textsc{ pl ill } \\
\underline{sep}\,$\oplus$\,piiz & \textsc{ pl ine } \\
\underline{sep}\,$\oplus$\,piiss & \textsc{ pl ela } \\
\underline{sep}\,$\oplus$\,piille & \textsc{ pl all } \\
\underline{sep}\,$\oplus$\,piill & \textsc{ pl ade } \\
\underline{sep}\,$\oplus$\,piilt & \textsc{ pl abl } \\
\underline{sep}\,$\oplus$\,piissi & \textsc{ pl tra } \\
\underline{sep}\,$\oplus$\,piissaa & \textsc{ pl ter } \\
\underline{sep}\,$\oplus$\,pijka & \textsc{ pl com } \\
\end{tabular}
\end{sideways}
\captionof{table}{Tüüpsõnamall \arabic{mallinumber}\,\vadja{seppe} ekstraheeritud muutvormimallid.}
\label{tab:tüüpsõnamall-seppe}

\end{minipage}

 
\vspace{1em}
\noindent Tüüpsõnamall \vadja{seppe} hõlmab vormisõnastikus 2 lekseemi: \vadja{\underline{sep}pe} ja \vadja{\underline{lep}pe}.

Tüüp\-sõna\-mall kirjeldab eespoolseid sõnu tüvemuutusega pp:p.


\vspace{3.5em}
\noindent \begin{minipage}{\textwidth}
\stepcounter{mallinumber}
\noindent \textbf{Tüüpsõnamall \arabic{mallinumber}\,\vadja{ärče}}\\

\begin{sideways}
\begin{tabular}{l l}
muutvormimall & tunnused \\
\hline
\underline{är}\,$\oplus$\,če & \textsc{ sg nom } \\
\underline{är}\,$\oplus$\,jä & \textsc{ sg gen } \\
\underline{är}\,$\oplus$\,čä & \textsc{ sg par } \\
\underline{är}\,$\oplus$\,čäse & \textsc{ sg ill } \\
\underline{är}\,$\oplus$\,jez & \textsc{ sg ine } \\
\underline{är}\,$\oplus$\,jess & \textsc{ sg ela } \\
\underline{är}\,$\oplus$\,jelle & \textsc{ sg all } \\
\underline{är}\,$\oplus$\,jell & \textsc{ sg ade } \\
\underline{är}\,$\oplus$\,jelt & \textsc{ sg abl } \\
\underline{är}\,$\oplus$\,jessi & \textsc{ sg tra } \\
\underline{är}\,$\oplus$\,jessaa & \textsc{ sg ter } \\
\underline{är}\,$\oplus$\,jeka & \textsc{ sg com } \\
\underline{är}\,$\oplus$\,jed & \textsc{ pl nom } \\
\underline{är}\,$\oplus$\,čije & \textsc{ pl gen } \\
\underline{är}\,$\oplus$\,čiit & \textsc{ pl par } \\
\underline{är}\,$\oplus$\,čiise & \textsc{ pl ill } \\
\underline{är}\,$\oplus$\,čiiz & \textsc{ pl ine } \\
\underline{är}\,$\oplus$\,čiiss & \textsc{ pl ela } \\
\underline{är}\,$\oplus$\,čiille & \textsc{ pl all } \\
\underline{är}\,$\oplus$\,čiill & \textsc{ pl ade } \\
\underline{är}\,$\oplus$\,čiilt & \textsc{ pl abl } \\
\underline{är}\,$\oplus$\,čiissi & \textsc{ pl tra } \\
\underline{är}\,$\oplus$\,čiissaa & \textsc{ pl ter } \\
\underline{är}\,$\oplus$\,čijka & \textsc{ pl com } \\
\end{tabular}
\end{sideways}
\captionof{table}{Tüüpsõnamall \arabic{mallinumber}\,\vadja{ärče} ekstraheeritud muutvormimallid.}
\label{tab:tüüpsõnamall-ärče}

\end{minipage}

 
\vspace{1em}
\noindent Tüüpsõnamall \vadja{ärče} hõlmab vormisõnastikus 2 lekseemi: \vadja{\underline{är}če} ja \vadja{\underline{mär}če}.

Tüüp\-sõna\-mall kirjeldab eespoolseid sõnu tüvemuutusega rč:rj.


\vspace{3.5em}
\noindent \begin{minipage}{\textwidth}
\stepcounter{mallinumber}
\noindent \textbf{Tüüpsõnamall \arabic{mallinumber}\,\vadja{pesä}}\\

\begin{sideways}
\begin{tabular}{l l}
muutvormimall & tunnused \\
\hline
\underline{pe}\,$\oplus$\,sä & \textsc{ sg nom } \\
\underline{pe}\,$\oplus$\,zä & \textsc{ sg gen } \\
\underline{pe}\,$\oplus$\,ssä & \textsc{ sg par } \\
\underline{pe}\,$\oplus$\,ssäse & \textsc{ sg ill } \\
\underline{pe}\,$\oplus$\,zäz & \textsc{ sg ine } \\
\underline{pe}\,$\oplus$\,zäss & \textsc{ sg ela } \\
\underline{pe}\,$\oplus$\,zälle & \textsc{ sg all } \\
\underline{pe}\,$\oplus$\,zäll & \textsc{ sg ade } \\
\underline{pe}\,$\oplus$\,zält & \textsc{ sg abl } \\
\underline{pe}\,$\oplus$\,zässi & \textsc{ sg tra } \\
\underline{pe}\,$\oplus$\,zässaa & \textsc{ sg ter } \\
\underline{pe}\,$\oplus$\,zäka & \textsc{ sg com } \\
\underline{pe}\,$\oplus$\,zäd & \textsc{ pl nom } \\
\underline{pe}\,$\oplus$\,sije & \textsc{ pl gen } \\
\underline{pe}\,$\oplus$\,siit & \textsc{ pl par } \\
\underline{pe}\,$\oplus$\,siise & \textsc{ pl ill } \\
\underline{pe}\,$\oplus$\,siiz & \textsc{ pl ine } \\
\underline{pe}\,$\oplus$\,siiss & \textsc{ pl ela } \\
\underline{pe}\,$\oplus$\,siille & \textsc{ pl all } \\
\underline{pe}\,$\oplus$\,siill & \textsc{ pl ade } \\
\underline{pe}\,$\oplus$\,siilt & \textsc{ pl abl } \\
\underline{pe}\,$\oplus$\,siissi & \textsc{ pl tra } \\
\underline{pe}\,$\oplus$\,siissaa & \textsc{ pl ter } \\
\underline{pe}\,$\oplus$\,sijka & \textsc{ pl com } \\
\end{tabular}
\end{sideways}
\captionof{table}{Tüüpsõnamall \arabic{mallinumber}\,\vadja{pesä} ekstraheeritud muutvormimallid.}
\label{tab:tüüpsõnamall-pesä}

\end{minipage}

 
\vspace{1em}
\noindent Tüüpsõnamall \vadja{pesä} hõlmab vormisõnastikus 3 lekseemi: \vadja{\underline{pe}sä, \underline{rissi}sä} ja \vadja{\underline{če}sä}.

Tüüp\-sõna\-mall kirjeldab eespoolseid sõnu tüvemuutusega s:z, mille mitmuse tüvi on -\textit{ii}-.


\vspace{3.5em}
\noindent \begin{minipage}{\textwidth}
\stepcounter{mallinumber}
\noindent \textbf{Tüüpsõnamall \arabic{mallinumber}\,\vadja{lisä}}\\

\begin{sideways}
\begin{tabular}{l l}
muutvormimall & tunnused \\
\hline
\underline{li}\,$\oplus$\,s\,$\oplus$\,\underline{ä} & \textsc{ sg nom } \\
\underline{li}\,$\oplus$\,z\,$\oplus$\,\underline{ä} & \textsc{ sg gen } \\
\underline{li}\,$\oplus$\,ss\,$\oplus$\,\underline{ä} & \textsc{ sg par } \\
\underline{li}\,$\oplus$\,ss\,$\oplus$\,\underline{ä}\,$\oplus$\,se & \textsc{ sg ill } \\
\underline{li}\,$\oplus$\,z\,$\oplus$\,\underline{ä}\,$\oplus$\,z & \textsc{ sg ine } \\
\underline{li}\,$\oplus$\,z\,$\oplus$\,\underline{ä}\,$\oplus$\,ss & \textsc{ sg ela } \\
\underline{li}\,$\oplus$\,z\,$\oplus$\,\underline{ä}\,$\oplus$\,lle & \textsc{ sg all } \\
\underline{li}\,$\oplus$\,z\,$\oplus$\,\underline{ä}\,$\oplus$\,ll & \textsc{ sg ade } \\
\underline{li}\,$\oplus$\,z\,$\oplus$\,\underline{ä}\,$\oplus$\,lt & \textsc{ sg abl } \\
\underline{li}\,$\oplus$\,z\,$\oplus$\,\underline{ä}\,$\oplus$\,ssi & \textsc{ sg tra } \\
\underline{li}\,$\oplus$\,z\,$\oplus$\,\underline{ä}\,$\oplus$\,ssaa & \textsc{ sg ter } \\
\underline{li}\,$\oplus$\,z\,$\oplus$\,\underline{ä}\,$\oplus$\,ka & \textsc{ sg com } \\
\underline{li}\,$\oplus$\,z\,$\oplus$\,\underline{ä}\,$\oplus$\,d & \textsc{ pl nom } \\
\underline{li}\,$\oplus$\,ss\,$\oplus$\,\underline{ä}\,$\oplus$\,ije & \textsc{ pl gen } \\
\underline{li}\,$\oplus$\,ss\,$\oplus$\,\underline{ä}\,$\oplus$\,it & \textsc{ pl par } \\
\underline{li}\,$\oplus$\,ss\,$\oplus$\,\underline{ä}\,$\oplus$\,ise & \textsc{ pl ill } \\
\underline{li}\,$\oplus$\,ss\,$\oplus$\,\underline{ä}\,$\oplus$\,iz & \textsc{ pl ine } \\
\underline{li}\,$\oplus$\,ss\,$\oplus$\,\underline{ä}\,$\oplus$\,iss & \textsc{ pl ela } \\
\underline{li}\,$\oplus$\,ss\,$\oplus$\,\underline{ä}\,$\oplus$\,ille & \textsc{ pl all } \\
\underline{li}\,$\oplus$\,ss\,$\oplus$\,\underline{ä}\,$\oplus$\,ill & \textsc{ pl ade } \\
\underline{li}\,$\oplus$\,ss\,$\oplus$\,\underline{ä}\,$\oplus$\,ilt & \textsc{ pl abl } \\
\underline{li}\,$\oplus$\,ss\,$\oplus$\,\underline{ä}\,$\oplus$\,issi & \textsc{ pl tra } \\
\underline{li}\,$\oplus$\,ss\,$\oplus$\,\underline{ä}\,$\oplus$\,issaa & \textsc{ pl ter } \\
\underline{li}\,$\oplus$\,ss\,$\oplus$\,\underline{ä}\,$\oplus$\,ika & \textsc{ pl com } \\
\end{tabular}
\end{sideways}
\captionof{table}{Tüüpsõnamall \arabic{mallinumber}\,\vadja{lisä} ekstraheeritud muutvormimallid.}
\label{tab:tüüpsõnamall-lisä}

\end{minipage}

 
\vspace{1em}
\noindent Tüüpsõnamall \vadja{lisä} ei hõlma teisi lekseeme vormi\-sõnastikus.

Tüüp\-sõna\-mall kirjeldab eespoolseid sõnu tüvemuutusega s:z, mille mitmuse tüvi on -\textit{äi}-.


\vspace{3.5em}
\noindent \begin{minipage}{\textwidth}
\stepcounter{mallinumber}
\noindent \textbf{Tüüpsõnamall \arabic{mallinumber}\,\vadja{mätä}}\\

\begin{sideways}
\begin{tabular}{l l}
muutvormimall & tunnused \\
\hline
\underline{mä}\,$\oplus$\,tä & \textsc{ sg nom } \\
\underline{mä}\,$\oplus$\,ä & \textsc{ sg gen } \\
\underline{mä}\,$\oplus$\,ttä & \textsc{ sg par } \\
\underline{mä}\,$\oplus$\,ttäse & \textsc{ sg ill } \\
\underline{mä}\,$\oplus$\,äz & \textsc{ sg ine } \\
\underline{mä}\,$\oplus$\,äss & \textsc{ sg ela } \\
\underline{mä}\,$\oplus$\,älle & \textsc{ sg all } \\
\underline{mä}\,$\oplus$\,äll & \textsc{ sg ade } \\
\underline{mä}\,$\oplus$\,ält & \textsc{ sg abl } \\
\underline{mä}\,$\oplus$\,ässi & \textsc{ sg tra } \\
\underline{mä}\,$\oplus$\,ässaa & \textsc{ sg ter } \\
\underline{mä}\,$\oplus$\,äka & \textsc{ sg com } \\
\underline{mä}\,$\oplus$\,äd & \textsc{ pl nom } \\
\underline{mä}\,$\oplus$\,ttije & \textsc{ pl gen } \\
\underline{mä}\,$\oplus$\,ttiit & \textsc{ pl par } \\
\underline{mä}\,$\oplus$\,ttiise & \textsc{ pl ill } \\
\underline{mä}\,$\oplus$\,ttiiz & \textsc{ pl ine } \\
\underline{mä}\,$\oplus$\,ttiiss & \textsc{ pl ela } \\
\underline{mä}\,$\oplus$\,ttiille & \textsc{ pl all } \\
\underline{mä}\,$\oplus$\,ttiill & \textsc{ pl ade } \\
\underline{mä}\,$\oplus$\,ttiilt & \textsc{ pl abl } \\
\underline{mä}\,$\oplus$\,ttiissi & \textsc{ pl tra } \\
\underline{mä}\,$\oplus$\,ttiissaa & \textsc{ pl ter } \\
\underline{mä}\,$\oplus$\,ttijka & \textsc{ pl com } \\
\end{tabular}
\end{sideways}
\captionof{table}{Tüüpsõnamall \arabic{mallinumber}\,\vadja{mätä} ekstraheeritud muutvormimallid.}
\label{tab:tüüpsõnamall-mätä}

\end{minipage}

 
\vspace{1em}
\noindent Tüüpsõnamall \vadja{mätä} ei hõlma teisi lekseeme vormi\-sõnastikus.

Tüüp\-sõna\-mall kirjeldab eespoolseid sõnu tüvemuutusega t:∅.


\vspace{3.5em}
\noindent \begin{minipage}{\textwidth}
\stepcounter{mallinumber}
\noindent \textbf{Tüüpsõnamall \arabic{mallinumber}\,\vadja{mettse}}\\

\begin{sideways}
\begin{tabular}{l l}
muutvormimall & tunnused \\
\hline
\underline{met}\,$\oplus$\,t\,$\oplus$\,\underline{s}\,$\oplus$\,e & \textsc{ sg nom } \\
\underline{met}\,$\oplus$\,\underline{s}\,$\oplus$\,ä & \textsc{ sg gen } \\
\underline{met}\,$\oplus$\,t\,$\oplus$\,\underline{s}\,$\oplus$\,ä & \textsc{ sg par } \\
\underline{met}\,$\oplus$\,t\,$\oplus$\,\underline{s}\,$\oplus$\,äse & \textsc{ sg ill } \\
\underline{met}\,$\oplus$\,t\,$\oplus$\,\underline{s}\,$\oplus$\,äz & \textsc{ sg ine } \\
\underline{met}\,$\oplus$\,\underline{s}\,$\oplus$\,äss & \textsc{ sg ela } \\
\underline{met}\,$\oplus$\,\underline{s}\,$\oplus$\,älle & \textsc{ sg all } \\
\underline{met}\,$\oplus$\,\underline{s}\,$\oplus$\,äll & \textsc{ sg ade } \\
\underline{met}\,$\oplus$\,\underline{s}\,$\oplus$\,ält & \textsc{ sg abl } \\
\underline{met}\,$\oplus$\,\underline{s}\,$\oplus$\,ässi & \textsc{ sg tra } \\
\underline{met}\,$\oplus$\,t\,$\oplus$\,\underline{s}\,$\oplus$\,ässaa & \textsc{ sg ter } \\
\underline{met}\,$\oplus$\,\underline{s}\,$\oplus$\,äka & \textsc{ sg com } \\
\underline{met}\,$\oplus$\,\underline{s}\,$\oplus$\,äd & \textsc{ pl nom } \\
\underline{met}\,$\oplus$\,t\,$\oplus$\,\underline{s}\,$\oplus$\,ije & \textsc{ pl gen } \\
\underline{met}\,$\oplus$\,t\,$\oplus$\,\underline{s}\,$\oplus$\,iit & \textsc{ pl par } \\
\underline{met}\,$\oplus$\,t\,$\oplus$\,\underline{s}\,$\oplus$\,iise & \textsc{ pl ill } \\
\underline{met}\,$\oplus$\,t\,$\oplus$\,\underline{s}\,$\oplus$\,iiz & \textsc{ pl ine } \\
\underline{met}\,$\oplus$\,t\,$\oplus$\,\underline{s}\,$\oplus$\,iiss & \textsc{ pl ela } \\
\underline{met}\,$\oplus$\,t\,$\oplus$\,\underline{s}\,$\oplus$\,iille & \textsc{ pl all } \\
\underline{met}\,$\oplus$\,t\,$\oplus$\,\underline{s}\,$\oplus$\,iill & \textsc{ pl ade } \\
\underline{met}\,$\oplus$\,t\,$\oplus$\,\underline{s}\,$\oplus$\,iilt & \textsc{ pl abl } \\
\underline{met}\,$\oplus$\,t\,$\oplus$\,\underline{s}\,$\oplus$\,iissi & \textsc{ pl tra } \\
\underline{met}\,$\oplus$\,t\,$\oplus$\,\underline{s}\,$\oplus$\,iissaa & \textsc{ pl ter } \\
\underline{met}\,$\oplus$\,t\,$\oplus$\,\underline{s}\,$\oplus$\,ijka & \textsc{ pl com } \\
\end{tabular}
\end{sideways}
\captionof{table}{Tüüpsõnamall \arabic{mallinumber}\,\vadja{mettse} ekstraheeritud muutvormimallid.}
\label{tab:tüüpsõnamall-mettse}

\end{minipage}

 
\vspace{1em}
\noindent Tüüpsõnamall \vadja{mettse} ei hõlma teisi lekseeme vormi\-sõnastikus.

Tüüp\-sõna\-mall kirjeldab eespoolseid sõnu tüvemuutusega tts:ts.


\vspace{3.5em}
\noindent \begin{minipage}{\textwidth}
\stepcounter{mallinumber}
\noindent \textbf{Tüüpsõnamall \arabic{mallinumber}\,\vadja{nenä}}\\

\begin{sideways}
\begin{tabular}{l l}
muutvormimall & tunnused \\
\hline
\underline{nen}\,$\oplus$\,ä & \textsc{ sg nom } \\
\underline{nen}\,$\oplus$\,ä & \textsc{ sg gen } \\
\underline{nen}\,$\oplus$\,nä & \textsc{ sg par } \\
\underline{nen}\,$\oplus$\,äse & \textsc{ sg ill } \\
\underline{nen}\,$\oplus$\,äz & \textsc{ sg ine } \\
\underline{nen}\,$\oplus$\,äss & \textsc{ sg ela } \\
\underline{nen}\,$\oplus$\,älle & \textsc{ sg all } \\
\underline{nen}\,$\oplus$\,äll & \textsc{ sg ade } \\
\underline{nen}\,$\oplus$\,ält & \textsc{ sg abl } \\
\underline{nen}\,$\oplus$\,ässi & \textsc{ sg tra } \\
\underline{nen}\,$\oplus$\,ässaa & \textsc{ sg ter } \\
\underline{nen}\,$\oplus$\,äka & \textsc{ sg com } \\
\underline{nen}\,$\oplus$\,äd & \textsc{ pl nom } \\
\underline{nen}\,$\oplus$\,ije & \textsc{ pl gen } \\
\underline{nen}\,$\oplus$\,iit & \textsc{ pl par } \\
\underline{nen}\,$\oplus$\,iise & \textsc{ pl ill } \\
\underline{nen}\,$\oplus$\,iiz & \textsc{ pl ine } \\
\underline{nen}\,$\oplus$\,iiss & \textsc{ pl ela } \\
\underline{nen}\,$\oplus$\,iille & \textsc{ pl all } \\
\underline{nen}\,$\oplus$\,iill & \textsc{ pl ade } \\
\underline{nen}\,$\oplus$\,iilt & \textsc{ pl abl } \\
\underline{nen}\,$\oplus$\,iissi & \textsc{ pl tra } \\
\underline{nen}\,$\oplus$\,iissaa & \textsc{ pl ter } \\
\underline{nen}\,$\oplus$\,ijka & \textsc{ pl com } \\
\end{tabular}
\end{sideways}
\captionof{table}{Tüüpsõnamall \arabic{mallinumber}\,\vadja{nenä} ekstraheeritud muutvormimallid.}
\label{tab:tüüpsõnamall-nenä}

\end{minipage}

 
\vspace{1em}
\noindent Tüüpsõnamall \vadja{nenä} ei hõlma teisi lekseeme vormi\-sõnastikus.

Tüüp\-sõna\-mall kirjeldab eespoolseid tüvemuutuseta sõnu, mille tüvi gemineerub -\textit{nn}-.


\vspace{3.5em}
\noindent \begin{minipage}{\textwidth}
\stepcounter{mallinumber}
\noindent \textbf{Tüüpsõnamall \arabic{mallinumber}\,\vadja{čülä}}\\

\begin{sideways}
\begin{tabular}{l l}
muutvormimall & tunnused \\
\hline
\underline{čül}\,$\oplus$\,ä & \textsc{ sg nom } \\
\underline{čül}\,$\oplus$\,ä & \textsc{ sg gen } \\
\underline{čül}\,$\oplus$\,lä & \textsc{ sg par } \\
\underline{čül}\,$\oplus$\,äse & \textsc{ sg ill } \\
\underline{čül}\,$\oplus$\,äz & \textsc{ sg ine } \\
\underline{čül}\,$\oplus$\,äss & \textsc{ sg ela } \\
\underline{čül}\,$\oplus$\,älle & \textsc{ sg all } \\
\underline{čül}\,$\oplus$\,äll & \textsc{ sg ade } \\
\underline{čül}\,$\oplus$\,ält & \textsc{ sg abl } \\
\underline{čül}\,$\oplus$\,ässi & \textsc{ sg tra } \\
\underline{čül}\,$\oplus$\,ässaa & \textsc{ sg ter } \\
\underline{čül}\,$\oplus$\,äka & \textsc{ sg com } \\
\underline{čül}\,$\oplus$\,äd & \textsc{ pl nom } \\
\underline{čül}\,$\oplus$\,ije & \textsc{ pl gen } \\
\underline{čül}\,$\oplus$\,iit & \textsc{ pl par } \\
\underline{čül}\,$\oplus$\,iise & \textsc{ pl ill } \\
\underline{čül}\,$\oplus$\,iiz & \textsc{ pl ine } \\
\underline{čül}\,$\oplus$\,iiss & \textsc{ pl ela } \\
\underline{čül}\,$\oplus$\,iille & \textsc{ pl all } \\
\underline{čül}\,$\oplus$\,iill & \textsc{ pl ade } \\
\underline{čül}\,$\oplus$\,iilt & \textsc{ pl abl } \\
\underline{čül}\,$\oplus$\,iissi & \textsc{ pl tra } \\
\underline{čül}\,$\oplus$\,iissaa & \textsc{ pl ter } \\
\underline{čül}\,$\oplus$\,ijka & \textsc{ pl com } \\
\end{tabular}
\end{sideways}
\captionof{table}{Tüüpsõnamall \arabic{mallinumber}\,\vadja{čülä} ekstraheeritud muutvormimallid.}
\label{tab:tüüpsõnamall-čülä}

\end{minipage}

 
\vspace{1em}
\noindent Tüüpsõnamall \vadja{čülä} ei hõlma teisi lekseeme vormi\-sõnastikus.

Tüüp\-sõna\-mall kirjeldab eespoolseid tüvemuutuseta sõnu, mille tüvi gemineerub -\textit{ll}- \msd{sg par} vormis.


\vspace{3.5em}
\noindent \begin{minipage}{\textwidth}
\stepcounter{mallinumber}
\noindent \textbf{Tüüpsõnamall \arabic{mallinumber}\,\vadja{pühä}}\\

\begin{sideways}
\begin{tabular}{l l}
muutvormimall & tunnused \\
\hline
\underline{püh}\,$\oplus$\,ä & \textsc{ sg nom } \\
\underline{püh}\,$\oplus$\,ä & \textsc{ sg gen } \\
\underline{püh}\,$\oplus$\,hä & \textsc{ sg par } \\
\underline{püh}\,$\oplus$\,häse & \textsc{ sg ill } \\
\underline{püh}\,$\oplus$\,äz & \textsc{ sg ine } \\
\underline{püh}\,$\oplus$\,äss & \textsc{ sg ela } \\
\underline{püh}\,$\oplus$\,älle & \textsc{ sg all } \\
\underline{püh}\,$\oplus$\,äll & \textsc{ sg ade } \\
\underline{püh}\,$\oplus$\,ält & \textsc{ sg abl } \\
\underline{püh}\,$\oplus$\,ässi & \textsc{ sg tra } \\
\underline{püh}\,$\oplus$\,ässaa & \textsc{ sg ter } \\
\underline{püh}\,$\oplus$\,äka & \textsc{ sg com } \\
\underline{püh}\,$\oplus$\,äd & \textsc{ pl nom } \\
\underline{püh}\,$\oplus$\,hije & \textsc{ pl gen } \\
\underline{püh}\,$\oplus$\,hiit & \textsc{ pl par } \\
\underline{püh}\,$\oplus$\,hiise & \textsc{ pl ill } \\
\underline{püh}\,$\oplus$\,hiiz & \textsc{ pl ine } \\
\underline{püh}\,$\oplus$\,hiiss & \textsc{ pl ela } \\
\underline{püh}\,$\oplus$\,hiille & \textsc{ pl all } \\
\underline{püh}\,$\oplus$\,hiill & \textsc{ pl ade } \\
\underline{püh}\,$\oplus$\,hiilt & \textsc{ pl abl } \\
\underline{püh}\,$\oplus$\,hiissi & \textsc{ pl tra } \\
\underline{püh}\,$\oplus$\,hiissaa & \textsc{ pl ter } \\
\underline{püh}\,$\oplus$\,hijka & \textsc{ pl com } \\
\end{tabular}
\end{sideways}
\captionof{table}{Tüüpsõnamall \arabic{mallinumber}\,\vadja{pühä} ekstraheeritud muutvormimallid.}
\label{tab:tüüpsõnamall-pühä}

\end{minipage}

 
\vspace{1em}
\noindent Tüüpsõnamall \vadja{pühä} ei hõlma teisi lekseeme vormi\-sõnastikus.

Tüüp\-sõna\-mall kirjeldab eespoolseid tüvemuutuseta sõnu, mille tüvi gemineerub -\textit{hh}- \msd{sg par} vormis.


\vspace{3.5em}
\noindent \begin{minipage}{\textwidth}
\stepcounter{mallinumber}
\noindent \textbf{Tüüpsõnamall \arabic{mallinumber}\,\vadja{üvä}}\\

\begin{sideways}
\begin{tabular}{l l}
muutvormimall & tunnused \\
\hline
\underline{üv}\,$\oplus$\,ä & \textsc{ sg nom } \\
\underline{üv}\,$\oplus$\,ä & \textsc{ sg gen } \\
\underline{üv}\,$\oplus$\,vä & \textsc{ sg par } \\
\underline{üv}\,$\oplus$\,väse & \textsc{ sg ill } \\
\underline{üv}\,$\oplus$\,äz & \textsc{ sg ine } \\
\underline{üv}\,$\oplus$\,äss & \textsc{ sg ela } \\
\underline{üv}\,$\oplus$\,älle & \textsc{ sg all } \\
\underline{üv}\,$\oplus$\,äll & \textsc{ sg ade } \\
\underline{üv}\,$\oplus$\,ält & \textsc{ sg abl } \\
\underline{üv}\,$\oplus$\,ässi & \textsc{ sg tra } \\
\underline{üv}\,$\oplus$\,ässaa & \textsc{ sg ter } \\
\underline{üv}\,$\oplus$\,äka & \textsc{ sg com } \\
\underline{üv}\,$\oplus$\,äd & \textsc{ pl nom } \\
\underline{üv}\,$\oplus$\,vije & \textsc{ pl gen } \\
\underline{üv}\,$\oplus$\,viit & \textsc{ pl par } \\
\underline{üv}\,$\oplus$\,viise & \textsc{ pl ill } \\
\underline{üv}\,$\oplus$\,viiz & \textsc{ pl ine } \\
\underline{üv}\,$\oplus$\,viiss & \textsc{ pl ela } \\
\underline{üv}\,$\oplus$\,viille & \textsc{ pl all } \\
\underline{üv}\,$\oplus$\,viill & \textsc{ pl ade } \\
\underline{üv}\,$\oplus$\,viilt & \textsc{ pl abl } \\
\underline{üv}\,$\oplus$\,viissi & \textsc{ pl tra } \\
\underline{üv}\,$\oplus$\,viissaa & \textsc{ pl ter } \\
\underline{üv}\,$\oplus$\,vijka & \textsc{ pl com } \\
\end{tabular}
\end{sideways}
\captionof{table}{Tüüpsõnamall \arabic{mallinumber}\,\vadja{üvä} ekstraheeritud muutvormimallid.}
\label{tab:tüüpsõnamall-üvä}

\end{minipage}

 
\vspace{1em}
\noindent Tüüpsõnamall \vadja{üvä} ei hõlma teisi lekseeme vormi\-sõnastikus.

Tüüp\-sõna\-mall kirjeldab eespoolseid tüvemuutuseta sõnu, mille tüvi gemineerub -\textit{vv}- \msd{sg par} vormis.


\vspace{3.5em}
\noindent \begin{minipage}{\textwidth}
\stepcounter{mallinumber}
\noindent \textbf{Tüüpsõnamall \arabic{mallinumber}\,\vadja{kõva}}\\

\begin{sideways}
\begin{tabular}{l l}
muutvormimall & tunnused \\
\hline
\underline{kõv}\,$\oplus$\,a & \textsc{ sg nom } \\
\underline{kõv}\,$\oplus$\,a & \textsc{ sg gen } \\
\underline{kõv}\,$\oplus$\,at & \textsc{ sg par } \\
\underline{kõv}\,$\oplus$\,asõ & \textsc{ sg ill } \\
\underline{kõv}\,$\oplus$\,az & \textsc{ sg ine } \\
\underline{kõv}\,$\oplus$\,ass & \textsc{ sg ela } \\
\underline{kõv}\,$\oplus$\,allõ & \textsc{ sg all } \\
\underline{kõv}\,$\oplus$\,all & \textsc{ sg ade } \\
\underline{kõv}\,$\oplus$\,alt & \textsc{ sg abl } \\
\underline{kõv}\,$\oplus$\,assi & \textsc{ sg tra } \\
\underline{kõv}\,$\oplus$\,assaa & \textsc{ sg ter } \\
\underline{kõv}\,$\oplus$\,aka & \textsc{ sg com } \\
\underline{kõv}\,$\oplus$\,ad & \textsc{ pl nom } \\
\underline{kõv}\,$\oplus$\,ijõ & \textsc{ pl gen } \\
\underline{kõv}\,$\oplus$\,iit & \textsc{ pl par } \\
\underline{kõv}\,$\oplus$\,iisõ & \textsc{ pl ill } \\
\underline{kõv}\,$\oplus$\,iiz & \textsc{ pl ine } \\
\underline{kõv}\,$\oplus$\,iiss & \textsc{ pl ela } \\
\underline{kõv}\,$\oplus$\,iillõ & \textsc{ pl all } \\
\underline{kõv}\,$\oplus$\,iill & \textsc{ pl ade } \\
\underline{kõv}\,$\oplus$\,iilt & \textsc{ pl abl } \\
\underline{kõv}\,$\oplus$\,iissi & \textsc{ pl tra } \\
\underline{kõv}\,$\oplus$\,iissaa & \textsc{ pl ter } \\
\underline{kõv}\,$\oplus$\,ijka & \textsc{ pl com } \\
\end{tabular}
\end{sideways}
\captionof{table}{Tüüpsõnamall \arabic{mallinumber}\,\vadja{kõva} ekstraheeritud muutvormimallid.}
\label{tab:tüüpsõnamall-kõva}

\end{minipage}

 
\vspace{1em}
\noindent Tüüpsõnamall \vadja{kõva} ei hõlma teisi lekseeme vormi\-sõnastikus.

Tüüp\-sõna\-mall kirjeldab tagapoolseid tüvemuutuseta sõnu, mille tüvi gemineerub -\textit{vv}- \msd{sg par} vormis.


\vspace{3.5em}
\noindent \begin{minipage}{\textwidth}
\stepcounter{mallinumber}
\noindent \textbf{Tüüpsõnamall \arabic{mallinumber}\,\vadja{õja}}\\

\begin{sideways}
\begin{tabular}{l l}
muutvormimall & tunnused \\
\hline
\underline{õj}\,$\oplus$\,a & \textsc{ sg nom } \\
\underline{õj}\,$\oplus$\,a & \textsc{ sg gen } \\
\underline{õj}\,$\oplus$\,ja & \textsc{ sg par } \\
\underline{õj}\,$\oplus$\,asõ & \textsc{ sg ill } \\
\underline{õj}\,$\oplus$\,az & \textsc{ sg ine } \\
\underline{õj}\,$\oplus$\,ass & \textsc{ sg ela } \\
\underline{õj}\,$\oplus$\,allõ & \textsc{ sg all } \\
\underline{õj}\,$\oplus$\,all & \textsc{ sg ade } \\
\underline{õj}\,$\oplus$\,alt & \textsc{ sg abl } \\
\underline{õj}\,$\oplus$\,assi & \textsc{ sg tra } \\
\underline{õj}\,$\oplus$\,assaa & \textsc{ sg ter } \\
\underline{õj}\,$\oplus$\,aka & \textsc{ sg com } \\
\underline{õj}\,$\oplus$\,ad & \textsc{ pl nom } \\
\underline{õj}\,$\oplus$\,ijõ & \textsc{ pl gen } \\
\underline{õj}\,$\oplus$\,it & \textsc{ pl par } \\
\underline{õj}\,$\oplus$\,isõ & \textsc{ pl ill } \\
\underline{õj}\,$\oplus$\,iz & \textsc{ pl ine } \\
\underline{õj}\,$\oplus$\,iss & \textsc{ pl ela } \\
\underline{õj}\,$\oplus$\,illõ & \textsc{ pl all } \\
\underline{õj}\,$\oplus$\,ill & \textsc{ pl ade } \\
\underline{õj}\,$\oplus$\,ilt & \textsc{ pl abl } \\
\underline{õj}\,$\oplus$\,issi & \textsc{ pl tra } \\
\underline{õj}\,$\oplus$\,issaa & \textsc{ pl ter } \\
\underline{õj}\,$\oplus$\,ika & \textsc{ pl com } \\
\end{tabular}
\end{sideways}
\captionof{table}{Tüüpsõnamall \arabic{mallinumber}\,\vadja{õja} ekstraheeritud muutvormimallid.}
\label{tab:tüüpsõnamall-õja}

\end{minipage}

 
\vspace{1em}
\noindent Tüüpsõnamall \vadja{õja} ei hõlma teisi lekseeme vormi\-sõnastikus.

Tüüp\-sõna\-mall kirjeldab tagapoolseid tüvemuutuseta sõnu, mille tüvi gemineerub -\textit{jj}- \msd{sg par} vormis.


\vspace{3.5em}
\noindent \begin{minipage}{\textwidth}
\stepcounter{mallinumber}
\noindent \textbf{Tüüpsõnamall \arabic{mallinumber}\,\vadja{õma}}\\

\begin{sideways}
\begin{tabular}{l l}
muutvormimall & tunnused \\
\hline
\underline{õm}\,$\oplus$\,a & \textsc{ sg nom } \\
\underline{õm}\,$\oplus$\,a & \textsc{ sg gen } \\
\underline{õm}\,$\oplus$\,ma & \textsc{ sg par } \\
\underline{õm}\,$\oplus$\,masõ & \textsc{ sg ill } \\
\underline{õm}\,$\oplus$\,õz & \textsc{ sg ine } \\
\underline{õm}\,$\oplus$\,õss & \textsc{ sg ela } \\
\underline{õm}\,$\oplus$\,õllõ & \textsc{ sg all } \\
\underline{õm}\,$\oplus$\,õll & \textsc{ sg ade } \\
\underline{õm}\,$\oplus$\,õlt & \textsc{ sg abl } \\
\underline{õm}\,$\oplus$\,õssi & \textsc{ sg tra } \\
\underline{õm}\,$\oplus$\,õssaa & \textsc{ sg ter } \\
\underline{õm}\,$\oplus$\,õka & \textsc{ sg com } \\
\underline{õm}\,$\oplus$\,õd & \textsc{ pl nom } \\
\underline{õm}\,$\oplus$\,mijõ & \textsc{ pl gen } \\
\underline{õm}\,$\oplus$\,miit & \textsc{ pl par } \\
\underline{õm}\,$\oplus$\,miisõ & \textsc{ pl ill } \\
\underline{õm}\,$\oplus$\,miiz & \textsc{ pl ine } \\
\underline{õm}\,$\oplus$\,miiss & \textsc{ pl ela } \\
\underline{õm}\,$\oplus$\,miillõ & \textsc{ pl all } \\
\underline{õm}\,$\oplus$\,miill & \textsc{ pl ade } \\
\underline{õm}\,$\oplus$\,miilt & \textsc{ pl abl } \\
\underline{õm}\,$\oplus$\,miissi & \textsc{ pl tra } \\
\underline{õm}\,$\oplus$\,miissaa & \textsc{ pl ter } \\
\underline{õm}\,$\oplus$\,mijka & \textsc{ pl com } \\
\end{tabular}
\end{sideways}
\captionof{table}{Tüüpsõnamall \arabic{mallinumber}\,\vadja{õma} ekstraheeritud muutvormimallid.}
\label{tab:tüüpsõnamall-õma}

\end{minipage}

 
\vspace{1em}
\noindent Tüüpsõnamall \vadja{õma} ei hõlma teisi lekseeme vormi\-sõnastikus.

Tüüp\-sõna\-mall kirjeldab tagapoolseid tüvemuutuseta sõnu, mille tüvi gemineerub -\textit{mm}- \msd{sg par} vormis.
\spacing{1.5}



\subsection{\RN{9} käändkond}
See käändkond on spetsiifiline Kattila murdele ja ei esine loodava vadja kirjakeele aluseks olevas murdes.


\subsection{\RN{10} käändkond}

Kümnendasse käändkonda koondub suur osa kahetüvelisi sõnu, mille \msd{sg nom} lõpp on \vadja{-i}, ent mille tüvevokaal on \vadja{-e/-õ} (\cite[47]{ariste_grammar_1968}). Ariste märgib, et kuigi \msd{sp par} vormil on mitu erinevat realisatsiooni, järgib nende moodustamis\-viis põhimõtteliselt sama malli (\cite[47]{ariste_grammar_1968}). Ekstrakt\-morfoloogiaga eraldatud muutsõnamallide vähene arv näitab, et meetod pädeb nende põhimõttete tuvastamises.

Käändkonna muutsõnamallides on kirjakeele järgi ühtlustatud kõikide sõnade lõpuvokaalid \textit{-i}:\textit{-õ}:\textit{-tõ} ja \textit{-i}:\textit{-e}:\textit{-te} vastavalt sõna vokaalharmooniale.

\subsubsection*{Ekstraktmorfoloogia tüüpsõnad}
\spacing{1.1}
\vspace{-3.5em}


\vspace{3.5em}
\noindent \begin{minipage}{\textwidth}
\stepcounter{mallinumber}
\noindent \textbf{Tüüpsõnamall \arabic{mallinumber}\,\vadja{lumi}}\\

\begin{sideways}
\begin{tabular}{l l}
muutvormimall & tunnused \\
\hline
\underline{lu}\,$\oplus$\,mi & \textsc{ sg nom } \\
\underline{lu}\,$\oplus$\,mõ & \textsc{ sg gen } \\
\underline{lu}\,$\oplus$\,ntõ & \textsc{ sg par } \\
\underline{lu}\,$\oplus$\,mõsõ & \textsc{ sg ill } \\
\underline{lu}\,$\oplus$\,mõz & \textsc{ sg ine } \\
\underline{lu}\,$\oplus$\,mõss & \textsc{ sg ela } \\
\underline{lu}\,$\oplus$\,mõllõ & \textsc{ sg all } \\
\underline{lu}\,$\oplus$\,mõll & \textsc{ sg ade } \\
\underline{lu}\,$\oplus$\,mõlt & \textsc{ sg abl } \\
\underline{lu}\,$\oplus$\,mõssi & \textsc{ sg tra } \\
\underline{lu}\,$\oplus$\,mõssaa & \textsc{ sg ter } \\
\underline{lu}\,$\oplus$\,mõka & \textsc{ sg com } \\
\underline{lu}\,$\oplus$\,mõd & \textsc{ pl nom } \\
\underline{lu}\,$\oplus$\,mijõ & \textsc{ pl gen } \\
\underline{lu}\,$\oplus$\,miit & \textsc{ pl par } \\
\underline{lu}\,$\oplus$\,miisõ & \textsc{ pl ill } \\
\underline{lu}\,$\oplus$\,miiz & \textsc{ pl ine } \\
\underline{lu}\,$\oplus$\,miiss & \textsc{ pl ela } \\
\underline{lu}\,$\oplus$\,miillõ & \textsc{ pl all } \\
\underline{lu}\,$\oplus$\,miill & \textsc{ pl ade } \\
\underline{lu}\,$\oplus$\,miilt & \textsc{ pl abl } \\
\underline{lu}\,$\oplus$\,miissi & \textsc{ pl tra } \\
\underline{lu}\,$\oplus$\,miissaa & \textsc{ pl ter } \\
\underline{lu}\,$\oplus$\,mijka & \textsc{ pl com } \\
\end{tabular}
\end{sideways}
\captionof{table}{Tüüpsõnamall \arabic{mallinumber}\,\vadja{lumi} ekstraheeritud muutvormimallid.}
\label{tab:tüüpsõnamall-lumi}

\end{minipage}

 
\vspace{1em}
\noindent Tüüpsõnamall \vadja{lumi} ei hõlma teisi lekseeme vormi\-sõnastikus.



\vspace{3.5em}
\noindent \begin{minipage}{\textwidth}
\stepcounter{mallinumber}
\noindent \textbf{Tüüpsõnamall \arabic{mallinumber}\,\vadja{iiri}}\\

\begin{sideways}
\begin{tabular}{l l}
muutvormimall & tunnused \\
\hline
\underline{iir}\,$\oplus$\,i & \textsc{ sg nom } \\
\underline{iir}\,$\oplus$\,e & \textsc{ sg gen } \\
\underline{iir}\,$\oplus$\,te & \textsc{ sg par } \\
\underline{iir}\,$\oplus$\,ese & \textsc{ sg ill } \\
\underline{iir}\,$\oplus$\,ez & \textsc{ sg ine } \\
\underline{iir}\,$\oplus$\,ess & \textsc{ sg ela } \\
\underline{iir}\,$\oplus$\,elle & \textsc{ sg all } \\
\underline{iir}\,$\oplus$\,ell & \textsc{ sg ade } \\
\underline{iir}\,$\oplus$\,elt & \textsc{ sg abl } \\
\underline{iir}\,$\oplus$\,essi & \textsc{ sg tra } \\
\underline{iir}\,$\oplus$\,essaa & \textsc{ sg ter } \\
\underline{iir}\,$\oplus$\,eka & \textsc{ sg com } \\
\underline{iir}\,$\oplus$\,ed & \textsc{ pl nom } \\
\underline{iir}\,$\oplus$\,ije & \textsc{ pl gen } \\
\underline{iir}\,$\oplus$\,iit & \textsc{ pl par } \\
\underline{iir}\,$\oplus$\,iise & \textsc{ pl ill } \\
\underline{iir}\,$\oplus$\,iiz & \textsc{ pl ine } \\
\underline{iir}\,$\oplus$\,iiss & \textsc{ pl ela } \\
\underline{iir}\,$\oplus$\,iille & \textsc{ pl all } \\
\underline{iir}\,$\oplus$\,iill & \textsc{ pl ade } \\
\underline{iir}\,$\oplus$\,iilt & \textsc{ pl abl } \\
\underline{iir}\,$\oplus$\,iissi & \textsc{ pl tra } \\
\underline{iir}\,$\oplus$\,iissaa & \textsc{ pl ter } \\
\underline{iir}\,$\oplus$\,ijka & \textsc{ pl com } \\
\end{tabular}
\end{sideways}
\captionof{table}{Tüüpsõnamall \arabic{mallinumber}\,\vadja{iiri} ekstraheeritud muutvormimallid.}
\label{tab:tüüpsõnamall-iiri}

\end{minipage}

 
\vspace{1em}
\noindent Tüüpsõnamall \vadja{iiri} hõlmab vormisõnastikus 9 lekseemi: \vadja{\underline{iir}i, \underline{meel}i, \underline{mer}i, \underline{peen}i, \underline{sül}i, \underline{ver}i, \underline{ään}i, \underline{äär}i} ja \vadja{\underline{čeel}i}.



\vspace{3.5em}
\noindent \begin{minipage}{\textwidth}
\stepcounter{mallinumber}
\noindent \textbf{Tüüpsõnamall \arabic{mallinumber}\,\vadja{juuri}}\\

\begin{sideways}
\begin{tabular}{l l}
muutvormimall & tunnused \\
\hline
\underline{juur}\,$\oplus$\,i & \textsc{ sg nom } \\
\underline{juur}\,$\oplus$\,õ & \textsc{ sg gen } \\
\underline{juur}\,$\oplus$\,tõ & \textsc{ sg par } \\
\underline{juur}\,$\oplus$\,õsõ & \textsc{ sg ill } \\
\underline{juur}\,$\oplus$\,õz & \textsc{ sg ine } \\
\underline{juur}\,$\oplus$\,õss & \textsc{ sg ela } \\
\underline{juur}\,$\oplus$\,õllõ & \textsc{ sg all } \\
\underline{juur}\,$\oplus$\,õll & \textsc{ sg ade } \\
\underline{juur}\,$\oplus$\,õlt & \textsc{ sg abl } \\
\underline{juur}\,$\oplus$\,õssi & \textsc{ sg tra } \\
\underline{juur}\,$\oplus$\,õssaa & \textsc{ sg ter } \\
\underline{juur}\,$\oplus$\,õka & \textsc{ sg com } \\
\underline{juur}\,$\oplus$\,õd & \textsc{ pl nom } \\
\underline{juur}\,$\oplus$\,ijõ & \textsc{ pl gen } \\
\underline{juur}\,$\oplus$\,iit & \textsc{ pl par } \\
\underline{juur}\,$\oplus$\,iisõ & \textsc{ pl ill } \\
\underline{juur}\,$\oplus$\,iiz & \textsc{ pl ine } \\
\underline{juur}\,$\oplus$\,iiss & \textsc{ pl ela } \\
\underline{juur}\,$\oplus$\,iillõ & \textsc{ pl all } \\
\underline{juur}\,$\oplus$\,iill & \textsc{ pl ade } \\
\underline{juur}\,$\oplus$\,iilt & \textsc{ pl abl } \\
\underline{juur}\,$\oplus$\,iissi & \textsc{ pl tra } \\
\underline{juur}\,$\oplus$\,iissaa & \textsc{ pl ter } \\
\underline{juur}\,$\oplus$\,ijka & \textsc{ pl com } \\
\end{tabular}
\end{sideways}
\captionof{table}{Tüüpsõnamall \arabic{mallinumber}\,\vadja{juuri} ekstraheeritud muutvormimallid.}
\label{tab:tüüpsõnamall-juuri}

\end{minipage}

 
\vspace{1em}
\noindent Tüüpsõnamall \vadja{juuri} hõlmab vormisõnastikus 13 lekseemi: \vadja{\underline{juur}i, \underline{kaan}i, \underline{koor}i, \underline{lõh}i, \underline{noor}i, \underline{ool}i, \underline{pool}i, \underline{sool}i, \underline{suur}i, \underline{tul}i, \underline{tuul}i, \underline{un}i} ja \vadja{\underline{hool}i}.



\vspace{3.5em}
\noindent \begin{minipage}{\textwidth}
\stepcounter{mallinumber}
\noindent \textbf{Tüüpsõnamall \arabic{mallinumber}\,\vadja{uhsi}}\\

\begin{sideways}
\begin{tabular}{l l}
muutvormimall & tunnused \\
\hline
\underline{u}\,$\oplus$\,hsi & \textsc{ sg nom } \\
\underline{u}\,$\oplus$\,hzõ & \textsc{ sg gen } \\
\underline{u}\,$\oplus$\,ssõ & \textsc{ sg par } \\
\underline{u}\,$\oplus$\,hsõsõ & \textsc{ sg ill } \\
\underline{u}\,$\oplus$\,hsõz & \textsc{ sg ine } \\
\underline{u}\,$\oplus$\,hzõss & \textsc{ sg ela } \\
\underline{u}\,$\oplus$\,hzõllõ & \textsc{ sg all } \\
\underline{u}\,$\oplus$\,hzõll & \textsc{ sg ade } \\
\underline{u}\,$\oplus$\,hzõlt & \textsc{ sg abl } \\
\underline{u}\,$\oplus$\,hzõssi & \textsc{ sg tra } \\
\underline{u}\,$\oplus$\,hsõssaa & \textsc{ sg ter } \\
\underline{u}\,$\oplus$\,hzõka & \textsc{ sg com } \\
\underline{u}\,$\oplus$\,hzõd & \textsc{ pl nom } \\
\underline{u}\,$\oplus$\,hsijõ & \textsc{ pl gen } \\
\underline{u}\,$\oplus$\,hsiit & \textsc{ pl par } \\
\underline{u}\,$\oplus$\,hsiisõ & \textsc{ pl ill } \\
\underline{u}\,$\oplus$\,hsiiz & \textsc{ pl ine } \\
\underline{u}\,$\oplus$\,hsiiss & \textsc{ pl ela } \\
\underline{u}\,$\oplus$\,hsiillõ & \textsc{ pl all } \\
\underline{u}\,$\oplus$\,hsiill & \textsc{ pl ade } \\
\underline{u}\,$\oplus$\,hsiilt & \textsc{ pl abl } \\
\underline{u}\,$\oplus$\,hsiissi & \textsc{ pl tra } \\
\underline{u}\,$\oplus$\,hsiissaa & \textsc{ pl ter } \\
\underline{u}\,$\oplus$\,hsijka & \textsc{ pl com } \\
\end{tabular}
\end{sideways}
\captionof{table}{Tüüpsõnamall \arabic{mallinumber}\,\vadja{uhsi} ekstraheeritud muutvormimallid.}
\label{tab:tüüpsõnamall-uhsi}

\end{minipage}

 
\vspace{1em}
\noindent Tüüpsõnamall \vadja{uhsi} hõlmab vormisõnastikus 2 lekseemi: \vadja{\underline{u}hsi} ja \vadja{\underline{la}hsi}.



\vspace{3.5em}
\noindent \begin{minipage}{\textwidth}
\stepcounter{mallinumber}
\noindent \textbf{Tüüpsõnamall \arabic{mallinumber}\,\vadja{kusi}}\\

\begin{sideways}
\begin{tabular}{l l}
muutvormimall & tunnused \\
\hline
\underline{ku}\,$\oplus$\,si & \textsc{ sg nom } \\
\underline{ku}\,$\oplus$\,zõ & \textsc{ sg gen } \\
\underline{ku}\,$\oplus$\,ssõ & \textsc{ sg par } \\
\underline{ku}\,$\oplus$\,ssõ & \textsc{ sg ill } \\
\underline{ku}\,$\oplus$\,zõz & \textsc{ sg ine } \\
\underline{ku}\,$\oplus$\,zõss & \textsc{ sg ela } \\
\underline{ku}\,$\oplus$\,zõllõ & \textsc{ sg all } \\
\underline{ku}\,$\oplus$\,zõll & \textsc{ sg ade } \\
\underline{ku}\,$\oplus$\,zõlt & \textsc{ sg abl } \\
\underline{ku}\,$\oplus$\,zõssi & \textsc{ sg tra } \\
\underline{ku}\,$\oplus$\,zõssaa & \textsc{ sg ter } \\
\underline{ku}\,$\oplus$\,zõka & \textsc{ sg com } \\
\underline{ku}\,$\oplus$\,zõd & \textsc{ pl nom } \\
\underline{ku}\,$\oplus$\,ssijõ & \textsc{ pl gen } \\
\underline{ku}\,$\oplus$\,ssiit & \textsc{ pl par } \\
\underline{ku}\,$\oplus$\,ssiisõ & \textsc{ pl ill } \\
\underline{ku}\,$\oplus$\,ssiiz & \textsc{ pl ine } \\
\underline{ku}\,$\oplus$\,ssiiss & \textsc{ pl ela } \\
\underline{ku}\,$\oplus$\,ssiillõ & \textsc{ pl all } \\
\underline{ku}\,$\oplus$\,ssiill & \textsc{ pl ade } \\
\underline{ku}\,$\oplus$\,ssiilt & \textsc{ pl abl } \\
\underline{ku}\,$\oplus$\,ssiissi & \textsc{ pl tra } \\
\underline{ku}\,$\oplus$\,ssiissaa & \textsc{ pl ter } \\
\underline{ku}\,$\oplus$\,ssijka & \textsc{ pl com } \\
\end{tabular}
\end{sideways}
\captionof{table}{Tüüpsõnamall \arabic{mallinumber}\,\vadja{kusi} ekstraheeritud muutvormimallid.}
\label{tab:tüüpsõnamall-kusi}

\end{minipage}

 
\vspace{1em}
\noindent Tüüpsõnamall \vadja{kusi} ei hõlma teisi lekseeme vormi\-sõnastikus.



\vspace{3.5em}
\noindent \begin{minipage}{\textwidth}
\stepcounter{mallinumber}
\noindent \textbf{Tüüpsõnamall \arabic{mallinumber}\,\vadja{uusi}}\\

\begin{sideways}
\begin{tabular}{l l}
muutvormimall & tunnused \\
\hline
\underline{uu}\,$\oplus$\,si & \textsc{ sg nom } \\
\underline{uu}\,$\oplus$\,vvõ & \textsc{ sg gen } \\
\underline{uu}\,$\oplus$\,ttõ & \textsc{ sg par } \\
\underline{uu}\,$\oplus$\,vvõsõ & \textsc{ sg ill } \\
\underline{uu}\,$\oplus$\,vvõz & \textsc{ sg ine } \\
\underline{uu}\,$\oplus$\,vvõss & \textsc{ sg ela } \\
\underline{uu}\,$\oplus$\,vvõllõ & \textsc{ sg all } \\
\underline{uu}\,$\oplus$\,vvõll & \textsc{ sg ade } \\
\underline{uu}\,$\oplus$\,vvõlt & \textsc{ sg abl } \\
\underline{uu}\,$\oplus$\,vvõssi & \textsc{ sg tra } \\
\underline{uu}\,$\oplus$\,vvõssaa & \textsc{ sg ter } \\
\underline{uu}\,$\oplus$\,vvõka & \textsc{ sg com } \\
\underline{uu}\,$\oplus$\,vvõd & \textsc{ pl nom } \\
\underline{uu}\,$\oplus$\,sijõ & \textsc{ pl gen } \\
\underline{uu}\,$\oplus$\,siit & \textsc{ pl par } \\
\underline{uu}\,$\oplus$\,siisõ & \textsc{ pl ill } \\
\underline{uu}\,$\oplus$\,siiz & \textsc{ pl ine } \\
\underline{uu}\,$\oplus$\,siiss & \textsc{ pl ela } \\
\underline{uu}\,$\oplus$\,siillõ & \textsc{ pl all } \\
\underline{uu}\,$\oplus$\,siill & \textsc{ pl ade } \\
\underline{uu}\,$\oplus$\,siilt & \textsc{ pl abl } \\
\underline{uu}\,$\oplus$\,siissi & \textsc{ pl tra } \\
\underline{uu}\,$\oplus$\,siissaa & \textsc{ pl ter } \\
\underline{uu}\,$\oplus$\,sijka & \textsc{ pl com } \\
\end{tabular}
\end{sideways}
\captionof{table}{Tüüpsõnamall \arabic{mallinumber}\,\vadja{uusi} ekstraheeritud muutvormimallid.}
\label{tab:tüüpsõnamall-uusi}

\end{minipage}

 
\vspace{1em}
\noindent Tüüpsõnamall \vadja{uusi} hõlmab vormisõnastikus 3 lekseemi: \vadja{\underline{uu}si, \underline{voo}si} ja \vadja{\underline{kuu}si}.

%\\
Märkus: kuigi lekseem \textit{voosi}:\textit{voovvõ} hääldub 'vuuvvõ' on see analoogiliselt teiste sõnade muutustega kirjakeeles märgitud \textit{voovvõ}.


\vspace{3.5em}
\noindent \begin{minipage}{\textwidth}
\stepcounter{mallinumber}
\noindent \textbf{Tüüpsõnamall \arabic{mallinumber}\,\vadja{ühs}}\\

\begin{sideways}
\begin{tabular}{l l}
muutvormimall & tunnused \\
\hline
\underline{üh}\,$\oplus$\,s & \textsc{ sg nom } \\
\underline{üh}\,$\oplus$\,e & \textsc{ sg gen } \\
\underline{üh}\,$\oplus$\,te & \textsc{ sg par } \\
\underline{üh}\,$\oplus$\,tese & \textsc{ sg ill } \\
\underline{üh}\,$\oplus$\,ez & \textsc{ sg ine } \\
\underline{üh}\,$\oplus$\,ess & \textsc{ sg ela } \\
\underline{üh}\,$\oplus$\,elle & \textsc{ sg all } \\
\underline{üh}\,$\oplus$\,ell & \textsc{ sg ade } \\
\underline{üh}\,$\oplus$\,elt & \textsc{ sg abl } \\
\underline{üh}\,$\oplus$\,essi & \textsc{ sg tra } \\
\underline{üh}\,$\oplus$\,essaa & \textsc{ sg ter } \\
\underline{üh}\,$\oplus$\,eka & \textsc{ sg com } \\
\underline{üh}\,$\oplus$\,ed & \textsc{ pl nom } \\
\underline{üh}\,$\oplus$\,sije & \textsc{ pl gen } \\
\underline{üh}\,$\oplus$\,siit & \textsc{ pl par } \\
\underline{üh}\,$\oplus$\,siise & \textsc{ pl ill } \\
\underline{üh}\,$\oplus$\,siiz & \textsc{ pl ine } \\
\underline{üh}\,$\oplus$\,siiss & \textsc{ pl ela } \\
\underline{üh}\,$\oplus$\,siille & \textsc{ pl all } \\
\underline{üh}\,$\oplus$\,siill & \textsc{ pl ade } \\
\underline{üh}\,$\oplus$\,siilt & \textsc{ pl abl } \\
\underline{üh}\,$\oplus$\,siissi & \textsc{ pl tra } \\
\underline{üh}\,$\oplus$\,siissaa & \textsc{ pl ter } \\
\underline{üh}\,$\oplus$\,sijka & \textsc{ pl com } \\
\end{tabular}
\end{sideways}
\captionof{table}{Tüüpsõnamall \arabic{mallinumber}\,\vadja{ühs} ekstraheeritud muutvormimallid.}
\label{tab:tüüpsõnamall-ühs}

\end{minipage}

 
\vspace{1em}
\noindent Tüüpsõnamall \vadja{ühs} ei hõlma teisi lekseeme vormi\-sõnastikus.



\vspace{3.5em}
\noindent \begin{minipage}{\textwidth}
\stepcounter{mallinumber}
\noindent \textbf{Tüüpsõnamall \arabic{mallinumber}\,\vadja{ülči}}\\

\begin{sideways}
\begin{tabular}{l l}
muutvormimall & tunnused \\
\hline
\underline{ül}\,$\oplus$\,či & \textsc{ sg nom } \\
\underline{ül}\,$\oplus$\,le & \textsc{ sg gen } \\
\underline{ül}\,$\oplus$\,čiä & \textsc{ sg par } \\
\underline{ül}\,$\oplus$\,čese & \textsc{ sg ill } \\
\underline{ül}\,$\oplus$\,lez & \textsc{ sg ine } \\
\underline{ül}\,$\oplus$\,less & \textsc{ sg ela } \\
\underline{ül}\,$\oplus$\,lelle & \textsc{ sg all } \\
\underline{ül}\,$\oplus$\,lell & \textsc{ sg ade } \\
\underline{ül}\,$\oplus$\,lelt & \textsc{ sg abl } \\
\underline{ül}\,$\oplus$\,lessi & \textsc{ sg tra } \\
\underline{ül}\,$\oplus$\,lessaa & \textsc{ sg ter } \\
\underline{ül}\,$\oplus$\,leka & \textsc{ sg com } \\
\underline{ül}\,$\oplus$\,led & \textsc{ pl nom } \\
\underline{ül}\,$\oplus$\,čije & \textsc{ pl gen } \\
\underline{ül}\,$\oplus$\,čiit & \textsc{ pl par } \\
\underline{ül}\,$\oplus$\,čiise & \textsc{ pl ill } \\
\underline{ül}\,$\oplus$\,čiiz & \textsc{ pl ine } \\
\underline{ül}\,$\oplus$\,čiiss & \textsc{ pl ela } \\
\underline{ül}\,$\oplus$\,čiille & \textsc{ pl all } \\
\underline{ül}\,$\oplus$\,čiill & \textsc{ pl ade } \\
\underline{ül}\,$\oplus$\,čiilt & \textsc{ pl abl } \\
\underline{ül}\,$\oplus$\,čiissi & \textsc{ pl tra } \\
\underline{ül}\,$\oplus$\,čiissaa & \textsc{ pl ter } \\
\underline{ül}\,$\oplus$\,čijka & \textsc{ pl com } \\
\end{tabular}
\end{sideways}
\captionof{table}{Tüüpsõnamall \arabic{mallinumber}\,\vadja{ülči} ekstraheeritud muutvormimallid.}
\label{tab:tüüpsõnamall-ülči}

\end{minipage}

 
\vspace{1em}
\noindent Tüüpsõnamall \vadja{ülči} hõlmab vormisõnastikus 2 lekseemi: \vadja{\underline{ül}či} ja \vadja{\underline{jäl}či}.



\vspace{3.5em}
\noindent \begin{minipage}{\textwidth}
\stepcounter{mallinumber}
\noindent \textbf{Tüüpsõnamall \arabic{mallinumber}\,\vadja{kahs}}\\

\begin{sideways}
\begin{tabular}{l l}
muutvormimall & tunnused \\
\hline
\underline{kah}\,$\oplus$\,s & \textsc{ sg nom } \\
\underline{kah}\,$\oplus$\,õ & \textsc{ sg gen } \\
\underline{kah}\,$\oplus$\,tõ & \textsc{ sg par } \\
\underline{kah}\,$\oplus$\,tõ & \textsc{ sg ill } \\
\underline{kah}\,$\oplus$\,õz & \textsc{ sg ine } \\
\underline{kah}\,$\oplus$\,õss & \textsc{ sg ela } \\
\underline{kah}\,$\oplus$\,õllõ & \textsc{ sg all } \\
\underline{kah}\,$\oplus$\,õll & \textsc{ sg ade } \\
\underline{kah}\,$\oplus$\,õlt & \textsc{ sg abl } \\
\underline{kah}\,$\oplus$\,õssi & \textsc{ sg tra } \\
\underline{kah}\,$\oplus$\,õssaa & \textsc{ sg ter } \\
\underline{kah}\,$\oplus$\,õka & \textsc{ sg com } \\
\underline{kah}\,$\oplus$\,õd & \textsc{ pl nom } \\
\underline{kah}\,$\oplus$\,sijõ & \textsc{ pl gen } \\
\underline{kah}\,$\oplus$\,siit & \textsc{ pl par } \\
\underline{kah}\,$\oplus$\,siisõ & \textsc{ pl ill } \\
\underline{kah}\,$\oplus$\,siiz & \textsc{ pl ine } \\
\underline{kah}\,$\oplus$\,siiss & \textsc{ pl ela } \\
\underline{kah}\,$\oplus$\,siillõ & \textsc{ pl all } \\
\underline{kah}\,$\oplus$\,siill & \textsc{ pl ade } \\
\underline{kah}\,$\oplus$\,siilt & \textsc{ pl abl } \\
\underline{kah}\,$\oplus$\,siissi & \textsc{ pl tra } \\
\underline{kah}\,$\oplus$\,siissaa & \textsc{ pl ter } \\
\underline{kah}\,$\oplus$\,sijka & \textsc{ pl com } \\
\end{tabular}
\end{sideways}
\captionof{table}{Tüüpsõnamall \arabic{mallinumber}\,\vadja{kahs} ekstraheeritud muutvormimallid.}
\label{tab:tüüpsõnamall-kahs}

\end{minipage}

 
\vspace{1em}
\noindent Tüüpsõnamall \vadja{kahs} ei hõlma teisi lekseeme vormi\-sõnastikus.



\vspace{3.5em}
\noindent \begin{minipage}{\textwidth}
\stepcounter{mallinumber}
\noindent \textbf{Tüüpsõnamall \arabic{mallinumber}\,\vadja{varsi}}\\

\begin{sideways}
\begin{tabular}{l l}
muutvormimall & tunnused \\
\hline
\underline{var}\,$\oplus$\,si & \textsc{ sg nom } \\
\underline{var}\,$\oplus$\,rõ & \textsc{ sg gen } \\
\underline{var}\,$\oplus$\,ttõ & \textsc{ sg par } \\
\underline{var}\,$\oplus$\,tõsõ & \textsc{ sg ill } \\
\underline{var}\,$\oplus$\,rõz & \textsc{ sg ine } \\
\underline{var}\,$\oplus$\,rõss & \textsc{ sg ela } \\
\underline{var}\,$\oplus$\,rõllõ & \textsc{ sg all } \\
\underline{var}\,$\oplus$\,rõll & \textsc{ sg ade } \\
\underline{var}\,$\oplus$\,rõlt & \textsc{ sg abl } \\
\underline{var}\,$\oplus$\,rõssi & \textsc{ sg tra } \\
\underline{var}\,$\oplus$\,rõssaa & \textsc{ sg ter } \\
\underline{var}\,$\oplus$\,rõka & \textsc{ sg com } \\
\underline{var}\,$\oplus$\,rõd & \textsc{ pl nom } \\
\underline{var}\,$\oplus$\,sijõ & \textsc{ pl gen } \\
\underline{var}\,$\oplus$\,siit & \textsc{ pl par } \\
\underline{var}\,$\oplus$\,siisõ & \textsc{ pl ill } \\
\underline{var}\,$\oplus$\,siiz & \textsc{ pl ine } \\
\underline{var}\,$\oplus$\,siiss & \textsc{ pl ela } \\
\underline{var}\,$\oplus$\,siillõ & \textsc{ pl all } \\
\underline{var}\,$\oplus$\,siill & \textsc{ pl ade } \\
\underline{var}\,$\oplus$\,siilt & \textsc{ pl abl } \\
\underline{var}\,$\oplus$\,siissi & \textsc{ pl tra } \\
\underline{var}\,$\oplus$\,siissaa & \textsc{ pl ter } \\
\underline{var}\,$\oplus$\,sijka & \textsc{ pl com } \\
\end{tabular}
\end{sideways}
\captionof{table}{Tüüpsõnamall \arabic{mallinumber}\,\vadja{varsi} ekstraheeritud muutvormimallid.}
\label{tab:tüüpsõnamall-varsi}

\end{minipage}

 
\vspace{1em}
\noindent Tüüpsõnamall \vadja{varsi} ei hõlma teisi lekseeme vormi\-sõnastikus.



\vspace{3.5em}
\noindent \begin{minipage}{\textwidth}
\stepcounter{mallinumber}
\noindent \textbf{Tüüpsõnamall \arabic{mallinumber}\,\vadja{mesi}}\\

\begin{sideways}
\begin{tabular}{l l}
muutvormimall & tunnused \\
\hline
\underline{me}\,$\oplus$\,si & \textsc{ sg nom } \\
\underline{me}\,$\oplus$\,e & \textsc{ sg gen } \\
\underline{me}\,$\oplus$\,tte & \textsc{ sg par } \\
\underline{me}\,$\oplus$\,ttese & \textsc{ sg ill } \\
\underline{me}\,$\oplus$\,ez & \textsc{ sg ine } \\
\underline{me}\,$\oplus$\,ess & \textsc{ sg ela } \\
\underline{me}\,$\oplus$\,elle & \textsc{ sg all } \\
\underline{me}\,$\oplus$\,ell & \textsc{ sg ade } \\
\underline{me}\,$\oplus$\,elt & \textsc{ sg abl } \\
\underline{me}\,$\oplus$\,essi & \textsc{ sg tra } \\
\underline{me}\,$\oplus$\,essaa & \textsc{ sg ter } \\
\underline{me}\,$\oplus$\,eka & \textsc{ sg com } \\
\underline{me}\,$\oplus$\,ed & \textsc{ pl nom } \\
\underline{me}\,$\oplus$\,sije & \textsc{ pl gen } \\
\underline{me}\,$\oplus$\,siit & \textsc{ pl par } \\
\underline{me}\,$\oplus$\,siise & \textsc{ pl ill } \\
\underline{me}\,$\oplus$\,siiz & \textsc{ pl ine } \\
\underline{me}\,$\oplus$\,siiss & \textsc{ pl ela } \\
\underline{me}\,$\oplus$\,siille & \textsc{ pl all } \\
\underline{me}\,$\oplus$\,siill & \textsc{ pl ade } \\
\underline{me}\,$\oplus$\,siilt & \textsc{ pl abl } \\
\underline{me}\,$\oplus$\,siissi & \textsc{ pl tra } \\
\underline{me}\,$\oplus$\,siissaa & \textsc{ pl ter } \\
\underline{me}\,$\oplus$\,sijka & \textsc{ pl com } \\
\end{tabular}
\end{sideways}
\captionof{table}{Tüüpsõnamall \arabic{mallinumber}\,\vadja{mesi} ekstraheeritud muutvormimallid.}
\label{tab:tüüpsõnamall-mesi}

\end{minipage}

 
\vspace{1em}
\noindent Tüüpsõnamall \vadja{mesi} hõlmab vormisõnastikus 4 lekseemi: \vadja{\underline{me}si, \underline{sü}si, \underline{ve}si} ja \vadja{\underline{čä}si}.



\vspace{3.5em}
\noindent \begin{minipage}{\textwidth}
\stepcounter{mallinumber}
\noindent \textbf{Tüüpsõnamall \arabic{mallinumber}\,\vadja{susi}}\\

\begin{sideways}
\begin{tabular}{l l}
muutvormimall & tunnused \\
\hline
\underline{su}\,$\oplus$\,si & \textsc{ sg nom } \\
\underline{su}\,$\oplus$\,õ & \textsc{ sg gen } \\
\underline{su}\,$\oplus$\,ttõ & \textsc{ sg par } \\
\underline{su}\,$\oplus$\,ttõsõ & \textsc{ sg ill } \\
\underline{su}\,$\oplus$\,õz & \textsc{ sg ine } \\
\underline{su}\,$\oplus$\,õss & \textsc{ sg ela } \\
\underline{su}\,$\oplus$\,õllõ & \textsc{ sg all } \\
\underline{su}\,$\oplus$\,õll & \textsc{ sg ade } \\
\underline{su}\,$\oplus$\,õlt & \textsc{ sg abl } \\
\underline{su}\,$\oplus$\,õssi & \textsc{ sg tra } \\
\underline{su}\,$\oplus$\,õssaa & \textsc{ sg ter } \\
\underline{su}\,$\oplus$\,õka & \textsc{ sg com } \\
\underline{su}\,$\oplus$\,õd & \textsc{ pl nom } \\
\underline{su}\,$\oplus$\,sijõ & \textsc{ pl gen } \\
\underline{su}\,$\oplus$\,siit & \textsc{ pl par } \\
\underline{su}\,$\oplus$\,siisõ & \textsc{ pl ill } \\
\underline{su}\,$\oplus$\,siiz & \textsc{ pl ine } \\
\underline{su}\,$\oplus$\,siiss & \textsc{ pl ela } \\
\underline{su}\,$\oplus$\,siillõ & \textsc{ pl all } \\
\underline{su}\,$\oplus$\,siill & \textsc{ pl ade } \\
\underline{su}\,$\oplus$\,siilt & \textsc{ pl abl } \\
\underline{su}\,$\oplus$\,siissi & \textsc{ pl tra } \\
\underline{su}\,$\oplus$\,siissaa & \textsc{ pl ter } \\
\underline{su}\,$\oplus$\,sijka & \textsc{ pl com } \\
\end{tabular}
\end{sideways}
\captionof{table}{Tüüpsõnamall \arabic{mallinumber}\,\vadja{susi} ekstraheeritud muutvormimallid.}
\label{tab:tüüpsõnamall-susi}

\end{minipage}

 
\vspace{1em}
\noindent Tüüpsõnamall \vadja{susi} ei hõlma teisi lekseeme vormi\-sõnastikus.

\spacing{1.5}


\subsection{\RN{11} käändkond}

Üheteistkümnendasse käändkonda liigitab Ariste (\citeyear[48]{ariste_grammar_1968}) need sõnad, mille \msd{sg nom} lõpp on \vadja{{-Z}}, ent mille vokaaltüvi sõltub murdest: Jõgõperä murdes realiseerub see kui \vadja{-s-}, Kattila murdes kui \textit{-hs-} ja teistes vadja murretes kui \textit{-ss-}. Vormisõnastikus on järgitud Jõgõperä murdele omast realisatsiooni \textit{-s-}.

Vadja keele sandhi nähtust on kirjakeeles ühtlustatud \vadja{-z} lõpulisteks.

%% Avatuid küsimusi-tähelepanekuid:
%% \begin{itemize}
%% \item \vadja{-Z}-lõpu sandhi nähtus on kõigi liikmete puhul ühtlustatud \vadja{-z} lõpulisteks
%% \item kas seda peab mainima, et Jõgõperä murdes on -s-, Kattila murdes on -hs- ja teistes murretes on -ss-
%% \end{itemize}

\subsubsection*{Ekstraktmorfoloogia tüüpsõnad}
\spacing{1.1}
\vspace{-3.5em}


\vspace{3.5em}
\noindent \begin{minipage}{\textwidth}
\stepcounter{mallinumber}
\noindent \textbf{Tüüpsõnamall \arabic{mallinumber}\,\vadja{makuz}}\\

\begin{sideways}
\begin{tabular}{l l}
muutvormimall & tunnused \\
\hline
\underline{maku}\,$\oplus$\,z & \textsc{ sg nom } \\
\underline{maku}\,$\oplus$\,sõ & \textsc{ sg gen } \\
\underline{maku}\,$\oplus$\,ssõ & \textsc{ sg par } \\
\underline{maku}\,$\oplus$\,zõsõ & \textsc{ sg ill } \\
\underline{maku}\,$\oplus$\,zõz & \textsc{ sg ine } \\
\underline{maku}\,$\oplus$\,zõss & \textsc{ sg ela } \\
\underline{maku}\,$\oplus$\,zõllõ & \textsc{ sg all } \\
\underline{maku}\,$\oplus$\,zõll & \textsc{ sg ade } \\
\underline{maku}\,$\oplus$\,zõlt & \textsc{ sg abl } \\
\underline{maku}\,$\oplus$\,zõssi & \textsc{ sg tra } \\
\underline{maku}\,$\oplus$\,zõssaa & \textsc{ sg ter } \\
\underline{maku}\,$\oplus$\,zõka & \textsc{ sg com } \\
\underline{maku}\,$\oplus$\,zõd & \textsc{ pl nom } \\
\underline{maku}\,$\oplus$\,sijõ & \textsc{ pl gen } \\
\underline{maku}\,$\oplus$\,siit & \textsc{ pl par } \\
\underline{maku}\,$\oplus$\,siisõ & \textsc{ pl ill } \\
\underline{maku}\,$\oplus$\,siiz & \textsc{ pl ine } \\
\underline{maku}\,$\oplus$\,siiss & \textsc{ pl ela } \\
\underline{maku}\,$\oplus$\,siillõ & \textsc{ pl all } \\
\underline{maku}\,$\oplus$\,siill & \textsc{ pl ade } \\
\underline{maku}\,$\oplus$\,siilt & \textsc{ pl abl } \\
\underline{maku}\,$\oplus$\,siissi & \textsc{ pl tra } \\
\underline{maku}\,$\oplus$\,siissaa & \textsc{ pl ter } \\
\underline{maku}\,$\oplus$\,sijka & \textsc{ pl com } \\
\end{tabular}
\end{sideways}
\captionof{table}{Tüüpsõnamall \arabic{mallinumber}\,\vadja{makuz} ekstraheeritud muutvormimallid.}
\label{tab:tüüpsõnamall-makuz}

\end{minipage}

 
\vspace{1em}
\noindent Tüüpsõnamall \vadja{makuz} hõlmab vormisõnastikus 4 lekseemi: \vadja{\underline{maku}z, \underline{nagri}z, \underline{paganu}z} ja \vadja{\underline{kolau}z}.



\vspace{3.5em}
\noindent \begin{minipage}{\textwidth}
\stepcounter{mallinumber}
\noindent \textbf{Tüüpsõnamall \arabic{mallinumber}\,\vadja{rihenneüz}}\\

\begin{sideways}
\begin{tabular}{l l}
muutvormimall & tunnused \\
\hline
\underline{rihenneü}\,$\oplus$\,z & \textsc{ sg nom } \\
\underline{rihenneü}\,$\oplus$\,se & \textsc{ sg gen } \\
\underline{rihenneü}\,$\oplus$\,sse & \textsc{ sg par } \\
\underline{rihenneü}\,$\oplus$\,sesse & \textsc{ sg ill } \\
\underline{rihenneü}\,$\oplus$\,sez & \textsc{ sg ine } \\
\underline{rihenneü}\,$\oplus$\,sess & \textsc{ sg ela } \\
\underline{rihenneü}\,$\oplus$\,selle & \textsc{ sg all } \\
\underline{rihenneü}\,$\oplus$\,sell & \textsc{ sg ade } \\
\underline{rihenneü}\,$\oplus$\,selt & \textsc{ sg abl } \\
\underline{rihenneü}\,$\oplus$\,sessi & \textsc{ sg tra } \\
\underline{rihenneü}\,$\oplus$\,sessaa & \textsc{ sg ter } \\
\underline{rihenneü}\,$\oplus$\,seka & \textsc{ sg com } \\
\underline{rihenneü}\,$\oplus$\,sed & \textsc{ pl nom } \\
\underline{rihenneü}\,$\oplus$\,sije & \textsc{ pl gen } \\
\underline{rihenneü}\,$\oplus$\,siit & \textsc{ pl par } \\
\underline{rihenneü}\,$\oplus$\,siise & \textsc{ pl ill } \\
\underline{rihenneü}\,$\oplus$\,siiz & \textsc{ pl ine } \\
\underline{rihenneü}\,$\oplus$\,siiss & \textsc{ pl ela } \\
\underline{rihenneü}\,$\oplus$\,siille & \textsc{ pl all } \\
\underline{rihenneü}\,$\oplus$\,siill & \textsc{ pl ade } \\
\underline{rihenneü}\,$\oplus$\,siilt & \textsc{ pl abl } \\
\underline{rihenneü}\,$\oplus$\,siissi & \textsc{ pl tra } \\
\underline{rihenneü}\,$\oplus$\,siissaa & \textsc{ pl ter } \\
\underline{rihenneü}\,$\oplus$\,sijka & \textsc{ pl com } \\
\end{tabular}
\end{sideways}
\captionof{table}{Tüüpsõnamall \arabic{mallinumber}\,\vadja{rihenneüz} ekstraheeritud muutvormimallid.}
\label{tab:tüüpsõnamall-rihenneüz}

\end{minipage}

 
\vspace{1em}
\noindent Tüüpsõnamall \vadja{rihenneüz} ei hõlma teisi lekseeme vormi\-sõnastikus.

\spacing{1.5}


\subsection{\RN{12} käändkond}

Kaheteistkümnes käändkond koondab need sõnad, mille \msd{sg nom} lõpp on \vadja{-n/-ne/-nõ}, ent mille vokaaltüves on \vadja{-se-}:\vadja{-ze-/-sö-}:\vadja{-zö-} sõltuvalt astmevaheldusest (\cite[49]{ariste_grammar_1968}).

Mitmuse vormide vokaaltüved on kunstlikult ühtlustatud tugevaastmeliseks (\textit{-s-}). Tsvetkovi sõnaraamatus esitatud sõnades esineb suur varieeruvus tüve\-astme puhul ja siin\-kirjutaja ei ole suutnud leida mingit reeglipära. Sellega seotud on küsimus, kas \msd{pl gen} tüvi peaks vahelduma teistest \msd{pl} käänete astmest.

Pika vokaaliga lõpevate sõnade puhul esineb ka varieeruvust: \textit{süsiin} ja \textit{õnnõliin} puhul on see pikk ainult \msd{sg nom} ja \msd{sg par} käänetes, ning \textit{magnettiin} ja \textit{talviin} puhul on see pikk igas käändes.

%% Avatuid küsimusi-tähelepanekuid:
%% \begin{itemize}
%% \item pl tüvi ühtlustatud -s- igal pool TODO üle vaadata s:z vaheldus pluuralis, kas see on s kui 1. silp on pikk v kinnine? (Tsvetkovil pole reeglipäraselt vaid variatsiooniline)
%% \item kas pl gen peaks vahelduma -z- (iloin)? või -s- (keskolin)?
%% \item talviisijõ,talviiziit
%% \item õpõin on väga erandlik sõna
%% \end{itemize}

\subsubsection*{Ekstraktmorfoloogia tüüpsõnad}
\spacing{1.1}
\vspace{-3.5em}


\vspace{3.5em}
\noindent \begin{minipage}{\textwidth}
\stepcounter{mallinumber}
\noindent \textbf{Tüüpsõnamall \arabic{mallinumber}\,\vadja{õpõin}}\\

\begin{sideways}
\begin{tabular}{l l}
muutvormimall & tunnused \\
\hline
\underline{õ}\,$\oplus$\,p\,$\oplus$\,\underline{õ}\,$\oplus$\,in & \textsc{ sg nom } \\
\underline{õ}\,$\oplus$\,p\,$\oplus$\,\underline{õ}\,$\oplus$\,izõ & \textsc{ sg gen } \\
\underline{õ}\,$\oplus$\,v\,$\oplus$\,\underline{õ}\,$\oplus$\,issõ & \textsc{ sg par } \\
\underline{õ}\,$\oplus$\,p\,$\oplus$\,\underline{õ}\,$\oplus$\,zõsõ & \textsc{ sg ill } \\
\underline{õ}\,$\oplus$\,p\,$\oplus$\,\underline{õ}\,$\oplus$\,izõz & \textsc{ sg ine } \\
\underline{õ}\,$\oplus$\,p\,$\oplus$\,\underline{õ}\,$\oplus$\,izõss & \textsc{ sg ela } \\
\underline{õ}\,$\oplus$\,p\,$\oplus$\,\underline{õ}\,$\oplus$\,izõllõ & \textsc{ sg all } \\
\underline{õ}\,$\oplus$\,p\,$\oplus$\,\underline{õ}\,$\oplus$\,izõll & \textsc{ sg ade } \\
\underline{õ}\,$\oplus$\,p\,$\oplus$\,\underline{õ}\,$\oplus$\,izõlt & \textsc{ sg abl } \\
\underline{õ}\,$\oplus$\,p\,$\oplus$\,\underline{õ}\,$\oplus$\,izõssi & \textsc{ sg tra } \\
\underline{õ}\,$\oplus$\,p\,$\oplus$\,\underline{õ}\,$\oplus$\,izõssaa & \textsc{ sg ter } \\
\underline{õ}\,$\oplus$\,p\,$\oplus$\,\underline{õ}\,$\oplus$\,izõka & \textsc{ sg com } \\
\underline{õ}\,$\oplus$\,p\,$\oplus$\,\underline{õ}\,$\oplus$\,izõd & \textsc{ pl nom } \\
\underline{õ}\,$\oplus$\,p\,$\oplus$\,\underline{õ}\,$\oplus$\,izijõ & \textsc{ pl gen } \\
\underline{õ}\,$\oplus$\,p\,$\oplus$\,\underline{õ}\,$\oplus$\,iziit & \textsc{ pl par } \\
\underline{õ}\,$\oplus$\,p\,$\oplus$\,\underline{õ}\,$\oplus$\,iziisõ & \textsc{ pl ill } \\
\underline{õ}\,$\oplus$\,p\,$\oplus$\,\underline{õ}\,$\oplus$\,iziiz & \textsc{ pl ine } \\
\underline{õ}\,$\oplus$\,p\,$\oplus$\,\underline{õ}\,$\oplus$\,iziiss & \textsc{ pl ela } \\
\underline{õ}\,$\oplus$\,p\,$\oplus$\,\underline{õ}\,$\oplus$\,iziillõ & \textsc{ pl all } \\
\underline{õ}\,$\oplus$\,p\,$\oplus$\,\underline{õ}\,$\oplus$\,iziill & \textsc{ pl ade } \\
\underline{õ}\,$\oplus$\,p\,$\oplus$\,\underline{õ}\,$\oplus$\,iziilt & \textsc{ pl abl } \\
\underline{õ}\,$\oplus$\,p\,$\oplus$\,\underline{õ}\,$\oplus$\,iziissi & \textsc{ pl tra } \\
\underline{õ}\,$\oplus$\,p\,$\oplus$\,\underline{õ}\,$\oplus$\,iziissaa & \textsc{ pl ter } \\
\underline{õ}\,$\oplus$\,p\,$\oplus$\,\underline{õ}\,$\oplus$\,izijka & \textsc{ pl com } \\
\end{tabular}
\end{sideways}
\captionof{table}{Tüüpsõnamall \arabic{mallinumber}\,\vadja{õpõin} ekstraheeritud muutvormimallid.}
\label{tab:tüüpsõnamall-õpõin}

\end{minipage}

 
\vspace{1em}
\noindent Tüüpsõnamall \vadja{õpõin} ei hõlma teisi lekseeme vormi\-sõnastikus.



\vspace{3.5em}
\noindent \begin{minipage}{\textwidth}
\stepcounter{mallinumber}
\noindent \textbf{Tüüpsõnamall \arabic{mallinumber}\,\vadja{entin}}\\

\begin{sideways}
\begin{tabular}{l l}
muutvormimall & tunnused \\
\hline
\underline{enti}\,$\oplus$\,n & \textsc{ sg nom } \\
\underline{enti}\,$\oplus$\,ze & \textsc{ sg gen } \\
\underline{enti}\,$\oplus$\,sse & \textsc{ sg par } \\
\underline{enti}\,$\oplus$\,zese & \textsc{ sg ill } \\
\underline{enti}\,$\oplus$\,zez & \textsc{ sg ine } \\
\underline{enti}\,$\oplus$\,zess & \textsc{ sg ela } \\
\underline{enti}\,$\oplus$\,zelle & \textsc{ sg all } \\
\underline{enti}\,$\oplus$\,zell & \textsc{ sg ade } \\
\underline{enti}\,$\oplus$\,zelt & \textsc{ sg abl } \\
\underline{enti}\,$\oplus$\,zessi & \textsc{ sg tra } \\
\underline{enti}\,$\oplus$\,zessaa & \textsc{ sg ter } \\
\underline{enti}\,$\oplus$\,zeka & \textsc{ sg com } \\
\underline{enti}\,$\oplus$\,zed & \textsc{ pl nom } \\
\underline{enti}\,$\oplus$\,sije & \textsc{ pl gen } \\
\underline{enti}\,$\oplus$\,siit & \textsc{ pl par } \\
\underline{enti}\,$\oplus$\,siise & \textsc{ pl ill } \\
\underline{enti}\,$\oplus$\,siiz & \textsc{ pl ine } \\
\underline{enti}\,$\oplus$\,siiss & \textsc{ pl ela } \\
\underline{enti}\,$\oplus$\,siille & \textsc{ pl all } \\
\underline{enti}\,$\oplus$\,siill & \textsc{ pl ade } \\
\underline{enti}\,$\oplus$\,siilt & \textsc{ pl abl } \\
\underline{enti}\,$\oplus$\,siissi & \textsc{ pl tra } \\
\underline{enti}\,$\oplus$\,siissaa & \textsc{ pl ter } \\
\underline{enti}\,$\oplus$\,sijka & \textsc{ pl com } \\
\end{tabular}
\end{sideways}
\captionof{table}{Tüüpsõnamall \arabic{mallinumber}\,\vadja{entin} ekstraheeritud muutvormimallid.}
\label{tab:tüüpsõnamall-entin}

\end{minipage}

 
\vspace{1em}
\noindent Tüüpsõnamall \vadja{entin} hõlmab vormisõnastikus 9 lekseemi: \vadja{\underline{enti}n, \underline{esimei}n, \underline{inimi}n, \underline{jäi}n, \underline{miltäi}n, \underline{reheli}n, \underline{sini}n, \underline{venäläi}n} ja \vadja{\underline{egli}n}.



\vspace{3.5em}
\noindent \begin{minipage}{\textwidth}
\stepcounter{mallinumber}
\noindent \textbf{Tüüpsõnamall \arabic{mallinumber}\,\vadja{süsiin}}\\

\begin{sideways}
\begin{tabular}{l l}
muutvormimall & tunnused \\
\hline
\underline{süsi}\,$\oplus$\,in & \textsc{ sg nom } \\
\underline{süsi}\,$\oplus$\,ze & \textsc{ sg gen } \\
\underline{süsi}\,$\oplus$\,isse & \textsc{ sg par } \\
\underline{süsi}\,$\oplus$\,zese & \textsc{ sg ill } \\
\underline{süsi}\,$\oplus$\,zez & \textsc{ sg ine } \\
\underline{süsi}\,$\oplus$\,zess & \textsc{ sg ela } \\
\underline{süsi}\,$\oplus$\,zelle & \textsc{ sg all } \\
\underline{süsi}\,$\oplus$\,zell & \textsc{ sg ade } \\
\underline{süsi}\,$\oplus$\,zelt & \textsc{ sg abl } \\
\underline{süsi}\,$\oplus$\,zessi & \textsc{ sg tra } \\
\underline{süsi}\,$\oplus$\,zessaa & \textsc{ sg ter } \\
\underline{süsi}\,$\oplus$\,zeka & \textsc{ sg com } \\
\underline{süsi}\,$\oplus$\,zed & \textsc{ pl nom } \\
\underline{süsi}\,$\oplus$\,sije & \textsc{ pl gen } \\
\underline{süsi}\,$\oplus$\,siit & \textsc{ pl par } \\
\underline{süsi}\,$\oplus$\,siise & \textsc{ pl ill } \\
\underline{süsi}\,$\oplus$\,siiz & \textsc{ pl ine } \\
\underline{süsi}\,$\oplus$\,siiss & \textsc{ pl ela } \\
\underline{süsi}\,$\oplus$\,siille & \textsc{ pl all } \\
\underline{süsi}\,$\oplus$\,siill & \textsc{ pl ade } \\
\underline{süsi}\,$\oplus$\,siilt & \textsc{ pl abl } \\
\underline{süsi}\,$\oplus$\,siissi & \textsc{ pl tra } \\
\underline{süsi}\,$\oplus$\,siissaa & \textsc{ pl ter } \\
\underline{süsi}\,$\oplus$\,sijka & \textsc{ pl com } \\
\end{tabular}
\end{sideways}
\captionof{table}{Tüüpsõnamall \arabic{mallinumber}\,\vadja{süsiin} ekstraheeritud muutvormimallid.}
\label{tab:tüüpsõnamall-süsiin}

\end{minipage}

 
\vspace{1em}
\noindent Tüüpsõnamall \vadja{süsiin} ei hõlma teisi lekseeme vormi\-sõnastikus.



\vspace{3.5em}
\noindent \begin{minipage}{\textwidth}
\stepcounter{mallinumber}
\noindent \textbf{Tüüpsõnamall \arabic{mallinumber}\,\vadja{õnnõliin}}\\

\begin{sideways}
\begin{tabular}{l l}
muutvormimall & tunnused \\
\hline
\underline{õnnõli}\,$\oplus$\,in & \textsc{ sg nom } \\
\underline{õnnõli}\,$\oplus$\,zõ & \textsc{ sg gen } \\
\underline{õnnõli}\,$\oplus$\,issõ & \textsc{ sg par } \\
\underline{õnnõli}\,$\oplus$\,zõsõ & \textsc{ sg ill } \\
\underline{õnnõli}\,$\oplus$\,zõz & \textsc{ sg ine } \\
\underline{õnnõli}\,$\oplus$\,zõss & \textsc{ sg ela } \\
\underline{õnnõli}\,$\oplus$\,zõllõ & \textsc{ sg all } \\
\underline{õnnõli}\,$\oplus$\,zõll & \textsc{ sg ade } \\
\underline{õnnõli}\,$\oplus$\,zõlt & \textsc{ sg abl } \\
\underline{õnnõli}\,$\oplus$\,zõssi & \textsc{ sg tra } \\
\underline{õnnõli}\,$\oplus$\,zõssaa & \textsc{ sg ter } \\
\underline{õnnõli}\,$\oplus$\,zõka & \textsc{ sg com } \\
\underline{õnnõli}\,$\oplus$\,zõd & \textsc{ pl nom } \\
\underline{õnnõli}\,$\oplus$\,sijõ & \textsc{ pl gen } \\
\underline{õnnõli}\,$\oplus$\,siit & \textsc{ pl par } \\
\underline{õnnõli}\,$\oplus$\,siisõ & \textsc{ pl ill } \\
\underline{õnnõli}\,$\oplus$\,siiz & \textsc{ pl ine } \\
\underline{õnnõli}\,$\oplus$\,siiss & \textsc{ pl ela } \\
\underline{õnnõli}\,$\oplus$\,siillõ & \textsc{ pl all } \\
\underline{õnnõli}\,$\oplus$\,siill & \textsc{ pl ade } \\
\underline{õnnõli}\,$\oplus$\,siilt & \textsc{ pl abl } \\
\underline{õnnõli}\,$\oplus$\,siissi & \textsc{ pl tra } \\
\underline{õnnõli}\,$\oplus$\,siissaa & \textsc{ pl ter } \\
\underline{õnnõli}\,$\oplus$\,sijka & \textsc{ pl com } \\
\end{tabular}
\end{sideways}
\captionof{table}{Tüüpsõnamall \arabic{mallinumber}\,\vadja{õnnõliin} ekstraheeritud muutvormimallid.}
\label{tab:tüüpsõnamall-õnnõliin}

\end{minipage}

 
\vspace{1em}
\noindent Tüüpsõnamall \vadja{õnnõliin} ei hõlma teisi lekseeme vormi\-sõnastikus.



\vspace{3.5em}
\noindent \begin{minipage}{\textwidth}
\stepcounter{mallinumber}
\noindent \textbf{Tüüpsõnamall \arabic{mallinumber}\,\vadja{čimolain}}\\

\begin{sideways}
\begin{tabular}{l l}
muutvormimall & tunnused \\
\hline
\underline{čimolai}\,$\oplus$\,n & \textsc{ sg nom } \\
\underline{čimolai}\,$\oplus$\,zõ & \textsc{ sg gen } \\
\underline{čimolai}\,$\oplus$\,ssõ & \textsc{ sg par } \\
\underline{čimolai}\,$\oplus$\,zõsõ & \textsc{ sg ill } \\
\underline{čimolai}\,$\oplus$\,zõz & \textsc{ sg ine } \\
\underline{čimolai}\,$\oplus$\,zõss & \textsc{ sg ela } \\
\underline{čimolai}\,$\oplus$\,zõllõ & \textsc{ sg all } \\
\underline{čimolai}\,$\oplus$\,zõll & \textsc{ sg ade } \\
\underline{čimolai}\,$\oplus$\,zõlt & \textsc{ sg abl } \\
\underline{čimolai}\,$\oplus$\,zõssi & \textsc{ sg tra } \\
\underline{čimolai}\,$\oplus$\,zõssaa & \textsc{ sg ter } \\
\underline{čimolai}\,$\oplus$\,zõka & \textsc{ sg com } \\
\underline{čimolai}\,$\oplus$\,zõd & \textsc{ pl nom } \\
\underline{čimolai}\,$\oplus$\,sijõ & \textsc{ pl gen } \\
\underline{čimolai}\,$\oplus$\,siit & \textsc{ pl par } \\
\underline{čimolai}\,$\oplus$\,siisõ & \textsc{ pl ill } \\
\underline{čimolai}\,$\oplus$\,siiz & \textsc{ pl ine } \\
\underline{čimolai}\,$\oplus$\,siiss & \textsc{ pl ela } \\
\underline{čimolai}\,$\oplus$\,siillõ & \textsc{ pl all } \\
\underline{čimolai}\,$\oplus$\,siill & \textsc{ pl ade } \\
\underline{čimolai}\,$\oplus$\,siilt & \textsc{ pl abl } \\
\underline{čimolai}\,$\oplus$\,siissi & \textsc{ pl tra } \\
\underline{čimolai}\,$\oplus$\,siissaa & \textsc{ pl ter } \\
\underline{čimolai}\,$\oplus$\,sijka & \textsc{ pl com } \\
\end{tabular}
\end{sideways}
\captionof{table}{Tüüpsõnamall \arabic{mallinumber}\,\vadja{čimolain} ekstraheeritud muutvormimallid.}
\label{tab:tüüpsõnamall-čimolain}

\end{minipage}

 
\vspace{1em}
\noindent Tüüpsõnamall \vadja{čimolain} hõlmab vormisõnastikus 37 lekseemi: \vadja{\underline{čimolai}n, \underline{greekklai}n, \underline{hatukkõi}n, \underline{iirikkõi}n, \underline{il̕l̕õkkõi}n, \underline{iloi}n, \underline{jõkai}n, \underline{kehnokkõi}n, \underline{keskoli}n, \underline{kõikõllai}n, \underline{kõrvõli}n, \underline{leivekkõi}n, \underline{lui}n, \underline{lättilai}n, \underline{magnettii}n, \underline{mai}n, \underline{mokomõi}n, \underline{mustõlai}n, \underline{nai}n, \underline{partõi}n, \underline{perennai}n, \underline{prikukkõi}n, \underline{pui}n, \underline{roottsilai}n, \underline{ruskolai}n, \underline{saunlai}n, \underline{soomõlai}n, \underline{sopui}n, \underline{sukulai}n, \underline{talvii}n, \underline{tarttulai}n, \underline{tõi}n, \underline{ukrainalai}n, \underline{virolai}n, \underline{õhtõgoi}n, \underline{ühellai}n} ja \vadja{\underline{audžikkõi}n}.



\vspace{3.5em}
\noindent \begin{minipage}{\textwidth}
\stepcounter{mallinumber}
\noindent \textbf{Tüüpsõnamall \arabic{mallinumber}\,\vadja{koivuin}}\\

\begin{sideways}
\begin{tabular}{l l}
muutvormimall & tunnused \\
\hline
\underline{koivui}\,$\oplus$\,n & \textsc{ sg nom } \\
\underline{koivui}\,$\oplus$\,zõ & \textsc{ sg gen } \\
\underline{koivui}\,$\oplus$\,ssõ & \textsc{ sg par } \\
\underline{koivui}\,$\oplus$\,zõsõ & \textsc{ sg ill } \\
\underline{koivui}\,$\oplus$\,zõz & \textsc{ sg ine } \\
\underline{koivui}\,$\oplus$\,zõss & \textsc{ sg ela } \\
\underline{koivui}\,$\oplus$\,zõllõ & \textsc{ sg all } \\
\underline{koivui}\,$\oplus$\,zõll & \textsc{ sg ade } \\
\underline{koivui}\,$\oplus$\,zõlt & \textsc{ sg abl } \\
\underline{koivui}\,$\oplus$\,zõssi & \textsc{ sg tra } \\
\underline{koivui}\,$\oplus$\,zõssaa & \textsc{ sg ter } \\
\underline{koivui}\,$\oplus$\,zõka & \textsc{ sg com } \\
\underline{koivui}\,$\oplus$\,zõd & \textsc{ pl nom } \\
\underline{koivui}\,$\oplus$\,zijõ & \textsc{ pl gen } \\
\underline{koivui}\,$\oplus$\,ziit & \textsc{ pl par } \\
\underline{koivui}\,$\oplus$\,ziisõ & \textsc{ pl ill } \\
\underline{koivui}\,$\oplus$\,ziiz & \textsc{ pl ine } \\
\underline{koivui}\,$\oplus$\,ziiss & \textsc{ pl ela } \\
\underline{koivui}\,$\oplus$\,ziillõ & \textsc{ pl all } \\
\underline{koivui}\,$\oplus$\,ziill & \textsc{ pl ade } \\
\underline{koivui}\,$\oplus$\,ziilt & \textsc{ pl abl } \\
\underline{koivui}\,$\oplus$\,ziissi & \textsc{ pl tra } \\
\underline{koivui}\,$\oplus$\,ziissaa & \textsc{ pl ter } \\
\underline{koivui}\,$\oplus$\,zijka & \textsc{ pl com } \\
\end{tabular}
\end{sideways}
\captionof{table}{Tüüpsõnamall \arabic{mallinumber}\,\vadja{koivuin} ekstraheeritud muutvormimallid.}
\label{tab:tüüpsõnamall-koivuin}

\end{minipage}

 
\vspace{1em}
\noindent Tüüpsõnamall \vadja{koivuin} hõlmab vormisõnastikus 8 lekseemi: \vadja{\underline{koivui}n, \underline{kultõi}n, \underline{kõltõi}n, \underline{pakkõi}n, \underline{rohoi}n, \underline{uuti}n, \underline{voosi}n} ja \vadja{\underline{kalttõi}n}.

\spacing{1.5}


\subsection{\RN{13} käändkond}

Kolmeteistkümnendasse käändkonda kuuluvad need sõnad, mis lõpevad pika vokaaliga \msd{sg nom} käändes. Lisaks kuuluvad siia mõned sõnad, mis lõpevad diftongiga \msd{sg nom}. (\cite[49]{ariste_grammar_1968})

Käesolevas töös on siia käändkonda koondatud need sõnad, mis \msd{sg gen} käändes lõpevad \textit{-me/-mä/-ma/-mõ/-ne/-re}. Suure variatiivsuse tõttu on käändkonda kuuluvad sõnad jäetud ühtlustamata. Variatiivsust esineb nii Tsvetkovi sõnaraamatu sees, kui ka õppevahendite vahel. % TODO lisada võti Tsvetkovilt ja Heinsoolt
%Konkoval (\citeyear{konkova_vaddceeli_2014}) on esitatud \textit{võttim}:\textit{võttimõ}:\textit{võttima} ent Aristel, Tsvetkovil ja Heinsool on \textit{võti}.

%Näiteks Tsvetkovil on \textit{süä}:\textit{süä}:\textit{süttä/süät}.

%% Avatuid küsimusi-tähelepanekuid:
%% \begin{itemize}
%% \item Aristel pole \vadja{seemen} vaid on seemee:seemenee:seemeetä
%% \item Tsvetkovil pole süä:süäme vaid on süä:süä:süttä/süät
%% \item Konkoval on võttim:võttimõ:võttima (Tsvetkovil on näitelauses võti)
%% \end{itemize}

%% Veel kuuluvad siia käändkonda ordinaalid kolmest edasi \cite[50]{ariste_grammar_1968}.
%% Numeraalide puhul on järgitud Rozhanskiy ja Markuse välja toodud:
%% \begin{itemize}
%% \item \msd{sg nom} lõpp on \vadja{-iz}
%% \end{itemize}


\subsubsection*{Ekstraktmorfoloogia tüüpsõnad}
\spacing{1}
\vspace{-3.5em}


\vspace{3.5em}
\noindent \begin{minipage}{\textwidth}
\stepcounter{mallinumber}
\noindent \textbf{Tüüpsõnamall \arabic{mallinumber}\,\vadja{čümme}}\\

\begin{sideways}
\begin{tabular}{l l}
muutvormimall & tunnused \\
\hline
\underline{čümme} & \textsc{ sg nom } \\
\underline{čümme}\,$\oplus$\,ne & \textsc{ sg gen } \\
\underline{čümme}\,$\oplus$\,nä & \textsc{ sg par } \\
\underline{čümme}\,$\oplus$\,nese & \textsc{ sg ill } \\
\underline{čümme}\,$\oplus$\,z & \textsc{ sg ine } \\
\underline{čümme}\,$\oplus$\,ss & \textsc{ sg ela } \\
\underline{čümme}\,$\oplus$\,lle & \textsc{ sg all } \\
\underline{čümme}\,$\oplus$\,ll & \textsc{ sg ade } \\
\underline{čümme}\,$\oplus$\,lt & \textsc{ sg abl } \\
\underline{čümme}\,$\oplus$\,ssi & \textsc{ sg tra } \\
\underline{čümme}\,$\oplus$\,ssaa & \textsc{ sg ter } \\
\underline{čümme}\,$\oplus$\,ka & \textsc{ sg com } \\
\underline{čümme}\,$\oplus$\,d & \textsc{ pl nom } \\
\underline{čümme}\,$\oplus$\,nije & \textsc{ pl gen } \\
\underline{čümme}\,$\oplus$\,niit & \textsc{ pl par } \\
\underline{čümme}\,$\oplus$\,niise & \textsc{ pl ill } \\
\underline{čümme}\,$\oplus$\,niiz & \textsc{ pl ine } \\
\underline{čümme}\,$\oplus$\,niiss & \textsc{ pl ela } \\
\underline{čümme}\,$\oplus$\,niille & \textsc{ pl all } \\
\underline{čümme}\,$\oplus$\,niill & \textsc{ pl ade } \\
\underline{čümme}\,$\oplus$\,niilt & \textsc{ pl abl } \\
\underline{čümme}\,$\oplus$\,niissi & \textsc{ pl tra } \\
\underline{čümme}\,$\oplus$\,niissaa & \textsc{ pl ter } \\
\underline{čümme}\,$\oplus$\,nijka & \textsc{ pl com } \\
\end{tabular}
\end{sideways}
\captionof{table}{Tüüpsõnamall \arabic{mallinumber}\,\vadja{čümme} ekstraheeritud muutvormimallid.}
\label{tab:tüüpsõnamall-čümme}

\end{minipage}

 
\vspace{1em}
\noindent Tüüpsõnamall \vadja{čümme} ei hõlma teisi lekseeme vormi\-sõnastikus.



\vspace{3.5em}
\noindent \begin{minipage}{\textwidth}
\stepcounter{mallinumber}
\noindent \textbf{Tüüpsõnamall \arabic{mallinumber}\,\vadja{čümmenäz}}\\

\begin{sideways}
\begin{tabular}{l l}
muutvormimall & tunnused \\
\hline
\underline{čümmen}\,$\oplus$\,äz & \textsc{ sg nom } \\
\underline{čümmen}\,$\oplus$\,ettemä & \textsc{ sg gen } \\
\underline{čümmen}\,$\oplus$\,eii & \textsc{ sg par } \\
\underline{čümmen}\,$\oplus$\,ettemäse & \textsc{ sg ill } \\
\underline{čümmen}\,$\oplus$\,ettemäz & \textsc{ sg ine } \\
\underline{čümmen}\,$\oplus$\,ettemäss & \textsc{ sg ela } \\
\underline{čümmen}\,$\oplus$\,ettemälle & \textsc{ sg all } \\
\underline{čümmen}\,$\oplus$\,ettemäll & \textsc{ sg ade } \\
\underline{čümmen}\,$\oplus$\,ettemält & \textsc{ sg abl } \\
\underline{čümmen}\,$\oplus$\,ettemässi & \textsc{ sg tra } \\
\underline{čümmen}\,$\oplus$\,ettemässaa & \textsc{ sg ter } \\
\underline{čümmen}\,$\oplus$\,ettemäka & \textsc{ sg com } \\
\underline{čümmen}\,$\oplus$\,ettemäd & \textsc{ pl nom } \\
\underline{čümmen}\,$\oplus$\,ettemije & \textsc{ pl gen } \\
\underline{čümmen}\,$\oplus$\,ettemiit & \textsc{ pl par } \\
\underline{čümmen}\,$\oplus$\,ettemiise & \textsc{ pl ill } \\
\underline{čümmen}\,$\oplus$\,ettemiiz & \textsc{ pl ine } \\
\underline{čümmen}\,$\oplus$\,ettemiiss & \textsc{ pl ela } \\
\underline{čümmen}\,$\oplus$\,ettemiille & \textsc{ pl all } \\
\underline{čümmen}\,$\oplus$\,ettemiill & \textsc{ pl ade } \\
\underline{čümmen}\,$\oplus$\,ettemiilt & \textsc{ pl abl } \\
\underline{čümmen}\,$\oplus$\,ettemiissi & \textsc{ pl tra } \\
\underline{čümmen}\,$\oplus$\,ettemiissaa & \textsc{ pl ter } \\
\underline{čümmen}\,$\oplus$\,ettemijka & \textsc{ pl com } \\
\end{tabular}
\end{sideways}
\captionof{table}{Tüüpsõnamall \arabic{mallinumber}\,\vadja{čümmenäz} ekstraheeritud muutvormimallid.}
\label{tab:tüüpsõnamall-čümmenäz}

\end{minipage}

 
\vspace{1em}
\noindent Tüüpsõnamall \vadja{čümmenäz} ei hõlma teisi lekseeme vormi\-sõnastikus.



\vspace{3.5em}
\noindent \begin{minipage}{\textwidth}
\stepcounter{mallinumber}
\noindent \textbf{Tüüpsõnamall \arabic{mallinumber}\,\vadja{nel̕l̕äz}}\\

\begin{sideways}
\begin{tabular}{l l}
muutvormimall & tunnused \\
\hline
\underline{nel̕l̕}\,$\oplus$\,äz & \textsc{ sg nom } \\
\underline{nel̕l̕}\,$\oplus$\,ettemä & \textsc{ sg gen } \\
\underline{nel̕l̕}\,$\oplus$\,että & \textsc{ sg par } \\
\underline{nel̕l̕}\,$\oplus$\,ettemäse & \textsc{ sg ill } \\
\underline{nel̕l̕}\,$\oplus$\,etteemäz & \textsc{ sg ine } \\
\underline{nel̕l̕}\,$\oplus$\,etteemäss & \textsc{ sg ela } \\
\underline{nel̕l̕}\,$\oplus$\,etteemälle & \textsc{ sg all } \\
\underline{nel̕l̕}\,$\oplus$\,etteemäll & \textsc{ sg ade } \\
\underline{nel̕l̕}\,$\oplus$\,etteemält & \textsc{ sg abl } \\
\underline{nel̕l̕}\,$\oplus$\,etteemässi & \textsc{ sg tra } \\
\underline{nel̕l̕}\,$\oplus$\,etteemässaa & \textsc{ sg ter } \\
\underline{nel̕l̕}\,$\oplus$\,etteemäka & \textsc{ sg com } \\
\underline{nel̕l̕}\,$\oplus$\,etteemäd & \textsc{ pl nom } \\
\underline{nel̕l̕}\,$\oplus$\,ettemije & \textsc{ pl gen } \\
\underline{nel̕l̕}\,$\oplus$\,ettemiit & \textsc{ pl par } \\
\underline{nel̕l̕}\,$\oplus$\,ettemiise & \textsc{ pl ill } \\
\underline{nel̕l̕}\,$\oplus$\,ettemiiz & \textsc{ pl ine } \\
\underline{nel̕l̕}\,$\oplus$\,ettemiiss & \textsc{ pl ela } \\
\underline{nel̕l̕}\,$\oplus$\,ettemiille & \textsc{ pl all } \\
\underline{nel̕l̕}\,$\oplus$\,ettemiill & \textsc{ pl ade } \\
\underline{nel̕l̕}\,$\oplus$\,ettemiilt & \textsc{ pl abl } \\
\underline{nel̕l̕}\,$\oplus$\,ettemiissi & \textsc{ pl tra } \\
\underline{nel̕l̕}\,$\oplus$\,ettemiissaa & \textsc{ pl ter } \\
\underline{nel̕l̕}\,$\oplus$\,ettemijka & \textsc{ pl com } \\
\end{tabular}
\end{sideways}
\captionof{table}{Tüüpsõnamall \arabic{mallinumber}\,\vadja{nel̕l̕äz} ekstraheeritud muutvormimallid.}
\label{tab:tüüpsõnamall-nel̕l̕äz}

\end{minipage}

 
\vspace{1em}
\noindent Tüüpsõnamall \vadja{nel̕l̕äz} ei hõlma teisi lekseeme vormi\-sõnastikus.



\vspace{3.5em}
\noindent \begin{minipage}{\textwidth}
\stepcounter{mallinumber}
\noindent \textbf{Tüüpsõnamall \arabic{mallinumber}\,\vadja{seemen}}\\

\begin{sideways}
\begin{tabular}{l l}
muutvormimall & tunnused \\
\hline
\underline{seem}\,$\oplus$\,e\,$\oplus$\,\underline{n} & \textsc{ sg nom } \\
\underline{seem}\,$\oplus$\,\underline{n}\,$\oplus$\,e & \textsc{ sg gen } \\
\underline{seem}\,$\oplus$\,e\,$\oplus$\,\underline{n}\,$\oplus$\,t & \textsc{ sg par } \\
\underline{seem}\,$\oplus$\,\underline{n}\,$\oplus$\,ese & \textsc{ sg ill } \\
\underline{seem}\,$\oplus$\,\underline{n}\,$\oplus$\,ez & \textsc{ sg ine } \\
\underline{seem}\,$\oplus$\,\underline{n}\,$\oplus$\,ess & \textsc{ sg ela } \\
\underline{seem}\,$\oplus$\,\underline{n}\,$\oplus$\,elle & \textsc{ sg all } \\
\underline{seem}\,$\oplus$\,\underline{n}\,$\oplus$\,ell & \textsc{ sg ade } \\
\underline{seem}\,$\oplus$\,\underline{n}\,$\oplus$\,elt & \textsc{ sg abl } \\
\underline{seem}\,$\oplus$\,\underline{n}\,$\oplus$\,essi & \textsc{ sg tra } \\
\underline{seem}\,$\oplus$\,\underline{n}\,$\oplus$\,essaa & \textsc{ sg ter } \\
\underline{seem}\,$\oplus$\,\underline{n}\,$\oplus$\,eka & \textsc{ sg com } \\
\underline{seem}\,$\oplus$\,\underline{n}\,$\oplus$\,ed & \textsc{ pl nom } \\
\underline{seem}\,$\oplus$\,\underline{n}\,$\oplus$\,ije & \textsc{ pl gen } \\
\underline{seem}\,$\oplus$\,\underline{n}\,$\oplus$\,iit & \textsc{ pl par } \\
\underline{seem}\,$\oplus$\,\underline{n}\,$\oplus$\,iise & \textsc{ pl ill } \\
\underline{seem}\,$\oplus$\,\underline{n}\,$\oplus$\,iiz & \textsc{ pl ine } \\
\underline{seem}\,$\oplus$\,\underline{n}\,$\oplus$\,iiss & \textsc{ pl ela } \\
\underline{seem}\,$\oplus$\,\underline{n}\,$\oplus$\,iille & \textsc{ pl all } \\
\underline{seem}\,$\oplus$\,\underline{n}\,$\oplus$\,iill & \textsc{ pl ade } \\
\underline{seem}\,$\oplus$\,\underline{n}\,$\oplus$\,iilt & \textsc{ pl abl } \\
\underline{seem}\,$\oplus$\,\underline{n}\,$\oplus$\,iissi & \textsc{ pl tra } \\
\underline{seem}\,$\oplus$\,\underline{n}\,$\oplus$\,iissaa & \textsc{ pl ter } \\
\underline{seem}\,$\oplus$\,\underline{n}\,$\oplus$\,ijka & \textsc{ pl com } \\
\end{tabular}
\end{sideways}
\captionof{table}{Tüüpsõnamall \arabic{mallinumber}\,\vadja{seemen} ekstraheeritud muutvormimallid.}
\label{tab:tüüpsõnamall-seemen}

\end{minipage}

 
\vspace{1em}
\noindent Tüüpsõnamall \vadja{seemen} ei hõlma teisi lekseeme vormi\-sõnastikus.



\vspace{3.5em}
\noindent \begin{minipage}{\textwidth}
\stepcounter{mallinumber}
\noindent \textbf{Tüüpsõnamall \arabic{mallinumber}\,\vadja{süä}}\\

\begin{sideways}
\begin{tabular}{l l}
muutvormimall & tunnused \\
\hline
\underline{sü}\,$\oplus$\,ä & \textsc{ sg nom } \\
\underline{sü}\,$\oplus$\,ä & \textsc{ sg gen } \\
\underline{sü}\,$\oplus$\,ttä & \textsc{ sg par } \\
\underline{sü}\,$\oplus$\,ttäse & \textsc{ sg ill } \\
\underline{sü}\,$\oplus$\,äz & \textsc{ sg ine } \\
\underline{sü}\,$\oplus$\,äss & \textsc{ sg ela } \\
\underline{sü}\,$\oplus$\,älle & \textsc{ sg all } \\
\underline{sü}\,$\oplus$\,äll & \textsc{ sg ade } \\
\underline{sü}\,$\oplus$\,ält & \textsc{ sg abl } \\
\underline{sü}\,$\oplus$\,ässi & \textsc{ sg tra } \\
\underline{sü}\,$\oplus$\,ässaa & \textsc{ sg ter } \\
\underline{sü}\,$\oplus$\,äka & \textsc{ sg com } \\
\underline{sü}\,$\oplus$\,äd & \textsc{ pl nom } \\
\underline{sü}\,$\oplus$\,ttije & \textsc{ pl gen } \\
\underline{sü}\,$\oplus$\,ttiit & \textsc{ pl par } \\
\underline{sü}\,$\oplus$\,ttiise & \textsc{ pl ill } \\
\underline{sü}\,$\oplus$\,ttiiz & \textsc{ pl ine } \\
\underline{sü}\,$\oplus$\,ttiiss & \textsc{ pl ela } \\
\underline{sü}\,$\oplus$\,ttiille & \textsc{ pl all } \\
\underline{sü}\,$\oplus$\,ttiill & \textsc{ pl ade } \\
\underline{sü}\,$\oplus$\,ttiilt & \textsc{ pl abl } \\
\underline{sü}\,$\oplus$\,ttiissi & \textsc{ pl tra } \\
\underline{sü}\,$\oplus$\,ttiissaa & \textsc{ pl ter } \\
\underline{sü}\,$\oplus$\,ttijka & \textsc{ pl com } \\
\end{tabular}
\end{sideways}
\captionof{table}{Tüüpsõnamall \arabic{mallinumber}\,\vadja{süä} ekstraheeritud muutvormimallid.}
\label{tab:tüüpsõnamall-süä}

\end{minipage}

 
\vspace{1em}
\noindent Tüüpsõnamall \vadja{süä} ei hõlma teisi lekseeme vormi\-sõnastikus.



\vspace{3.5em}
\noindent \begin{minipage}{\textwidth}
\stepcounter{mallinumber}
\noindent \textbf{Tüüpsõnamall \arabic{mallinumber}\,\vadja{tütär}}\\

\begin{sideways}
\begin{tabular}{l l}
muutvormimall & tunnused \\
\hline
\underline{tüt}\,$\oplus$\,\underline{är} & \textsc{ sg nom } \\
\underline{tüt}\,$\oplus$\,t\,$\oplus$\,\underline{är}\,$\oplus$\,e & \textsc{ sg gen } \\
\underline{tüt}\,$\oplus$\,\underline{är}\,$\oplus$\,te & \textsc{ sg par } \\
\underline{tüt}\,$\oplus$\,t\,$\oplus$\,\underline{är}\,$\oplus$\,ese & \textsc{ sg ill } \\
\underline{tüt}\,$\oplus$\,t\,$\oplus$\,\underline{är}\,$\oplus$\,ez & \textsc{ sg ine } \\
\underline{tüt}\,$\oplus$\,t\,$\oplus$\,\underline{är}\,$\oplus$\,ess & \textsc{ sg ela } \\
\underline{tüt}\,$\oplus$\,t\,$\oplus$\,\underline{är}\,$\oplus$\,elle & \textsc{ sg all } \\
\underline{tüt}\,$\oplus$\,t\,$\oplus$\,\underline{är}\,$\oplus$\,ell & \textsc{ sg ade } \\
\underline{tüt}\,$\oplus$\,t\,$\oplus$\,\underline{är}\,$\oplus$\,elt & \textsc{ sg abl } \\
\underline{tüt}\,$\oplus$\,t\,$\oplus$\,\underline{är}\,$\oplus$\,essi & \textsc{ sg tra } \\
\underline{tüt}\,$\oplus$\,t\,$\oplus$\,\underline{är}\,$\oplus$\,essaa & \textsc{ sg ter } \\
\underline{tüt}\,$\oplus$\,t\,$\oplus$\,\underline{är}\,$\oplus$\,eka & \textsc{ sg com } \\
\underline{tüt}\,$\oplus$\,t\,$\oplus$\,\underline{är}\,$\oplus$\,ed & \textsc{ pl nom } \\
\underline{tüt}\,$\oplus$\,t\,$\oplus$\,\underline{är}\,$\oplus$\,ije & \textsc{ pl gen } \\
\underline{tüt}\,$\oplus$\,t\,$\oplus$\,\underline{är}\,$\oplus$\,iit & \textsc{ pl par } \\
\underline{tüt}\,$\oplus$\,t\,$\oplus$\,\underline{är}\,$\oplus$\,iise & \textsc{ pl ill } \\
\underline{tüt}\,$\oplus$\,t\,$\oplus$\,\underline{är}\,$\oplus$\,iiz & \textsc{ pl ine } \\
\underline{tüt}\,$\oplus$\,t\,$\oplus$\,\underline{är}\,$\oplus$\,iiss & \textsc{ pl ela } \\
\underline{tüt}\,$\oplus$\,t\,$\oplus$\,\underline{är}\,$\oplus$\,iille & \textsc{ pl all } \\
\underline{tüt}\,$\oplus$\,t\,$\oplus$\,\underline{är}\,$\oplus$\,iill & \textsc{ pl ade } \\
\underline{tüt}\,$\oplus$\,t\,$\oplus$\,\underline{är}\,$\oplus$\,iilt & \textsc{ pl abl } \\
\underline{tüt}\,$\oplus$\,t\,$\oplus$\,\underline{är}\,$\oplus$\,iissi & \textsc{ pl tra } \\
\underline{tüt}\,$\oplus$\,t\,$\oplus$\,\underline{är}\,$\oplus$\,iissaa & \textsc{ pl ter } \\
\underline{tüt}\,$\oplus$\,t\,$\oplus$\,\underline{är}\,$\oplus$\,ijka & \textsc{ pl com } \\
\end{tabular}
\end{sideways}
\captionof{table}{Tüüpsõnamall \arabic{mallinumber}\,\vadja{tütär} ekstraheeritud muutvormimallid.}
\label{tab:tüüpsõnamall-tütär}

\end{minipage}

 
\vspace{1em}
\noindent Tüüpsõnamall \vadja{tütär} ei hõlma teisi lekseeme vormi\-sõnastikus.



\vspace{3.5em}
\noindent \begin{minipage}{\textwidth}
\stepcounter{mallinumber}
\noindent \textbf{Tüüpsõnamall \arabic{mallinumber}\,\vadja{õnnõto}}\\

\begin{sideways}
\begin{tabular}{l l}
muutvormimall & tunnused \\
\hline
\underline{õnnõt}\,$\oplus$\,\underline{o} & \textsc{ sg nom } \\
\underline{õnnõt}\,$\oplus$\,t\,$\oplus$\,\underline{o}\,$\oplus$\,ma & \textsc{ sg gen } \\
\underline{õnnõt}\,$\oplus$\,\underline{o}\,$\oplus$\,ta & \textsc{ sg par } \\
\underline{õnnõt}\,$\oplus$\,t\,$\oplus$\,\underline{o}\,$\oplus$\,masõ & \textsc{ sg ill } \\
\underline{õnnõt}\,$\oplus$\,t\,$\oplus$\,\underline{o}\,$\oplus$\,maz & \textsc{ sg ine } \\
\underline{õnnõt}\,$\oplus$\,t\,$\oplus$\,\underline{o}\,$\oplus$\,mass & \textsc{ sg ela } \\
\underline{õnnõt}\,$\oplus$\,t\,$\oplus$\,\underline{o}\,$\oplus$\,mallõ & \textsc{ sg all } \\
\underline{õnnõt}\,$\oplus$\,t\,$\oplus$\,\underline{o}\,$\oplus$\,mall & \textsc{ sg ade } \\
\underline{õnnõt}\,$\oplus$\,t\,$\oplus$\,\underline{o}\,$\oplus$\,malt & \textsc{ sg abl } \\
\underline{õnnõt}\,$\oplus$\,t\,$\oplus$\,\underline{o}\,$\oplus$\,massi & \textsc{ sg tra } \\
\underline{õnnõt}\,$\oplus$\,t\,$\oplus$\,\underline{o}\,$\oplus$\,massaa & \textsc{ sg ter } \\
\underline{õnnõt}\,$\oplus$\,t\,$\oplus$\,\underline{o}\,$\oplus$\,maka & \textsc{ sg com } \\
\underline{õnnõt}\,$\oplus$\,t\,$\oplus$\,\underline{o}\,$\oplus$\,mad & \textsc{ pl nom } \\
\underline{õnnõt}\,$\oplus$\,t\,$\oplus$\,\underline{o}\,$\oplus$\,mijõ & \textsc{ pl gen } \\
\underline{õnnõt}\,$\oplus$\,t\,$\oplus$\,\underline{o}\,$\oplus$\,miit & \textsc{ pl par } \\
\underline{õnnõt}\,$\oplus$\,t\,$\oplus$\,\underline{o}\,$\oplus$\,miisõ & \textsc{ pl ill } \\
\underline{õnnõt}\,$\oplus$\,t\,$\oplus$\,\underline{o}\,$\oplus$\,miiz & \textsc{ pl ine } \\
\underline{õnnõt}\,$\oplus$\,t\,$\oplus$\,\underline{o}\,$\oplus$\,miiss & \textsc{ pl ela } \\
\underline{õnnõt}\,$\oplus$\,t\,$\oplus$\,\underline{o}\,$\oplus$\,miillõ & \textsc{ pl all } \\
\underline{õnnõt}\,$\oplus$\,t\,$\oplus$\,\underline{o}\,$\oplus$\,miill & \textsc{ pl ade } \\
\underline{õnnõt}\,$\oplus$\,t\,$\oplus$\,\underline{o}\,$\oplus$\,miilt & \textsc{ pl abl } \\
\underline{õnnõt}\,$\oplus$\,t\,$\oplus$\,\underline{o}\,$\oplus$\,miissi & \textsc{ pl tra } \\
\underline{õnnõt}\,$\oplus$\,t\,$\oplus$\,\underline{o}\,$\oplus$\,miissaa & \textsc{ pl ter } \\
\underline{õnnõt}\,$\oplus$\,t\,$\oplus$\,\underline{o}\,$\oplus$\,mijka & \textsc{ pl com } \\
\end{tabular}
\end{sideways}
\captionof{table}{Tüüpsõnamall \arabic{mallinumber}\,\vadja{õnnõto} ekstraheeritud muutvormimallid.}
\label{tab:tüüpsõnamall-õnnõto}

\end{minipage}

 
\vspace{1em}
\noindent Tüüpsõnamall \vadja{õnnõto} hõlmab vormisõnastikus 2 lekseemi: \vadja{\underline{õnnõt}\underline{o}} ja \vadja{\underline{hoolit}\underline{o}}.



\vspace{3.5em}
\noindent \begin{minipage}{\textwidth}
\stepcounter{mallinumber}
\noindent \textbf{Tüüpsõnamall \arabic{mallinumber}\,\vadja{kahõsa}}\\

\begin{sideways}
\begin{tabular}{l l}
muutvormimall & tunnused \\
\hline
\underline{kahõs}\,$\oplus$\,a & \textsc{ sg nom } \\
\underline{kahõs}\,$\oplus$\,sõmõ & \textsc{ sg gen } \\
\underline{kahõs}\,$\oplus$\,sõma & \textsc{ sg par } \\
\underline{kahõs}\,$\oplus$\,sõmasõ & \textsc{ sg ill } \\
\underline{kahõs}\,$\oplus$\,sõmaz & \textsc{ sg ine } \\
\underline{kahõs}\,$\oplus$\,sõmass & \textsc{ sg ela } \\
\underline{kahõs}\,$\oplus$\,sõmallõ & \textsc{ sg all } \\
\underline{kahõs}\,$\oplus$\,sõmall & \textsc{ sg ade } \\
\underline{kahõs}\,$\oplus$\,sõmalt & \textsc{ sg abl } \\
\underline{kahõs}\,$\oplus$\,sõmassi & \textsc{ sg tra } \\
\underline{kahõs}\,$\oplus$\,sõmassaa & \textsc{ sg ter } \\
\underline{kahõs}\,$\oplus$\,sõmaka & \textsc{ sg com } \\
\underline{kahõs}\,$\oplus$\,sõmad & \textsc{ pl nom } \\
\underline{kahõs}\,$\oplus$\,sõmijõ & \textsc{ pl gen } \\
\underline{kahõs}\,$\oplus$\,sõmiit & \textsc{ pl par } \\
\underline{kahõs}\,$\oplus$\,sõmiisõ & \textsc{ pl ill } \\
\underline{kahõs}\,$\oplus$\,sõmiiz & \textsc{ pl ine } \\
\underline{kahõs}\,$\oplus$\,sõmiiss & \textsc{ pl ela } \\
\underline{kahõs}\,$\oplus$\,sõmiillõ & \textsc{ pl all } \\
\underline{kahõs}\,$\oplus$\,sõmiill & \textsc{ pl ade } \\
\underline{kahõs}\,$\oplus$\,sõmiilt & \textsc{ pl abl } \\
\underline{kahõs}\,$\oplus$\,sõmiissi & \textsc{ pl tra } \\
\underline{kahõs}\,$\oplus$\,sõmiissaa & \textsc{ pl ter } \\
\underline{kahõs}\,$\oplus$\,sõmijka & \textsc{ pl com } \\
\end{tabular}
\end{sideways}
\captionof{table}{Tüüpsõnamall \arabic{mallinumber}\,\vadja{kahõsa} ekstraheeritud muutvormimallid.}
\label{tab:tüüpsõnamall-kahõsa}

\end{minipage}

 
\vspace{1em}
\noindent Tüüpsõnamall \vadja{kahõsa} ei hõlma teisi lekseeme vormi\-sõnastikus.



\vspace{3.5em}
\noindent \begin{minipage}{\textwidth}
\stepcounter{mallinumber}
\noindent \textbf{Tüüpsõnamall \arabic{mallinumber}\,\vadja{kuuvvaiz}}\\

\begin{sideways}
\begin{tabular}{l l}
muutvormimall & tunnused \\
\hline
\underline{kuuvv}\,$\oplus$\,aiz & \textsc{ sg nom } \\
\underline{kuuvv}\,$\oplus$\,õttõma & \textsc{ sg gen } \\
\underline{kuuvv}\,$\oplus$\,õt & \textsc{ sg par } \\
\underline{kuuvv}\,$\oplus$\,õttõmasõ & \textsc{ sg ill } \\
\underline{kuuvv}\,$\oplus$\,õttõmaz & \textsc{ sg ine } \\
\underline{kuuvv}\,$\oplus$\,õttõmass & \textsc{ sg ela } \\
\underline{kuuvv}\,$\oplus$\,õttõmallõ & \textsc{ sg all } \\
\underline{kuuvv}\,$\oplus$\,õttõmall & \textsc{ sg ade } \\
\underline{kuuvv}\,$\oplus$\,õttõmalt & \textsc{ sg abl } \\
\underline{kuuvv}\,$\oplus$\,õttõmassi & \textsc{ sg tra } \\
\underline{kuuvv}\,$\oplus$\,õttõmassaa & \textsc{ sg ter } \\
\underline{kuuvv}\,$\oplus$\,õttõmaka & \textsc{ sg com } \\
\underline{kuuvv}\,$\oplus$\,õttõmad & \textsc{ pl nom } \\
\underline{kuuvv}\,$\oplus$\,õttõmijõ & \textsc{ pl gen } \\
\underline{kuuvv}\,$\oplus$\,õttõmiit & \textsc{ pl par } \\
\underline{kuuvv}\,$\oplus$\,õttõmiisõ & \textsc{ pl ill } \\
\underline{kuuvv}\,$\oplus$\,õttõmiiz & \textsc{ pl ine } \\
\underline{kuuvv}\,$\oplus$\,õttõmiiss & \textsc{ pl ela } \\
\underline{kuuvv}\,$\oplus$\,õttõmiillõ & \textsc{ pl all } \\
\underline{kuuvv}\,$\oplus$\,õttõmiill & \textsc{ pl ade } \\
\underline{kuuvv}\,$\oplus$\,õttõmiilt & \textsc{ pl abl } \\
\underline{kuuvv}\,$\oplus$\,õttõmiissi & \textsc{ pl tra } \\
\underline{kuuvv}\,$\oplus$\,õttõmiissaa & \textsc{ pl ter } \\
\underline{kuuvv}\,$\oplus$\,õttõmijka & \textsc{ pl com } \\
\end{tabular}
\end{sideways}
\captionof{table}{Tüüpsõnamall \arabic{mallinumber}\,\vadja{kuuvvaiz} ekstraheeritud muutvormimallid.}
\label{tab:tüüpsõnamall-kuuvvaiz}

\end{minipage}

 
\vspace{1em}
\noindent Tüüpsõnamall \vadja{kuuvvaiz} ei hõlma teisi lekseeme vormi\-sõnastikus.



\vspace{3.5em}
\noindent \begin{minipage}{\textwidth}
\stepcounter{mallinumber}
\noindent \textbf{Tüüpsõnamall \arabic{mallinumber}\,\vadja{kõlmaz}}\\

\begin{sideways}
\begin{tabular}{l l}
muutvormimall & tunnused \\
\hline
\underline{kõlm}\,$\oplus$\,az & \textsc{ sg nom } \\
\underline{kõlm}\,$\oplus$\,õttõma & \textsc{ sg gen } \\
\underline{kõlm}\,$\oplus$\,aissõ & \textsc{ sg par } \\
\underline{kõlm}\,$\oplus$\,õttõmasõ & \textsc{ sg ill } \\
\underline{kõlm}\,$\oplus$\,õttõmaz & \textsc{ sg ine } \\
\underline{kõlm}\,$\oplus$\,õttõmass & \textsc{ sg ela } \\
\underline{kõlm}\,$\oplus$\,õttõmallõ & \textsc{ sg all } \\
\underline{kõlm}\,$\oplus$\,õttõmall & \textsc{ sg ade } \\
\underline{kõlm}\,$\oplus$\,õttõmalt & \textsc{ sg abl } \\
\underline{kõlm}\,$\oplus$\,õttõmassi & \textsc{ sg tra } \\
\underline{kõlm}\,$\oplus$\,õttõmassaa & \textsc{ sg ter } \\
\underline{kõlm}\,$\oplus$\,õttõmaka & \textsc{ sg com } \\
\underline{kõlm}\,$\oplus$\,õttõmad & \textsc{ pl nom } \\
\underline{kõlm}\,$\oplus$\,õttõmijõ & \textsc{ pl gen } \\
\underline{kõlm}\,$\oplus$\,õttõmiit & \textsc{ pl par } \\
\underline{kõlm}\,$\oplus$\,õttõmiisõ & \textsc{ pl ill } \\
\underline{kõlm}\,$\oplus$\,õttõmiiz & \textsc{ pl ine } \\
\underline{kõlm}\,$\oplus$\,õttõmiiss & \textsc{ pl ela } \\
\underline{kõlm}\,$\oplus$\,õttõmiillõ & \textsc{ pl all } \\
\underline{kõlm}\,$\oplus$\,õttõmiill & \textsc{ pl ade } \\
\underline{kõlm}\,$\oplus$\,õttõmiilt & \textsc{ pl abl } \\
\underline{kõlm}\,$\oplus$\,õttõmiissi & \textsc{ pl tra } \\
\underline{kõlm}\,$\oplus$\,õttõmiissaa & \textsc{ pl ter } \\
\underline{kõlm}\,$\oplus$\,õttõmijka & \textsc{ pl com } \\
\end{tabular}
\end{sideways}
\captionof{table}{Tüüpsõnamall \arabic{mallinumber}\,\vadja{kõlmaz} ekstraheeritud muutvormimallid.}
\label{tab:tüüpsõnamall-kõlmaz}

\end{minipage}

 
\vspace{1em}
\noindent Tüüpsõnamall \vadja{kõlmaz} ei hõlma teisi lekseeme vormi\-sõnastikus.

\spacing{1.5}


\subsection{\RN{14} käändkond}

Neljateistkümnenda käändkonna sõnad lõpevad Ariste (\citeyear[50]{ariste_grammar_1968}) järgi \vadja{-aZ/-äZ, -iZ} või \vadja{-e/-õ}. % \cite[50]{ariste_grammar_1968}.

Vadja keele \vadja{-Z}-lõpu sandhi nähtus on kõigi liikmete puhul ühtlustatud \vadja{-z}-lõpulisteks \msd{sg nom} käändes.

Juhul kui Tsvetkovi sõnaraamatus on antud paralleelvorme, on neist püütud valida selline, mis esineb kas Heinsoo ja Konkova sõnastikes, või VKSi Vaipooli murdelistes näite\-lausetes. Näiteks on valitud \vadja{lähe} tugeva\-astmeline \msd{sg} tüvi näite\-lausete põhjal.

%% Avatuid küsimusi-tähelepanekuid:
%% \begin{itemize}
%% \item \vadja{-Z}-lõpu sandhi nähtus on kõigi liikmete puhul ühtlustatud \vadja{-z} lõpulisteks
%% \item plurale tantum 'ivusõd' kustutatud sest 'ivuz' olemas
%% \item Tsvetkovi antud paralleelvariantidest on valitud vaid üks (korpuse, analoogsete sõnade ülekaalu ning Heinsoo ja Konkova põhjal)
%% \item valitud 'lähe' tugevaastmeline sg tüvi, sest VKSis esineb ühes Li näitelauses
%% \item -kõz-liides muudetud eespoolseks vastavate sõnade juures
%% \end{itemize}


\subsubsection*{Ekstraktmorfoloogia tüüpsõnad}
\spacing{1.1}
\vspace{-3.5em}


\vspace{3.5em}
\noindent \begin{minipage}{\textwidth}
\stepcounter{mallinumber}
\noindent \textbf{Tüüpsõnamall \arabic{mallinumber}\,\vadja{õgaz}}\\

\begin{sideways}
\begin{tabular}{l l}
muutvormimall & tunnused \\
\hline
\underline{õ}\,$\oplus$\,g\,$\oplus$\,\underline{a}\,$\oplus$\,z & \textsc{ sg nom } \\
\underline{õ}\,$\oplus$\,kk\,$\oplus$\,\underline{a} & \textsc{ sg gen } \\
\underline{õ}\,$\oplus$\,g\,$\oplus$\,\underline{a}\,$\oplus$\,ssõ & \textsc{ sg par } \\
\underline{õ}\,$\oplus$\,kk\,$\oplus$\,\underline{a}\,$\oplus$\,sõ & \textsc{ sg ill } \\
\underline{õ}\,$\oplus$\,kk\,$\oplus$\,\underline{a}\,$\oplus$\,z & \textsc{ sg ine } \\
\underline{õ}\,$\oplus$\,kk\,$\oplus$\,\underline{a}\,$\oplus$\,ss & \textsc{ sg ela } \\
\underline{õ}\,$\oplus$\,kk\,$\oplus$\,\underline{a}\,$\oplus$\,llõ & \textsc{ sg all } \\
\underline{õ}\,$\oplus$\,kk\,$\oplus$\,\underline{a}\,$\oplus$\,ll & \textsc{ sg ade } \\
\underline{õ}\,$\oplus$\,kk\,$\oplus$\,\underline{a}\,$\oplus$\,lt & \textsc{ sg abl } \\
\underline{õ}\,$\oplus$\,kk\,$\oplus$\,\underline{a}\,$\oplus$\,ssi & \textsc{ sg tra } \\
\underline{õ}\,$\oplus$\,kk\,$\oplus$\,\underline{a}\,$\oplus$\,ssaa & \textsc{ sg ter } \\
\underline{õ}\,$\oplus$\,kk\,$\oplus$\,\underline{a}\,$\oplus$\,ka & \textsc{ sg com } \\
\underline{õ}\,$\oplus$\,kk\,$\oplus$\,\underline{a}\,$\oplus$\,d & \textsc{ pl nom } \\
\underline{õ}\,$\oplus$\,kk\,$\oplus$\,\underline{a}\,$\oplus$\,jõ & \textsc{ pl gen } \\
\underline{õ}\,$\oplus$\,kk\,$\oplus$\,\underline{a}\,$\oplus$\,it & \textsc{ pl par } \\
\underline{õ}\,$\oplus$\,kk\,$\oplus$\,\underline{a}\,$\oplus$\,isõ & \textsc{ pl ill } \\
\underline{õ}\,$\oplus$\,kk\,$\oplus$\,\underline{a}\,$\oplus$\,iz & \textsc{ pl ine } \\
\underline{õ}\,$\oplus$\,kk\,$\oplus$\,\underline{a}\,$\oplus$\,iss & \textsc{ pl ela } \\
\underline{õ}\,$\oplus$\,kk\,$\oplus$\,\underline{a}\,$\oplus$\,illõ & \textsc{ pl all } \\
\underline{õ}\,$\oplus$\,kk\,$\oplus$\,\underline{a}\,$\oplus$\,ill & \textsc{ pl ade } \\
\underline{õ}\,$\oplus$\,kk\,$\oplus$\,\underline{a}\,$\oplus$\,ilt & \textsc{ pl abl } \\
\underline{õ}\,$\oplus$\,kk\,$\oplus$\,\underline{a}\,$\oplus$\,issi & \textsc{ pl tra } \\
\underline{õ}\,$\oplus$\,kk\,$\oplus$\,\underline{a}\,$\oplus$\,issaa & \textsc{ pl ter } \\
\underline{õ}\,$\oplus$\,kk\,$\oplus$\,\underline{a}\,$\oplus$\,ika & \textsc{ pl com } \\
\end{tabular}
\end{sideways}
\captionof{table}{Tüüpsõnamall \arabic{mallinumber}\,\vadja{õgaz} ekstraheeritud muutvormimallid.}
\label{tab:tüüpsõnamall-õgaz}

\end{minipage}

 
\vspace{1em}
\noindent Tüüpsõnamall \vadja{õgaz} ei hõlma teisi lekseeme vormi\-sõnastikus.



\vspace{3.5em}
\noindent \begin{minipage}{\textwidth}
\stepcounter{mallinumber}
\noindent \textbf{Tüüpsõnamall \arabic{mallinumber}\,\vadja{puhaz}}\\

\begin{sideways}
\begin{tabular}{l l}
muutvormimall & tunnused \\
\hline
\underline{puh}\,$\oplus$\,\underline{a}\,$\oplus$\,z & \textsc{ sg nom } \\
\underline{puh}\,$\oplus$\,t\,$\oplus$\,\underline{a} & \textsc{ sg gen } \\
\underline{puh}\,$\oplus$\,\underline{a}\,$\oplus$\,ssõ & \textsc{ sg par } \\
\underline{puh}\,$\oplus$\,t\,$\oplus$\,\underline{a}\,$\oplus$\,sõ & \textsc{ sg ill } \\
\underline{puh}\,$\oplus$\,t\,$\oplus$\,\underline{a}\,$\oplus$\,z & \textsc{ sg ine } \\
\underline{puh}\,$\oplus$\,t\,$\oplus$\,\underline{a}\,$\oplus$\,ss & \textsc{ sg ela } \\
\underline{puh}\,$\oplus$\,t\,$\oplus$\,\underline{a}\,$\oplus$\,llõ & \textsc{ sg all } \\
\underline{puh}\,$\oplus$\,t\,$\oplus$\,\underline{a}\,$\oplus$\,ll & \textsc{ sg ade } \\
\underline{puh}\,$\oplus$\,t\,$\oplus$\,\underline{a}\,$\oplus$\,lt & \textsc{ sg abl } \\
\underline{puh}\,$\oplus$\,t\,$\oplus$\,\underline{a}\,$\oplus$\,ssi & \textsc{ sg tra } \\
\underline{puh}\,$\oplus$\,t\,$\oplus$\,\underline{a}\,$\oplus$\,ssaa & \textsc{ sg ter } \\
\underline{puh}\,$\oplus$\,t\,$\oplus$\,\underline{a}\,$\oplus$\,ka & \textsc{ sg com } \\
\underline{puh}\,$\oplus$\,t\,$\oplus$\,\underline{a}\,$\oplus$\,d & \textsc{ pl nom } \\
\underline{puh}\,$\oplus$\,t\,$\oplus$\,\underline{a}\,$\oplus$\,jõ & \textsc{ pl gen } \\
\underline{puh}\,$\oplus$\,t\,$\oplus$\,\underline{a}\,$\oplus$\,it & \textsc{ pl par } \\
\underline{puh}\,$\oplus$\,t\,$\oplus$\,\underline{a}\,$\oplus$\,isõ & \textsc{ pl ill } \\
\underline{puh}\,$\oplus$\,t\,$\oplus$\,\underline{a}\,$\oplus$\,iz & \textsc{ pl ine } \\
\underline{puh}\,$\oplus$\,t\,$\oplus$\,\underline{a}\,$\oplus$\,iss & \textsc{ pl ela } \\
\underline{puh}\,$\oplus$\,t\,$\oplus$\,\underline{a}\,$\oplus$\,illõ & \textsc{ pl all } \\
\underline{puh}\,$\oplus$\,t\,$\oplus$\,\underline{a}\,$\oplus$\,ill & \textsc{ pl ade } \\
\underline{puh}\,$\oplus$\,t\,$\oplus$\,\underline{a}\,$\oplus$\,ilt & \textsc{ pl abl } \\
\underline{puh}\,$\oplus$\,t\,$\oplus$\,\underline{a}\,$\oplus$\,issi & \textsc{ pl tra } \\
\underline{puh}\,$\oplus$\,t\,$\oplus$\,\underline{a}\,$\oplus$\,issaa & \textsc{ pl ter } \\
\underline{puh}\,$\oplus$\,t\,$\oplus$\,\underline{a}\,$\oplus$\,ika & \textsc{ pl com } \\
\end{tabular}
\end{sideways}
\captionof{table}{Tüüpsõnamall \arabic{mallinumber}\,\vadja{puhaz} ekstraheeritud muutvormimallid.}
\label{tab:tüüpsõnamall-puhaz}

\end{minipage}

 
\vspace{1em}
\noindent Tüüpsõnamall \vadja{puhaz} hõlmab vormisõnastikus 2 lekseemi: \vadja{\underline{puh}\underline{a}z} ja \vadja{\underline{ah}\underline{a}z}.



\vspace{3.5em}
\noindent \begin{minipage}{\textwidth}
\stepcounter{mallinumber}
\noindent \textbf{Tüüpsõnamall \arabic{mallinumber}\,\vadja{lähe}}\\

\begin{sideways}
\begin{tabular}{l l}
muutvormimall & tunnused \\
\hline
\underline{läh}\,$\oplus$\,\underline{e} & \textsc{ sg nom } \\
\underline{läh}\,$\oplus$\,t\,$\oplus$\,\underline{e} & \textsc{ sg gen } \\
\underline{läh}\,$\oplus$\,\underline{e}\,$\oplus$\,tt & \textsc{ sg par } \\
\underline{läh}\,$\oplus$\,t\,$\oplus$\,\underline{e}\,$\oplus$\,se & \textsc{ sg ill } \\
\underline{läh}\,$\oplus$\,t\,$\oplus$\,\underline{e}\,$\oplus$\,z & \textsc{ sg ine } \\
\underline{läh}\,$\oplus$\,t\,$\oplus$\,\underline{e}\,$\oplus$\,ss & \textsc{ sg ela } \\
\underline{läh}\,$\oplus$\,t\,$\oplus$\,\underline{e}\,$\oplus$\,lle & \textsc{ sg all } \\
\underline{läh}\,$\oplus$\,t\,$\oplus$\,\underline{e}\,$\oplus$\,ll & \textsc{ sg ade } \\
\underline{läh}\,$\oplus$\,t\,$\oplus$\,\underline{e}\,$\oplus$\,lt & \textsc{ sg abl } \\
\underline{läh}\,$\oplus$\,t\,$\oplus$\,\underline{e}\,$\oplus$\,ssi & \textsc{ sg tra } \\
\underline{läh}\,$\oplus$\,t\,$\oplus$\,\underline{e}\,$\oplus$\,ssaa & \textsc{ sg ter } \\
\underline{läh}\,$\oplus$\,t\,$\oplus$\,\underline{e}\,$\oplus$\,ka & \textsc{ sg com } \\
\underline{läh}\,$\oplus$\,t\,$\oplus$\,\underline{e}\,$\oplus$\,d & \textsc{ pl nom } \\
\underline{läh}\,$\oplus$\,t\,$\oplus$\,\underline{e}\,$\oplus$\,je & \textsc{ pl gen } \\
\underline{läh}\,$\oplus$\,t\,$\oplus$\,\underline{e}\,$\oplus$\,it & \textsc{ pl par } \\
\underline{läh}\,$\oplus$\,t\,$\oplus$\,\underline{e}\,$\oplus$\,ise & \textsc{ pl ill } \\
\underline{läh}\,$\oplus$\,t\,$\oplus$\,\underline{e}\,$\oplus$\,iz & \textsc{ pl ine } \\
\underline{läh}\,$\oplus$\,t\,$\oplus$\,\underline{e}\,$\oplus$\,iss & \textsc{ pl ela } \\
\underline{läh}\,$\oplus$\,t\,$\oplus$\,\underline{e}\,$\oplus$\,ille & \textsc{ pl all } \\
\underline{läh}\,$\oplus$\,t\,$\oplus$\,\underline{e}\,$\oplus$\,ill & \textsc{ pl ade } \\
\underline{läh}\,$\oplus$\,t\,$\oplus$\,\underline{e}\,$\oplus$\,ilt & \textsc{ pl abl } \\
\underline{läh}\,$\oplus$\,t\,$\oplus$\,\underline{e}\,$\oplus$\,issi & \textsc{ pl tra } \\
\underline{läh}\,$\oplus$\,t\,$\oplus$\,\underline{e}\,$\oplus$\,issaa & \textsc{ pl ter } \\
\underline{läh}\,$\oplus$\,t\,$\oplus$\,\underline{e}\,$\oplus$\,ika & \textsc{ pl com } \\
\end{tabular}
\end{sideways}
\captionof{table}{Tüüpsõnamall \arabic{mallinumber}\,\vadja{lähe} ekstraheeritud muutvormimallid.}
\label{tab:tüüpsõnamall-lähe}

\end{minipage}

 
\vspace{1em}
\noindent Tüüpsõnamall \vadja{lähe} ei hõlma teisi lekseeme vormi\-sõnastikus.



\vspace{3.5em}
\noindent \begin{minipage}{\textwidth}
\stepcounter{mallinumber}
\noindent \textbf{Tüüpsõnamall \arabic{mallinumber}\,\vadja{rüiz}}\\

\begin{sideways}
\begin{tabular}{l l}
muutvormimall & tunnused \\
\hline
\underline{rü}\,$\oplus$\,iz & \textsc{ sg nom } \\
\underline{rü}\,$\oplus$\,čče & \textsc{ sg gen } \\
\underline{rü}\,$\oplus$\,isse & \textsc{ sg par } \\
\underline{rü}\,$\oplus$\,ččese & \textsc{ sg ill } \\
\underline{rü}\,$\oplus$\,ččez & \textsc{ sg ine } \\
\underline{rü}\,$\oplus$\,ččess & \textsc{ sg ela } \\
\underline{rü}\,$\oplus$\,ččelle & \textsc{ sg all } \\
\underline{rü}\,$\oplus$\,ččell & \textsc{ sg ade } \\
\underline{rü}\,$\oplus$\,ččelt & \textsc{ sg abl } \\
\underline{rü}\,$\oplus$\,ččessi & \textsc{ sg tra } \\
\underline{rü}\,$\oplus$\,ččessaa & \textsc{ sg ter } \\
\underline{rü}\,$\oplus$\,ččeka & \textsc{ sg com } \\
\underline{rü}\,$\oplus$\,ččed & \textsc{ pl nom } \\
\underline{rü}\,$\oplus$\,ččije & \textsc{ pl gen } \\
\underline{rü}\,$\oplus$\,ččiit & \textsc{ pl par } \\
\underline{rü}\,$\oplus$\,ččiise & \textsc{ pl ill } \\
\underline{rü}\,$\oplus$\,ččiiz & \textsc{ pl ine } \\
\underline{rü}\,$\oplus$\,ččiiss & \textsc{ pl ela } \\
\underline{rü}\,$\oplus$\,ččiille & \textsc{ pl all } \\
\underline{rü}\,$\oplus$\,ččiill & \textsc{ pl ade } \\
\underline{rü}\,$\oplus$\,ččiilt & \textsc{ pl abl } \\
\underline{rü}\,$\oplus$\,ččiissi & \textsc{ pl tra } \\
\underline{rü}\,$\oplus$\,ččiissaa & \textsc{ pl ter } \\
\underline{rü}\,$\oplus$\,ččijka & \textsc{ pl com } \\
\end{tabular}
\end{sideways}
\captionof{table}{Tüüpsõnamall \arabic{mallinumber}\,\vadja{rüiz} ekstraheeritud muutvormimallid.}
\label{tab:tüüpsõnamall-rüiz}

\end{minipage}

 
\vspace{1em}
\noindent Tüüpsõnamall \vadja{rüiz} ei hõlma teisi lekseeme vormi\-sõnastikus.



\vspace{3.5em}
\noindent \begin{minipage}{\textwidth}
\stepcounter{mallinumber}
\noindent \textbf{Tüüpsõnamall \arabic{mallinumber}\,\vadja{rakõ}}\\

\begin{sideways}
\begin{tabular}{l l}
muutvormimall & tunnused \\
\hline
\underline{rak}\,$\oplus$\,\underline{õ} & \textsc{ sg nom } \\
\underline{rak}\,$\oplus$\,k\,$\oplus$\,\underline{õ} & \textsc{ sg gen } \\
\underline{rak}\,$\oplus$\,\underline{õ}\,$\oplus$\,ttõ & \textsc{ sg par } \\
\underline{rak}\,$\oplus$\,k\,$\oplus$\,\underline{õ}\,$\oplus$\,sõ & \textsc{ sg ill } \\
\underline{rak}\,$\oplus$\,k\,$\oplus$\,\underline{õ}\,$\oplus$\,z & \textsc{ sg ine } \\
\underline{rak}\,$\oplus$\,k\,$\oplus$\,\underline{õ}\,$\oplus$\,ss & \textsc{ sg ela } \\
\underline{rak}\,$\oplus$\,k\,$\oplus$\,\underline{õ}\,$\oplus$\,llõ & \textsc{ sg all } \\
\underline{rak}\,$\oplus$\,k\,$\oplus$\,\underline{õ}\,$\oplus$\,ll & \textsc{ sg ade } \\
\underline{rak}\,$\oplus$\,k\,$\oplus$\,\underline{õ}\,$\oplus$\,lt & \textsc{ sg abl } \\
\underline{rak}\,$\oplus$\,k\,$\oplus$\,\underline{õ}\,$\oplus$\,ssi & \textsc{ sg tra } \\
\underline{rak}\,$\oplus$\,k\,$\oplus$\,\underline{õ}\,$\oplus$\,ssaa & \textsc{ sg ter } \\
\underline{rak}\,$\oplus$\,k\,$\oplus$\,\underline{õ}\,$\oplus$\,ka & \textsc{ sg com } \\
\underline{rak}\,$\oplus$\,k\,$\oplus$\,\underline{õ}\,$\oplus$\,d & \textsc{ pl nom } \\
\underline{rak}\,$\oplus$\,k\,$\oplus$\,\underline{õ}\,$\oplus$\,jõ & \textsc{ pl gen } \\
\underline{rak}\,$\oplus$\,k\,$\oplus$\,\underline{õ}\,$\oplus$\,it & \textsc{ pl par } \\
\underline{rak}\,$\oplus$\,k\,$\oplus$\,\underline{õ}\,$\oplus$\,isõ & \textsc{ pl ill } \\
\underline{rak}\,$\oplus$\,k\,$\oplus$\,\underline{õ}\,$\oplus$\,iz & \textsc{ pl ine } \\
\underline{rak}\,$\oplus$\,k\,$\oplus$\,\underline{õ}\,$\oplus$\,iss & \textsc{ pl ela } \\
\underline{rak}\,$\oplus$\,k\,$\oplus$\,\underline{õ}\,$\oplus$\,illõ & \textsc{ pl all } \\
\underline{rak}\,$\oplus$\,k\,$\oplus$\,\underline{õ}\,$\oplus$\,ill & \textsc{ pl ade } \\
\underline{rak}\,$\oplus$\,k\,$\oplus$\,\underline{õ}\,$\oplus$\,ilt & \textsc{ pl abl } \\
\underline{rak}\,$\oplus$\,k\,$\oplus$\,\underline{õ}\,$\oplus$\,issi & \textsc{ pl tra } \\
\underline{rak}\,$\oplus$\,k\,$\oplus$\,\underline{õ}\,$\oplus$\,issaa & \textsc{ pl ter } \\
\underline{rak}\,$\oplus$\,k\,$\oplus$\,\underline{õ}\,$\oplus$\,ika & \textsc{ pl com } \\
\end{tabular}
\end{sideways}
\captionof{table}{Tüüpsõnamall \arabic{mallinumber}\,\vadja{rakõ} ekstraheeritud muutvormimallid.}
\label{tab:tüüpsõnamall-rakõ}

\end{minipage}

 
\vspace{1em}
\noindent Tüüpsõnamall \vadja{rakõ} ei hõlma teisi lekseeme vormi\-sõnastikus.



\vspace{3.5em}
\noindent \begin{minipage}{\textwidth}
\stepcounter{mallinumber}
\noindent \textbf{Tüüpsõnamall \arabic{mallinumber}\,\vadja{rikaz}}\\

\begin{sideways}
\begin{tabular}{l l}
muutvormimall & tunnused \\
\hline
\underline{rik}\,$\oplus$\,\underline{a}\,$\oplus$\,z & \textsc{ sg nom } \\
\underline{rik}\,$\oplus$\,k\,$\oplus$\,\underline{a} & \textsc{ sg gen } \\
\underline{rik}\,$\oplus$\,\underline{a}\,$\oplus$\,ssõ & \textsc{ sg par } \\
\underline{rik}\,$\oplus$\,k\,$\oplus$\,\underline{a}\,$\oplus$\,sõ & \textsc{ sg ill } \\
\underline{rik}\,$\oplus$\,k\,$\oplus$\,\underline{a}\,$\oplus$\,z & \textsc{ sg ine } \\
\underline{rik}\,$\oplus$\,k\,$\oplus$\,\underline{a}\,$\oplus$\,ss & \textsc{ sg ela } \\
\underline{rik}\,$\oplus$\,k\,$\oplus$\,\underline{a}\,$\oplus$\,llõ & \textsc{ sg all } \\
\underline{rik}\,$\oplus$\,k\,$\oplus$\,\underline{a}\,$\oplus$\,ll & \textsc{ sg ade } \\
\underline{rik}\,$\oplus$\,k\,$\oplus$\,\underline{a}\,$\oplus$\,lt & \textsc{ sg abl } \\
\underline{rik}\,$\oplus$\,k\,$\oplus$\,\underline{a}\,$\oplus$\,ssi & \textsc{ sg tra } \\
\underline{rik}\,$\oplus$\,k\,$\oplus$\,\underline{a}\,$\oplus$\,ssaa & \textsc{ sg ter } \\
\underline{rik}\,$\oplus$\,k\,$\oplus$\,\underline{a}\,$\oplus$\,ka & \textsc{ sg com } \\
\underline{rik}\,$\oplus$\,k\,$\oplus$\,\underline{a}\,$\oplus$\,d & \textsc{ pl nom } \\
\underline{rik}\,$\oplus$\,k\,$\oplus$\,\underline{a}\,$\oplus$\,jõ & \textsc{ pl gen } \\
\underline{rik}\,$\oplus$\,k\,$\oplus$\,\underline{a}\,$\oplus$\,it & \textsc{ pl par } \\
\underline{rik}\,$\oplus$\,k\,$\oplus$\,\underline{a}\,$\oplus$\,isõ & \textsc{ pl ill } \\
\underline{rik}\,$\oplus$\,k\,$\oplus$\,\underline{a}\,$\oplus$\,iz & \textsc{ pl ine } \\
\underline{rik}\,$\oplus$\,k\,$\oplus$\,\underline{a}\,$\oplus$\,iss & \textsc{ pl ela } \\
\underline{rik}\,$\oplus$\,k\,$\oplus$\,\underline{a}\,$\oplus$\,illõ & \textsc{ pl all } \\
\underline{rik}\,$\oplus$\,k\,$\oplus$\,\underline{a}\,$\oplus$\,ill & \textsc{ pl ade } \\
\underline{rik}\,$\oplus$\,k\,$\oplus$\,\underline{a}\,$\oplus$\,ilt & \textsc{ pl abl } \\
\underline{rik}\,$\oplus$\,k\,$\oplus$\,\underline{a}\,$\oplus$\,issi & \textsc{ pl tra } \\
\underline{rik}\,$\oplus$\,k\,$\oplus$\,\underline{a}\,$\oplus$\,issaa & \textsc{ pl ter } \\
\underline{rik}\,$\oplus$\,k\,$\oplus$\,\underline{a}\,$\oplus$\,ika & \textsc{ pl com } \\
\end{tabular}
\end{sideways}
\captionof{table}{Tüüpsõnamall \arabic{mallinumber}\,\vadja{rikaz} ekstraheeritud muutvormimallid.}
\label{tab:tüüpsõnamall-rikaz}

\end{minipage}

 
\vspace{1em}
\noindent Tüüpsõnamall \vadja{rikaz} ei hõlma teisi lekseeme vormi\-sõnastikus.



\vspace{3.5em}
\noindent \begin{minipage}{\textwidth}
\stepcounter{mallinumber}
\noindent \textbf{Tüüpsõnamall \arabic{mallinumber}\,\vadja{bul̕bukõz}}\\

\begin{sideways}
\begin{tabular}{l l}
muutvormimall & tunnused \\
\hline
\underline{bul̕buk}\,$\oplus$\,õz & \textsc{ sg nom } \\
\underline{bul̕buk}\,$\oplus$\,ka & \textsc{ sg gen } \\
\underline{bul̕buk}\,$\oplus$\,assõ & \textsc{ sg par } \\
\underline{bul̕buk}\,$\oplus$\,kasõ & \textsc{ sg ill } \\
\underline{bul̕buk}\,$\oplus$\,kaz & \textsc{ sg ine } \\
\underline{bul̕buk}\,$\oplus$\,kass & \textsc{ sg ela } \\
\underline{bul̕buk}\,$\oplus$\,kallõ & \textsc{ sg all } \\
\underline{bul̕buk}\,$\oplus$\,kall & \textsc{ sg ade } \\
\underline{bul̕buk}\,$\oplus$\,kalt & \textsc{ sg abl } \\
\underline{bul̕buk}\,$\oplus$\,kassi & \textsc{ sg tra } \\
\underline{bul̕buk}\,$\oplus$\,kassaa & \textsc{ sg ter } \\
\underline{bul̕buk}\,$\oplus$\,kaka & \textsc{ sg com } \\
\underline{bul̕buk}\,$\oplus$\,kad & \textsc{ pl nom } \\
\underline{bul̕buk}\,$\oplus$\,kajõ & \textsc{ pl gen } \\
\underline{bul̕buk}\,$\oplus$\,kait & \textsc{ pl par } \\
\underline{bul̕buk}\,$\oplus$\,kaisõ & \textsc{ pl ill } \\
\underline{bul̕buk}\,$\oplus$\,kaiz & \textsc{ pl ine } \\
\underline{bul̕buk}\,$\oplus$\,kaiss & \textsc{ pl ela } \\
\underline{bul̕buk}\,$\oplus$\,kaillõ & \textsc{ pl all } \\
\underline{bul̕buk}\,$\oplus$\,kaill & \textsc{ pl ade } \\
\underline{bul̕buk}\,$\oplus$\,kailt & \textsc{ pl abl } \\
\underline{bul̕buk}\,$\oplus$\,kaissi & \textsc{ pl tra } \\
\underline{bul̕buk}\,$\oplus$\,kaissaa & \textsc{ pl ter } \\
\underline{bul̕buk}\,$\oplus$\,kaika & \textsc{ pl com } \\
\end{tabular}
\end{sideways}
\captionof{table}{Tüüpsõnamall \arabic{mallinumber}\,\vadja{bul̕bukõz} ekstraheeritud muutvormimallid.}
\label{tab:tüüpsõnamall-bul̕bukõz}

\end{minipage}

 
\vspace{1em}
\noindent Tüüpsõnamall \vadja{bul̕bukõz} hõlmab vormisõnastikus 8 lekseemi: \vadja{\underline{bul̕buk}õz, \underline{čirk}õz, \underline{liivõk}õz, \underline{mansik}õz, \underline{musik}õz, \underline{nenäk}õz, \underline{õnnõk}õz} ja \vadja{\underline{baabuk}õz}.



\vspace{3.5em}
\noindent \begin{minipage}{\textwidth}
\stepcounter{mallinumber}
\noindent \textbf{Tüüpsõnamall \arabic{mallinumber}\,\vadja{kalliz}}\\

\begin{sideways}
\begin{tabular}{l l}
muutvormimall & tunnused \\
\hline
\underline{kall}\,$\oplus$\,iz & \textsc{ sg nom } \\
\underline{kall}\,$\oplus$\,i & \textsc{ sg gen } \\
\underline{kall}\,$\oplus$\,issõ & \textsc{ sg par } \\
\underline{kall}\,$\oplus$\,isõ & \textsc{ sg ill } \\
\underline{kall}\,$\oplus$\,iz & \textsc{ sg ine } \\
\underline{kall}\,$\oplus$\,iss & \textsc{ sg ela } \\
\underline{kall}\,$\oplus$\,illõ & \textsc{ sg all } \\
\underline{kall}\,$\oplus$\,ill & \textsc{ sg ade } \\
\underline{kall}\,$\oplus$\,ilt & \textsc{ sg abl } \\
\underline{kall}\,$\oplus$\,issi & \textsc{ sg tra } \\
\underline{kall}\,$\oplus$\,issaa & \textsc{ sg ter } \\
\underline{kall}\,$\oplus$\,ika & \textsc{ sg com } \\
\underline{kall}\,$\oplus$\,id & \textsc{ pl nom } \\
\underline{kall}\,$\oplus$\,ejõ & \textsc{ pl gen } \\
\underline{kall}\,$\oplus$\,eit & \textsc{ pl par } \\
\underline{kall}\,$\oplus$\,eisõ & \textsc{ pl ill } \\
\underline{kall}\,$\oplus$\,eiz & \textsc{ pl ine } \\
\underline{kall}\,$\oplus$\,eiss & \textsc{ pl ela } \\
\underline{kall}\,$\oplus$\,eillõ & \textsc{ pl all } \\
\underline{kall}\,$\oplus$\,eill & \textsc{ pl ade } \\
\underline{kall}\,$\oplus$\,eilt & \textsc{ pl abl } \\
\underline{kall}\,$\oplus$\,eissi & \textsc{ pl tra } \\
\underline{kall}\,$\oplus$\,eissaa & \textsc{ pl ter } \\
\underline{kall}\,$\oplus$\,eika & \textsc{ pl com } \\
\end{tabular}
\end{sideways}
\captionof{table}{Tüüpsõnamall \arabic{mallinumber}\,\vadja{kalliz} ekstraheeritud muutvormimallid.}
\label{tab:tüüpsõnamall-kalliz}

\end{minipage}

 
\vspace{1em}
\noindent Tüüpsõnamall \vadja{kalliz} ei hõlma teisi lekseeme vormi\-sõnastikus.



\vspace{3.5em}
\noindent \begin{minipage}{\textwidth}
\stepcounter{mallinumber}
\noindent \textbf{Tüüpsõnamall \arabic{mallinumber}\,\vadja{pal̕l̕õz}}\\

\begin{sideways}
\begin{tabular}{l l}
muutvormimall & tunnused \\
\hline
\underline{pal̕l̕}\,$\oplus$\,õz & \textsc{ sg nom } \\
\underline{pal̕l̕}\,$\oplus$\,a & \textsc{ sg gen } \\
\underline{pal̕l̕}\,$\oplus$\,assõ & \textsc{ sg par } \\
\underline{pal̕l̕}\,$\oplus$\,asõ & \textsc{ sg ill } \\
\underline{pal̕l̕}\,$\oplus$\,az & \textsc{ sg ine } \\
\underline{pal̕l̕}\,$\oplus$\,ass & \textsc{ sg ela } \\
\underline{pal̕l̕}\,$\oplus$\,allõ & \textsc{ sg all } \\
\underline{pal̕l̕}\,$\oplus$\,all & \textsc{ sg ade } \\
\underline{pal̕l̕}\,$\oplus$\,alt & \textsc{ sg abl } \\
\underline{pal̕l̕}\,$\oplus$\,assi & \textsc{ sg tra } \\
\underline{pal̕l̕}\,$\oplus$\,assaa & \textsc{ sg ter } \\
\underline{pal̕l̕}\,$\oplus$\,aka & \textsc{ sg com } \\
\underline{pal̕l̕}\,$\oplus$\,ad & \textsc{ pl nom } \\
\underline{pal̕l̕}\,$\oplus$\,ajõ & \textsc{ pl gen } \\
\underline{pal̕l̕}\,$\oplus$\,ait & \textsc{ pl par } \\
\underline{pal̕l̕}\,$\oplus$\,aisõ & \textsc{ pl ill } \\
\underline{pal̕l̕}\,$\oplus$\,aiz & \textsc{ pl ine } \\
\underline{pal̕l̕}\,$\oplus$\,aiss & \textsc{ pl ela } \\
\underline{pal̕l̕}\,$\oplus$\,aillõ & \textsc{ pl all } \\
\underline{pal̕l̕}\,$\oplus$\,aill & \textsc{ pl ade } \\
\underline{pal̕l̕}\,$\oplus$\,ailt & \textsc{ pl abl } \\
\underline{pal̕l̕}\,$\oplus$\,aissi & \textsc{ pl tra } \\
\underline{pal̕l̕}\,$\oplus$\,aissaa & \textsc{ pl ter } \\
\underline{pal̕l̕}\,$\oplus$\,aika & \textsc{ pl com } \\
\end{tabular}
\end{sideways}
\captionof{table}{Tüüpsõnamall \arabic{mallinumber}\,\vadja{pal̕l̕õz} ekstraheeritud muutvormimallid.}
\label{tab:tüüpsõnamall-pal̕l̕õz}

\end{minipage}

 
\vspace{1em}
\noindent Tüüpsõnamall \vadja{pal̕l̕õz} hõlmab vormisõnastikus 5 lekseemi: \vadja{\underline{pal̕l̕}õz, \underline{rahv}õz, \underline{taiv}õz, \underline{võõr}õz} ja \vadja{\underline{ahn}õz}.



\vspace{3.5em}
\noindent \begin{minipage}{\textwidth}
\stepcounter{mallinumber}
\noindent \textbf{Tüüpsõnamall \arabic{mallinumber}\,\vadja{poolõz}}\\

\begin{sideways}
\begin{tabular}{l l}
muutvormimall & tunnused \\
\hline
\underline{pool}\,$\oplus$\,õz & \textsc{ sg nom } \\
\underline{pool}\,$\oplus$\,a & \textsc{ sg gen } \\
\underline{pool}\,$\oplus$\,assõ & \textsc{ sg par } \\
\underline{pool}\,$\oplus$\,asõ & \textsc{ sg ill } \\
\underline{pool}\,$\oplus$\,az & \textsc{ sg ine } \\
\underline{pool}\,$\oplus$\,ass & \textsc{ sg ela } \\
\underline{pool}\,$\oplus$\,allõ & \textsc{ sg all } \\
\underline{pool}\,$\oplus$\,all & \textsc{ sg ade } \\
\underline{pool}\,$\oplus$\,alt & \textsc{ sg abl } \\
\underline{pool}\,$\oplus$\,assi & \textsc{ sg tra } \\
\underline{pool}\,$\oplus$\,assaa & \textsc{ sg ter } \\
\underline{pool}\,$\oplus$\,aka & \textsc{ sg com } \\
\underline{pool}\,$\oplus$\,ad & \textsc{ pl nom } \\
\underline{pool}\,$\oplus$\,ajõ & \textsc{ pl gen } \\
\underline{pool}\,$\oplus$\,oit & \textsc{ pl par } \\
\underline{pool}\,$\oplus$\,oisõ & \textsc{ pl ill } \\
\underline{pool}\,$\oplus$\,oiz & \textsc{ pl ine } \\
\underline{pool}\,$\oplus$\,oiss & \textsc{ pl ela } \\
\underline{pool}\,$\oplus$\,oillõ & \textsc{ pl all } \\
\underline{pool}\,$\oplus$\,oill & \textsc{ pl ade } \\
\underline{pool}\,$\oplus$\,oilt & \textsc{ pl abl } \\
\underline{pool}\,$\oplus$\,oissi & \textsc{ pl tra } \\
\underline{pool}\,$\oplus$\,oissaa & \textsc{ pl ter } \\
\underline{pool}\,$\oplus$\,oika & \textsc{ pl com } \\
\end{tabular}
\end{sideways}
\captionof{table}{Tüüpsõnamall \arabic{mallinumber}\,\vadja{poolõz} ekstraheeritud muutvormimallid.}
\label{tab:tüüpsõnamall-poolõz}

\end{minipage}

 
\vspace{1em}
\noindent Tüüpsõnamall \vadja{poolõz} ei hõlma teisi lekseeme vormi\-sõnastikus.



\vspace{3.5em}
\noindent \begin{minipage}{\textwidth}
\stepcounter{mallinumber}
\noindent \textbf{Tüüpsõnamall \arabic{mallinumber}\,\vadja{hammõz}}\\

\begin{sideways}
\begin{tabular}{l l}
muutvormimall & tunnused \\
\hline
\underline{ham}\,$\oplus$\,mõz & \textsc{ sg nom } \\
\underline{ham}\,$\oplus$\,pa & \textsc{ sg gen } \\
\underline{ham}\,$\oplus$\,massõ & \textsc{ sg par } \\
\underline{ham}\,$\oplus$\,pasõ & \textsc{ sg ill } \\
\underline{ham}\,$\oplus$\,paz & \textsc{ sg ine } \\
\underline{ham}\,$\oplus$\,pass & \textsc{ sg ela } \\
\underline{ham}\,$\oplus$\,pallõ & \textsc{ sg all } \\
\underline{ham}\,$\oplus$\,pall & \textsc{ sg ade } \\
\underline{ham}\,$\oplus$\,palt & \textsc{ sg abl } \\
\underline{ham}\,$\oplus$\,passi & \textsc{ sg tra } \\
\underline{ham}\,$\oplus$\,passaa & \textsc{ sg ter } \\
\underline{ham}\,$\oplus$\,paka & \textsc{ sg com } \\
\underline{ham}\,$\oplus$\,pad & \textsc{ pl nom } \\
\underline{ham}\,$\oplus$\,pajõ & \textsc{ pl gen } \\
\underline{ham}\,$\oplus$\,pait & \textsc{ pl par } \\
\underline{ham}\,$\oplus$\,paisõ & \textsc{ pl ill } \\
\underline{ham}\,$\oplus$\,paiz & \textsc{ pl ine } \\
\underline{ham}\,$\oplus$\,paiss & \textsc{ pl ela } \\
\underline{ham}\,$\oplus$\,paillõ & \textsc{ pl all } \\
\underline{ham}\,$\oplus$\,paill & \textsc{ pl ade } \\
\underline{ham}\,$\oplus$\,pailt & \textsc{ pl abl } \\
\underline{ham}\,$\oplus$\,paissi & \textsc{ pl tra } \\
\underline{ham}\,$\oplus$\,paissaa & \textsc{ pl ter } \\
\underline{ham}\,$\oplus$\,paika & \textsc{ pl com } \\
\end{tabular}
\end{sideways}
\captionof{table}{Tüüpsõnamall \arabic{mallinumber}\,\vadja{hammõz} ekstraheeritud muutvormimallid.}
\label{tab:tüüpsõnamall-hammõz}

\end{minipage}

 
\vspace{1em}
\noindent Tüüpsõnamall \vadja{hammõz} hõlmab vormisõnastikus 3 lekseemi: \vadja{\underline{ham}mõz, \underline{lam}mõz} ja \vadja{\underline{am}mõz}.



\vspace{3.5em}
\noindent \begin{minipage}{\textwidth}
\stepcounter{mallinumber}
\noindent \textbf{Tüüpsõnamall \arabic{mallinumber}\,\vadja{lõunõ}}\\

\begin{sideways}
\begin{tabular}{l l}
muutvormimall & tunnused \\
\hline
\underline{lõun}\,$\oplus$\,õ & \textsc{ sg nom } \\
\underline{lõun}\,$\oplus$\,a & \textsc{ sg gen } \\
\underline{lõun}\,$\oplus$\,attõ & \textsc{ sg par } \\
\underline{lõun}\,$\oplus$\,asõ & \textsc{ sg ill } \\
\underline{lõun}\,$\oplus$\,az & \textsc{ sg ine } \\
\underline{lõun}\,$\oplus$\,ass & \textsc{ sg ela } \\
\underline{lõun}\,$\oplus$\,allõ & \textsc{ sg all } \\
\underline{lõun}\,$\oplus$\,all & \textsc{ sg ade } \\
\underline{lõun}\,$\oplus$\,alt & \textsc{ sg abl } \\
\underline{lõun}\,$\oplus$\,assi & \textsc{ sg tra } \\
\underline{lõun}\,$\oplus$\,assaa & \textsc{ sg ter } \\
\underline{lõun}\,$\oplus$\,aka & \textsc{ sg com } \\
\underline{lõun}\,$\oplus$\,ad & \textsc{ pl nom } \\
\underline{lõun}\,$\oplus$\,ajõ & \textsc{ pl gen } \\
\underline{lõun}\,$\oplus$\,ait & \textsc{ pl par } \\
\underline{lõun}\,$\oplus$\,aisõ & \textsc{ pl ill } \\
\underline{lõun}\,$\oplus$\,aiz & \textsc{ pl ine } \\
\underline{lõun}\,$\oplus$\,aiss & \textsc{ pl ela } \\
\underline{lõun}\,$\oplus$\,aillõ & \textsc{ pl all } \\
\underline{lõun}\,$\oplus$\,aill & \textsc{ pl ade } \\
\underline{lõun}\,$\oplus$\,ailt & \textsc{ pl abl } \\
\underline{lõun}\,$\oplus$\,aissi & \textsc{ pl tra } \\
\underline{lõun}\,$\oplus$\,aissaa & \textsc{ pl ter } \\
\underline{lõun}\,$\oplus$\,aika & \textsc{ pl com } \\
\end{tabular}
\end{sideways}
\captionof{table}{Tüüpsõnamall \arabic{mallinumber}\,\vadja{lõunõ} ekstraheeritud muutvormimallid.}
\label{tab:tüüpsõnamall-lõunõ}

\end{minipage}

 
\vspace{1em}
\noindent Tüüpsõnamall \vadja{lõunõ} ei hõlma teisi lekseeme vormi\-sõnastikus.



\vspace{3.5em}
\noindent \begin{minipage}{\textwidth}
\stepcounter{mallinumber}
\noindent \textbf{Tüüpsõnamall \arabic{mallinumber}\,\vadja{kangõz}}\\

\begin{sideways}
\begin{tabular}{l l}
muutvormimall & tunnused \\
\hline
\underline{kan}\,$\oplus$\,gõz & \textsc{ sg nom } \\
\underline{kan}\,$\oplus$\,ka & \textsc{ sg gen } \\
\underline{kan}\,$\oplus$\,gõssõ & \textsc{ sg par } \\
\underline{kan}\,$\oplus$\,kasõ & \textsc{ sg ill } \\
\underline{kan}\,$\oplus$\,kaz & \textsc{ sg ine } \\
\underline{kan}\,$\oplus$\,kass & \textsc{ sg ela } \\
\underline{kan}\,$\oplus$\,kallõ & \textsc{ sg all } \\
\underline{kan}\,$\oplus$\,kall & \textsc{ sg ade } \\
\underline{kan}\,$\oplus$\,kalt & \textsc{ sg abl } \\
\underline{kan}\,$\oplus$\,kassi & \textsc{ sg tra } \\
\underline{kan}\,$\oplus$\,kassaa & \textsc{ sg ter } \\
\underline{kan}\,$\oplus$\,kaka & \textsc{ sg com } \\
\underline{kan}\,$\oplus$\,kad & \textsc{ pl nom } \\
\underline{kan}\,$\oplus$\,kajõ & \textsc{ pl gen } \\
\underline{kan}\,$\oplus$\,kait & \textsc{ pl par } \\
\underline{kan}\,$\oplus$\,kaisõ & \textsc{ pl ill } \\
\underline{kan}\,$\oplus$\,kaiz & \textsc{ pl ine } \\
\underline{kan}\,$\oplus$\,kaiss & \textsc{ pl ela } \\
\underline{kan}\,$\oplus$\,kaillõ & \textsc{ pl all } \\
\underline{kan}\,$\oplus$\,kaill & \textsc{ pl ade } \\
\underline{kan}\,$\oplus$\,kailt & \textsc{ pl abl } \\
\underline{kan}\,$\oplus$\,kaissi & \textsc{ pl tra } \\
\underline{kan}\,$\oplus$\,kaissaa & \textsc{ pl ter } \\
\underline{kan}\,$\oplus$\,kaika & \textsc{ pl com } \\
\end{tabular}
\end{sideways}
\captionof{table}{Tüüpsõnamall \arabic{mallinumber}\,\vadja{kangõz} ekstraheeritud muutvormimallid.}
\label{tab:tüüpsõnamall-kangõz}

\end{minipage}

 
\vspace{1em}
\noindent Tüüpsõnamall \vadja{kangõz} ei hõlma teisi lekseeme vormi\-sõnastikus.



\vspace{3.5em}
\noindent \begin{minipage}{\textwidth}
\stepcounter{mallinumber}
\noindent \textbf{Tüüpsõnamall \arabic{mallinumber}\,\vadja{kauniz}}\\

\begin{sideways}
\begin{tabular}{l l}
muutvormimall & tunnused \\
\hline
\underline{kauni}\,$\oplus$\,z & \textsc{ sg nom } \\
\underline{kauni} & \textsc{ sg gen } \\
\underline{kauni}\,$\oplus$\,ssõ & \textsc{ sg par } \\
\underline{kauni}\,$\oplus$\,sõ & \textsc{ sg ill } \\
\underline{kauni}\,$\oplus$\,z & \textsc{ sg ine } \\
\underline{kauni}\,$\oplus$\,ss & \textsc{ sg ela } \\
\underline{kauni}\,$\oplus$\,llõ & \textsc{ sg all } \\
\underline{kauni}\,$\oplus$\,ll & \textsc{ sg ade } \\
\underline{kauni}\,$\oplus$\,lt & \textsc{ sg abl } \\
\underline{kauni}\,$\oplus$\,ssi & \textsc{ sg tra } \\
\underline{kauni}\,$\oplus$\,ssaa & \textsc{ sg ter } \\
\underline{kauni}\,$\oplus$\,ka & \textsc{ sg com } \\
\underline{kauni}\,$\oplus$\,d & \textsc{ pl nom } \\
\underline{kauni}\,$\oplus$\,jõ & \textsc{ pl gen } \\
\underline{kauni}\,$\oplus$\,it & \textsc{ pl par } \\
\underline{kauni}\,$\oplus$\,isõ & \textsc{ pl ill } \\
\underline{kauni}\,$\oplus$\,iz & \textsc{ pl ine } \\
\underline{kauni}\,$\oplus$\,iss & \textsc{ pl ela } \\
\underline{kauni}\,$\oplus$\,illõ & \textsc{ pl all } \\
\underline{kauni}\,$\oplus$\,ill & \textsc{ pl ade } \\
\underline{kauni}\,$\oplus$\,ilt & \textsc{ pl abl } \\
\underline{kauni}\,$\oplus$\,issi & \textsc{ pl tra } \\
\underline{kauni}\,$\oplus$\,issaa & \textsc{ pl ter } \\
\underline{kauni}\,$\oplus$\,jka & \textsc{ pl com } \\
\end{tabular}
\end{sideways}
\captionof{table}{Tüüpsõnamall \arabic{mallinumber}\,\vadja{kauniz} ekstraheeritud muutvormimallid.}
\label{tab:tüüpsõnamall-kauniz}

\end{minipage}

 
\vspace{1em}
\noindent Tüüpsõnamall \vadja{kauniz} ei hõlma teisi lekseeme vormi\-sõnastikus.



\vspace{3.5em}
\noindent \begin{minipage}{\textwidth}
\stepcounter{mallinumber}
\noindent \textbf{Tüüpsõnamall \arabic{mallinumber}\,\vadja{angõriaz}}\\

\begin{sideways}
\begin{tabular}{l l}
muutvormimall & tunnused \\
\hline
\underline{angõri}\,$\oplus$\,az & \textsc{ sg nom } \\
\underline{angõri}\,$\oplus$\,a & \textsc{ sg gen } \\
\underline{angõri}\,$\oplus$\,assõ & \textsc{ sg par } \\
\underline{angõri}\,$\oplus$\,asõ & \textsc{ sg ill } \\
\underline{angõri}\,$\oplus$\,az & \textsc{ sg ine } \\
\underline{angõri}\,$\oplus$\,ass & \textsc{ sg ela } \\
\underline{angõri}\,$\oplus$\,allõ & \textsc{ sg all } \\
\underline{angõri}\,$\oplus$\,all & \textsc{ sg ade } \\
\underline{angõri}\,$\oplus$\,alt & \textsc{ sg abl } \\
\underline{angõri}\,$\oplus$\,assi & \textsc{ sg tra } \\
\underline{angõri}\,$\oplus$\,assaa & \textsc{ sg ter } \\
\underline{angõri}\,$\oplus$\,aka & \textsc{ sg com } \\
\underline{angõri}\,$\oplus$\,ad & \textsc{ pl nom } \\
\underline{angõri}\,$\oplus$\,jõ & \textsc{ pl gen } \\
\underline{angõri}\,$\oplus$\,it & \textsc{ pl par } \\
\underline{angõri}\,$\oplus$\,isõ & \textsc{ pl ill } \\
\underline{angõri}\,$\oplus$\,iz & \textsc{ pl ine } \\
\underline{angõri}\,$\oplus$\,iss & \textsc{ pl ela } \\
\underline{angõri}\,$\oplus$\,illõ & \textsc{ pl all } \\
\underline{angõri}\,$\oplus$\,ill & \textsc{ pl ade } \\
\underline{angõri}\,$\oplus$\,ilt & \textsc{ pl abl } \\
\underline{angõri}\,$\oplus$\,issi & \textsc{ pl tra } \\
\underline{angõri}\,$\oplus$\,issaa & \textsc{ pl ter } \\
\underline{angõri}\,$\oplus$\,jka & \textsc{ pl com } \\
\end{tabular}
\end{sideways}
\captionof{table}{Tüüpsõnamall \arabic{mallinumber}\,\vadja{angõriaz} ekstraheeritud muutvormimallid.}
\label{tab:tüüpsõnamall-angõriaz}

\end{minipage}

 
\vspace{1em}
\noindent Tüüpsõnamall \vadja{angõriaz} ei hõlma teisi lekseeme vormi\-sõnastikus.



\vspace{3.5em}
\noindent \begin{minipage}{\textwidth}
\stepcounter{mallinumber}
\noindent \textbf{Tüüpsõnamall \arabic{mallinumber}\,\vadja{raskõz}}\\

\begin{sideways}
\begin{tabular}{l l}
muutvormimall & tunnused \\
\hline
\underline{rask}\,$\oplus$\,õz & \textsc{ sg nom } \\
\underline{rask}\,$\oplus$\,a & \textsc{ sg gen } \\
\underline{rask}\,$\oplus$\,assõ & \textsc{ sg par } \\
\underline{rask}\,$\oplus$\,asõ & \textsc{ sg ill } \\
\underline{rask}\,$\oplus$\,az & \textsc{ sg ine } \\
\underline{rask}\,$\oplus$\,ass & \textsc{ sg ela } \\
\underline{rask}\,$\oplus$\,allõ & \textsc{ sg all } \\
\underline{rask}\,$\oplus$\,all & \textsc{ sg ade } \\
\underline{rask}\,$\oplus$\,alt & \textsc{ sg abl } \\
\underline{rask}\,$\oplus$\,assi & \textsc{ sg tra } \\
\underline{rask}\,$\oplus$\,assaa & \textsc{ sg ter } \\
\underline{rask}\,$\oplus$\,aka & \textsc{ sg com } \\
\underline{rask}\,$\oplus$\,ad & \textsc{ pl nom } \\
\underline{rask}\,$\oplus$\,ojõ & \textsc{ pl gen } \\
\underline{rask}\,$\oplus$\,ait & \textsc{ pl par } \\
\underline{rask}\,$\oplus$\,aisõ & \textsc{ pl ill } \\
\underline{rask}\,$\oplus$\,aiz & \textsc{ pl ine } \\
\underline{rask}\,$\oplus$\,aiss & \textsc{ pl ela } \\
\underline{rask}\,$\oplus$\,aillõ & \textsc{ pl all } \\
\underline{rask}\,$\oplus$\,aill & \textsc{ pl ade } \\
\underline{rask}\,$\oplus$\,ailt & \textsc{ pl abl } \\
\underline{rask}\,$\oplus$\,aissi & \textsc{ pl tra } \\
\underline{rask}\,$\oplus$\,aissaa & \textsc{ pl ter } \\
\underline{rask}\,$\oplus$\,aika & \textsc{ pl com } \\
\end{tabular}
\end{sideways}
\captionof{table}{Tüüpsõnamall \arabic{mallinumber}\,\vadja{raskõz} ekstraheeritud muutvormimallid.}
\label{tab:tüüpsõnamall-raskõz}

\end{minipage}

 
\vspace{1em}
\noindent Tüüpsõnamall \vadja{raskõz} ei hõlma teisi lekseeme vormi\-sõnastikus.



\vspace{3.5em}
\noindent \begin{minipage}{\textwidth}
\stepcounter{mallinumber}
\noindent \textbf{Tüüpsõnamall \arabic{mallinumber}\,\vadja{kasõ}}\\

\begin{sideways}
\begin{tabular}{l l}
muutvormimall & tunnused \\
\hline
\underline{kas}\,$\oplus$\,\underline{õ} & \textsc{ sg nom } \\
\underline{kas}\,$\oplus$\,s\,$\oplus$\,\underline{õ} & \textsc{ sg gen } \\
\underline{kas}\,$\oplus$\,\underline{õ}\,$\oplus$\,ttõ & \textsc{ sg par } \\
\underline{kas}\,$\oplus$\,s\,$\oplus$\,\underline{õ}\,$\oplus$\,sõ & \textsc{ sg ill } \\
\underline{kas}\,$\oplus$\,s\,$\oplus$\,\underline{õ}\,$\oplus$\,z & \textsc{ sg ine } \\
\underline{kas}\,$\oplus$\,s\,$\oplus$\,\underline{õ}\,$\oplus$\,ss & \textsc{ sg ela } \\
\underline{kas}\,$\oplus$\,s\,$\oplus$\,\underline{õ}\,$\oplus$\,llõ & \textsc{ sg all } \\
\underline{kas}\,$\oplus$\,s\,$\oplus$\,\underline{õ}\,$\oplus$\,ll & \textsc{ sg ade } \\
\underline{kas}\,$\oplus$\,s\,$\oplus$\,\underline{õ}\,$\oplus$\,lt & \textsc{ sg abl } \\
\underline{kas}\,$\oplus$\,s\,$\oplus$\,\underline{õ}\,$\oplus$\,ssi & \textsc{ sg tra } \\
\underline{kas}\,$\oplus$\,s\,$\oplus$\,\underline{õ}\,$\oplus$\,ssaa & \textsc{ sg ter } \\
\underline{kas}\,$\oplus$\,s\,$\oplus$\,\underline{õ}\,$\oplus$\,ka & \textsc{ sg com } \\
\underline{kas}\,$\oplus$\,s\,$\oplus$\,\underline{õ}\,$\oplus$\,d & \textsc{ pl nom } \\
\underline{kas}\,$\oplus$\,s\,$\oplus$\,\underline{õ}\,$\oplus$\,jõ & \textsc{ pl gen } \\
\underline{kas}\,$\oplus$\,s\,$\oplus$\,\underline{õ}\,$\oplus$\,it & \textsc{ pl par } \\
\underline{kas}\,$\oplus$\,s\,$\oplus$\,\underline{õ}\,$\oplus$\,isõ & \textsc{ pl ill } \\
\underline{kas}\,$\oplus$\,s\,$\oplus$\,\underline{õ}\,$\oplus$\,iz & \textsc{ pl ine } \\
\underline{kas}\,$\oplus$\,s\,$\oplus$\,\underline{õ}\,$\oplus$\,iss & \textsc{ pl ela } \\
\underline{kas}\,$\oplus$\,s\,$\oplus$\,\underline{õ}\,$\oplus$\,illõ & \textsc{ pl all } \\
\underline{kas}\,$\oplus$\,s\,$\oplus$\,\underline{õ}\,$\oplus$\,ill & \textsc{ pl ade } \\
\underline{kas}\,$\oplus$\,s\,$\oplus$\,\underline{õ}\,$\oplus$\,ilt & \textsc{ pl abl } \\
\underline{kas}\,$\oplus$\,s\,$\oplus$\,\underline{õ}\,$\oplus$\,issi & \textsc{ pl tra } \\
\underline{kas}\,$\oplus$\,s\,$\oplus$\,\underline{õ}\,$\oplus$\,issaa & \textsc{ pl ter } \\
\underline{kas}\,$\oplus$\,s\,$\oplus$\,\underline{õ}\,$\oplus$\,ika & \textsc{ pl com } \\
\end{tabular}
\end{sideways}
\captionof{table}{Tüüpsõnamall \arabic{mallinumber}\,\vadja{kasõ} ekstraheeritud muutvormimallid.}
\label{tab:tüüpsõnamall-kasõ}

\end{minipage}

 
\vspace{1em}
\noindent Tüüpsõnamall \vadja{kasõ} ei hõlma teisi lekseeme vormi\-sõnastikus.



\vspace{3.5em}
\noindent \begin{minipage}{\textwidth}
\stepcounter{mallinumber}
\noindent \textbf{Tüüpsõnamall \arabic{mallinumber}\,\vadja{vetelüz}}\\

\begin{sideways}
\begin{tabular}{l l}
muutvormimall & tunnused \\
\hline
\underline{vetelü}\,$\oplus$\,z & \textsc{ sg nom } \\
\underline{vetelü}\,$\oplus$\,se & \textsc{ sg gen } \\
\underline{vetelü}\,$\oplus$\,sse & \textsc{ sg par } \\
\underline{vetelü}\,$\oplus$\,sse & \textsc{ sg ill } \\
\underline{vetelü}\,$\oplus$\,sez & \textsc{ sg ine } \\
\underline{vetelü}\,$\oplus$\,sess & \textsc{ sg ela } \\
\underline{vetelü}\,$\oplus$\,selle & \textsc{ sg all } \\
\underline{vetelü}\,$\oplus$\,sell & \textsc{ sg ade } \\
\underline{vetelü}\,$\oplus$\,selt & \textsc{ sg abl } \\
\underline{vetelü}\,$\oplus$\,sessi & \textsc{ sg tra } \\
\underline{vetelü}\,$\oplus$\,sessaa & \textsc{ sg ter } \\
\underline{vetelü}\,$\oplus$\,seka & \textsc{ sg com } \\
\underline{vetelü}\,$\oplus$\,sed & \textsc{ pl nom } \\
\underline{vetelü}\,$\oplus$\,ssije & \textsc{ pl gen } \\
\underline{vetelü}\,$\oplus$\,ssiit & \textsc{ pl par } \\
\underline{vetelü}\,$\oplus$\,ssiise & \textsc{ pl ill } \\
\underline{vetelü}\,$\oplus$\,ssiiz & \textsc{ pl ine } \\
\underline{vetelü}\,$\oplus$\,ssiiss & \textsc{ pl ela } \\
\underline{vetelü}\,$\oplus$\,ssiille & \textsc{ pl all } \\
\underline{vetelü}\,$\oplus$\,ssiill & \textsc{ pl ade } \\
\underline{vetelü}\,$\oplus$\,ssiilt & \textsc{ pl abl } \\
\underline{vetelü}\,$\oplus$\,ssiissi & \textsc{ pl tra } \\
\underline{vetelü}\,$\oplus$\,ssiissaa & \textsc{ pl ter } \\
\underline{vetelü}\,$\oplus$\,ssijka & \textsc{ pl com } \\
\end{tabular}
\end{sideways}
\captionof{table}{Tüüpsõnamall \arabic{mallinumber}\,\vadja{vetelüz} ekstraheeritud muutvormimallid.}
\label{tab:tüüpsõnamall-vetelüz}

\end{minipage}

 
\vspace{1em}
\noindent Tüüpsõnamall \vadja{vetelüz} hõlmab vormisõnastikus 2 lekseemi: \vadja{\underline{vetelü}z} ja \vadja{\underline{jäne}z}.



\vspace{3.5em}
\noindent \begin{minipage}{\textwidth}
\stepcounter{mallinumber}
\noindent \textbf{Tüüpsõnamall \arabic{mallinumber}\,\vadja{põrzõz}}\\

\begin{sideways}
\begin{tabular}{l l}
muutvormimall & tunnused \\
\hline
\underline{põr}\,$\oplus$\,zõz & \textsc{ sg nom } \\
\underline{põr}\,$\oplus$\,sa & \textsc{ sg gen } \\
\underline{põr}\,$\oplus$\,zassõ & \textsc{ sg par } \\
\underline{põr}\,$\oplus$\,sasõ & \textsc{ sg ill } \\
\underline{põr}\,$\oplus$\,saz & \textsc{ sg ine } \\
\underline{põr}\,$\oplus$\,sass & \textsc{ sg ela } \\
\underline{põr}\,$\oplus$\,sallõ & \textsc{ sg all } \\
\underline{põr}\,$\oplus$\,sall & \textsc{ sg ade } \\
\underline{põr}\,$\oplus$\,salt & \textsc{ sg abl } \\
\underline{põr}\,$\oplus$\,sassi & \textsc{ sg tra } \\
\underline{põr}\,$\oplus$\,sassaa & \textsc{ sg ter } \\
\underline{põr}\,$\oplus$\,saka & \textsc{ sg com } \\
\underline{põr}\,$\oplus$\,sad & \textsc{ pl nom } \\
\underline{põr}\,$\oplus$\,sojõ & \textsc{ pl gen } \\
\underline{põr}\,$\oplus$\,soit & \textsc{ pl par } \\
\underline{põr}\,$\oplus$\,soisõ & \textsc{ pl ill } \\
\underline{põr}\,$\oplus$\,soiz & \textsc{ pl ine } \\
\underline{põr}\,$\oplus$\,soiss & \textsc{ pl ela } \\
\underline{põr}\,$\oplus$\,soillõ & \textsc{ pl all } \\
\underline{põr}\,$\oplus$\,soill & \textsc{ pl ade } \\
\underline{põr}\,$\oplus$\,soilt & \textsc{ pl abl } \\
\underline{põr}\,$\oplus$\,soissi & \textsc{ pl tra } \\
\underline{põr}\,$\oplus$\,soissaa & \textsc{ pl ter } \\
\underline{põr}\,$\oplus$\,soika & \textsc{ pl com } \\
\end{tabular}
\end{sideways}
\captionof{table}{Tüüpsõnamall \arabic{mallinumber}\,\vadja{põrzõz} ekstraheeritud muutvormimallid.}
\label{tab:tüüpsõnamall-põrzõz}

\end{minipage}

 
\vspace{1em}
\noindent Tüüpsõnamall \vadja{põrzõz} ei hõlma teisi lekseeme vormi\-sõnastikus.



\vspace{3.5em}
\noindent \begin{minipage}{\textwidth}
\stepcounter{mallinumber}
\noindent \textbf{Tüüpsõnamall \arabic{mallinumber}\,\vadja{ratiz}}\\

\begin{sideways}
\begin{tabular}{l l}
muutvormimall & tunnused \\
\hline
\underline{rat}\,$\oplus$\,\underline{i}\,$\oplus$\,z & \textsc{ sg nom } \\
\underline{rat}\,$\oplus$\,t\,$\oplus$\,\underline{i} & \textsc{ sg gen } \\
\underline{rat}\,$\oplus$\,\underline{i}\,$\oplus$\,ssõ & \textsc{ sg par } \\
\underline{rat}\,$\oplus$\,t\,$\oplus$\,\underline{i}\,$\oplus$\,sõ & \textsc{ sg ill } \\
\underline{rat}\,$\oplus$\,t\,$\oplus$\,\underline{i}\,$\oplus$\,z & \textsc{ sg ine } \\
\underline{rat}\,$\oplus$\,t\,$\oplus$\,\underline{i}\,$\oplus$\,ss & \textsc{ sg ela } \\
\underline{rat}\,$\oplus$\,t\,$\oplus$\,\underline{i}\,$\oplus$\,llõ & \textsc{ sg all } \\
\underline{rat}\,$\oplus$\,t\,$\oplus$\,\underline{i}\,$\oplus$\,ll & \textsc{ sg ade } \\
\underline{rat}\,$\oplus$\,t\,$\oplus$\,\underline{i}\,$\oplus$\,lt & \textsc{ sg abl } \\
\underline{rat}\,$\oplus$\,t\,$\oplus$\,\underline{i}\,$\oplus$\,ssi & \textsc{ sg tra } \\
\underline{rat}\,$\oplus$\,t\,$\oplus$\,\underline{i}\,$\oplus$\,ssaa & \textsc{ sg ter } \\
\underline{rat}\,$\oplus$\,t\,$\oplus$\,\underline{i}\,$\oplus$\,ka & \textsc{ sg com } \\
\underline{rat}\,$\oplus$\,t\,$\oplus$\,\underline{i}\,$\oplus$\,d & \textsc{ pl nom } \\
\underline{rat}\,$\oplus$\,t\,$\oplus$\,\underline{i}\,$\oplus$\,jõ & \textsc{ pl gen } \\
\underline{rat}\,$\oplus$\,t\,$\oplus$\,\underline{i}\,$\oplus$\,it & \textsc{ pl par } \\
\underline{rat}\,$\oplus$\,t\,$\oplus$\,\underline{i}\,$\oplus$\,isõ & \textsc{ pl ill } \\
\underline{rat}\,$\oplus$\,t\,$\oplus$\,\underline{i}\,$\oplus$\,iz & \textsc{ pl ine } \\
\underline{rat}\,$\oplus$\,t\,$\oplus$\,\underline{i}\,$\oplus$\,iss & \textsc{ pl ela } \\
\underline{rat}\,$\oplus$\,t\,$\oplus$\,\underline{i}\,$\oplus$\,illõ & \textsc{ pl all } \\
\underline{rat}\,$\oplus$\,t\,$\oplus$\,\underline{i}\,$\oplus$\,ill & \textsc{ pl ade } \\
\underline{rat}\,$\oplus$\,t\,$\oplus$\,\underline{i}\,$\oplus$\,ilt & \textsc{ pl abl } \\
\underline{rat}\,$\oplus$\,t\,$\oplus$\,\underline{i}\,$\oplus$\,issi & \textsc{ pl tra } \\
\underline{rat}\,$\oplus$\,t\,$\oplus$\,\underline{i}\,$\oplus$\,issaa & \textsc{ pl ter } \\
\underline{rat}\,$\oplus$\,t\,$\oplus$\,\underline{i}\,$\oplus$\,jka & \textsc{ pl com } \\
\end{tabular}
\end{sideways}
\captionof{table}{Tüüpsõnamall \arabic{mallinumber}\,\vadja{ratiz} ekstraheeritud muutvormimallid.}
\label{tab:tüüpsõnamall-ratiz}

\end{minipage}

 
\vspace{1em}
\noindent Tüüpsõnamall \vadja{ratiz} ei hõlma teisi lekseeme vormi\-sõnastikus.



\vspace{3.5em}
\noindent \begin{minipage}{\textwidth}
\stepcounter{mallinumber}
\noindent \textbf{Tüüpsõnamall \arabic{mallinumber}\,\vadja{katõ}}\\

\begin{sideways}
\begin{tabular}{l l}
muutvormimall & tunnused \\
\hline
\underline{kat}\,$\oplus$\,\underline{õ} & \textsc{ sg nom } \\
\underline{kat}\,$\oplus$\,t\,$\oplus$\,\underline{õ} & \textsc{ sg gen } \\
\underline{kat}\,$\oplus$\,\underline{õ}\,$\oplus$\,ttõ & \textsc{ sg par } \\
\underline{kat}\,$\oplus$\,t\,$\oplus$\,\underline{õ}\,$\oplus$\,sõ & \textsc{ sg ill } \\
\underline{kat}\,$\oplus$\,t\,$\oplus$\,\underline{õ}\,$\oplus$\,z & \textsc{ sg ine } \\
\underline{kat}\,$\oplus$\,t\,$\oplus$\,\underline{õ}\,$\oplus$\,ss & \textsc{ sg ela } \\
\underline{kat}\,$\oplus$\,t\,$\oplus$\,\underline{õ}\,$\oplus$\,llõ & \textsc{ sg all } \\
\underline{kat}\,$\oplus$\,t\,$\oplus$\,\underline{õ}\,$\oplus$\,ll & \textsc{ sg ade } \\
\underline{kat}\,$\oplus$\,t\,$\oplus$\,\underline{õ}\,$\oplus$\,lt & \textsc{ sg abl } \\
\underline{kat}\,$\oplus$\,t\,$\oplus$\,\underline{õ}\,$\oplus$\,ssi & \textsc{ sg tra } \\
\underline{kat}\,$\oplus$\,t\,$\oplus$\,\underline{õ}\,$\oplus$\,ssaa & \textsc{ sg ter } \\
\underline{kat}\,$\oplus$\,t\,$\oplus$\,\underline{õ}\,$\oplus$\,ka & \textsc{ sg com } \\
\underline{kat}\,$\oplus$\,t\,$\oplus$\,\underline{õ}\,$\oplus$\,d & \textsc{ pl nom } \\
\underline{kat}\,$\oplus$\,t\,$\oplus$\,\underline{õ}\,$\oplus$\,jõ & \textsc{ pl gen } \\
\underline{kat}\,$\oplus$\,t\,$\oplus$\,\underline{õ}\,$\oplus$\,it & \textsc{ pl par } \\
\underline{kat}\,$\oplus$\,t\,$\oplus$\,\underline{õ}\,$\oplus$\,isõ & \textsc{ pl ill } \\
\underline{kat}\,$\oplus$\,t\,$\oplus$\,\underline{õ}\,$\oplus$\,iz & \textsc{ pl ine } \\
\underline{kat}\,$\oplus$\,t\,$\oplus$\,\underline{õ}\,$\oplus$\,iss & \textsc{ pl ela } \\
\underline{kat}\,$\oplus$\,t\,$\oplus$\,\underline{õ}\,$\oplus$\,illõ & \textsc{ pl all } \\
\underline{kat}\,$\oplus$\,t\,$\oplus$\,\underline{õ}\,$\oplus$\,ill & \textsc{ pl ade } \\
\underline{kat}\,$\oplus$\,t\,$\oplus$\,\underline{õ}\,$\oplus$\,ilt & \textsc{ pl abl } \\
\underline{kat}\,$\oplus$\,t\,$\oplus$\,\underline{õ}\,$\oplus$\,issi & \textsc{ pl tra } \\
\underline{kat}\,$\oplus$\,t\,$\oplus$\,\underline{õ}\,$\oplus$\,issaa & \textsc{ pl ter } \\
\underline{kat}\,$\oplus$\,t\,$\oplus$\,\underline{õ}\,$\oplus$\,ika & \textsc{ pl com } \\
\end{tabular}
\end{sideways}
\captionof{table}{Tüüpsõnamall \arabic{mallinumber}\,\vadja{katõ} ekstraheeritud muutvormimallid.}
\label{tab:tüüpsõnamall-katõ}

\end{minipage}

 
\vspace{1em}
\noindent Tüüpsõnamall \vadja{katõ} ei hõlma teisi lekseeme vormi\-sõnastikus.



\vspace{3.5em}
\noindent \begin{minipage}{\textwidth}
\stepcounter{mallinumber}
\noindent \textbf{Tüüpsõnamall \arabic{mallinumber}\,\vadja{kuõ}}\\

\begin{sideways}
\begin{tabular}{l l}
muutvormimall & tunnused \\
\hline
\underline{ku}\,$\oplus$\,\underline{õ} & \textsc{ sg nom } \\
\underline{ku}\,$\oplus$\,t\,$\oplus$\,\underline{õ} & \textsc{ sg gen } \\
\underline{ku}\,$\oplus$\,\underline{õ}\,$\oplus$\,ttõ & \textsc{ sg par } \\
\underline{ku}\,$\oplus$\,t\,$\oplus$\,\underline{õ}\,$\oplus$\,sõ & \textsc{ sg ill } \\
\underline{ku}\,$\oplus$\,tt\,$\oplus$\,\underline{õ}\,$\oplus$\,z & \textsc{ sg ine } \\
\underline{ku}\,$\oplus$\,tt\,$\oplus$\,\underline{õ}\,$\oplus$\,ss & \textsc{ sg ela } \\
\underline{ku}\,$\oplus$\,tt\,$\oplus$\,\underline{õ}\,$\oplus$\,llõ & \textsc{ sg all } \\
\underline{ku}\,$\oplus$\,tt\,$\oplus$\,\underline{õ}\,$\oplus$\,ll & \textsc{ sg ade } \\
\underline{ku}\,$\oplus$\,tt\,$\oplus$\,\underline{õ}\,$\oplus$\,lt & \textsc{ sg abl } \\
\underline{ku}\,$\oplus$\,tt\,$\oplus$\,\underline{õ}\,$\oplus$\,ssi & \textsc{ sg tra } \\
\underline{ku}\,$\oplus$\,tt\,$\oplus$\,\underline{õ}\,$\oplus$\,ssaa & \textsc{ sg ter } \\
\underline{ku}\,$\oplus$\,tt\,$\oplus$\,\underline{õ}\,$\oplus$\,ka & \textsc{ sg com } \\
\underline{ku}\,$\oplus$\,tt\,$\oplus$\,\underline{õ}\,$\oplus$\,d & \textsc{ pl nom } \\
\underline{ku}\,$\oplus$\,tt\,$\oplus$\,\underline{õ}\,$\oplus$\,jõ & \textsc{ pl gen } \\
\underline{ku}\,$\oplus$\,tt\,$\oplus$\,\underline{õ}\,$\oplus$\,it & \textsc{ pl par } \\
\underline{ku}\,$\oplus$\,tt\,$\oplus$\,\underline{õ}\,$\oplus$\,isõ & \textsc{ pl ill } \\
\underline{ku}\,$\oplus$\,tt\,$\oplus$\,\underline{õ}\,$\oplus$\,iz & \textsc{ pl ine } \\
\underline{ku}\,$\oplus$\,tt\,$\oplus$\,\underline{õ}\,$\oplus$\,iss & \textsc{ pl ela } \\
\underline{ku}\,$\oplus$\,tt\,$\oplus$\,\underline{õ}\,$\oplus$\,illõ & \textsc{ pl all } \\
\underline{ku}\,$\oplus$\,tt\,$\oplus$\,\underline{õ}\,$\oplus$\,ill & \textsc{ pl ade } \\
\underline{ku}\,$\oplus$\,tt\,$\oplus$\,\underline{õ}\,$\oplus$\,ilt & \textsc{ pl abl } \\
\underline{ku}\,$\oplus$\,tt\,$\oplus$\,\underline{õ}\,$\oplus$\,issi & \textsc{ pl tra } \\
\underline{ku}\,$\oplus$\,tt\,$\oplus$\,\underline{õ}\,$\oplus$\,issaa & \textsc{ pl ter } \\
\underline{ku}\,$\oplus$\,tt\,$\oplus$\,\underline{õ}\,$\oplus$\,ika & \textsc{ pl com } \\
\end{tabular}
\end{sideways}
\captionof{table}{Tüüpsõnamall \arabic{mallinumber}\,\vadja{kuõ} ekstraheeritud muutvormimallid.}
\label{tab:tüüpsõnamall-kuõ}

\end{minipage}

 
\vspace{1em}
\noindent Tüüpsõnamall \vadja{kuõ} ei hõlma teisi lekseeme vormi\-sõnastikus.



\vspace{3.5em}
\noindent \begin{minipage}{\textwidth}
\stepcounter{mallinumber}
\noindent \textbf{Tüüpsõnamall \arabic{mallinumber}\,\vadja{varvõz}}\\

\begin{sideways}
\begin{tabular}{l l}
muutvormimall & tunnused \\
\hline
\underline{var}\,$\oplus$\,võz & \textsc{ sg nom } \\
\underline{var}\,$\oplus$\,pa & \textsc{ sg gen } \\
\underline{var}\,$\oplus$\,vassõ & \textsc{ sg par } \\
\underline{var}\,$\oplus$\,pasõ & \textsc{ sg ill } \\
\underline{var}\,$\oplus$\,paz & \textsc{ sg ine } \\
\underline{var}\,$\oplus$\,pass & \textsc{ sg ela } \\
\underline{var}\,$\oplus$\,pallõ & \textsc{ sg all } \\
\underline{var}\,$\oplus$\,pall & \textsc{ sg ade } \\
\underline{var}\,$\oplus$\,palt & \textsc{ sg abl } \\
\underline{var}\,$\oplus$\,passi & \textsc{ sg tra } \\
\underline{var}\,$\oplus$\,passaa & \textsc{ sg ter } \\
\underline{var}\,$\oplus$\,paka & \textsc{ sg com } \\
\underline{var}\,$\oplus$\,pad & \textsc{ pl nom } \\
\underline{var}\,$\oplus$\,pajõ & \textsc{ pl gen } \\
\underline{var}\,$\oplus$\,pait & \textsc{ pl par } \\
\underline{var}\,$\oplus$\,paisõ & \textsc{ pl ill } \\
\underline{var}\,$\oplus$\,paiz & \textsc{ pl ine } \\
\underline{var}\,$\oplus$\,paiss & \textsc{ pl ela } \\
\underline{var}\,$\oplus$\,paillõ & \textsc{ pl all } \\
\underline{var}\,$\oplus$\,paill & \textsc{ pl ade } \\
\underline{var}\,$\oplus$\,pailt & \textsc{ pl abl } \\
\underline{var}\,$\oplus$\,paissi & \textsc{ pl tra } \\
\underline{var}\,$\oplus$\,paissaa & \textsc{ pl ter } \\
\underline{var}\,$\oplus$\,paika & \textsc{ pl com } \\
\end{tabular}
\end{sideways}
\captionof{table}{Tüüpsõnamall \arabic{mallinumber}\,\vadja{varvõz} ekstraheeritud muutvormimallid.}
\label{tab:tüüpsõnamall-varvõz}

\end{minipage}

 
\vspace{1em}
\noindent Tüüpsõnamall \vadja{varvõz} hõlmab vormisõnastikus 2 lekseemi: \vadja{\underline{var}võz} ja \vadja{\underline{tur}võz}.



\vspace{3.5em}
\noindent \begin{minipage}{\textwidth}
\stepcounter{mallinumber}
\noindent \textbf{Tüüpsõnamall \arabic{mallinumber}\,\vadja{pere}}\\

\begin{sideways}
\begin{tabular}{l l}
muutvormimall & tunnused \\
\hline
\underline{pere} & \textsc{ sg nom } \\
\underline{pere} & \textsc{ sg gen } \\
\underline{pere}\,$\oplus$\,tte & \textsc{ sg par } \\
\underline{pere}\,$\oplus$\,se & \textsc{ sg ill } \\
\underline{pere}\,$\oplus$\,z & \textsc{ sg ine } \\
\underline{pere}\,$\oplus$\,ss & \textsc{ sg ela } \\
\underline{pere}\,$\oplus$\,lle & \textsc{ sg all } \\
\underline{pere}\,$\oplus$\,ll & \textsc{ sg ade } \\
\underline{pere}\,$\oplus$\,lt & \textsc{ sg abl } \\
\underline{pere}\,$\oplus$\,ssi & \textsc{ sg tra } \\
\underline{pere}\,$\oplus$\,ssaa & \textsc{ sg ter } \\
\underline{pere}\,$\oplus$\,ka & \textsc{ sg com } \\
\underline{pere}\,$\oplus$\,d & \textsc{ pl nom } \\
\underline{pere}\,$\oplus$\,je & \textsc{ pl gen } \\
\underline{pere}\,$\oplus$\,it & \textsc{ pl par } \\
\underline{pere}\,$\oplus$\,ise & \textsc{ pl ill } \\
\underline{pere}\,$\oplus$\,iz & \textsc{ pl ine } \\
\underline{pere}\,$\oplus$\,iss & \textsc{ pl ela } \\
\underline{pere}\,$\oplus$\,ille & \textsc{ pl all } \\
\underline{pere}\,$\oplus$\,ill & \textsc{ pl ade } \\
\underline{pere}\,$\oplus$\,ilt & \textsc{ pl abl } \\
\underline{pere}\,$\oplus$\,issi & \textsc{ pl tra } \\
\underline{pere}\,$\oplus$\,issaa & \textsc{ pl ter } \\
\underline{pere}\,$\oplus$\,ika & \textsc{ pl com } \\
\end{tabular}
\end{sideways}
\captionof{table}{Tüüpsõnamall \arabic{mallinumber}\,\vadja{pere} ekstraheeritud muutvormimallid.}
\label{tab:tüüpsõnamall-pere}

\end{minipage}

 
\vspace{1em}
\noindent Tüüpsõnamall \vadja{pere} hõlmab vormisõnastikus 3 lekseemi: \vadja{\underline{pere}, \underline{vene}} ja \vadja{\underline{erne}}.

\spacing{1.5}



\subsection{\RN{15} käändkond}

Viieteistkümnendasse käändkonda koondab Ariste (\citeyear[51]{ariste_grammar_1968}) sõnu nagu \vadja{lühüd}, \vadja{õhud}, \vadja{koollu}, \vadja{ilozuZ}, \vadja{rikkauZ}.

Ka selle käändkonna liikmete \vadja{-Z}-lõpu sandhi nähtus on ühtlustatud \vadja{-z}-lõpuliseks \msd{sg nom} käändes.


\subsubsection*{Ekstraktmorfoloogia tüüpsõnad}
\spacing{1.1}
\vspace{-3.5em}


\vspace{3.5em}
\noindent \begin{minipage}{\textwidth}
\stepcounter{mallinumber}
\noindent \textbf{Tüüpsõnamall \arabic{mallinumber}\,\vadja{čämmel}}\\

\begin{sideways}
\begin{tabular}{l l}
muutvormimall & tunnused \\
\hline
\underline{čämmel} & \textsc{ sg nom } \\
\underline{čämmel}\,$\oplus$\,e & \textsc{ sg gen } \\
\underline{čämmel}\,$\oplus$\,te & \textsc{ sg par } \\
\underline{čämmel}\,$\oplus$\,äse & \textsc{ sg ill } \\
\underline{čämmel}\,$\oplus$\,ez & \textsc{ sg ine } \\
\underline{čämmel}\,$\oplus$\,ess & \textsc{ sg ela } \\
\underline{čämmel}\,$\oplus$\,elle & \textsc{ sg all } \\
\underline{čämmel}\,$\oplus$\,ell & \textsc{ sg ade } \\
\underline{čämmel}\,$\oplus$\,elt & \textsc{ sg abl } \\
\underline{čämmel}\,$\oplus$\,essi & \textsc{ sg tra } \\
\underline{čämmel}\,$\oplus$\,essaa & \textsc{ sg ter } \\
\underline{čämmel}\,$\oplus$\,eka & \textsc{ sg com } \\
\underline{čämmel}\,$\oplus$\,ed & \textsc{ pl nom } \\
\underline{čämmel}\,$\oplus$\,ije & \textsc{ pl gen } \\
\underline{čämmel}\,$\oplus$\,iit & \textsc{ pl par } \\
\underline{čämmel}\,$\oplus$\,iise & \textsc{ pl ill } \\
\underline{čämmel}\,$\oplus$\,iiz & \textsc{ pl ine } \\
\underline{čämmel}\,$\oplus$\,iiss & \textsc{ pl ela } \\
\underline{čämmel}\,$\oplus$\,iille & \textsc{ pl all } \\
\underline{čämmel}\,$\oplus$\,iill & \textsc{ pl ade } \\
\underline{čämmel}\,$\oplus$\,iilt & \textsc{ pl abl } \\
\underline{čämmel}\,$\oplus$\,iissi & \textsc{ pl tra } \\
\underline{čämmel}\,$\oplus$\,iissaa & \textsc{ pl ter } \\
\underline{čämmel}\,$\oplus$\,ijka & \textsc{ pl com } \\
\end{tabular}
\end{sideways}
\captionof{table}{Tüüpsõnamall \arabic{mallinumber}\,\vadja{čämmel} ekstraheeritud muutvormimallid.}
\label{tab:tüüpsõnamall-čämmel}

\end{minipage}

 
\vspace{1em}
\noindent Tüüpsõnamall \vadja{čämmel} ei hõlma teisi lekseeme vormi\-sõnastikus.



\vspace{3.5em}
\noindent \begin{minipage}{\textwidth}
\stepcounter{mallinumber}
\noindent \textbf{Tüüpsõnamall \arabic{mallinumber}\,\vadja{oonõ}}\\

\begin{sideways}
\begin{tabular}{l l}
muutvormimall & tunnused \\
\hline
\underline{oon}\,$\oplus$\,õ & \textsc{ sg nom } \\
\underline{oon}\,$\oplus$\,õ & \textsc{ sg gen } \\
\underline{oon}\,$\oplus$\,õt & \textsc{ sg par } \\
\underline{oon}\,$\oplus$\,õsõ & \textsc{ sg ill } \\
\underline{oon}\,$\oplus$\,õz & \textsc{ sg ine } \\
\underline{oon}\,$\oplus$\,õss & \textsc{ sg ela } \\
\underline{oon}\,$\oplus$\,õllõ & \textsc{ sg all } \\
\underline{oon}\,$\oplus$\,õll & \textsc{ sg ade } \\
\underline{oon}\,$\oplus$\,õlt & \textsc{ sg abl } \\
\underline{oon}\,$\oplus$\,õssi & \textsc{ sg tra } \\
\underline{oon}\,$\oplus$\,õssaa & \textsc{ sg ter } \\
\underline{oon}\,$\oplus$\,õka & \textsc{ sg com } \\
\underline{oon}\,$\oplus$\,õd & \textsc{ pl nom } \\
\underline{oon}\,$\oplus$\,ijõ & \textsc{ pl gen } \\
\underline{oon}\,$\oplus$\,iit & \textsc{ pl par } \\
\underline{oon}\,$\oplus$\,iisõ & \textsc{ pl ill } \\
\underline{oon}\,$\oplus$\,iiz & \textsc{ pl ine } \\
\underline{oon}\,$\oplus$\,iiss & \textsc{ pl ela } \\
\underline{oon}\,$\oplus$\,iillõ & \textsc{ pl all } \\
\underline{oon}\,$\oplus$\,iill & \textsc{ pl ade } \\
\underline{oon}\,$\oplus$\,iilt & \textsc{ pl abl } \\
\underline{oon}\,$\oplus$\,iissi & \textsc{ pl tra } \\
\underline{oon}\,$\oplus$\,iissaa & \textsc{ pl ter } \\
\underline{oon}\,$\oplus$\,ijka & \textsc{ pl com } \\
\end{tabular}
\end{sideways}
\captionof{table}{Tüüpsõnamall \arabic{mallinumber}\,\vadja{oonõ} ekstraheeritud muutvormimallid.}
\label{tab:tüüpsõnamall-oonõ}

\end{minipage}

 
\vspace{1em}
\noindent Tüüpsõnamall \vadja{oonõ} ei hõlma teisi lekseeme vormi\-sõnastikus.



\vspace{3.5em}
\noindent \begin{minipage}{\textwidth}
\stepcounter{mallinumber}
\noindent \textbf{Tüüpsõnamall \arabic{mallinumber}\,\vadja{kannõl}}\\

\begin{sideways}
\begin{tabular}{l l}
muutvormimall & tunnused \\
\hline
\underline{kannõl} & \textsc{ sg nom } \\
\underline{kannõl}\,$\oplus$\,õ & \textsc{ sg gen } \\
\underline{kannõl}\,$\oplus$\,tõ & \textsc{ sg par } \\
\underline{kannõl}\,$\oplus$\,asõ & \textsc{ sg ill } \\
\underline{kannõl}\,$\oplus$\,õz & \textsc{ sg ine } \\
\underline{kannõl}\,$\oplus$\,õss & \textsc{ sg ela } \\
\underline{kannõl}\,$\oplus$\,õllõ & \textsc{ sg all } \\
\underline{kannõl}\,$\oplus$\,õll & \textsc{ sg ade } \\
\underline{kannõl}\,$\oplus$\,õlt & \textsc{ sg abl } \\
\underline{kannõl}\,$\oplus$\,õssi & \textsc{ sg tra } \\
\underline{kannõl}\,$\oplus$\,õssaa & \textsc{ sg ter } \\
\underline{kannõl}\,$\oplus$\,õka & \textsc{ sg com } \\
\underline{kannõl}\,$\oplus$\,õd & \textsc{ pl nom } \\
\underline{kannõl}\,$\oplus$\,ijõ & \textsc{ pl gen } \\
\underline{kannõl}\,$\oplus$\,iit & \textsc{ pl par } \\
\underline{kannõl}\,$\oplus$\,iisõ & \textsc{ pl ill } \\
\underline{kannõl}\,$\oplus$\,iiz & \textsc{ pl ine } \\
\underline{kannõl}\,$\oplus$\,iiss & \textsc{ pl ela } \\
\underline{kannõl}\,$\oplus$\,iillõ & \textsc{ pl all } \\
\underline{kannõl}\,$\oplus$\,iill & \textsc{ pl ade } \\
\underline{kannõl}\,$\oplus$\,iilt & \textsc{ pl abl } \\
\underline{kannõl}\,$\oplus$\,iissi & \textsc{ pl tra } \\
\underline{kannõl}\,$\oplus$\,iissaa & \textsc{ pl ter } \\
\underline{kannõl}\,$\oplus$\,ijka & \textsc{ pl com } \\
\end{tabular}
\end{sideways}
\captionof{table}{Tüüpsõnamall \arabic{mallinumber}\,\vadja{kannõl} ekstraheeritud muutvormimallid.}
\label{tab:tüüpsõnamall-kannõl}

\end{minipage}

 
\vspace{1em}
\noindent Tüüpsõnamall \vadja{kannõl} ei hõlma teisi lekseeme vormi\-sõnastikus.



\vspace{3.5em}
\noindent \begin{minipage}{\textwidth}
\stepcounter{mallinumber}
\noindent \textbf{Tüüpsõnamall \arabic{mallinumber}\,\vadja{peremmeez}}\\

\begin{sideways}
\begin{tabular}{l l}
muutvormimall & tunnused \\
\hline
\underline{peremme}\,$\oplus$\,ez & \textsc{ sg nom } \\
\underline{peremme}\,$\oplus$\,he & \textsc{ sg gen } \\
\underline{peremme}\,$\oplus$\,esse & \textsc{ sg par } \\
\underline{peremme}\,$\oplus$\,hese & \textsc{ sg ill } \\
\underline{peremme}\,$\oplus$\,hez & \textsc{ sg ine } \\
\underline{peremme}\,$\oplus$\,hess & \textsc{ sg ela } \\
\underline{peremme}\,$\oplus$\,helle & \textsc{ sg all } \\
\underline{peremme}\,$\oplus$\,hell & \textsc{ sg ade } \\
\underline{peremme}\,$\oplus$\,helt & \textsc{ sg abl } \\
\underline{peremme}\,$\oplus$\,hessi & \textsc{ sg tra } \\
\underline{peremme}\,$\oplus$\,hessaa & \textsc{ sg ter } \\
\underline{peremme}\,$\oplus$\,heka & \textsc{ sg com } \\
\underline{peremme}\,$\oplus$\,hed & \textsc{ pl nom } \\
\underline{peremme}\,$\oplus$\,hije & \textsc{ pl gen } \\
\underline{peremme}\,$\oplus$\,hiit & \textsc{ pl par } \\
\underline{peremme}\,$\oplus$\,hiise & \textsc{ pl ill } \\
\underline{peremme}\,$\oplus$\,hiiz & \textsc{ pl ine } \\
\underline{peremme}\,$\oplus$\,hiiss & \textsc{ pl ela } \\
\underline{peremme}\,$\oplus$\,hiille & \textsc{ pl all } \\
\underline{peremme}\,$\oplus$\,hiill & \textsc{ pl ade } \\
\underline{peremme}\,$\oplus$\,hiilt & \textsc{ pl abl } \\
\underline{peremme}\,$\oplus$\,hiissi & \textsc{ pl tra } \\
\underline{peremme}\,$\oplus$\,hiissaa & \textsc{ pl ter } \\
\underline{peremme}\,$\oplus$\,hijka & \textsc{ pl com } \\
\end{tabular}
\end{sideways}
\captionof{table}{Tüüpsõnamall \arabic{mallinumber}\,\vadja{peremmeez} ekstraheeritud muutvormimallid.}
\label{tab:tüüpsõnamall-peremmeez}

\end{minipage}

 
\vspace{1em}
\noindent Tüüpsõnamall \vadja{peremmeez} hõlmab vormisõnastikus 2 lekseemi: \vadja{\underline{peremme}ez} ja \vadja{\underline{me}ez}.



\vspace{3.5em}
\noindent \begin{minipage}{\textwidth}
\stepcounter{mallinumber}
\noindent \textbf{Tüüpsõnamall \arabic{mallinumber}\,\vadja{märännü}}\\

\begin{sideways}
\begin{tabular}{l l}
muutvormimall & tunnused \\
\hline
\underline{märänn}\,$\oplus$\,ü & \textsc{ sg nom } \\
\underline{märänn}\,$\oplus$\,ü & \textsc{ sg gen } \\
\underline{märänn}\,$\oplus$\,üt & \textsc{ sg par } \\
\underline{märänn}\,$\oplus$\,üse & \textsc{ sg ill } \\
\underline{märänn}\,$\oplus$\,ez & \textsc{ sg ine } \\
\underline{märänn}\,$\oplus$\,ess & \textsc{ sg ela } \\
\underline{märänn}\,$\oplus$\,elle & \textsc{ sg all } \\
\underline{märänn}\,$\oplus$\,ell & \textsc{ sg ade } \\
\underline{märänn}\,$\oplus$\,elt & \textsc{ sg abl } \\
\underline{märänn}\,$\oplus$\,essi & \textsc{ sg tra } \\
\underline{märänn}\,$\oplus$\,essaa & \textsc{ sg ter } \\
\underline{märänn}\,$\oplus$\,eka & \textsc{ sg com } \\
\underline{märänn}\,$\oplus$\,ed & \textsc{ pl nom } \\
\underline{märänn}\,$\oplus$\,ije & \textsc{ pl gen } \\
\underline{märänn}\,$\oplus$\,eit & \textsc{ pl par } \\
\underline{märänn}\,$\oplus$\,eise & \textsc{ pl ill } \\
\underline{märänn}\,$\oplus$\,eiz & \textsc{ pl ine } \\
\underline{märänn}\,$\oplus$\,eiss & \textsc{ pl ela } \\
\underline{märänn}\,$\oplus$\,eille & \textsc{ pl all } \\
\underline{märänn}\,$\oplus$\,eill & \textsc{ pl ade } \\
\underline{märänn}\,$\oplus$\,eilt & \textsc{ pl abl } \\
\underline{märänn}\,$\oplus$\,eissi & \textsc{ pl tra } \\
\underline{märänn}\,$\oplus$\,eissaa & \textsc{ pl ter } \\
\underline{märänn}\,$\oplus$\,eika & \textsc{ pl com } \\
\end{tabular}
\end{sideways}
\captionof{table}{Tüüpsõnamall \arabic{mallinumber}\,\vadja{märännü} ekstraheeritud muutvormimallid.}
\label{tab:tüüpsõnamall-märännü}

\end{minipage}

 
\vspace{1em}
\noindent Tüüpsõnamall \vadja{märännü} ei hõlma teisi lekseeme vormi\-sõnastikus.



\vspace{3.5em}
\noindent \begin{minipage}{\textwidth}
\stepcounter{mallinumber}
\noindent \textbf{Tüüpsõnamall \arabic{mallinumber}\,\vadja{čeväd}}\\

\begin{sideways}
\begin{tabular}{l l}
muutvormimall & tunnused \\
\hline
\underline{čev}\,$\oplus$\,äd & \textsc{ sg nom } \\
\underline{čev}\,$\oplus$\,vä & \textsc{ sg gen } \\
\underline{čev}\,$\oplus$\,ätte & \textsc{ sg par } \\
\underline{čev}\,$\oplus$\,ättese & \textsc{ sg ill } \\
\underline{čev}\,$\oplus$\,äz & \textsc{ sg ine } \\
\underline{čev}\,$\oplus$\,äss & \textsc{ sg ela } \\
\underline{čev}\,$\oplus$\,älle & \textsc{ sg all } \\
\underline{čev}\,$\oplus$\,äll & \textsc{ sg ade } \\
\underline{čev}\,$\oplus$\,ält & \textsc{ sg abl } \\
\underline{čev}\,$\oplus$\,ässi & \textsc{ sg tra } \\
\underline{čev}\,$\oplus$\,ässaa & \textsc{ sg ter } \\
\underline{čev}\,$\oplus$\,äka & \textsc{ sg com } \\
\underline{čev}\,$\oplus$\,äd & \textsc{ pl nom } \\
\underline{čev}\,$\oplus$\,vije & \textsc{ pl gen } \\
\underline{čev}\,$\oplus$\,viit & \textsc{ pl par } \\
\underline{čev}\,$\oplus$\,viise & \textsc{ pl ill } \\
\underline{čev}\,$\oplus$\,viiz & \textsc{ pl ine } \\
\underline{čev}\,$\oplus$\,viiss & \textsc{ pl ela } \\
\underline{čev}\,$\oplus$\,viille & \textsc{ pl all } \\
\underline{čev}\,$\oplus$\,viill & \textsc{ pl ade } \\
\underline{čev}\,$\oplus$\,viilt & \textsc{ pl abl } \\
\underline{čev}\,$\oplus$\,viissi & \textsc{ pl tra } \\
\underline{čev}\,$\oplus$\,viissaa & \textsc{ pl ter } \\
\underline{čev}\,$\oplus$\,vijka & \textsc{ pl com } \\
\end{tabular}
\end{sideways}
\captionof{table}{Tüüpsõnamall \arabic{mallinumber}\,\vadja{čeväd} ekstraheeritud muutvormimallid.}
\label{tab:tüüpsõnamall-čeväd}

\end{minipage}

 
\vspace{1em}
\noindent Tüüpsõnamall \vadja{čeväd} ei hõlma teisi lekseeme vormi\-sõnastikus.



\vspace{3.5em}
\noindent \begin{minipage}{\textwidth}
\stepcounter{mallinumber}
\noindent \textbf{Tüüpsõnamall \arabic{mallinumber}\,\vadja{čäčüd}}\\

\begin{sideways}
\begin{tabular}{l l}
muutvormimall & tunnused \\
\hline
\underline{čäč}\,$\oplus$\,\underline{ü}\,$\oplus$\,d & \textsc{ sg nom } \\
\underline{čäč}\,$\oplus$\,č\,$\oplus$\,\underline{ü} & \textsc{ sg gen } \\
\underline{čäč}\,$\oplus$\,\underline{ü}\,$\oplus$\,tt & \textsc{ sg par } \\
\underline{čäč}\,$\oplus$\,č\,$\oplus$\,\underline{ü}\,$\oplus$\,se & \textsc{ sg ill } \\
\underline{čäč}\,$\oplus$\,č\,$\oplus$\,\underline{ü}\,$\oplus$\,z & \textsc{ sg ine } \\
\underline{čäč}\,$\oplus$\,č\,$\oplus$\,\underline{ü}\,$\oplus$\,ss & \textsc{ sg ela } \\
\underline{čäč}\,$\oplus$\,č\,$\oplus$\,\underline{ü}\,$\oplus$\,lle & \textsc{ sg all } \\
\underline{čäč}\,$\oplus$\,č\,$\oplus$\,\underline{ü}\,$\oplus$\,ll & \textsc{ sg ade } \\
\underline{čäč}\,$\oplus$\,č\,$\oplus$\,\underline{ü}\,$\oplus$\,lt & \textsc{ sg abl } \\
\underline{čäč}\,$\oplus$\,č\,$\oplus$\,\underline{ü}\,$\oplus$\,ssi & \textsc{ sg tra } \\
\underline{čäč}\,$\oplus$\,č\,$\oplus$\,\underline{ü}\,$\oplus$\,ssaa & \textsc{ sg ter } \\
\underline{čäč}\,$\oplus$\,č\,$\oplus$\,\underline{ü}\,$\oplus$\,ka & \textsc{ sg com } \\
\underline{čäč}\,$\oplus$\,č\,$\oplus$\,\underline{ü}\,$\oplus$\,d & \textsc{ pl nom } \\
\underline{čäč}\,$\oplus$\,č\,$\oplus$\,\underline{ü}\,$\oplus$\,je & \textsc{ pl gen } \\
\underline{čäč}\,$\oplus$\,č\,$\oplus$\,\underline{ü}\,$\oplus$\,it & \textsc{ pl par } \\
\underline{čäč}\,$\oplus$\,č\,$\oplus$\,\underline{ü}\,$\oplus$\,ise & \textsc{ pl ill } \\
\underline{čäč}\,$\oplus$\,č\,$\oplus$\,\underline{ü}\,$\oplus$\,iz & \textsc{ pl ine } \\
\underline{čäč}\,$\oplus$\,č\,$\oplus$\,\underline{ü}\,$\oplus$\,iss & \textsc{ pl ela } \\
\underline{čäč}\,$\oplus$\,č\,$\oplus$\,\underline{ü}\,$\oplus$\,ille & \textsc{ pl all } \\
\underline{čäč}\,$\oplus$\,č\,$\oplus$\,\underline{ü}\,$\oplus$\,ill & \textsc{ pl ade } \\
\underline{čäč}\,$\oplus$\,č\,$\oplus$\,\underline{ü}\,$\oplus$\,ilt & \textsc{ pl abl } \\
\underline{čäč}\,$\oplus$\,č\,$\oplus$\,\underline{ü}\,$\oplus$\,issi & \textsc{ pl tra } \\
\underline{čäč}\,$\oplus$\,č\,$\oplus$\,\underline{ü}\,$\oplus$\,issaa & \textsc{ pl ter } \\
\underline{čäč}\,$\oplus$\,č\,$\oplus$\,\underline{ü}\,$\oplus$\,ika & \textsc{ pl com } \\
\end{tabular}
\end{sideways}
\captionof{table}{Tüüpsõnamall \arabic{mallinumber}\,\vadja{čäčüd} ekstraheeritud muutvormimallid.}
\label{tab:tüüpsõnamall-čäčüd}

\end{minipage}

 
\vspace{1em}
\noindent Tüüpsõnamall \vadja{čäčüd} ei hõlma teisi lekseeme vormi\-sõnastikus.



\vspace{3.5em}
\noindent \begin{minipage}{\textwidth}
\stepcounter{mallinumber}
\noindent \textbf{Tüüpsõnamall \arabic{mallinumber}\,\vadja{ivusõd}}\\

\begin{sideways}
\begin{tabular}{l l}
muutvormimall & tunnused \\
\hline
\underline{ivus}\,$\oplus$\,õd & \textsc{ sg nom } \\
\underline{ivus}\,$\oplus$\,sijõ & \textsc{ sg gen } \\
\underline{ivus}\,$\oplus$\,siit & \textsc{ sg par } \\
\underline{ivus}\,$\oplus$\,sisõ & \textsc{ sg ill } \\
\underline{ivus}\,$\oplus$\,õz & \textsc{ sg ine } \\
\underline{ivus}\,$\oplus$\,õss & \textsc{ sg ela } \\
\underline{ivus}\,$\oplus$\,õllõ & \textsc{ sg all } \\
\underline{ivus}\,$\oplus$\,õll & \textsc{ sg ade } \\
\underline{ivus}\,$\oplus$\,õlt & \textsc{ sg abl } \\
\underline{ivus}\,$\oplus$\,õssi & \textsc{ sg tra } \\
\underline{ivus}\,$\oplus$\,õssaa & \textsc{ sg ter } \\
\underline{ivus}\,$\oplus$\,õka & \textsc{ sg com } \\
\underline{ivus}\,$\oplus$\,õd & \textsc{ pl nom } \\
\underline{ivus}\,$\oplus$\,sijõ & \textsc{ pl gen } \\
\underline{ivus}\,$\oplus$\,siit & \textsc{ pl par } \\
\underline{ivus}\,$\oplus$\,siisõ & \textsc{ pl ill } \\
\underline{ivus}\,$\oplus$\,siiz & \textsc{ pl ine } \\
\underline{ivus}\,$\oplus$\,siiss & \textsc{ pl ela } \\
\underline{ivus}\,$\oplus$\,siillõ & \textsc{ pl all } \\
\underline{ivus}\,$\oplus$\,siill & \textsc{ pl ade } \\
\underline{ivus}\,$\oplus$\,siilt & \textsc{ pl abl } \\
\underline{ivus}\,$\oplus$\,siissi & \textsc{ pl tra } \\
\underline{ivus}\,$\oplus$\,siissaa & \textsc{ pl ter } \\
\underline{ivus}\,$\oplus$\,sijka & \textsc{ pl com } \\
\end{tabular}
\end{sideways}
\captionof{table}{Tüüpsõnamall \arabic{mallinumber}\,\vadja{ivusõd} ekstraheeritud muutvormimallid.}
\label{tab:tüüpsõnamall-ivusõd}

\end{minipage}

 
\vspace{1em}
\noindent Tüüpsõnamall \vadja{ivusõd} ei hõlma teisi lekseeme vormi\-sõnastikus.



\vspace{3.5em}
\noindent \begin{minipage}{\textwidth}
\stepcounter{mallinumber}
\noindent \textbf{Tüüpsõnamall \arabic{mallinumber}\,\vadja{kaatsõd}}\\

\begin{sideways}
\begin{tabular}{l l}
muutvormimall & tunnused \\
\hline
\underline{kaat}\,$\oplus$\,\underline{s}\,$\oplus$\,õd & \textsc{ sg nom } \\
\underline{kaat}\,$\oplus$\,t\,$\oplus$\,\underline{s}\,$\oplus$\,ojõ & \textsc{ sg gen } \\
\underline{kaat}\,$\oplus$\,t\,$\oplus$\,\underline{s}\,$\oplus$\,oit & \textsc{ sg par } \\
\underline{kaat}\,$\oplus$\,t\,$\oplus$\,\underline{s}\,$\oplus$\,oisõ & \textsc{ sg ill } \\
\underline{kaat}\,$\oplus$\,t\,$\oplus$\,\underline{s}\,$\oplus$\,õz & \textsc{ sg ine } \\
\underline{kaat}\,$\oplus$\,\underline{s}\,$\oplus$\,õss & \textsc{ sg ela } \\
\underline{kaat}\,$\oplus$\,\underline{s}\,$\oplus$\,õllõ & \textsc{ sg all } \\
\underline{kaat}\,$\oplus$\,\underline{s}\,$\oplus$\,õll & \textsc{ sg ade } \\
\underline{kaat}\,$\oplus$\,\underline{s}\,$\oplus$\,õlt & \textsc{ sg abl } \\
\underline{kaat}\,$\oplus$\,\underline{s}\,$\oplus$\,õssi & \textsc{ sg tra } \\
\underline{kaat}\,$\oplus$\,t\,$\oplus$\,\underline{s}\,$\oplus$\,õssaa & \textsc{ sg ter } \\
\underline{kaat}\,$\oplus$\,\underline{s}\,$\oplus$\,õka & \textsc{ sg com } \\
\underline{kaat}\,$\oplus$\,\underline{s}\,$\oplus$\,õd & \textsc{ pl nom } \\
\underline{kaat}\,$\oplus$\,\underline{s}\,$\oplus$\,ojõ & \textsc{ pl gen } \\
\underline{kaat}\,$\oplus$\,t\,$\oplus$\,\underline{s}\,$\oplus$\,oit & \textsc{ pl par } \\
\underline{kaat}\,$\oplus$\,t\,$\oplus$\,\underline{s}\,$\oplus$\,oisõ & \textsc{ pl ill } \\
\underline{kaat}\,$\oplus$\,t\,$\oplus$\,\underline{s}\,$\oplus$\,oiz & \textsc{ pl ine } \\
\underline{kaat}\,$\oplus$\,t\,$\oplus$\,\underline{s}\,$\oplus$\,oiss & \textsc{ pl ela } \\
\underline{kaat}\,$\oplus$\,t\,$\oplus$\,\underline{s}\,$\oplus$\,oillõ & \textsc{ pl all } \\
\underline{kaat}\,$\oplus$\,t\,$\oplus$\,\underline{s}\,$\oplus$\,oill & \textsc{ pl ade } \\
\underline{kaat}\,$\oplus$\,t\,$\oplus$\,\underline{s}\,$\oplus$\,oilt & \textsc{ pl abl } \\
\underline{kaat}\,$\oplus$\,t\,$\oplus$\,\underline{s}\,$\oplus$\,oissi & \textsc{ pl tra } \\
\underline{kaat}\,$\oplus$\,t\,$\oplus$\,\underline{s}\,$\oplus$\,oissaa & \textsc{ pl ter } \\
\underline{kaat}\,$\oplus$\,t\,$\oplus$\,\underline{s}\,$\oplus$\,oika & \textsc{ pl com } \\
\end{tabular}
\end{sideways}
\captionof{table}{Tüüpsõnamall \arabic{mallinumber}\,\vadja{kaatsõd} ekstraheeritud muutvormimallid.}
\label{tab:tüüpsõnamall-kaatsõd}

\end{minipage}

 
\vspace{1em}
\noindent Tüüpsõnamall \vadja{kaatsõd} ei hõlma teisi lekseeme vormi\-sõnastikus.



\vspace{3.5em}
\noindent \begin{minipage}{\textwidth}
\stepcounter{mallinumber}
\noindent \textbf{Tüüpsõnamall \arabic{mallinumber}\,\vadja{koollud}}\\

\begin{sideways}
\begin{tabular}{l l}
muutvormimall & tunnused \\
\hline
\underline{kooll}\,$\oplus$\,ud & \textsc{ sg nom } \\
\underline{kooll}\,$\oplus$\,õ & \textsc{ sg gen } \\
\underline{kooll}\,$\oplus$\,uttõ & \textsc{ sg par } \\
\underline{kooll}\,$\oplus$\,õsõ & \textsc{ sg ill } \\
\underline{kooll}\,$\oplus$\,õz & \textsc{ sg ine } \\
\underline{kooll}\,$\oplus$\,õss & \textsc{ sg ela } \\
\underline{kooll}\,$\oplus$\,õllõ & \textsc{ sg all } \\
\underline{kooll}\,$\oplus$\,õll & \textsc{ sg ade } \\
\underline{kooll}\,$\oplus$\,õlt & \textsc{ sg abl } \\
\underline{kooll}\,$\oplus$\,õssi & \textsc{ sg tra } \\
\underline{kooll}\,$\oplus$\,õssaa & \textsc{ sg ter } \\
\underline{kooll}\,$\oplus$\,õka & \textsc{ sg com } \\
\underline{kooll}\,$\oplus$\,õd & \textsc{ pl nom } \\
\underline{kooll}\,$\oplus$\,ujõ & \textsc{ pl gen } \\
\underline{kooll}\,$\oplus$\,uit & \textsc{ pl par } \\
\underline{kooll}\,$\oplus$\,uisõ & \textsc{ pl ill } \\
\underline{kooll}\,$\oplus$\,uiz & \textsc{ pl ine } \\
\underline{kooll}\,$\oplus$\,uiss & \textsc{ pl ela } \\
\underline{kooll}\,$\oplus$\,uillõ & \textsc{ pl all } \\
\underline{kooll}\,$\oplus$\,uill & \textsc{ pl ade } \\
\underline{kooll}\,$\oplus$\,uilt & \textsc{ pl abl } \\
\underline{kooll}\,$\oplus$\,uissi & \textsc{ pl tra } \\
\underline{kooll}\,$\oplus$\,uissaa & \textsc{ pl ter } \\
\underline{kooll}\,$\oplus$\,uika & \textsc{ pl com } \\
\end{tabular}
\end{sideways}
\captionof{table}{Tüüpsõnamall \arabic{mallinumber}\,\vadja{koollud} ekstraheeritud muutvormimallid.}
\label{tab:tüüpsõnamall-koollud}

\end{minipage}

 
\vspace{1em}
\noindent Tüüpsõnamall \vadja{koollud} ei hõlma teisi lekseeme vormi\-sõnastikus.



\vspace{3.5em}
\noindent \begin{minipage}{\textwidth}
\stepcounter{mallinumber}
\noindent \textbf{Tüüpsõnamall \arabic{mallinumber}\,\vadja{koorrõ}}\\

\begin{sideways}
\begin{tabular}{l l}
muutvormimall & tunnused \\
\hline
\underline{koor}\,$\oplus$\,r\,$\oplus$\,\underline{õ} & \textsc{ sg nom } \\
\underline{koor}\,$\oplus$\,r\,$\oplus$\,\underline{õ} & \textsc{ sg gen } \\
\underline{koor}\,$\oplus$\,r\,$\oplus$\,\underline{õ}\,$\oplus$\,t & \textsc{ sg par } \\
\underline{koor}\,$\oplus$\,t\,$\oplus$\,\underline{õ}\,$\oplus$\,sõ & \textsc{ sg ill } \\
\underline{koor}\,$\oplus$\,r\,$\oplus$\,\underline{õ}\,$\oplus$\,z & \textsc{ sg ine } \\
\underline{koor}\,$\oplus$\,r\,$\oplus$\,\underline{õ}\,$\oplus$\,ss & \textsc{ sg ela } \\
\underline{koor}\,$\oplus$\,r\,$\oplus$\,\underline{õ}\,$\oplus$\,llõ & \textsc{ sg all } \\
\underline{koor}\,$\oplus$\,r\,$\oplus$\,\underline{õ}\,$\oplus$\,ll & \textsc{ sg ade } \\
\underline{koor}\,$\oplus$\,r\,$\oplus$\,\underline{õ}\,$\oplus$\,lt & \textsc{ sg abl } \\
\underline{koor}\,$\oplus$\,r\,$\oplus$\,\underline{õ}\,$\oplus$\,ssi & \textsc{ sg tra } \\
\underline{koor}\,$\oplus$\,r\,$\oplus$\,\underline{õ}\,$\oplus$\,ssaa & \textsc{ sg ter } \\
\underline{koor}\,$\oplus$\,r\,$\oplus$\,\underline{õ}\,$\oplus$\,ka & \textsc{ sg com } \\
\underline{koor}\,$\oplus$\,r\,$\oplus$\,\underline{õ}\,$\oplus$\,d & \textsc{ pl nom } \\
\underline{koor}\,$\oplus$\,t\,$\oplus$\,\underline{õ}\,$\oplus$\,jõ & \textsc{ pl gen } \\
\underline{koor}\,$\oplus$\,t\,$\oplus$\,\underline{õ}\,$\oplus$\,it & \textsc{ pl par } \\
\underline{koor}\,$\oplus$\,t\,$\oplus$\,\underline{õ}\,$\oplus$\,isõ & \textsc{ pl ill } \\
\underline{koor}\,$\oplus$\,t\,$\oplus$\,\underline{õ}\,$\oplus$\,iz & \textsc{ pl ine } \\
\underline{koor}\,$\oplus$\,t\,$\oplus$\,\underline{õ}\,$\oplus$\,iss & \textsc{ pl ela } \\
\underline{koor}\,$\oplus$\,t\,$\oplus$\,\underline{õ}\,$\oplus$\,illõ & \textsc{ pl all } \\
\underline{koor}\,$\oplus$\,t\,$\oplus$\,\underline{õ}\,$\oplus$\,ill & \textsc{ pl ade } \\
\underline{koor}\,$\oplus$\,t\,$\oplus$\,\underline{õ}\,$\oplus$\,ilt & \textsc{ pl abl } \\
\underline{koor}\,$\oplus$\,t\,$\oplus$\,\underline{õ}\,$\oplus$\,issi & \textsc{ pl tra } \\
\underline{koor}\,$\oplus$\,t\,$\oplus$\,\underline{õ}\,$\oplus$\,issaa & \textsc{ pl ter } \\
\underline{koor}\,$\oplus$\,t\,$\oplus$\,\underline{õ}\,$\oplus$\,ika & \textsc{ pl com } \\
\end{tabular}
\end{sideways}
\captionof{table}{Tüüpsõnamall \arabic{mallinumber}\,\vadja{koorrõ} ekstraheeritud muutvormimallid.}
\label{tab:tüüpsõnamall-koorrõ}

\end{minipage}

 
\vspace{1em}
\noindent Tüüpsõnamall \vadja{koorrõ} ei hõlma teisi lekseeme vormi\-sõnastikus.



\vspace{3.5em}
\noindent \begin{minipage}{\textwidth}
\stepcounter{mallinumber}
\noindent \textbf{Tüüpsõnamall \arabic{mallinumber}\,\vadja{kõlmõd}}\\

\begin{sideways}
\begin{tabular}{l l}
muutvormimall & tunnused \\
\hline
\underline{kõlm}\,$\oplus$\,õd & \textsc{ sg nom } \\
\underline{kõlm}\,$\oplus$\,õ & \textsc{ sg gen } \\
\underline{kõlm}\,$\oplus$\,a & \textsc{ sg par } \\
\underline{kõlm}\,$\oplus$\,õsõ & \textsc{ sg ill } \\
\underline{kõlm}\,$\oplus$\,õz & \textsc{ sg ine } \\
\underline{kõlm}\,$\oplus$\,õss & \textsc{ sg ela } \\
\underline{kõlm}\,$\oplus$\,õllõ & \textsc{ sg all } \\
\underline{kõlm}\,$\oplus$\,õll & \textsc{ sg ade } \\
\underline{kõlm}\,$\oplus$\,õlt & \textsc{ sg abl } \\
\underline{kõlm}\,$\oplus$\,õssi & \textsc{ sg tra } \\
\underline{kõlm}\,$\oplus$\,õssaa & \textsc{ sg ter } \\
\underline{kõlm}\,$\oplus$\,õka & \textsc{ sg com } \\
\underline{kõlm}\,$\oplus$\,õd & \textsc{ pl nom } \\
\underline{kõlm}\,$\oplus$\,ijõ & \textsc{ pl gen } \\
\underline{kõlm}\,$\oplus$\,iit & \textsc{ pl par } \\
\underline{kõlm}\,$\oplus$\,iisõ & \textsc{ pl ill } \\
\underline{kõlm}\,$\oplus$\,iiz & \textsc{ pl ine } \\
\underline{kõlm}\,$\oplus$\,iiss & \textsc{ pl ela } \\
\underline{kõlm}\,$\oplus$\,iillõ & \textsc{ pl all } \\
\underline{kõlm}\,$\oplus$\,iill & \textsc{ pl ade } \\
\underline{kõlm}\,$\oplus$\,iilt & \textsc{ pl abl } \\
\underline{kõlm}\,$\oplus$\,iissi & \textsc{ pl tra } \\
\underline{kõlm}\,$\oplus$\,iissaa & \textsc{ pl ter } \\
\underline{kõlm}\,$\oplus$\,ijka & \textsc{ pl com } \\
\end{tabular}
\end{sideways}
\captionof{table}{Tüüpsõnamall \arabic{mallinumber}\,\vadja{kõlmõd} ekstraheeritud muutvormimallid.}
\label{tab:tüüpsõnamall-kõlmõd}

\end{minipage}

 
\vspace{1em}
\noindent Tüüpsõnamall \vadja{kõlmõd} ei hõlma teisi lekseeme vormi\-sõnastikus.



\vspace{3.5em}
\noindent \begin{minipage}{\textwidth}
\stepcounter{mallinumber}
\noindent \textbf{Tüüpsõnamall \arabic{mallinumber}\,\vadja{olud}}\\

\begin{sideways}
\begin{tabular}{l l}
muutvormimall & tunnused \\
\hline
\underline{olu}\,$\oplus$\,d & \textsc{ sg nom } \\
\underline{olu} & \textsc{ sg gen } \\
\underline{olu}\,$\oplus$\,ttõ & \textsc{ sg par } \\
\underline{olu}\,$\oplus$\,sõ & \textsc{ sg ill } \\
\underline{olu}\,$\oplus$\,z & \textsc{ sg ine } \\
\underline{olu}\,$\oplus$\,ss & \textsc{ sg ela } \\
\underline{olu}\,$\oplus$\,llõ & \textsc{ sg all } \\
\underline{olu}\,$\oplus$\,ll & \textsc{ sg ade } \\
\underline{olu}\,$\oplus$\,lt & \textsc{ sg abl } \\
\underline{olu}\,$\oplus$\,ssi & \textsc{ sg tra } \\
\underline{olu}\,$\oplus$\,ssaa & \textsc{ sg ter } \\
\underline{olu}\,$\oplus$\,ka & \textsc{ sg com } \\
\underline{olu}\,$\oplus$\,d & \textsc{ pl nom } \\
\underline{olu}\,$\oplus$\,jõ & \textsc{ pl gen } \\
\underline{olu}\,$\oplus$\,it & \textsc{ pl par } \\
\underline{olu}\,$\oplus$\,isõ & \textsc{ pl ill } \\
\underline{olu}\,$\oplus$\,iz & \textsc{ pl ine } \\
\underline{olu}\,$\oplus$\,iss & \textsc{ pl ela } \\
\underline{olu}\,$\oplus$\,illõ & \textsc{ pl all } \\
\underline{olu}\,$\oplus$\,ill & \textsc{ pl ade } \\
\underline{olu}\,$\oplus$\,ilt & \textsc{ pl abl } \\
\underline{olu}\,$\oplus$\,issi & \textsc{ pl tra } \\
\underline{olu}\,$\oplus$\,issaa & \textsc{ pl ter } \\
\underline{olu}\,$\oplus$\,ika & \textsc{ pl com } \\
\end{tabular}
\end{sideways}
\captionof{table}{Tüüpsõnamall \arabic{mallinumber}\,\vadja{olud} ekstraheeritud muutvormimallid.}
\label{tab:tüüpsõnamall-olud}

\end{minipage}

 
\vspace{1em}
\noindent Tüüpsõnamall \vadja{olud} ei hõlma teisi lekseeme vormi\-sõnastikus.



\vspace{3.5em}
\noindent \begin{minipage}{\textwidth}
\stepcounter{mallinumber}
\noindent \textbf{Tüüpsõnamall \arabic{mallinumber}\,\vadja{toho}}\\

\begin{sideways}
\begin{tabular}{l l}
muutvormimall & tunnused \\
\hline
\underline{toh}\,$\oplus$\,o & \textsc{ sg nom } \\
\underline{toh}\,$\oplus$\,o & \textsc{ sg gen } \\
\underline{toh}\,$\oplus$\,tõ & \textsc{ sg par } \\
\underline{toh}\,$\oplus$\,tosõ & \textsc{ sg ill } \\
\underline{toh}\,$\oplus$\,oz & \textsc{ sg ine } \\
\underline{toh}\,$\oplus$\,oss & \textsc{ sg ela } \\
\underline{toh}\,$\oplus$\,ollõ & \textsc{ sg all } \\
\underline{toh}\,$\oplus$\,oll & \textsc{ sg ade } \\
\underline{toh}\,$\oplus$\,olt & \textsc{ sg abl } \\
\underline{toh}\,$\oplus$\,ossi & \textsc{ sg tra } \\
\underline{toh}\,$\oplus$\,ossaa & \textsc{ sg ter } \\
\underline{toh}\,$\oplus$\,oka & \textsc{ sg com } \\
\underline{toh}\,$\oplus$\,od & \textsc{ pl nom } \\
\underline{toh}\,$\oplus$\,ojõ & \textsc{ pl gen } \\
\underline{toh}\,$\oplus$\,oit & \textsc{ pl par } \\
\underline{toh}\,$\oplus$\,oisõ & \textsc{ pl ill } \\
\underline{toh}\,$\oplus$\,oiz & \textsc{ pl ine } \\
\underline{toh}\,$\oplus$\,oiss & \textsc{ pl ela } \\
\underline{toh}\,$\oplus$\,oillõ & \textsc{ pl all } \\
\underline{toh}\,$\oplus$\,oill & \textsc{ pl ade } \\
\underline{toh}\,$\oplus$\,oilt & \textsc{ pl abl } \\
\underline{toh}\,$\oplus$\,oissi & \textsc{ pl tra } \\
\underline{toh}\,$\oplus$\,oissaa & \textsc{ pl ter } \\
\underline{toh}\,$\oplus$\,oika & \textsc{ pl com } \\
\end{tabular}
\end{sideways}
\captionof{table}{Tüüpsõnamall \arabic{mallinumber}\,\vadja{toho} ekstraheeritud muutvormimallid.}
\label{tab:tüüpsõnamall-toho}

\end{minipage}

 
\vspace{1em}
\noindent Tüüpsõnamall \vadja{toho} hõlmab vormisõnastikus 2 lekseemi: \vadja{\underline{toh}o} ja \vadja{\underline{roh}o}.

\spacing{1.5}


\subsection{Klassifitseerimata tüüpsõnad}

Siin esitatakse kõik tüüpsõnad, mille käändkonna klassifitseerimisega on töö autor jäänud hätta.

\spacing{1.1}
\vspace{-3.5em}


\vspace{3.5em}
\noindent \begin{minipage}{\textwidth}
\stepcounter{mallinumber}
\noindent \textbf{Tüüpsõnamall \arabic{mallinumber}\,\vadja{alõin}}\\

\begin{sideways}
\begin{tabular}{l l}
muutvormimall & tunnused \\
\hline
\underline{alõ}\,$\oplus$\,in & \textsc{ sg nom } \\
\underline{alõ}\,$\oplus$\,tsõ & \textsc{ sg gen } \\
\underline{alõ}\,$\oplus$\,issõ & \textsc{ sg par } \\
\underline{alõ}\,$\oplus$\,ttsõ & \textsc{ sg ill } \\
\underline{alõ}\,$\oplus$\,tsõz & \textsc{ sg ine } \\
\underline{alõ}\,$\oplus$\,tsõss & \textsc{ sg ela } \\
\underline{alõ}\,$\oplus$\,tsõllõ & \textsc{ sg all } \\
\underline{alõ}\,$\oplus$\,tsõll & \textsc{ sg ade } \\
\underline{alõ}\,$\oplus$\,tsõlt & \textsc{ sg abl } \\
\underline{alõ}\,$\oplus$\,tsõssi & \textsc{ sg tra } \\
\underline{alõ}\,$\oplus$\,tsõssaa & \textsc{ sg ter } \\
\underline{alõ}\,$\oplus$\,tsõka & \textsc{ sg com } \\
\underline{alõ}\,$\oplus$\,tsõd & \textsc{ pl nom } \\
\underline{alõ}\,$\oplus$\,ttsijõ & \textsc{ pl gen } \\
\underline{alõ}\,$\oplus$\,ttsiit & \textsc{ pl par } \\
\underline{alõ}\,$\oplus$\,ttsiisõ & \textsc{ pl ill } \\
\underline{alõ}\,$\oplus$\,ttsiiz & \textsc{ pl ine } \\
\underline{alõ}\,$\oplus$\,ttsiiss & \textsc{ pl ela } \\
\underline{alõ}\,$\oplus$\,ttsiillõ & \textsc{ pl all } \\
\underline{alõ}\,$\oplus$\,ttsiill & \textsc{ pl ade } \\
\underline{alõ}\,$\oplus$\,ttsiilt & \textsc{ pl abl } \\
\underline{alõ}\,$\oplus$\,ttsiissi & \textsc{ pl tra } \\
\underline{alõ}\,$\oplus$\,ttsiissaa & \textsc{ pl ter } \\
\underline{alõ}\,$\oplus$\,ttsijka & \textsc{ pl com } \\
\end{tabular}
\end{sideways}
\captionof{table}{Tüüpsõnamall \arabic{mallinumber}\,\vadja{alõin} ekstraheeritud muutvormimallid.}
\label{tab:tüüpsõnamall-alõin}

\end{minipage}

 
\vspace{1em}
\noindent Tüüpsõnamall \vadja{alõin} ei hõlma teisi lekseeme vormi\-sõnastikus.



\vspace{3.5em}
\noindent \begin{minipage}{\textwidth}
\stepcounter{mallinumber}
\noindent \textbf{Tüüpsõnamall \arabic{mallinumber}\,\vadja{häülütüs}}\\

\begin{sideways}
\begin{tabular}{l l}
muutvormimall & tunnused \\
\hline
\underline{häülütüs} & \textsc{ sg nom } \\
\underline{häülütüs}\,$\oplus$\,e & \textsc{ sg gen } \\
\underline{häülütüs}\,$\oplus$\,se & \textsc{ sg par } \\
\underline{häülütüs}\,$\oplus$\,see & \textsc{ sg ill } \\
\underline{häülütüs}\,$\oplus$\,ez & \textsc{ sg ine } \\
\underline{häülütüs}\,$\oplus$\,ess & \textsc{ sg ela } \\
\underline{häülütüs}\,$\oplus$\,elle & \textsc{ sg all } \\
\underline{häülütüs}\,$\oplus$\,ell & \textsc{ sg ade } \\
\underline{häülütüs}\,$\oplus$\,elt & \textsc{ sg abl } \\
\underline{häülütüs}\,$\oplus$\,essi & \textsc{ sg tra } \\
\underline{häülütüs}\,$\oplus$\,essaa & \textsc{ sg ter } \\
\underline{häülütüs}\,$\oplus$\,eka & \textsc{ sg com } \\
\underline{häülütüs}\,$\oplus$\,ed & \textsc{ pl nom } \\
\underline{häülütüs}\,$\oplus$\,sije & \textsc{ pl gen } \\
\underline{häülütüs}\,$\oplus$\,siit & \textsc{ pl par } \\
\underline{häülütüs}\,$\oplus$\,siise & \textsc{ pl ill } \\
\underline{häülütüs}\,$\oplus$\,siiz & \textsc{ pl ine } \\
\underline{häülütüs}\,$\oplus$\,siiss & \textsc{ pl ela } \\
\underline{häülütüs}\,$\oplus$\,siille & \textsc{ pl all } \\
\underline{häülütüs}\,$\oplus$\,siill & \textsc{ pl ade } \\
\underline{häülütüs}\,$\oplus$\,siilt & \textsc{ pl abl } \\
\underline{häülütüs}\,$\oplus$\,siissi & \textsc{ pl tra } \\
\underline{häülütüs}\,$\oplus$\,siissaa & \textsc{ pl ter } \\
\underline{häülütüs}\,$\oplus$\,sijka & \textsc{ pl com } \\
\end{tabular}
\end{sideways}
\captionof{table}{Tüüpsõnamall \arabic{mallinumber}\,\vadja{häülütüs} ekstraheeritud muutvormimallid.}
\label{tab:tüüpsõnamall-häülütüs}

\end{minipage}

 
\vspace{1em}
\noindent Tüüpsõnamall \vadja{häülütüs} ei hõlma teisi lekseeme vormi\-sõnastikus.



\vspace{3.5em}
\noindent \begin{minipage}{\textwidth}
\stepcounter{mallinumber}
\noindent \textbf{Tüüpsõnamall \arabic{mallinumber}\,\vadja{jalgõz}}\\

\begin{sideways}
\begin{tabular}{l l}
muutvormimall & tunnused \\
\hline
\underline{jalgõ}\,$\oplus$\,z & \textsc{ sg nom } \\
\underline{jalgõ}\,$\oplus$\,sõ & \textsc{ sg gen } \\
\underline{jalgõ}\,$\oplus$\,ssõ & \textsc{ sg par } \\
\underline{jalgõ}\,$\oplus$\,ssõ & \textsc{ sg ill } \\
\underline{jalgõ}\,$\oplus$\,sõz & \textsc{ sg ine } \\
\underline{jalgõ}\,$\oplus$\,sõss & \textsc{ sg ela } \\
\underline{jalgõ}\,$\oplus$\,sõllõ & \textsc{ sg all } \\
\underline{jalgõ}\,$\oplus$\,sõll & \textsc{ sg ade } \\
\underline{jalgõ}\,$\oplus$\,sõlt & \textsc{ sg abl } \\
\underline{jalgõ}\,$\oplus$\,sõssi & \textsc{ sg tra } \\
\underline{jalgõ}\,$\oplus$\,sõssaa & \textsc{ sg ter } \\
\underline{jalgõ}\,$\oplus$\,sõka & \textsc{ sg com } \\
\underline{jalgõ}\,$\oplus$\,sõd & \textsc{ pl nom } \\
\underline{jalgõ}\,$\oplus$\,ssijõ & \textsc{ pl gen } \\
\underline{jalgõ}\,$\oplus$\,ssiit & \textsc{ pl par } \\
\underline{jalgõ}\,$\oplus$\,ssiisõ & \textsc{ pl ill } \\
\underline{jalgõ}\,$\oplus$\,ssiiz & \textsc{ pl ine } \\
\underline{jalgõ}\,$\oplus$\,ssiiss & \textsc{ pl ela } \\
\underline{jalgõ}\,$\oplus$\,ssiillõ & \textsc{ pl all } \\
\underline{jalgõ}\,$\oplus$\,ssiill & \textsc{ pl ade } \\
\underline{jalgõ}\,$\oplus$\,ssiilt & \textsc{ pl abl } \\
\underline{jalgõ}\,$\oplus$\,ssiissi & \textsc{ pl tra } \\
\underline{jalgõ}\,$\oplus$\,ssiissaa & \textsc{ pl ter } \\
\underline{jalgõ}\,$\oplus$\,ssijka & \textsc{ pl com } \\
\end{tabular}
\end{sideways}
\captionof{table}{Tüüpsõnamall \arabic{mallinumber}\,\vadja{jalgõz} ekstraheeritud muutvormimallid.}
\label{tab:tüüpsõnamall-jalgõz}

\end{minipage}

 
\vspace{1em}
\noindent Tüüpsõnamall \vadja{jalgõz} hõlmab vormisõnastikus 10 lekseemi: \vadja{\underline{jalgõ}z, \underline{kaglu}z, \underline{kavalu}z, \underline{lad̕d̕u}z, \underline{l̕innõ}z, \underline{porotu}z, \underline{raskõu}z, \underline{sõrmu}z, \underline{varõ}z} ja \vadja{\underline{alu}z}.



\vspace{3.5em}
\noindent \begin{minipage}{\textwidth}
\stepcounter{mallinumber}
\noindent \textbf{Tüüpsõnamall \arabic{mallinumber}\,\vadja{kaõ}}\\

\begin{sideways}
\begin{tabular}{l l}
muutvormimall & tunnused \\
\hline
\underline{kaõ} & \textsc{ sg nom } \\
\underline{kaõ} & \textsc{ sg gen } \\
\underline{kaõ}\,$\oplus$\,ttõ & \textsc{ sg par } \\
\underline{kaõ}\,$\oplus$\,sõ & \textsc{ sg ill } \\
\underline{kaõ}\,$\oplus$\,z & \textsc{ sg ine } \\
\underline{kaõ}\,$\oplus$\,ss & \textsc{ sg ela } \\
\underline{kaõ}\,$\oplus$\,llõ & \textsc{ sg all } \\
\underline{kaõ}\,$\oplus$\,ll & \textsc{ sg ade } \\
\underline{kaõ}\,$\oplus$\,lt & \textsc{ sg abl } \\
\underline{kaõ}\,$\oplus$\,ssi & \textsc{ sg tra } \\
\underline{kaõ}\,$\oplus$\,ssaa & \textsc{ sg ter } \\
\underline{kaõ}\,$\oplus$\,ka & \textsc{ sg com } \\
\underline{kaõ}\,$\oplus$\,d & \textsc{ pl nom } \\
\underline{kaõ}\,$\oplus$\,jõ & \textsc{ pl gen } \\
\underline{kaõ}\,$\oplus$\,it & \textsc{ pl par } \\
\underline{kaõ}\,$\oplus$\,isõ & \textsc{ pl ill } \\
\underline{kaõ}\,$\oplus$\,iz & \textsc{ pl ine } \\
\underline{kaõ}\,$\oplus$\,iss & \textsc{ pl ela } \\
\underline{kaõ}\,$\oplus$\,illõ & \textsc{ pl all } \\
\underline{kaõ}\,$\oplus$\,ill & \textsc{ pl ade } \\
\underline{kaõ}\,$\oplus$\,ilt & \textsc{ pl abl } \\
\underline{kaõ}\,$\oplus$\,issi & \textsc{ pl tra } \\
\underline{kaõ}\,$\oplus$\,issaa & \textsc{ pl ter } \\
\underline{kaõ}\,$\oplus$\,ika & \textsc{ pl com } \\
\end{tabular}
\end{sideways}
\captionof{table}{Tüüpsõnamall \arabic{mallinumber}\,\vadja{kaõ} ekstraheeritud muutvormimallid.}
\label{tab:tüüpsõnamall-kaõ}

\end{minipage}

 
\vspace{1em}
\noindent Tüüpsõnamall \vadja{kaõ} ei hõlma teisi lekseeme vormi\-sõnastikus.



\vspace{3.5em}
\noindent \begin{minipage}{\textwidth}
\stepcounter{mallinumber}
\noindent \textbf{Tüüpsõnamall \arabic{mallinumber}\,\vadja{kuto}}\\

\begin{sideways}
\begin{tabular}{l l}
muutvormimall & tunnused \\
\hline
\underline{ku}\,$\oplus$\,\underline{t}\,$\oplus$\,o & \textsc{ sg nom } \\
\underline{ku}\,$\oplus$\,\underline{t}\,$\oplus$\,tõ & \textsc{ sg gen } \\
\underline{ku}\,$\oplus$\,õ\,$\oplus$\,\underline{t}\,$\oplus$\,t & \textsc{ sg par } \\
\underline{ku}\,$\oplus$\,\underline{t}\,$\oplus$\,tõ & \textsc{ sg ill } \\
\underline{ku}\,$\oplus$\,\underline{t}\,$\oplus$\,tõz & \textsc{ sg ine } \\
\underline{ku}\,$\oplus$\,\underline{t}\,$\oplus$\,tõss & \textsc{ sg ela } \\
\underline{ku}\,$\oplus$\,\underline{t}\,$\oplus$\,tõllõ & \textsc{ sg all } \\
\underline{ku}\,$\oplus$\,\underline{t}\,$\oplus$\,tõll & \textsc{ sg ade } \\
\underline{ku}\,$\oplus$\,\underline{t}\,$\oplus$\,tõlt & \textsc{ sg abl } \\
\underline{ku}\,$\oplus$\,\underline{t}\,$\oplus$\,tõssi & \textsc{ sg tra } \\
\underline{ku}\,$\oplus$\,\underline{t}\,$\oplus$\,tõssaa & \textsc{ sg ter } \\
\underline{ku}\,$\oplus$\,\underline{t}\,$\oplus$\,tõka & \textsc{ sg com } \\
\underline{ku}\,$\oplus$\,\underline{t}\,$\oplus$\,tõd & \textsc{ pl nom } \\
\underline{ku}\,$\oplus$\,\underline{t}\,$\oplus$\,tõjõ & \textsc{ pl gen } \\
\underline{ku}\,$\oplus$\,\underline{t}\,$\oplus$\,tõit & \textsc{ pl par } \\
\underline{ku}\,$\oplus$\,\underline{t}\,$\oplus$\,tõisõ & \textsc{ pl ill } \\
\underline{ku}\,$\oplus$\,\underline{t}\,$\oplus$\,tõiz & \textsc{ pl ine } \\
\underline{ku}\,$\oplus$\,\underline{t}\,$\oplus$\,tõiss & \textsc{ pl ela } \\
\underline{ku}\,$\oplus$\,\underline{t}\,$\oplus$\,tõillõ & \textsc{ pl all } \\
\underline{ku}\,$\oplus$\,\underline{t}\,$\oplus$\,tõill & \textsc{ pl ade } \\
\underline{ku}\,$\oplus$\,\underline{t}\,$\oplus$\,tõilt & \textsc{ pl abl } \\
\underline{ku}\,$\oplus$\,\underline{t}\,$\oplus$\,tõissi & \textsc{ pl tra } \\
\underline{ku}\,$\oplus$\,\underline{t}\,$\oplus$\,tõissaa & \textsc{ pl ter } \\
\underline{ku}\,$\oplus$\,\underline{t}\,$\oplus$\,tõika & \textsc{ pl com } \\
\end{tabular}
\end{sideways}
\captionof{table}{Tüüpsõnamall \arabic{mallinumber}\,\vadja{kuto} ekstraheeritud muutvormimallid.}
\label{tab:tüüpsõnamall-kuto}

\end{minipage}

 
\vspace{1em}
\noindent Tüüpsõnamall \vadja{kuto} ei hõlma teisi lekseeme vormi\-sõnastikus.



\vspace{3.5em}
\noindent \begin{minipage}{\textwidth}
\stepcounter{mallinumber}
\noindent \textbf{Tüüpsõnamall \arabic{mallinumber}\,\vadja{lauto}}\\

\begin{sideways}
\begin{tabular}{l l}
muutvormimall & tunnused \\
\hline
\underline{lauto} & \textsc{ sg nom } \\
\underline{lauto} & \textsc{ sg gen } \\
\underline{lauto}\,$\oplus$\,t & \textsc{ sg par } \\
\underline{lauto}\,$\oplus$\,sõ & \textsc{ sg ill } \\
\underline{lauto}\,$\oplus$\,z & \textsc{ sg ine } \\
\underline{lauto}\,$\oplus$\,ss & \textsc{ sg ela } \\
\underline{lauto}\,$\oplus$\,llõ & \textsc{ sg all } \\
\underline{lauto}\,$\oplus$\,ll & \textsc{ sg ade } \\
\underline{lauto}\,$\oplus$\,lt & \textsc{ sg abl } \\
\underline{lauto}\,$\oplus$\,ssi & \textsc{ sg tra } \\
\underline{lauto}\,$\oplus$\,ssaa & \textsc{ sg ter } \\
\underline{lauto}\,$\oplus$\,ka & \textsc{ sg com } \\
\underline{lauto}\,$\oplus$\,d & \textsc{ pl nom } \\
\underline{lauto}\,$\oplus$\,jõ & \textsc{ pl gen } \\
\underline{lauto}\,$\oplus$\,it & \textsc{ pl par } \\
\underline{lauto}\,$\oplus$\,isõ & \textsc{ pl ill } \\
\underline{lauto}\,$\oplus$\,iz & \textsc{ pl ine } \\
\underline{lauto}\,$\oplus$\,iss & \textsc{ pl ela } \\
\underline{lauto}\,$\oplus$\,illõ & \textsc{ pl all } \\
\underline{lauto}\,$\oplus$\,ill & \textsc{ pl ade } \\
\underline{lauto}\,$\oplus$\,ilt & \textsc{ pl abl } \\
\underline{lauto}\,$\oplus$\,issi & \textsc{ pl tra } \\
\underline{lauto}\,$\oplus$\,issaa & \textsc{ pl ter } \\
\underline{lauto}\,$\oplus$\,ika & \textsc{ pl com } \\
\end{tabular}
\end{sideways}
\captionof{table}{Tüüpsõnamall \arabic{mallinumber}\,\vadja{lauto} ekstraheeritud muutvormimallid.}
\label{tab:tüüpsõnamall-lauto}

\end{minipage}

 
\vspace{1em}
\noindent Tüüpsõnamall \vadja{lauto} hõlmab vormisõnastikus 2 lekseemi: \vadja{\underline{lauto}} ja \vadja{\underline{ilka}}.



\vspace{3.5em}
\noindent \begin{minipage}{\textwidth}
\stepcounter{mallinumber}
\noindent \textbf{Tüüpsõnamall \arabic{mallinumber}\,\vadja{mato}}\\

\begin{sideways}
\begin{tabular}{l l}
muutvormimall & tunnused \\
\hline
\underline{ma}\,$\oplus$\,t\,$\oplus$\,\underline{o} & \textsc{ sg nom } \\
\underline{ma}\,$\oplus$\,\underline{o} & \textsc{ sg gen } \\
\underline{ma}\,$\oplus$\,tt\,$\oplus$\,\underline{o}\,$\oplus$\,a & \textsc{ sg par } \\
\underline{ma}\,$\oplus$\,tt\,$\oplus$\,\underline{o}\,$\oplus$\,sõ & \textsc{ sg ill } \\
\underline{ma}\,$\oplus$\,\underline{o}\,$\oplus$\,z & \textsc{ sg ine } \\
\underline{ma}\,$\oplus$\,\underline{o}\,$\oplus$\,ss & \textsc{ sg ela } \\
\underline{ma}\,$\oplus$\,\underline{o}\,$\oplus$\,llõ & \textsc{ sg all } \\
\underline{ma}\,$\oplus$\,\underline{o}\,$\oplus$\,ll & \textsc{ sg ade } \\
\underline{ma}\,$\oplus$\,\underline{o}\,$\oplus$\,lt & \textsc{ sg abl } \\
\underline{ma}\,$\oplus$\,\underline{o}\,$\oplus$\,ssi & \textsc{ sg tra } \\
\underline{ma}\,$\oplus$\,\underline{o}\,$\oplus$\,ssaa & \textsc{ sg ter } \\
\underline{ma}\,$\oplus$\,\underline{o}\,$\oplus$\,ka & \textsc{ sg com } \\
\underline{ma}\,$\oplus$\,\underline{o}\,$\oplus$\,d & \textsc{ pl nom } \\
\underline{ma}\,$\oplus$\,t\,$\oplus$\,\underline{o}\,$\oplus$\,jõ & \textsc{ pl gen } \\
\underline{ma}\,$\oplus$\,t\,$\oplus$\,\underline{o}\,$\oplus$\,it & \textsc{ pl par } \\
\underline{ma}\,$\oplus$\,t\,$\oplus$\,\underline{o}\,$\oplus$\,isõ & \textsc{ pl ill } \\
\underline{ma}\,$\oplus$\,t\,$\oplus$\,\underline{o}\,$\oplus$\,iz & \textsc{ pl ine } \\
\underline{ma}\,$\oplus$\,t\,$\oplus$\,\underline{o}\,$\oplus$\,iss & \textsc{ pl ela } \\
\underline{ma}\,$\oplus$\,t\,$\oplus$\,\underline{o}\,$\oplus$\,illõ & \textsc{ pl all } \\
\underline{ma}\,$\oplus$\,t\,$\oplus$\,\underline{o}\,$\oplus$\,ill & \textsc{ pl ade } \\
\underline{ma}\,$\oplus$\,t\,$\oplus$\,\underline{o}\,$\oplus$\,ilt & \textsc{ pl abl } \\
\underline{ma}\,$\oplus$\,t\,$\oplus$\,\underline{o}\,$\oplus$\,issi & \textsc{ pl tra } \\
\underline{ma}\,$\oplus$\,t\,$\oplus$\,\underline{o}\,$\oplus$\,issaa & \textsc{ pl ter } \\
\underline{ma}\,$\oplus$\,t\,$\oplus$\,\underline{o}\,$\oplus$\,ika & \textsc{ pl com } \\
\end{tabular}
\end{sideways}
\captionof{table}{Tüüpsõnamall \arabic{mallinumber}\,\vadja{mato} ekstraheeritud muutvormimallid.}
\label{tab:tüüpsõnamall-mato}

\end{minipage}

 
\vspace{1em}
\noindent Tüüpsõnamall \vadja{mato} hõlmab vormisõnastikus 5 lekseemi: \vadja{\underline{ma}t\underline{o}, \underline{na}t\underline{o}, \underline{sa}t\underline{o}, \underline{ve}t\underline{o}} ja \vadja{\underline{ko}t\underline{o}}.



\vspace{3.5em}
\noindent \begin{minipage}{\textwidth}
\stepcounter{mallinumber}
\noindent \textbf{Tüüpsõnamall \arabic{mallinumber}\,\vadja{moodnoi}}\\

\begin{sideways}
\begin{tabular}{l l}
muutvormimall & tunnused \\
\hline
\underline{moodno}\,$\oplus$\,i & \textsc{ sg nom } \\
\underline{moodno}\,$\oplus$\,i & \textsc{ sg gen } \\
\underline{moodno}\,$\oplus$\,it & \textsc{ sg par } \\
\underline{moodno}\,$\oplus$\,isõ & \textsc{ sg ill } \\
\underline{moodno}\,$\oplus$\,iz & \textsc{ sg ine } \\
\underline{moodno}\,$\oplus$\,iss & \textsc{ sg ela } \\
\underline{moodno}\,$\oplus$\,illõ & \textsc{ sg all } \\
\underline{moodno}\,$\oplus$\,ill & \textsc{ sg ade } \\
\underline{moodno}\,$\oplus$\,ilt & \textsc{ sg abl } \\
\underline{moodno}\,$\oplus$\,issi & \textsc{ sg tra } \\
\underline{moodno}\,$\oplus$\,issaa & \textsc{ sg ter } \\
\underline{moodno}\,$\oplus$\,ika & \textsc{ sg com } \\
\underline{moodno}\,$\oplus$\,id & \textsc{ pl nom } \\
\underline{moodno}\,$\oplus$\,jõ & \textsc{ pl gen } \\
\underline{moodno}\,$\oplus$\,it & \textsc{ pl par } \\
\underline{moodno}\,$\oplus$\,isõ & \textsc{ pl ill } \\
\underline{moodno}\,$\oplus$\,iz & \textsc{ pl ine } \\
\underline{moodno}\,$\oplus$\,iss & \textsc{ pl ela } \\
\underline{moodno}\,$\oplus$\,illõ & \textsc{ pl all } \\
\underline{moodno}\,$\oplus$\,ill & \textsc{ pl ade } \\
\underline{moodno}\,$\oplus$\,ilt & \textsc{ pl abl } \\
\underline{moodno}\,$\oplus$\,issi & \textsc{ pl tra } \\
\underline{moodno}\,$\oplus$\,issaa & \textsc{ pl ter } \\
\underline{moodno}\,$\oplus$\,ika & \textsc{ pl com } \\
\end{tabular}
\end{sideways}
\captionof{table}{Tüüpsõnamall \arabic{mallinumber}\,\vadja{moodnoi} ekstraheeritud muutvormimallid.}
\label{tab:tüüpsõnamall-moodnoi}

\end{minipage}

 
\vspace{1em}
\noindent Tüüpsõnamall \vadja{moodnoi} hõlmab vormisõnastikus 5 lekseemi: \vadja{\underline{moodno}i, \underline{portno}i, \underline{požarno}i, \underline{prosto}i} ja \vadja{\underline{bodro}i}.



\vspace{3.5em}
\noindent \begin{minipage}{\textwidth}
\stepcounter{mallinumber}
\noindent \textbf{Tüüpsõnamall \arabic{mallinumber}\,\vadja{murhõ}}\\

\begin{sideways}
\begin{tabular}{l l}
muutvormimall & tunnused \\
\hline
\underline{murh}\,$\oplus$\,õ & \textsc{ sg nom } \\
\underline{murh}\,$\oplus$\,õ & \textsc{ sg gen } \\
\underline{murh}\,$\oplus$\,õttõ & \textsc{ sg par } \\
\underline{murh}\,$\oplus$\,õsõ & \textsc{ sg ill } \\
\underline{murh}\,$\oplus$\,õz & \textsc{ sg ine } \\
\underline{murh}\,$\oplus$\,õss & \textsc{ sg ela } \\
\underline{murh}\,$\oplus$\,õllõ & \textsc{ sg all } \\
\underline{murh}\,$\oplus$\,õll & \textsc{ sg ade } \\
\underline{murh}\,$\oplus$\,õlt & \textsc{ sg abl } \\
\underline{murh}\,$\oplus$\,õssi & \textsc{ sg tra } \\
\underline{murh}\,$\oplus$\,õssaa & \textsc{ sg ter } \\
\underline{murh}\,$\oplus$\,õka & \textsc{ sg com } \\
\underline{murh}\,$\oplus$\,õd & \textsc{ pl nom } \\
\underline{murh}\,$\oplus$\,ijõ & \textsc{ pl gen } \\
\underline{murh}\,$\oplus$\,iit & \textsc{ pl par } \\
\underline{murh}\,$\oplus$\,iisõ & \textsc{ pl ill } \\
\underline{murh}\,$\oplus$\,iiz & \textsc{ pl ine } \\
\underline{murh}\,$\oplus$\,iiss & \textsc{ pl ela } \\
\underline{murh}\,$\oplus$\,iillõ & \textsc{ pl all } \\
\underline{murh}\,$\oplus$\,iill & \textsc{ pl ade } \\
\underline{murh}\,$\oplus$\,iilt & \textsc{ pl abl } \\
\underline{murh}\,$\oplus$\,iissi & \textsc{ pl tra } \\
\underline{murh}\,$\oplus$\,iissaa & \textsc{ pl ter } \\
\underline{murh}\,$\oplus$\,ijka & \textsc{ pl com } \\
\end{tabular}
\end{sideways}
\captionof{table}{Tüüpsõnamall \arabic{mallinumber}\,\vadja{murhõ} ekstraheeritud muutvormimallid.}
\label{tab:tüüpsõnamall-murhõ}

\end{minipage}

 
\vspace{1em}
\noindent Tüüpsõnamall \vadja{murhõ} hõlmab vormisõnastikus 4 lekseemi: \vadja{\underline{murh}õ, \underline{mur}õ, \underline{vah}õ} ja \vadja{\underline{jum}õ}.



\vspace{3.5em}
\noindent \begin{minipage}{\textwidth}
\stepcounter{mallinumber}
\noindent \textbf{Tüüpsõnamall \arabic{mallinumber}\,\vadja{mõnikõz}}\\

\begin{sideways}
\begin{tabular}{l l}
muutvormimall & tunnused \\
\hline
\underline{mõnik}\,$\oplus$\,õz & \textsc{ sg nom } \\
\underline{mõnik}\,$\oplus$\,ka & \textsc{ sg gen } \\
\underline{mõnik}\,$\oplus$\,õssõ & \textsc{ sg par } \\
\underline{mõnik}\,$\oplus$\,kasõ & \textsc{ sg ill } \\
\underline{mõnik}\,$\oplus$\,kaz & \textsc{ sg ine } \\
\underline{mõnik}\,$\oplus$\,kass & \textsc{ sg ela } \\
\underline{mõnik}\,$\oplus$\,kallõ & \textsc{ sg all } \\
\underline{mõnik}\,$\oplus$\,kall & \textsc{ sg ade } \\
\underline{mõnik}\,$\oplus$\,kalt & \textsc{ sg abl } \\
\underline{mõnik}\,$\oplus$\,kassi & \textsc{ sg tra } \\
\underline{mõnik}\,$\oplus$\,kassaa & \textsc{ sg ter } \\
\underline{mõnik}\,$\oplus$\,kaka & \textsc{ sg com } \\
\underline{mõnik}\,$\oplus$\,kad & \textsc{ pl nom } \\
\underline{mõnik}\,$\oplus$\,kajõ & \textsc{ pl gen } \\
\underline{mõnik}\,$\oplus$\,kait & \textsc{ pl par } \\
\underline{mõnik}\,$\oplus$\,kaisõ & \textsc{ pl ill } \\
\underline{mõnik}\,$\oplus$\,kaiz & \textsc{ pl ine } \\
\underline{mõnik}\,$\oplus$\,kaiss & \textsc{ pl ela } \\
\underline{mõnik}\,$\oplus$\,kaillõ & \textsc{ pl all } \\
\underline{mõnik}\,$\oplus$\,kaill & \textsc{ pl ade } \\
\underline{mõnik}\,$\oplus$\,kailt & \textsc{ pl abl } \\
\underline{mõnik}\,$\oplus$\,kaissi & \textsc{ pl tra } \\
\underline{mõnik}\,$\oplus$\,kaissaa & \textsc{ pl ter } \\
\underline{mõnik}\,$\oplus$\,kaika & \textsc{ pl com } \\
\end{tabular}
\end{sideways}
\captionof{table}{Tüüpsõnamall \arabic{mallinumber}\,\vadja{mõnikõz} ekstraheeritud muutvormimallid.}
\label{tab:tüüpsõnamall-mõnikõz}

\end{minipage}

 
\vspace{1em}
\noindent Tüüpsõnamall \vadja{mõnikõz} hõlmab vormisõnastikus 4 lekseemi: \vadja{\underline{mõnik}õz, \underline{rahak}õz, \underline{roojõk}õz} ja \vadja{\underline{jevik}õz}.



\vspace{3.5em}
\noindent \begin{minipage}{\textwidth}
\stepcounter{mallinumber}
\noindent \textbf{Tüüpsõnamall \arabic{mallinumber}\,\vadja{nüčüin}}\\

\begin{sideways}
\begin{tabular}{l l}
muutvormimall & tunnused \\
\hline
\underline{nüčüi}\,$\oplus$\,n & \textsc{ sg nom } \\
\underline{nüčüi}\,$\oplus$\,ze & \textsc{ sg gen } \\
\underline{nüčüi}\,$\oplus$\,sse & \textsc{ sg par } \\
\underline{nüčüi}\,$\oplus$\,zese & \textsc{ sg ill } \\
\underline{nüčüi}\,$\oplus$\,zez & \textsc{ sg ine } \\
\underline{nüčüi}\,$\oplus$\,zess & \textsc{ sg ela } \\
\underline{nüčüi}\,$\oplus$\,zelle & \textsc{ sg all } \\
\underline{nüčüi}\,$\oplus$\,zell & \textsc{ sg ade } \\
\underline{nüčüi}\,$\oplus$\,zelt & \textsc{ sg abl } \\
\underline{nüčüi}\,$\oplus$\,zessi & \textsc{ sg tra } \\
\underline{nüčüi}\,$\oplus$\,zessaa & \textsc{ sg ter } \\
\underline{nüčüi}\,$\oplus$\,zeka & \textsc{ sg com } \\
\underline{nüčüi}\,$\oplus$\,zed & \textsc{ pl nom } \\
\underline{nüčüi}\,$\oplus$\,zije & \textsc{ pl gen } \\
\underline{nüčüi}\,$\oplus$\,ziit & \textsc{ pl par } \\
\underline{nüčüi}\,$\oplus$\,ziise & \textsc{ pl ill } \\
\underline{nüčüi}\,$\oplus$\,ziiz & \textsc{ pl ine } \\
\underline{nüčüi}\,$\oplus$\,ziiss & \textsc{ pl ela } \\
\underline{nüčüi}\,$\oplus$\,ziille & \textsc{ pl all } \\
\underline{nüčüi}\,$\oplus$\,ziill & \textsc{ pl ade } \\
\underline{nüčüi}\,$\oplus$\,ziilt & \textsc{ pl abl } \\
\underline{nüčüi}\,$\oplus$\,ziissi & \textsc{ pl tra } \\
\underline{nüčüi}\,$\oplus$\,ziissaa & \textsc{ pl ter } \\
\underline{nüčüi}\,$\oplus$\,zijka & \textsc{ pl com } \\
\end{tabular}
\end{sideways}
\captionof{table}{Tüüpsõnamall \arabic{mallinumber}\,\vadja{nüčüin} ekstraheeritud muutvormimallid.}
\label{tab:tüüpsõnamall-nüčüin}

\end{minipage}

 
\vspace{1em}
\noindent Tüüpsõnamall \vadja{nüčüin} hõlmab vormisõnastikus 2 lekseemi: \vadja{\underline{nüčüi}n} ja \vadja{\underline{čärpei}n}.



\vspace{3.5em}
\noindent \begin{minipage}{\textwidth}
\stepcounter{mallinumber}
\noindent \textbf{Tüüpsõnamall \arabic{mallinumber}\,\vadja{portugaalija}}\\

\begin{sideways}
\begin{tabular}{l l}
muutvormimall & tunnused \\
\hline
\underline{Portugaalij}\,$\oplus$\,a & \textsc{ sg nom } \\
\underline{Portugaalij}\,$\oplus$\,a & \textsc{ sg gen } \\
\underline{Portugaalij}\,$\oplus$\,at & \textsc{ sg par } \\
\underline{Portugaalij}\,$\oplus$\,asõ & \textsc{ sg ill } \\
\underline{Portugaalij}\,$\oplus$\,az & \textsc{ sg ine } \\
\underline{Portugaalij}\,$\oplus$\,ass & \textsc{ sg ela } \\
\underline{Portugaalij}\,$\oplus$\,allõ & \textsc{ sg all } \\
\underline{Portugaalij}\,$\oplus$\,all & \textsc{ sg ade } \\
\underline{Portugaalij}\,$\oplus$\,alt & \textsc{ sg abl } \\
\underline{Portugaalij}\,$\oplus$\,assi & \textsc{ sg tra } \\
\underline{Portugaalij}\,$\oplus$\,assaa & \textsc{ sg ter } \\
\underline{Portugaalij}\,$\oplus$\,aka & \textsc{ sg com } \\
\underline{Portugaalij}\,$\oplus$\,ad & \textsc{ pl nom } \\
\underline{Portugaalij}\,$\oplus$\,ojõ & \textsc{ pl gen } \\
\underline{Portugaalij}\,$\oplus$\,oit & \textsc{ pl par } \\
\underline{Portugaalij}\,$\oplus$\,oisõ & \textsc{ pl ill } \\
\underline{Portugaalij}\,$\oplus$\,oiz & \textsc{ pl ine } \\
\underline{Portugaalij}\,$\oplus$\,oiss & \textsc{ pl ela } \\
\underline{Portugaalij}\,$\oplus$\,oillõ & \textsc{ pl all } \\
\underline{Portugaalij}\,$\oplus$\,oill & \textsc{ pl ade } \\
\underline{Portugaalij}\,$\oplus$\,oilt & \textsc{ pl abl } \\
\underline{Portugaalij}\,$\oplus$\,oissi & \textsc{ pl tra } \\
\underline{Portugaalij}\,$\oplus$\,oissaa & \textsc{ pl ter } \\
\underline{Portugaalij}\,$\oplus$\,oika & \textsc{ pl com } \\
\end{tabular}
\end{sideways}
\captionof{table}{Tüüpsõnamall \arabic{mallinumber}\,\vadja{portugaalija} ekstraheeritud muutvormimallid.}
\label{tab:tüüpsõnamall-portugaalija}

\end{minipage}

 
\vspace{1em}
\noindent Tüüpsõnamall \vadja{portugaalija} ei hõlma teisi lekseeme vormi\-sõnastikus.



\vspace{3.5em}
\noindent \begin{minipage}{\textwidth}
\stepcounter{mallinumber}
\noindent \textbf{Tüüpsõnamall \arabic{mallinumber}\,\vadja{püütö}}\\

\begin{sideways}
\begin{tabular}{l l}
muutvormimall & tunnused \\
\hline
\underline{püü}\,$\oplus$\,t\,$\oplus$\,\underline{ö} & \textsc{ sg nom } \\
\underline{püü}\,$\oplus$\,vv\,$\oplus$\,\underline{ö} & \textsc{ sg gen } \\
\underline{püü}\,$\oplus$\,t\,$\oplus$\,\underline{ö}\,$\oplus$\,ä & \textsc{ sg par } \\
\underline{püü}\,$\oplus$\,t\,$\oplus$\,\underline{ö}\,$\oplus$\,se & \textsc{ sg ill } \\
\underline{püü}\,$\oplus$\,vv\,$\oplus$\,\underline{ö}\,$\oplus$\,z & \textsc{ sg ine } \\
\underline{püü}\,$\oplus$\,vv\,$\oplus$\,\underline{ö}\,$\oplus$\,ss & \textsc{ sg ela } \\
\underline{püü}\,$\oplus$\,vv\,$\oplus$\,\underline{ö}\,$\oplus$\,lle & \textsc{ sg all } \\
\underline{püü}\,$\oplus$\,vv\,$\oplus$\,\underline{ö}\,$\oplus$\,ll & \textsc{ sg ade } \\
\underline{püü}\,$\oplus$\,vv\,$\oplus$\,\underline{ö}\,$\oplus$\,lt & \textsc{ sg abl } \\
\underline{püü}\,$\oplus$\,vv\,$\oplus$\,\underline{ö}\,$\oplus$\,ssi & \textsc{ sg tra } \\
\underline{püü}\,$\oplus$\,vv\,$\oplus$\,\underline{ö}\,$\oplus$\,ssaa & \textsc{ sg ter } \\
\underline{püü}\,$\oplus$\,vv\,$\oplus$\,\underline{ö}\,$\oplus$\,ka & \textsc{ sg com } \\
\underline{püü}\,$\oplus$\,vv\,$\oplus$\,\underline{ö}\,$\oplus$\,d & \textsc{ pl nom } \\
\underline{püü}\,$\oplus$\,t\,$\oplus$\,\underline{ö}\,$\oplus$\,je & \textsc{ pl gen } \\
\underline{püü}\,$\oplus$\,t\,$\oplus$\,\underline{ö}\,$\oplus$\,it & \textsc{ pl par } \\
\underline{püü}\,$\oplus$\,t\,$\oplus$\,\underline{ö}\,$\oplus$\,ise & \textsc{ pl ill } \\
\underline{püü}\,$\oplus$\,t\,$\oplus$\,\underline{ö}\,$\oplus$\,iz & \textsc{ pl ine } \\
\underline{püü}\,$\oplus$\,t\,$\oplus$\,\underline{ö}\,$\oplus$\,iss & \textsc{ pl ela } \\
\underline{püü}\,$\oplus$\,t\,$\oplus$\,\underline{ö}\,$\oplus$\,ille & \textsc{ pl all } \\
\underline{püü}\,$\oplus$\,t\,$\oplus$\,\underline{ö}\,$\oplus$\,ill & \textsc{ pl ade } \\
\underline{püü}\,$\oplus$\,t\,$\oplus$\,\underline{ö}\,$\oplus$\,ilt & \textsc{ pl abl } \\
\underline{püü}\,$\oplus$\,t\,$\oplus$\,\underline{ö}\,$\oplus$\,issi & \textsc{ pl tra } \\
\underline{püü}\,$\oplus$\,t\,$\oplus$\,\underline{ö}\,$\oplus$\,issaa & \textsc{ pl ter } \\
\underline{püü}\,$\oplus$\,t\,$\oplus$\,\underline{ö}\,$\oplus$\,ika & \textsc{ pl com } \\
\end{tabular}
\end{sideways}
\captionof{table}{Tüüpsõnamall \arabic{mallinumber}\,\vadja{püütö} ekstraheeritud muutvormimallid.}
\label{tab:tüüpsõnamall-püütö}

\end{minipage}

 
\vspace{1em}
\noindent Tüüpsõnamall \vadja{püütö} ei hõlma teisi lekseeme vormi\-sõnastikus.



\vspace{3.5em}
\noindent \begin{minipage}{\textwidth}
\stepcounter{mallinumber}
\noindent \textbf{Tüüpsõnamall \arabic{mallinumber}\,\vadja{raadio}}\\

\begin{sideways}
\begin{tabular}{l l}
muutvormimall & tunnused \\
\hline
\underline{raadio} & \textsc{ sg nom } \\
\underline{raadio} & \textsc{ sg gen } \\
\underline{raadio}\,$\oplus$\,ta & \textsc{ sg par } \\
\underline{raadio}\,$\oplus$\,sõ & \textsc{ sg ill } \\
\underline{raadio}\,$\oplus$\,z & \textsc{ sg ine } \\
\underline{raadio}\,$\oplus$\,ss & \textsc{ sg ela } \\
\underline{raadio}\,$\oplus$\,llõ & \textsc{ sg all } \\
\underline{raadio}\,$\oplus$\,ll & \textsc{ sg ade } \\
\underline{raadio}\,$\oplus$\,lt & \textsc{ sg abl } \\
\underline{raadio}\,$\oplus$\,ssi & \textsc{ sg tra } \\
\underline{raadio}\,$\oplus$\,ssaa & \textsc{ sg ter } \\
\underline{raadio}\,$\oplus$\,ka & \textsc{ sg com } \\
\underline{raadio}\,$\oplus$\,d & \textsc{ pl nom } \\
\underline{raadio}\,$\oplus$\,jõ & \textsc{ pl gen } \\
\underline{raadio}\,$\oplus$\,it & \textsc{ pl par } \\
\underline{raadio}\,$\oplus$\,isõ & \textsc{ pl ill } \\
\underline{raadio}\,$\oplus$\,iz & \textsc{ pl ine } \\
\underline{raadio}\,$\oplus$\,iss & \textsc{ pl ela } \\
\underline{raadio}\,$\oplus$\,illõ & \textsc{ pl all } \\
\underline{raadio}\,$\oplus$\,ill & \textsc{ pl ade } \\
\underline{raadio}\,$\oplus$\,ilt & \textsc{ pl abl } \\
\underline{raadio}\,$\oplus$\,issi & \textsc{ pl tra } \\
\underline{raadio}\,$\oplus$\,issaa & \textsc{ pl ter } \\
\underline{raadio}\,$\oplus$\,ika & \textsc{ pl com } \\
\end{tabular}
\end{sideways}
\captionof{table}{Tüüpsõnamall \arabic{mallinumber}\,\vadja{raadio} ekstraheeritud muutvormimallid.}
\label{tab:tüüpsõnamall-raadio}

\end{minipage}

 
\vspace{1em}
\noindent Tüüpsõnamall \vadja{raadio} ei hõlma teisi lekseeme vormi\-sõnastikus.



\vspace{3.5em}
\noindent \begin{minipage}{\textwidth}
\stepcounter{mallinumber}
\noindent \textbf{Tüüpsõnamall \arabic{mallinumber}\,\vadja{tarkuz}}\\

\begin{sideways}
\begin{tabular}{l l}
muutvormimall & tunnused \\
\hline
\underline{tarku}\,$\oplus$\,z & \textsc{ sg nom } \\
\underline{tarku}\,$\oplus$\,sõ & \textsc{ sg gen } \\
\underline{tarku}\,$\oplus$\,ssõ & \textsc{ sg par } \\
\underline{tarku}\,$\oplus$\,ssõsõ & \textsc{ sg ill } \\
\underline{tarku}\,$\oplus$\,sõz & \textsc{ sg ine } \\
\underline{tarku}\,$\oplus$\,sõss & \textsc{ sg ela } \\
\underline{tarku}\,$\oplus$\,sõllõ & \textsc{ sg all } \\
\underline{tarku}\,$\oplus$\,sõll & \textsc{ sg ade } \\
\underline{tarku}\,$\oplus$\,sõlt & \textsc{ sg abl } \\
\underline{tarku}\,$\oplus$\,sõssi & \textsc{ sg tra } \\
\underline{tarku}\,$\oplus$\,sõssaa & \textsc{ sg ter } \\
\underline{tarku}\,$\oplus$\,sõka & \textsc{ sg com } \\
\underline{tarku}\,$\oplus$\,sõd & \textsc{ pl nom } \\
\underline{tarku}\,$\oplus$\,ssijõ & \textsc{ pl gen } \\
\underline{tarku}\,$\oplus$\,ssiit & \textsc{ pl par } \\
\underline{tarku}\,$\oplus$\,ssiisõ & \textsc{ pl ill } \\
\underline{tarku}\,$\oplus$\,ssiiz & \textsc{ pl ine } \\
\underline{tarku}\,$\oplus$\,ssiiss & \textsc{ pl ela } \\
\underline{tarku}\,$\oplus$\,ssiillõ & \textsc{ pl all } \\
\underline{tarku}\,$\oplus$\,ssiill & \textsc{ pl ade } \\
\underline{tarku}\,$\oplus$\,ssiilt & \textsc{ pl abl } \\
\underline{tarku}\,$\oplus$\,ssiissi & \textsc{ pl tra } \\
\underline{tarku}\,$\oplus$\,ssiissaa & \textsc{ pl ter } \\
\underline{tarku}\,$\oplus$\,ssijka & \textsc{ pl com } \\
\end{tabular}
\end{sideways}
\captionof{table}{Tüüpsõnamall \arabic{mallinumber}\,\vadja{tarkuz} ekstraheeritud muutvormimallid.}
\label{tab:tüüpsõnamall-tarkuz}

\end{minipage}

 
\vspace{1em}
\noindent Tüüpsõnamall \vadja{tarkuz} hõlmab vormisõnastikus 2 lekseemi: \vadja{\underline{tarku}z} ja \vadja{\underline{ivu}z}.



\vspace{3.5em}
\noindent \begin{minipage}{\textwidth}
\stepcounter{mallinumber}
\noindent \textbf{Tüüpsõnamall \arabic{mallinumber}\,\vadja{ugurittsõ}}\\

\begin{sideways}
\begin{tabular}{l l}
muutvormimall & tunnused \\
\hline
\underline{ugurit}\,$\oplus$\,t\,$\oplus$\,\underline{s}\,$\oplus$\,õ & \textsc{ sg nom } \\
\underline{ugurit}\,$\oplus$\,\underline{s}\,$\oplus$\,a & \textsc{ sg gen } \\
\underline{ugurit}\,$\oplus$\,t\,$\oplus$\,\underline{s}\,$\oplus$\,a & \textsc{ sg par } \\
\underline{ugurit}\,$\oplus$\,t\,$\oplus$\,\underline{s}\,$\oplus$\,asõ & \textsc{ sg ill } \\
\underline{ugurit}\,$\oplus$\,t\,$\oplus$\,\underline{s}\,$\oplus$\,az & \textsc{ sg ine } \\
\underline{ugurit}\,$\oplus$\,\underline{s}\,$\oplus$\,ass & \textsc{ sg ela } \\
\underline{ugurit}\,$\oplus$\,\underline{s}\,$\oplus$\,allõ & \textsc{ sg all } \\
\underline{ugurit}\,$\oplus$\,\underline{s}\,$\oplus$\,all & \textsc{ sg ade } \\
\underline{ugurit}\,$\oplus$\,\underline{s}\,$\oplus$\,alt & \textsc{ sg abl } \\
\underline{ugurit}\,$\oplus$\,\underline{s}\,$\oplus$\,assi & \textsc{ sg tra } \\
\underline{ugurit}\,$\oplus$\,t\,$\oplus$\,\underline{s}\,$\oplus$\,assaa & \textsc{ sg ter } \\
\underline{ugurit}\,$\oplus$\,\underline{s}\,$\oplus$\,aka & \textsc{ sg com } \\
\underline{ugurit}\,$\oplus$\,\underline{s}\,$\oplus$\,ad & \textsc{ pl nom } \\
\underline{ugurit}\,$\oplus$\,t\,$\oplus$\,\underline{s}\,$\oplus$\,ojõ & \textsc{ pl gen } \\
\underline{ugurit}\,$\oplus$\,t\,$\oplus$\,\underline{s}\,$\oplus$\,oit & \textsc{ pl par } \\
\underline{ugurit}\,$\oplus$\,t\,$\oplus$\,\underline{s}\,$\oplus$\,oisõ & \textsc{ pl ill } \\
\underline{ugurit}\,$\oplus$\,t\,$\oplus$\,\underline{s}\,$\oplus$\,oiz & \textsc{ pl ine } \\
\underline{ugurit}\,$\oplus$\,t\,$\oplus$\,\underline{s}\,$\oplus$\,oiss & \textsc{ pl ela } \\
\underline{ugurit}\,$\oplus$\,t\,$\oplus$\,\underline{s}\,$\oplus$\,oillõ & \textsc{ pl all } \\
\underline{ugurit}\,$\oplus$\,t\,$\oplus$\,\underline{s}\,$\oplus$\,oill & \textsc{ pl ade } \\
\underline{ugurit}\,$\oplus$\,t\,$\oplus$\,\underline{s}\,$\oplus$\,oilt & \textsc{ pl abl } \\
\underline{ugurit}\,$\oplus$\,t\,$\oplus$\,\underline{s}\,$\oplus$\,oissi & \textsc{ pl tra } \\
\underline{ugurit}\,$\oplus$\,t\,$\oplus$\,\underline{s}\,$\oplus$\,oissaa & \textsc{ pl ter } \\
\underline{ugurit}\,$\oplus$\,t\,$\oplus$\,\underline{s}\,$\oplus$\,oika & \textsc{ pl com } \\
\end{tabular}
\end{sideways}
\captionof{table}{Tüüpsõnamall \arabic{mallinumber}\,\vadja{ugurittsõ} ekstraheeritud muutvormimallid.}
\label{tab:tüüpsõnamall-ugurittsõ}

\end{minipage}

 
\vspace{1em}
\noindent Tüüpsõnamall \vadja{ugurittsõ} ei hõlma teisi lekseeme vormi\-sõnastikus.



\vspace{3.5em}
\noindent \begin{minipage}{\textwidth}
\stepcounter{mallinumber}
\noindent \textbf{Tüüpsõnamall \arabic{mallinumber}\,\vadja{vaka}}\\

\begin{sideways}
\begin{tabular}{l l}
muutvormimall & tunnused \\
\hline
\underline{va}\,$\oplus$\,k\,$\oplus$\,\underline{a} & \textsc{ sg nom } \\
\underline{va}\,$\oplus$\,g\,$\oplus$\,\underline{a} & \textsc{ sg gen } \\
\underline{va}\,$\oplus$\,k\,$\oplus$\,\underline{a}\,$\oplus$\,t & \textsc{ sg par } \\
\underline{va}\,$\oplus$\,k\,$\oplus$\,\underline{a}\,$\oplus$\,sõ & \textsc{ sg ill } \\
\underline{va}\,$\oplus$\,g\,$\oplus$\,\underline{a}\,$\oplus$\,z & \textsc{ sg ine } \\
\underline{va}\,$\oplus$\,g\,$\oplus$\,\underline{a}\,$\oplus$\,ss & \textsc{ sg ela } \\
\underline{va}\,$\oplus$\,g\,$\oplus$\,\underline{a}\,$\oplus$\,llõ & \textsc{ sg all } \\
\underline{va}\,$\oplus$\,g\,$\oplus$\,\underline{a}\,$\oplus$\,ll & \textsc{ sg ade } \\
\underline{va}\,$\oplus$\,g\,$\oplus$\,\underline{a}\,$\oplus$\,lt & \textsc{ sg abl } \\
\underline{va}\,$\oplus$\,g\,$\oplus$\,\underline{a}\,$\oplus$\,ssi & \textsc{ sg tra } \\
\underline{va}\,$\oplus$\,g\,$\oplus$\,\underline{a}\,$\oplus$\,ssaa & \textsc{ sg ter } \\
\underline{va}\,$\oplus$\,g\,$\oplus$\,\underline{a}\,$\oplus$\,ka & \textsc{ sg com } \\
\underline{va}\,$\oplus$\,g\,$\oplus$\,\underline{a}\,$\oplus$\,d & \textsc{ pl nom } \\
\underline{va}\,$\oplus$\,k\,$\oplus$\,\underline{a}\,$\oplus$\,jõ & \textsc{ pl gen } \\
\underline{va}\,$\oplus$\,k\,$\oplus$\,\underline{a}\,$\oplus$\,it & \textsc{ pl par } \\
\underline{va}\,$\oplus$\,k\,$\oplus$\,\underline{a}\,$\oplus$\,isõ & \textsc{ pl ill } \\
\underline{va}\,$\oplus$\,k\,$\oplus$\,\underline{a}\,$\oplus$\,iz & \textsc{ pl ine } \\
\underline{va}\,$\oplus$\,k\,$\oplus$\,\underline{a}\,$\oplus$\,iss & \textsc{ pl ela } \\
\underline{va}\,$\oplus$\,k\,$\oplus$\,\underline{a}\,$\oplus$\,illõ & \textsc{ pl all } \\
\underline{va}\,$\oplus$\,k\,$\oplus$\,\underline{a}\,$\oplus$\,ill & \textsc{ pl ade } \\
\underline{va}\,$\oplus$\,k\,$\oplus$\,\underline{a}\,$\oplus$\,ilt & \textsc{ pl abl } \\
\underline{va}\,$\oplus$\,k\,$\oplus$\,\underline{a}\,$\oplus$\,issi & \textsc{ pl tra } \\
\underline{va}\,$\oplus$\,k\,$\oplus$\,\underline{a}\,$\oplus$\,issaa & \textsc{ pl ter } \\
\underline{va}\,$\oplus$\,k\,$\oplus$\,\underline{a}\,$\oplus$\,ika & \textsc{ pl com } \\
\end{tabular}
\end{sideways}
\captionof{table}{Tüüpsõnamall \arabic{mallinumber}\,\vadja{vaka} ekstraheeritud muutvormimallid.}
\label{tab:tüüpsõnamall-vaka}

\end{minipage}

 
\vspace{1em}
\noindent Tüüpsõnamall \vadja{vaka} ei hõlma teisi lekseeme vormi\-sõnastikus.

\spacing{1.5}


% \subsection{Ekstraktmorfoloogiaga leitud tüüpsõnad}
% 
% See alaosa loendab leitud tüüpsõnad sõnaliigiti. Analüüsitakse tüüpsõnade alla kuuluvaid sõnu struktuurselt (kui mitu silpi, silpide struktuur).
% 
% (Analüüsidest on võimalik luua arvutikirjeldus hüpoteetilise vadjakeelse sõna üle õigekirjakontrollija jaoks.)
% 
% \paragraph*{bad̕d̕õ} kuulub käändkonda \RN{3} sest \msd{pl} -oi-
% \paragraph*{airo} kuulub käändkonda \RN{2} sest \msd{sg par} lõpp on -oa või -ua
% \paragraph*{bagaži} kuulub käändkonda \RN{2} ja \msd{pl} on -ii-
% \paragraph*{bank} kuulub käändkonda \RN{3} kuigi esimeses silbis esineb ka \vadja{u} peale \vadja{a, õ ja i} NB! \vadja{bank} ehk ei kuulugi siia või on paha nimetaja
% \paragraph*{čimolain} kuulub käändkonda \RN{12}
% \paragraph*{fartukkõ} kuulub käändkonda \RN{3} sest \msd{pl} -oi-
% \paragraph*{duumõ} kuulub käändkonda \RN{5} aga mida teha \msd{pl} -ii- (Aristel on -õi-)
% \paragraph*{flakku} kuulub käändkonda \RN{2} sest \msd{sg par} lõpp on -oa või -ua
% \paragraph*{baldõhina} kuulub käändkonda \RN{3} aga \vadja{suma} ei peaks siin olema?
% \paragraph*{bankrutti} kuulub käändkonda \RN{2}
% \paragraph*{eine} kuulub käändkonda \RN{8} -ä-tüvevokaaliga NB! aga Aristel \vadja{eine} hoopis \RN{4}, kus \msd{pl} on tagapoolne -oi-
% \paragraph*{greebeni} kuulub käändkonda \RN{2}
% \paragraph*{ivuz} kuulub käändkonda \RN{11} ja Jõgõperä moodi on vokaalitüvel -s-, mitte -hs- ega -ss-
% \paragraph*{aikõ} kuulub käändkonda \RN{3} 
% \paragraph*{hattu} kuulub käändkonda \RN{2}
% \paragraph*{sarvi} kuulub käändkonda \RN{7} aga -i:-õ:-ia, Aristel -i:-õ:-õa
% \paragraph*{alku} kuulub käändkonda \RN{2}
% \paragraph*{koivuin} kuulub käändkonda \RN{12}
% \paragraph*{lako} kuulub käändkonda \RN{2}
% \paragraph*{hoolitoi} kuulub käändkonda \RN{13}
% \paragraph*{irvi} kuulub käändkonda \RN{7}
% \paragraph*{kokki} kuulub käändkonda \RN{2} aga \msd{pl} on -ii-?
% \paragraph*{juuri} kuulub käändkonda \RN{10}
% \paragraph*{kraaskõ} kuulub käändkonda \RN{3}
% \paragraph*{jaanikukkõ} kuulub käändkonda \RN{5} aga mida teha -ii- Aristel on -õi-
% \paragraph*{iiri} kuulub käändkonda \RN{10}
% \paragraph*{eglin} kuulub käändkonda \RN{12}
% \paragraph*{liivõkõz} kuulub käändkonda \RN{14}(?)
% \paragraph*{lamppi} kuulub käändkonda \RN{2}
% \paragraph*{mato} kuulub käändkonda \RN{2} (või \RN{9}?)
% \paragraph*{kittsi} kuulub käändkonda \RN{2}
% \paragraph*{inostranttsõ} kuulub käändkonda \RN{3} NB! \msd{sg ill} valesti -sasõ?
% \paragraph*{karjušši} kuulub käändkonda \RN{2}
% \paragraph*{ikolookkõ} kuulub käändkonda \RN{5}
% \paragraph*{kanka} kuulub käändkonda \RN{6} ehk -\vadja{õa}-lõpulised
% \paragraph*{jäin} kuulub käändkonda \RN{12}
% \paragraph*{kompjutera} kuulub käändkonda \RN{5} kas ühtlustada -õ:-a:-a
% \paragraph*{murhõ} kuulub käändkonda \RN{10}
% \paragraph*{lento} kuulub käändkonda \RN{2}
% \paragraph*{lippu} kuulub käändkonda \RN{2}
% \paragraph*{riittõ} kuulub käändkonda \RN{3}
% \paragraph*{rusko} kuulub käändkonda \RN{2}
% \paragraph*{tüü} kuulub käändkonda \RN{1}
% \paragraph*{pal̕l̕õz} kuulub käändkonda \RN{14}
% \paragraph*{hammõz} kuulub käändkonda \RN{14}
% \paragraph*{pää} kuulub käändkonda \RN{1}
% \paragraph*{magnetti} kuulub käändkonda \RN{2}
% \paragraph*{mõiznikkõ} kuulub käändkonda \RN{3}
% \paragraph*{pere} kuulub käändkonda \RN{14} või \RN{6}(?) (\vadja{erne} ja \vadja{pere} kindlasti eri etümoloogiatega)
% \paragraph*{valka} kuulub käändkonda \RN{6}
% \paragraph*{kotko} kuulub käändkonda \RN{2}
% \paragraph*{propkõ} kuulub käändkonda \RN{5}
% \paragraph*{pehmiä} kuulub käändkonda \RN{6}
% \paragraph*{pihlpuu} kuulub käändkonda \RN{1}
% \paragraph*{meri} kuulub käändkonda \RN{10} NB! ühtlustada \msd{sg par} vokaaliga
% \paragraph*{süsi} kuulub käändkonda \RN{7}
% \paragraph*{sata} kuulub käändkonda \RN{3}
% \paragraph*{rapa} kuulub käändkonda \RN{3}
% \paragraph*{rätte} kuulub käändkonda \RN{2}
% \paragraph*{rantõ} kuulub käändkonda \RN{3}
% \paragraph*{trubõ} kuulub käändkonda \RN{3}
% \paragraph*{l̕iitkõ} kuulub käändkonda \RN{3}
% \paragraph*{rissisä, česä} kuulub käändkonda \RN{11} NB! \vadja{rissisä} teha \vadja{isä} järgi
% \paragraph*{musikõz} kuulub käändkonda \RN{14} või peaks ümber tegema \RN{15} järgi?
% \paragraph*{nenä, čülä} kuulub käändkonda \RN{8}
% \paragraph*{läkine} kuulub käändkonda \RN{8}
% \paragraph*{rissimä, emä} kuulub käändkonda \RN{8}
% \paragraph*{pliittõ} kuulub käändkonda \RN{3}
% \paragraph*{kaamenšikka} kuulub käändkonda \RN{3}
% \paragraph*{pähčen, ičäv} kuulub käändkonda \RN{8} ???
% \paragraph*{ülči, jälči} kuulub käändkonda \RN{2}
% \paragraph*{vetelüz, jänez} kuulub käändkonda \RN{14}
% \paragraph*{löülü, jürü} kuulub käändkonda \RN{2}
% \paragraph*{pen̕sioner} kuulub käändkonda \RN{3} (mil moel erineb \vadja{bank}ast?)
% \paragraph*{mussõ} kuulub käändkonda \RN{5} kuigi on -õi- mitte -ii-
% \paragraph*{liippõ} kuulub käändkonda \RN{3}
% \paragraph*{koomikk} kuulub käändkonda \RN{3} mis \msd{sg nom} lõpuga teha?
% \paragraph*{või} kuulub käändkonda \RN{1}
% \paragraph*{passi} kuulub käändkonda \RN{5}
% \paragraph*{moškõ} kuulub käändkonda \RN{5}
% \paragraph*{õnki} kuulub käändkonda \RN{5}
% \paragraph*{kõva} kuulub käändkonda \RN{6} (\vadja{kõrka} kuulub, aga \vadja{kõva} \msd{sg par} võiks muuta?
% \paragraph*{partõ} kuulub käändkonda \RN{3}
% \paragraph*{tečejä} kuulub käändkonda \RN{8}
% \paragraph*{uhsi} kuulub käändkonda \RN{10}
% \paragraph*{poutõ} kuulub käändkonda \RN{3}
% \paragraph*{seppe} kuulub käändkonda \RN{8}
% \paragraph*{suukkurliivõ} kuulub käändkonda \RN{3}
% \paragraph*{noori, lõhi} kuulub käändkonda \RN{10}
% \paragraph*{poikõ} kuulub käändkonda \RN{5}
% \paragraph*{lähe} kuulub käändkonda \RN{14}
% \paragraph*{lähe} kuulub käändkonda \RN{14} valida emb-kumb
% \paragraph*{läsijõ} kuulub käändkonda \RN{8} kas ühtlustada eespoolseks niku \vadja{tečejä}
% \paragraph*{läsijõ} kuulub käändkonda \RN{8} või ühtlustada tagapoolseks \vadja{-ja} liiteks?
% \paragraph*{muna} kuulub käändkonda \RN{5} kuigi -õi- mitte -ii-
% \paragraph*{musikko} kuulub käändkonda \RN{2}
% \paragraph*{peremmeez} kuulub käändkonda \RN{15}
% \paragraph*{trubačist, mokom} kuulub käändkonda \RN{5}
% \paragraph*{põlto} kuulub käändkonda \RN{2}
% \paragraph*{märännü} kuulub käändkonda \RN{15} või
% \paragraph*{märännü} kuulub käändkonda \RN{2} emb-kumb valida
% \paragraph*{siso} kuulub käändkonda \RN{2}
% \paragraph*{sooli} kuulub käändkonda \RN{10} ühtlustada \msd{sg par} lõpud
% \paragraph*{vaahto} kuulub käändkonda \RN{2}
% \paragraph*{vilppi} kuulub käändkonda \RN{2}
% \paragraph*{võõrõz} kuulub käändkonda \RN{14}
% \paragraph*{tauti} kuulub käändkonda \RN{2} vali emb-kumb tüvemuutus
% \paragraph*{tauti} kuulub käändkonda \RN{2} vali emb-kumb
% \paragraph*{vattsõ} kuulub käändkonda \RN{3}
% \paragraph*{voosi} kuulub käändkonda \RN{10}
% \paragraph*{aapõ} kuulub käändkonda \RN{3}
% \paragraph*{ahas} kuulub käändkonda \RN{14}
% \paragraph*{aitõ} kuulub käändkonda \RN{3}
% \paragraph*{aloi} kuulub käändkonda \RN{8} ??? \RN{5} jään hätta \msd{sg nom} \vadja{alojõ}? vrd \vadja{kitai}
% \paragraph*{alõin} kuulub käändkonda \RN{12} ???
% \paragraph*{angõriaz} kuulub käändkonda \RN{14} ühtlüstada tüvevokaali
% \paragraph*{auči} kuulub käändkonda \RN{2}
% \paragraph*{baabukõz} kuulub käändkonda \RN{14} ühtlüstada tüvevokaali
% \paragraph*{bašmuk} kuulub käändkonda \RN{3}
% \paragraph*{bašn̕i} kuulub käändkonda \RN{2} sest on palataliseeritud???
% \paragraph*{biblioteek} kuulub käändkonda \RN{5}
% \paragraph*{biskvittõ} kuulub käändkonda \RN{5}
% \paragraph*{borovikkõ} kuulub käändkonda \RN{3}
% \paragraph*{bruuss} kuulub käändkonda \RN{14} või \vadja{bruussõ} \RN{3}?
% \paragraph*{bukvõ} kuulub käändkonda \RN{5}
% \paragraph*{bul̕bukõz} kuulub käändkonda \RN{14}
% \paragraph*{čenče} kuulub käändkonda \RN{8}
% \paragraph*{čeväd} kuulub käändkonda \RN{15}
% \paragraph*{čämmel} kuulub käändkonda \RN{15}
% \paragraph*{čäsi} kuulub käändkonda \RN{10}
% \paragraph*{čäčüd} kuulub käändkonda \RN{15} Ariste mainib, et Jõgõperäl \msd{sg nom} vorm \vadja{čäčü}
% \paragraph*{čümme} kuulub käändkonda \RN{13} või \RN{10} NB! aga \msd{sg par} muuta \vadja{-ntä}? ja \msd{sg gen} peaks lõppema -ne?
% \paragraph*{čümmenäz} kuulub käändkonda \RN{13} NB! ordinaalid >3 ühtlustada selle käändkonna järgi
% \paragraph*{dovariššõ} kuulub käändkonda \RN{3} kas \msd{sg ine} ka gemineerub?
% \paragraph*{enči} kuulub käändkonda \RN{7}
% \paragraph*{esimein} kuulub käändkonda \RN{12}
% \paragraph*{famil̕} kuulub käändkonda \RN{2} sest on palataliseeritud???
% \paragraph*{fartõl} kuulub käändkonda \RN{3} NB! \msd{sg ill} peab valesti olema?
% \paragraph*{fookusnik} kuulub käändkonda \RN{3}
% \paragraph*{fotokartočka} kuulub käändkonda \RN{3} või peaks \msd{pl} muutma \RN{5} järgi?
% \paragraph*{fraak} kuulub käändkonda \RN{5}
% \paragraph*{frikad̕el̕k} kuulub käändkonda \RN{3} NB! kas muuta astmevahelduslikuks?
% \paragraph*{haisu} kuulub käändkonda \RN{2}
% \paragraph*{hapo} kuulub käändkonda \RN{2} aga \msd{sg nom} nõrgas astmes sest *<~hapan?
% \paragraph*{häülütüs} kuulub käändkonda \RN{14}
% \paragraph*{hüppü} kuulub käändkonda \RN{2}
% \paragraph*{ičä} kuulub käändkonda \RN{8} kuigi võiks olla \RN{9}
% \paragraph*{iloin} kuulub käändkonda \RN{12}
% \paragraph*{itikkõ} kuulub käändkonda \RN{3}
% \paragraph*{ivusõd} kuulub käändkonda \RN{15} ??? aga mul on plurale tantum?
% \paragraph*{joožikkõ} kuulub käändkonda \RN{3} NB! mul on \msd{pl} valesti nõrgas astmes?
% \paragraph*{jõki} kuulub käändkonda \RN{7} aga miks mitte \RN{2}?
% \paragraph*{jõutu} kuulub käändkonda \RN{3}
% \paragraph*{järčü} kuulub käändkonda \RN{3}
% \paragraph*{kaani} kuulub käändkonda \RN{10} aga kas \msd{sg gen} tüvevokaal muutub?
% \paragraph*{kaatsõd} kuulub käändkonda \RN{15} ??? aga mul on plurale tantum?
% \paragraph*{kahs} kuulub käändkonda \RN{10} kas parem oleks siiski \vadja{kahsi}?
% \paragraph*{kahõsa} kuulub käändkonda \RN{13} ?? vt ka Rozhanskiyst üle
% \paragraph*{kal̕indora} kuulub käändkonda \RN{5} või \RN{9} aga -õi- mitte -ii-? kas kõik -õi- hoopis \RN{9} alla??
% \paragraph*{kalliz} kuulub käändkonda \RN{14}
% \paragraph*{kamal̕ikka} kuulub käändkonda \RN{3}
% \paragraph*{kand̕idaat} kuulub käändkonda \RN{3}
% \paragraph*{kangõz} kuulub käändkonda \RN{14}
% \paragraph*{kanki} kuulub käändkonda \RN{2}
% \paragraph*{kannõl} kuulub käändkonda \RN{14}
% \paragraph*{kant} kuulub käändkonda \RN{3}
% \paragraph*{kasõ} kuulub käändkonda \RN{14}
% \paragraph*{katol̕ikk} kuulub käändkonda \RN{3}
% \paragraph*{katõ} kuulub käändkonda \RN{14}
% \paragraph*{kauniz} kuulub käändkonda \RN{14} NB! muuta Tsvetkovi -e-lõpp -i vastu nagu (Konkovalgi) AGA mida teha \msd{pl} tüvevokaaliga?
% \paragraph*{kaõ} kuulub käändkonda \RN{10} NB! kuigi \msd{sg nom} pole -i-lõpuline!
% \paragraph*{kerkä} kuulub käändkonda \RN{6}
% \paragraph*{keskolin} kuulub käändkonda \RN{12}
% \paragraph*{kitai} kuulub käändkonda \RN{8}
% \paragraph*{klaass} kuulub käändkonda \RN{3}
% \paragraph*{koffi} kuulub käändkonda \RN{2}
% \paragraph*{kolaus} kuulub käändkonda \RN{5}
% \paragraph*{komit̕et} kuulub käändkonda \RN{2}
% \paragraph*{koollud} kuulub käändkonda \RN{15} aga kas tüvevokaali peaks kuidagi ühtlustama?
% \paragraph*{koori} kuulub käändkonda \RN{2}
% \paragraph*{koorrõ} kuulub käändkonda \RN{15} ??? mida \msd{sg par} lõpuga teha?
% \paragraph*{kuha} kuulub käändkonda \RN{5} ???
% \paragraph*{kultõ} kuulub käändkonda \RN{5} või \RN{9} ??
% \paragraph*{kumpõ} kuulub käändkonda \RN{5} aga \msd{pl}-tüvi ühtlustada pikaks?
% \paragraph*{kurkku} kuulub käändkonda \RN{2}
% \paragraph*{kurp} kuulub käändkonda \RN{5}
% \paragraph*{kursi} kuulub käändkonda \RN{2}
% \paragraph*{kusi} kuulub käändkonda \RN{10} NB! ühtlusta \msd{sg ill}
% \paragraph*{kutõ} kuulub käändkonda \RN{14} ??
% \paragraph*{kuus} kuulub käändkonda \RN{7}
% \paragraph*{kuusi} kuulub käändkonda \RN{7}
% \paragraph*{kuuvvaiz} kuulub käändkonda \RN{13} NB! ühtlustada
% \paragraph*{kuõ} kuulub käändkonda \RN{14}
% \paragraph*{kõik} kuulub käändkonda \RN{5}
% \paragraph*{kõlmaz} kuulub käändkonda \RN{13} NB! ühtlustada
% \paragraph*{kõlmõd} kuulub käändkonda \RN{15} NB! \msd{sg par} muuta \vadja{kõlmõttõ}
% \paragraph*{lafkõ} kuulub käändkonda \RN{3}
% \paragraph*{lahti} kuulub käändkonda \RN{2} NB! lühike illatiiv muuta \vadja{lahtisõ}?
% \paragraph*{laki} kuulub käändkonda \RN{5}
% \paragraph*{lehto} kuulub käändkonda \RN{2}
% \paragraph*{leipe} kuulub käändkonda \RN{8}
% \paragraph*{lisä} kuulub käändkonda \RN{8} vrd \vadja{isä}?
% \paragraph*{lootõ} kuulub käändkonda \RN{5}
% \paragraph*{luiskõ} kuulub käändkonda \RN{5}
% \paragraph*{lumi} kuulub käändkonda \RN{10}
% \paragraph*{luukkõ} kuulub käändkonda \RN{9}?? -õi-mitmus
% \paragraph*{lõunõ} kuulub käändkonda \RN{14}??
% \paragraph*{läikk} kuulub käändkonda \RN{2}
% \paragraph*{läikkiv} kuulub käändkonda \RN{8}
% \paragraph*{läsive} kuulub käändkonda \RN{8}
% \paragraph*{läsü} kuulub käändkonda \RN{2}
% \paragraph*{lühüd} kuulub käändkonda \RN{15}
% \paragraph*{maa} kuulub käändkonda \RN{1}
% \paragraph*{magnettiin} kuulub käändkonda \RN{12}?
% \paragraph*{mahsu} kuulub käändkonda \RN{2}
% \paragraph*{mahsõ} kuulub käändkonda \RN{3}
% \paragraph*{main} kuulub käändkonda \RN{12}
% \paragraph*{makka} kuulub käändkonda \RN{6}
% \paragraph*{makuz} kuulub käändkonda \RN{11}
% \paragraph*{mansikõz} kuulub käändkonda \RN{14}
% \paragraph*{mettse} kuulub käändkonda \RN{8}
% \paragraph*{moodõ} kuulub käändkonda \RN{9}?? -õi-mitmus?
% \paragraph*{mõlõpi} kuulub käändkonda \RN{} 
% \paragraph*{mäči} kuulub käändkonda \RN{7}
% \paragraph*{märče} kuulub käändkonda \RN{8}
% \paragraph*{mätä} kuulub käändkonda \RN{8}
% \paragraph*{müüjõ} kuulub käändkonda \RN{8} kas ühtlustada -jõ eespoolseks?
% \paragraph*{nagriz} kuulub käändkonda \RN{11}
% \paragraph*{nagru} kuulub käändkonda \RN{2}
% \paragraph*{nel̕l̕äz} kuulub käändkonda \RN{13}
% \paragraph*{nenäkõz} kuulub käändkonda \RN{14} kas ühtlustada eespoolseks?
% \paragraph*{nõki} kuulub käändkonda \RN{12}
% \paragraph*{olud} kuulub käändkonda \RN{15}
% \paragraph*{ooli} kuulub käändkonda \RN{10} lisada \msd{sg par} lõpuvokaal
% \paragraph*{oonõ} kuulub käändkonda \RN{9}??
% \paragraph*{paganus} kuulub käändkonda \RN{11}
% \paragraph*{paha} kuulub käändkonda \RN{3},???
% \paragraph*{parad} kuulub käändkonda \RN{2}
% \paragraph*{partõin} kuulub käändkonda \RN{12}
% \paragraph*{parõpi} kuulub käändkonda \RN{}???
% \paragraph*{peenepi} kuulub käändkonda \RN{}???
% \paragraph*{pesä} kuulub käändkonda \RN{8}
% \paragraph*{pii} kuulub käändkonda \RN{1}
% \paragraph*{poduškõ} kuulub käändkonda \RN{3}
% \paragraph*{pojo} kuulub käändkonda \RN{9} aga mitmuse tüvi -ai-?
% \paragraph*{poolõz} kuulub käändkonda \RN{14}
% \paragraph*{Portugaalija} kuulub käändkonda \RN{6}???
% \paragraph*{poštaljon} kuulub käändkonda \RN{2} aga kas \msd{sg nom} -i-lõpuga?
% \paragraph*{programmõ} kuulub käändkonda \RN{5} NB! muuda mitmusetüvi pikaks -ii-
% \paragraph*{puhaz} kuulub käändkonda \RN{14}
% \paragraph*{põrzõz} kuulub käändkonda \RN{14}
% \paragraph*{põski} kuulub käändkonda \RN{7}
% \paragraph*{pühä} kuulub käändkonda \RN{8}
% \paragraph*{püütö} kuulub käändkonda \RN{2}
% \paragraph*{raadio} kuulub käändkonda \RN{6}??
% \paragraph*{rakõ} kuulub käändkonda \RN{14}
% \paragraph*{raskõz} kuulub käändkonda \RN{14}
% \paragraph*{ratis} kuulub käändkonda \RN{14} aga Aristel on \vadja{ratiz}?
% \paragraph*{rihenneüz} kuulub käändkonda \RN{11} aga see pole ilus liitsõna
% \paragraph*{rikaz} kuulub käändkonda \RN{14}
% \paragraph*{rissi} kuulub käändkonda \RN{2}
% \paragraph*{roho} kuulub käändkonda \RN{15}? NB! \msd{sg par} lõpuvokaal lisada
% \paragraph*{rookõ} kuulub käändkonda \RN{5} või \RN{9}
% \paragraph*{rooppõ} kuulub käändkonda \RN{5} või \RN{9}
% \paragraph*{ruska} kuulub käändkonda \RN{6}
% \paragraph*{räpäl} kuulub käändkonda \RN{8}
% \paragraph*{rüiz} kuulub käändkonda \RN{14}??
% \paragraph*{seemen} kuulub käändkonda \RN{13}
% \paragraph*{selče} kuulub käändkonda \RN{8}
% \paragraph*{sese} kuulub käändkonda \RN{2}????
% \paragraph*{siipi} kuulub käändkonda \RN{2}??
% \paragraph*{siitiä} kuulub käändkonda \RN{6}
% \paragraph*{sika} kuulub käändkonda \RN{3}? aga miks geminatsioon \msd{sg par}?
% \paragraph*{siltõ} kuulub käändkonda \RN{3}
% \paragraph*{sinin} kuulub käändkonda \RN{12}
% \paragraph*{slona} kuulub käändkonda \RN{8}??
% \paragraph*{soo} kuulub käändkonda \RN{1}
% \paragraph*{sopuin} kuulub käändkonda \RN{12}
% \paragraph*{sorsõ} kuulub käändkonda \RN{5}
% \paragraph*{susi} kuulub käändkonda \RN{7}
% \paragraph*{süčüzü} kuulub käändkonda \RN{2}
% \paragraph*{sünti} kuulub käändkonda \RN{2}
% \paragraph*{süsiin} kuulub käändkonda \RN{12}??
% \paragraph*{süä} kuulub käändkonda \RN{13}?? kuigi Tsetkovil pole \msd{sg gen} -me-lõpuline? või on see \RN{6} järgi?
% \paragraph*{süüčči} kuulub käändkonda \RN{2}
% \paragraph*{štanad} kuulub käändkonda \RN{3} aga on plurale tantum
% \paragraph*{taimi} kuulub käändkonda \RN{7}
% \paragraph*{talviin} kuulub käändkonda \RN{12}
% \paragraph*{tarkuz} kuulub käändkonda \RN{11}
% \paragraph*{tee} kuulub käändkonda \RN{1}
% \paragraph*{terve} kuulub käändkonda \RN{6}
% \paragraph*{toho} kuulub käändkonda \RN{15}? NB! \msd{sg par} lõpuvokaal lisada
% \paragraph*{tuhattõ} kuulub käändkonda \RN{3}?
% \paragraph*{turvõz} kuulub käändkonda \RN{14}
% \paragraph*{täti} kuulub käändkonda \RN{2}
% \paragraph*{tühjõ} kuulub käändkonda \RN{4} aga see käändkond on nii ebaproduktiivne, et tõsta ümber \RN{8}  NB! muuda \vadja{tühje}??
% \paragraph*{tükkü} kuulub käändkonda \RN{2}
% \paragraph*{tünke} kuulub käändkonda \RN{8}
% \paragraph*{tütär} kuulub käändkonda \RN{13}???
% \paragraph*{usa} kuulub käändkonda \RN{5}
% \paragraph*{vahti} kuulub käändkonda \RN{2}
% \paragraph*{vaka} kuulub käändkonda \RN{6}
% \paragraph*{varsi} kuulub käändkonda \RN{10}
% \paragraph*{varvõz} kuulub käändkonda \RN{14}
% \paragraph*{velosipedõ} kuulub käändkonda \RN{5}
% \paragraph*{vihtõ} kuulub käändkonda \RN{3}
% \paragraph*{vikahtõ} kuulub käändkonda \RN{3}
% \paragraph*{viki} kuulub käändkonda \RN{2}
% \paragraph*{viks} kuulub käändkonda \RN{3}
% \paragraph*{villõ} kuulub käändkonda \RN{3}
% \paragraph*{vimpõ} kuulub käändkonda \RN{5}
% \paragraph*{vipu} kuulub käändkonda \RN{2}
% \paragraph*{väči} kuulub käändkonda \RN{7}
% \paragraph*{õgaz} kuulub käändkonda \RN{14}
% \paragraph*{ohsõ} kuulub käändkonda \RN{5}
% \paragraph*{õhtõgoin} kuulub käändkonda \RN{12}
% \paragraph*{õja} kuulub käändkonda \RN{8}
% \paragraph*{õma} kuulub käändkonda \RN{8}
% \paragraph*{õmpõjõ} kuulub käändkonda \RN{8}
% \paragraph*{õmpõlia} kuulub käändkonda \RN{8} -lia aga -lija?
% \paragraph*{õnnõkõz} kuulub käändkonda \RN{14}
% \paragraph*{õnnõliin} kuulub käändkonda \RN{12}
% \paragraph*{õnnõtoi} kuulub käändkonda \RN{13}
% \paragraph*{õpõin} kuulub käändkonda \RN{12}
% \paragraph*{õpõttõja} kuulub käändkonda \RN{8}
% \paragraph*{õttsõ} kuulub käändkonda \RN{3}
% \paragraph*{änte} kuulub käändkonda \RN{8}
% \paragraph*{ärčä} kuulub käändkonda \RN{8} aga võiks ka \RN{7}
% \paragraph*{ääri} kuulub käändkonda \RN{7} aga \msd{sg par} \vadja{ääreä}?
% \paragraph*{ühs} kuulub käändkonda \RN{10}
% \paragraph*{üvä} kuulub käändkonda \RN{8}




%\subsection{Ekstraktmorfoloogia üldistatud muuttüüpide algoritm}
%\label{sec:muuttüüpide-süsteem}

% tundub, et algoritm on kallutatud regulaarsustele, mis esinevad kirjakeeltes-ühiskeeltes, aga ei tööta nii hästi variatiivse keeleainese peal

%Silfverberg ja Hulden (\citeyear{silfverberg_computational_2018}) on kirjeldanud üht formaalset viisi, kuidas ekstrakt\-morfoloogia tüüpsõnu kokku grupeerida ja seega nende arvu vähendada. Siin alaosas rakendatakse meetodit leitud tüüpsõnadele ja esitatakse selle põhjal loodud vadja muuttüübistik ja võrreldatakse leitud muuttüübistikku Ariste käändkondadega.



% Eesti muuttüüpide traditsioonist on kirjutanud mh \cite{viks_muuttuubid_nodate}.



%\subsubsection{Muuttüüp \RN{1}}
% Muuttüüp \RN{1} koondab enda alla kõige suurema hulga sõnu (303) ja tüüpsõnu (??). Hõlmatud tüüpsõnad jagunevad järgnevatesse \cite[85]{__2011} välja toodud muutkondadesse (klassi).
% 
% \paragraph*{Klass 1} \vadja{aapõ}, \vadja{aikõ}, \vadja{aitõ}, \vadja{butkõ}, \vadja{čenče}, \vadja{dovariššõ(?)}, \vadja{enči}, \vadja{propkõ}, \vadja{ülči}, \vadja{laiskõ}, \vadja{kultõ}, \vadja{kumpõ}, \vadja{õnki}, \vadja{kurp}, \vadja{partõ}, \vadja{lafkõ}, \vadja{lahti}, \vadja{lautõ}, \vadja{leipe}, \vadja{lootõ}, \vadja{luiskõ}, \vadja{poikõ}, \vadja{märče}, \vadja{selče}, \vadja{mätä}, \vadja{rantõ}, \vadja{põlvi}(?), \vadja{kanki}, \vadja{kant}, \vadja{põski}, \vadja{rookõ}, \vadja{siipi}, \vadja{siltõ}, \vadja{sorsõ}, \vadja{taimi}(?), \vadja{tünko}, \vadja{vihtõ}, \vadja{vimpõ}, \vadja{õhsõ}, \vadja{änte}, \vadja{ääri}(?)
% 
% \paragraph*{Klass 1a} \vadja{kukkõ}, \vadja{biskvittõ}, \vadja{borovikkõ}, \vadja{bruuss}, \vadja{mõiznikkõ}, \vadja{fartukkõ}, \vadja{pliittõ}, \vadja{ikolookkõ}, \vadja{itikkõ}, \vadja{joožikkõ}, \vadja{katol̕ikk}, \vadja{liippõ}, \vadja{koomikk}, \vadja{klaass}, \vadja{riittõ}, \vadja{seppe}, \vadja{luukkõ}, \vadja{mahsõ}, \vadja{musikko}, \vadja{rooppõ}, \vadja{tükku}, 
% 
% \paragraph*{Klass 2}
% 
% \paragraph*{Klass 2$^\ast$} \vadja{česä}, \vadja{ičä}, \vadja{jõki}, \vadja{sõta(?)}, \vadja{mäči}, \vadja{nõki}, \vadja{rapa}, \vadja{sika}, \vadja{usa}, \vadja{väči} 
% 
% \paragraph*{Klass 3} \vadja{iiri}, \vadja{irvi(?)}, \vadja{suuri}, \vadja{kahs}, \vadja{kahõsa(?)}, \vadja{koollud(?)}, \vadja{koori}, \vadja{kuus(?)}, \vadja{uusi}(?), \vadja{kuusi}, \vadja{lumi}, \vadja{noori}, \vadja{lõunõ}, \vadja{läsijõ}, \vadja{lühüd}, \vadja{meri}, \vadja{peeni}, \vadja{meeli}, \vadja{meez}, \vadja{ooli}, \vadja{poduškõ}(?), \vadja{pooli}, \vadja{uhsi}, \vadja{varsi}, \vadja{õmpõlia}, \vadja{ühs}
% 
% \paragraph*{Klass 3$^\ast$} \vadja{čäsi}, \vadja{laki}, \vadja{süsi}, \vadja{vesi}, \vadja{susi}
% 
% \paragraph*{Klass 4} \vadja{pal̕l̕õz}, \vadja{alõin}, \vadja{ammõz}, \vadja{čimolain}, \vadja{baabukõz}, \vadja{bul̕bukõz}, \vadja{musikõz}, \vadja{eglin}, \vadja{čümmenäz}, \vadja{esimein}, \vadja{iirikkõin}, \vadja{iloin}, \vadja{jäin}, \vadja{liivõkõz}, \vadja{kangõz}, \vadja{kauniz}, \vadja{keskolin}, \vadja{kultõin}, \vadja{kuuvvaiz}, \vadja{kõlmaz}, \vadja{makuz}, \vadja{mansikõz}, \vadja{nagriz}, \vadja{nel̕l̕äz}, \vadja{nenäkõz}, \vadja{nain}, \vadja{poolõz}, \vadja{põrzõz}, \vadja{raskõz}, \vadja{rihenneüz}, \vadja{rüiz}, \vadja{sinin}, \vadja{sopuin}, \vadja{süsiin}, \vadja{võõrõz}, \vadja{talviin}, \vadja{tarkuz}, \vadja{turvõz}, \vadja{varvõz}, \vadja{õhtõgoin}, \vadja{õnnõkõz}, \vadja{õnnõliin}



%\subsection{Põhivormid ja analoogiavormid}

%Selles osas selgitatakse välja vadja keele tüüpsõnade põhi- ja analoogiavormid sõnaliigiti. Seda püütakse teha formaalselt põhinedes vaid ekstrakt\-morfoloogiaga leitud tüüpsõnamallidele.

%\cite{erelt_eesti_2007} järgi ``[p]õhivormid on need vormid, mida pole võimalik teiste vormide alusel tuletada ning mille moodustamiseks tuleb iga sõnatüübi korral anda vastavad reeglid.'' ja ``[a]naloogiavormid on vormid, mida saab moodustada mingi põhivormi analoogial.''

%Tegelikult on ekstraktmorfoloogia leitud LCS ainus põhivorm ja kõik muutvormid on sellest tuletatud analoogiavormid. Kuna aga läänemeresoome keelte keeleteaduses ei ole katkendlike põhivormide kasutamine traditsioonis (nagu seda on nt araabia keelte puhul), püütakse siin leida traditsioonilise käsitluse järgi põhi- ja analoogiavormid.

%\subsubsection{Käändsõnad}
% Eesti keele käändsõna põhivormid on ainsuse nimetav, ainsuse omastav, ainsuse osastav, mitmuse omastav ja mitmuse osastav. Põhivormiks tuleb tingimisi lugeda ka ainsuse lühikest sisseütlevat.

%\subsubsection{Tegusõnad}



%\subsection{Muuttüüpide produktiivsus}
%
%Kristiina Kross (Ross) nimetab produktiivsuseks ``mingi morfoloogilise nähtuse võimet allutada endale uusi sõnu'' (\cite{kross_eesti_1984}). Siin allosas seatakse eelmises osas leitud muuttüübid pingeritta selle järgi, kui mitu tüüpsõna nendele allub.
%
%Kas selleks on vaja defineerida, mis on \textit{uus sõna}? Näiteks kõik uuemad vene keele laenud.
%
%Kas produktiivsuse pingerida on võimalik jagada mingi kriteeriumi järgi avatuteks ja suletuteks muuttüüpideks?





\newpage
\section{Programmkoodi tuletamine}
\label{sec:programmkoodi-tuletamine}


Programmkoodi tuletamise (ingl. \textit{source code generation}) all peetakse siin töös silmas mistahes protsessi, mille käigus tuletatakse mingi üldisema kirjelduse põhjal programmkoodi ühe või mitme konkreetse programmeerimis\-keskkonna jaoks.

Üldine kirjeldus (või teisisõnu ontoloogia) kirjeldab faktuaalset \textit{mis}-laadi teadmist ning tuletatud programmkood kirjeldab konkreetselt \textit{kuidas} neid teadmisi rakendada.

Töös kasutatakse keskseks kirjelduseks vormisõnastikku ja sellest ekstrakt\-morfo\-loogiaga eraldatud tüüpsõna\-malle. Kirjeldus on vormistatud rahvusvahelise standardi \textit{Lexical Markup Framework} (LMF) (\cite{iso/tc_37/sc_4_language_2007}) järgi leksikaalseks ressurssiks, mis on XML vormingus.

Programmkoodi tuletavad nn generaatorid. Töös esitatakse kaht generaatorit, üks programmeerimiskeele Grammatical Framework jaoks ning teine Giella keeletehnoloogilise taristu integreerimise jaoks. Generaatorid on kirjutatud XQuery programmeerimiskeeles.

Mõlema programmkoodi\-generaatori ühine arhitektuurne omadus on terminite tõlke\-tabelite kasutamine. Tõlke\-tabelite põhjal asendatakse leksikaalses ressursis kasutatud terminid vastavalt integreeritavas taristus kasutatud terminitega. See võimaldab järgida ja austada eri taristute terminoloogilisi traditsioone. Näiteks nimetatakse tüüpsõnu LMFis ingliskeelse prefiksiga \textit{as} (kui \textit{asHattu}), aga Grammatical Framework'is eesliitega \textit{mk} (inglis\-keelsest tegusõnast \textit{make}) kui \textit{mkHattu} ja Giella taristus hoopis sõnaliigiga (\textit{N\_HATTU}).

Samuti erinevad eri taristutes kasutatud grammatiliste tunnuste märgendid (nt ainsuse nominatiiv on LMFis ja GFis \textit{singular nominative}, aga Giellas \textit{+SG+NOM}).
% A shared architectural feature of both code generators is the use of translation
% tables for translating terms used in the LMF ontology to their corresponding terms
% used in the host environment. For example the names of paradigms are prefixed with
% as in the LMF (e.g. asTšiutto), but named like actions in the GF (mkTšiutto), and
% prefixed by their part of speech in Giellatekno (N_TŠIUTTO). Also the terminology for
% grammatical features differ between the environments.
% In this way different terminological traditions are supported and respected.



\subsection{Keskne kirjeldus Lexical Markup Framework vormingus}
\label{sec:lmf}

% 3.1
% Lexical Markup Framework
% The Lexical Markup Framework (LMF, ISO 24613:2008) is an ISO standard for natural
% language processing lexicons and machine readable dictionaries. It provides a com-
% mon model for managing exchange of data and enables merging of different resources.
% (Francopoulo, 2013).
% The LMF standard consists of a core model and several extensions. In our work we
% use two extensions: the NLP Morphological Pattern extension to model the extracted
% paradigm information, and the Morphology extension to represent each lexeme’s in-
% flected wordforms.
% Representing the lexeme’s morphology with both paradigms (describing in inten-
% sion) and inflected wordform tables (describing in extension) might seem superfluous.
% But both representations serve their own purpose.
% Listing extensionally all inflectional wordforms for each lexeme is in this work
% considered part of documentation and what adds value to the work’s 50-year per-
% spective (discussed in section 4).
% Recording lexeme’s all inflected wordforms extensionally also creates the possi-
% bility to further annotate the individual wordforms, such as real attested corpus at-
% testations, or other meta-linguistic information such as judgements.
% Listing wordform information explicitly is also beneficial for the dictionary sys-
% tem, enabling quick searches and statistics.
% Next, we will introduce our data and how it is represented in the LMF.

% sissejuhatav tekst
Lexical Markup Framework (LMF) on loomuliku keeletöötluse leksikonidele ja masin\-loetavatele sõna\-raamatutele mõeldud rahvusvaheline ISO-standard. Standard koosneb märgenduskeelega defineeritud ühisest mudelist andmevahetuse juhtimiseks ja võimaldab erinevate ressursside ühendamist. (\cite[1]{francopoulo_lmf_2013})
% TODO kas Francopoulo peatükist ei leidu natuke pikemalt lahtivõetud kirjeldus?

% laiendimoodulid
Standard koosneb põhimoodulist, mis määrab leksikaalse ressursi, ja mitmest eri\-otstarbelisest laiendi\-moodulist (\cite[22]{francopoulo_lmf_2013}). Magistritöös kasutatakse peale põhimooduli veel kahte moodulit: morfoloogia moodul (\textit{LMF Morphology Extension}) ja morfoloogiliste paradigmade moodul (\textit{LMF Morphological Pattern Extension}). Valitud moodulitega on olnud võimalik kirjeldada ja salvestada nii vormisõnastiku andmed (lekseemide muutvormid) kui ka ekstrakt\-morfoloogiaga eraldatud andmed (tüüpsõnade protsessuaalsed koostamis\-mallid ja tehniliste tüvede muutujad).

% morfoloogiamoodul eesmärk
Morfoloogiamooduli eesmärgiks on kirjeldada morfoloogiat mahu kaudu, s.o kirjeldada lekseemi loendades kõik selle muutvormid.

% paradigmamooduli eesmärk
Morfoloogiliste paradigmade mooduli eesmärgiks on seevastu kirjeldada morfoloogiat sisu kaudu, s.o kirjeldada neid kriteeriume ja reegleid, millega saab moodustada kõik ühe lekseemi muutvormid. Selles töös kirjeldatakse ekstrakt\-morfoloogia tüüpsõna\-mallid antud mooduliga.

% topeltkirjeldus ju liigne?
Sama nähtuse kirjeldamine nii mahus kui ka sisus võib tunduda liigsena, ent mõlemal kirjeldusviisil on omad head küljed. Lekseemide iga muutvormi loendamist peetakse magistritöös eeskätt dokumenteeriva ja arhiveeritava väärtusena, mis ei kahane pikaajalises perspektiivis -- muutvormide loendamine on ka 50 või 150 aasta pärast informatiivne, keele\-tehnoloogia programm\-kood ei pruugi olla arusaadav ega jooksutatav. Veel võimaldab muutvormide loendamine edaspidises sõnastikutöös ka igale muutvormile lisada informatsiooni, nt selle reaalsete korpusesinemuste kohta. %Samas on just loendatud muutvormid ekstraktmorfoloogia meetodi sisend, ja võimaldab .

% TODO sõnatüübimalli hoidmise häid külgi pole kirjeldatud
Ekstraktmorfoloogiaga eraldatud tüüpsõnamallide kirjeldamine samas vormingus võimaldab neid protsessuaalseid koostamis\-eeskirju teisendada programm\-koodi generaatoritega. Generaatoreid võib hiljem lisada juurde uue keeletehnoloogia tuletamise jaoks ilma, et lähte\-andmestikku peaks muutma. See teebki LMFi kirjelduse nö tehnoloogia\-neutraalseks standardiks.

% mis seal veel hoitakse?
Peale sõnaartiklite ja morfoloogilise informatsiooni hoitakse leksikaalses ressursis ka globaalset informatsiooni, nagu keele nimetust ja keele\-koodi, mida kasutatakse tuletatud programmkoodis peamiselt failide nimetamisel.

% TODO kirjutada outro ja intro järgmise kahe alatüki jaoks


\subsubsection{Sõnaartikli esitamine LMFis}
Lekseemide sõnaartiklid LMFi esituses koosnevad muutvormide loendist, lemmaks valitud muutvormist, sõnaliigist ja tüüpsõnast. Paralleelvormide puhul on tüüpsõnu rohkem kui üks.

% TODO
Andmete struktuuri näitlikustav sõnaartikkel on LMFi XML märgendus\-formaadis esitatud joonisel~\ref{code:lmf-lexicalentry}.


\spacing{1}
\begin{figure}[h]
  \center
\begin{minted}[frame=single,fontsize=\small,framesep=10pt]{XML}
<LexicalEntry morphologicalPatterns="asHattu">
  <feat att="partOfSpeech" val="commonNoun"/>
  <Lemma>
    <feat att="writtenForm" val="hattu"/>
  </Lemma>
  <WordForm>
    <feat att="writtenForm" val="hattu"/>
    <feat att="grammaticalNumber" val="singular"/>
    <feat att="grammaticalCase" val="nominative"/>
  </WordForm>
  <WordForm>
    <feat att="writtenForm" val="hatud"/>
    <feat att="grammaticalNumber" val="plural"/>
    <feat att="grammaticalCase" val="nominative"/>
  </WordForm>
</LexicalEntry>
\end{minted}
\caption{Sõnaartikli \textit{hattu} esitamine LMFis (muutvormid on kajastatud vaid osaliselt).
  \label{code:lmf-lexicalentry}}
\end{figure}
\spacing{1.5}

Tüüp\-sõnamall \textit{asHattu} on märgitud \textit{morphological\-Patterns} atribuudis. Mitme tüüpsõna puhul loendatakse need tühikutega eraldatuna. Sõnaliik on märgitud \textit{part\-Of\-Speech} elemendis (\textit{noun}). Element \textit{Lemma} sisaldab lemmaks valitud sõnavormi. \textit{Word\-Form}-elemendid loendavad muutvorme koos nende grammatiliste tunnustega: arv (\textit{grammatical\-Number} ja kääne (\textit{grammatical\-Case}).

Sõnavormide modaalsust on täpsustatud kirjakeelseks kujuks (\textit{writtenForm}).




\subsubsection{Tüüpsõnamalli esitamine LMFis}

%The extracted paradigms are represented as LMF Morphological Patterns. These hold
%information about their part of speech and are name-tagged with an ID. The names
%follow the LMF tradition and are prefixed with as, such as asTšiutto.
Ekstraktmorfoloogiaga eraldatud tüüpsõnad kirjeldatakse LMFi morfoloogiliste paradigmade mooduli elementidega (\textit{LMF Morphological Pattern Extension}, varem nimetatud ka \textit{LMF Paradigm Pattern}). Iga tüüpsõna on märgitud identifitseeriva nimetusega, mille puhul on järgitud LMFi tava lisada tüüpsõna ette ingliskeelne eeslide \textit{as} (nt \textit{asHattu}).

%The LMF Morphological Patterns model all the information extracted by Extract
%Morphology. The attested variable values, i.e. the technical stems, of all lexemes
%added in the Morphology Lab is saved. This information could be utilized to inte-
%grate prediction models into the generated source code, as have been demonstrated
%by Forsberg and Hulden (2016). This has not been done, as the work has focused on
%integrating the lexical resources as a first stage.
Peale nimetuse ja sõnaliigi kirjeldatakse LMF morfoloogiliste paradigmadega veel kõik ekstraktmorfoloogiaga eraldatud informatsioon: salvestatakse lekseemide tehniliste tüvede muutujad ja muutvormimallid.

Muutvormallid on LMFis kirjeldatud nn transformatsiooni\-elementidega (\textit{TransformSet}), mis koosnevad muutvormi morfoloogilistest tunnustest (\textit{GrammaticalFeatures}) ja muutvormi koostamiseks vajalikest protsessidest (\textit{Process}). Koostamis\-prot\-sessid osutavad lihtsatele järjestatud \glslink{konkatenatsioon}{konkatenatsiooni\-tehetele} (\textit{addAfter}), mis liidavad lükkimise teel tehnilise tüve muutujad ja muutvormi ülejäänud tähtkoostised kokku.

Andmete struktuuri näitlikustav tüüpsõna on LMFi XML märgendus\-formaadis esitatud joonisel~\ref{code:lmf-paradigmpattern}.

\spacing{1}
\begin{figure}[h]
  \center
\begin{minted}[frame=single,fontsize=\footnotesize,framesep=10pt]{XML}
<MorphologicalPattern>
  <feat att="id" val="asHattu"/>
  <feat att="partOfSpeech" val="commonNoun"/>
  <TransformSet>
    <GrammaticalFeatures>
      <feat att="grammaticalNumber" val="singular"/>
      <feat att="grammaticalCase" val="nominative"/>
    </GrammaticalFeatures>
    <Process>
      <feat att="operator" val="addAfter"/>
      <feat att="processType" val="pextractAddVariable"/>
      <feat att="variableNum" val="1"/>
    </Process>
    <Process>
      <feat att="operator" val="addAfter"/>
      <feat att="processType" val="pextractAddConstant"/>
      <feat att="stringValue" val="t"/>
    </Process>
    <Process>
      <feat att="operator" val="addAfter"/>
      <feat att="processType" val="pextractAddVariable"/>
      <feat att="variableNum" val="2"/>
    </Process>
  </TransformSet>
  <TransformSet>
    <GrammaticalFeatures>
      <feat att="grammaticalNumber" val="plural"/>
      <feat att="grammaticalCase" val="nominative"/>
    </GrammaticalFeatures>
    <Process>
      <feat att="operator" val="addAfter"/>
      <feat att="processType" val="pextractAddVariable"/>
      <feat att="variableNum" val="1"/>
    </Process>
    <Process>
      <feat att="operator" val="addAfter"/>
      <feat att="processType" val="pextractAddVariable"/>
      <feat att="variableNum" val="2"/>
    </Process>
    <Process>
      <feat att="operator" val="addAfter"/>
      <feat att="processType" val="pextractAddConstant"/>
      <feat att="stringValue" val="d"/>
    </Process>
  </TransformSet>
<MorphologicalPattern>
\end{minted}
\caption{Tüüpsõnamalli \vadja{hattu} osaline esitus LMF\-is kahe muutvormimalliga, $x_1\,\oplus\,\textbf{t}\,\oplus\,x_2$ (\msd{sg nom}) ja $x_1 \oplus x_2 \oplus \textbf{d}$ (\msd{pl nom}).
  \label{code:lmf-paradigmpattern}}
\end{figure}
\spacing{1.5}

LMFis on Morfoloogiliste paradigmade \textit{Process}-elemendid üks\-üheses vastavuses ekstrakt\-morfoloogia meetodiga eraldatud tüüpsõna\-mallide muutvormi\-mallidega.

% TODO process elementide processType kirjeldus!

% TODO kas ma lihtsalt 'unustan' tehniliste tüvede kirjeldamise?


\FloatBarrier
\subsection{Integreerimine Grammatical Framework'iga}
\label{sec:gf}

Grammatical Framework (GF) on eriotstarbeline programmeerimis\-keel, mis on loodud loomulike (ja ka tehislike) keelte grammatikate kirjeldamiseks. GFi iseloomustab programmeerimis\-keelena muuhulgas see, et kuulub funktsionaalsete programmeerimis\-keelte paradigmasse ja on rajatud tüübiteooriale. (\cite[\RN{7}]{ranta_grammatical_2011}) % TODO kas on õige lk. VII

Teoreetiliselt on GF formaal\-grammatikana väljendus\-rikkuse poolest ekvivalentne PMCFG tüüpi grammatikaga (\textit{Parallell Multiple Context-Free Grammar}), mis jääb mõõdukalt konteksti\-tundlike (\textit{mildly context-sensitive}) ja täielikult konteksti\-tundlike (\textit{fully context-sensitive}) grammatikate vahele (\cite[10]{ranta_grammatical_2011}).

Grammatical Framework toetab keele\-spetsiifilise koodi eraldamist teekidesse, mida nimetatakse ressursi\-teekideks (\cites{ranta_grammars_2009}[97]{ranta_grammatical_2011}). GFil on üle 32~keele tugi on ressursi\-teegi kujul. Eesti keele ressursi\-teegi on loonud Inari Listenmaa ja Kaarel Kaljurand (\citeyear{listenmaa_computational_2014}). %Teegid võivad ka jagada ühiseid grammatilisi jooni eriti süntaksis. On olemas soome süntaksi 

Siin esitatud programm\-koodi generaator loob vormisõnastikust automaatselt vadja keele ressursi\-teegi alam\-moodulid morfoloogia ja leksikoni jaoks. Loodud vadja ressursi\-teek on internetis saadaval\footnote{\url{https://github.com/keeleleek/GF-Votic}} ja ressursi\-teegi põhjal on edukalt katsetatud ehitada lihtne vadja--vene turismi\-vestmik\footnote{\url{https://github.com/keeleleek/GF-Votic-Phrasebook}} Grammatical Framework'is.

Ressursi\-teegi alustamist morfoloogia\-moodulist on peetud heaks viisiks uue keele toe lisamisel GFi, kuigi on võimalik alustada ka süntaksi moodulist (\cite[209]{ranta_grammatical_2011}). % Usually the best way to start resource grammar writing is with the morphology. \cite[209]{ranta_grammatical_2011}

% TODO järgmine kopeerida ka programmkoodi genereerimise intro alla
Loodud generaator näitlikustab üht integreerimis\-viisi kuidas vormisõnastiku siduda keele\-tehnoloogiaga, mida iseloomustab pideva paigaldamise tarkvara arendusmeetod (ingl. \textit{Continuous Integration}). Vormisõnastiku täiendamist (ja ka muutmist) saab kasutada eesmärgipäraselt selleks, et viia need täiendused sisse Grammatical Framework'i vadja morfoloogia\-teeki ja sealt\-kaudu sellele põhinevatele rakendustele. Kui leksikonist või morfoloogiast on midagi puudu või on nendes midagi valesti, saab täiendused viia sisse vormi\-sõnastikku ja ei pea otse muutma, s.o programmeerima, GF ressursi\-teegi mooduleid.

GF programmkoodi generaator on jaotatav kaheks mõtteliseks osaks, millest üks genereerib vadja morfoloogia\-mooduli ja teine vadja leksikoni\-mooduli.

Generaator on ehitatud võimalikult universaalseks ja ei sisalda midagi vadja keelele spetsiifilist. Lemmavormi valikut ei ole generaatorisse jäigalt sisse kodeeritud, vaid on vahetatav generaatori parameetritega. Veel koostatakse moodulite faili\-nimed vastavalt keele\-koodile, mis on kirjas leksikaalses ressursis.



\subsubsection{Morfoloogia moodul}
\label{sec:gf-tüüpsõnad}
Grammatical Framework'i morfoloogiamoodul koosneb nn paradigma\-funktsioonidest, mille ülesandeks on luua sõnade paradigmad ehk kõik muutvormid (\cites[248]{ranta_grammatical_2011}[645]{detrez_smart_2012}).

Ekstrakt\-morfoloogiaga eraldatud tüüpsõna\-mallide funktsioone on olnud lihtne teostada paradigma\-funktsioonidena. Igale tüüpsõnale on teostatud kaks eraldi funktsiooni: üks üldine, mis võtab sisendiks sõna lemmaks valitud muutvormi tervikuna ja väljastab sõna tehnilise tüve osad, teine on konkreetne ja sisendiks on tehnilise tüve osad ja väljundiks on muutvormide tabel. Kuigi kahest funktsioonist on võimalik moodustada üks, on kaks funktsiooni eraldi hoitud võimalike tarkvara\-vigade otsimise tarbeks, et arendajal oleks võimalik testida funktsioone enda moodustatud tehniliste tüvedega.

% To generate the function for an extracted paradigm description, we need to specify its name, interface and body.
Tüüp\-sõnamallist paradigmafunktsiooni programmkoodi genereerimiseks on vaja määrata funktsiooni nimi, liides (\textit{interface}) ja keha (\textit{body}). Genereeritud programm\-koodi illustreeritakse joonisel~\ref{code:gf-morfoloogia}. % TODO joonis?

Programmkoodis on morfoloogiamooduli failinimi, \textit{MorphoVot}, koostatud vastavalt vadja keele\-koodile, mis on leksikaalses ressursis antud.

\spacing{1}
\begin{figure}[!ht]
  \center
\begin{minted}[frame=single,fontsize=\small,framesep=10pt]{haskell}
resource MorphoVot = {

param
  Number = singular | plural ;
  Case = nominative ;
  NForm = NF Number Case ;

oper
  Noun : Type = {s : NForm => Str} ;

------------------------------------------------
-- Start of Noun section
------------------------------------------------

  mkHattu : Str -> Noun = \hattu -> 
    case hattu of {
      hat + "t" + u@(-(_+"t"+_)) => mkHattuConcrete hat u ;
      _ => Predef.error "Unsuitable lemma for mkHattu"
    } ;
  
  mkHattuConcrete : Str -> Str -> Noun = \hat,u -> 
    { s =
      table {
        NF singular nominative => hat + "t" + u ;
        NF plural nominative => hat + u + "d"
      }
    } ;
}
\end{minted}
\caption{Vadja keele GF morfoloogia\-mooduli genereeritud programmkood (näites piiratud käänetega \msd{sg nom} ja \msd{pl nom}).
  \label{code:gf-morfoloogia}}
\end{figure}
\spacing{1.5}

Paradigmjafunktsiooni nimetuseks on tüüpsõna identifikaator LMFi leksikaalsest ressursist, mille eesliide \textit{as} vahetatakse GFi traditsiooni järgi \textit{mk}.

% The function’s interface is declared in the oper section and parameter types in
% the param section. The names and values of the parameters reflect the attributes and values of the grammatical features in the lMF, with minor modifications. The oper
% definition is largely templatic, reflecting only the parts of speech from the LMF.
Funktsiooni liides on deklareeritud programmkoodi \textit{oper} osas ja parameetrite tüübid \textit{param} osas. Parameetrite nimetused ja kategooriad peegeldavad grammatilisi tunnuseid LMFis.

% To help readers navigate the code, the paradigms are ordered by their part of
% speech and headers are generated in the form of comments for each part of speech
% section. These are the only comments generated at the moment.
Kuigi vadja vormisõnastik sisaldab magistritöö valmimise hetkel ainult nimisõnu, arvestab koodi\-generaator teiste sõna\-liikidega ja grupeerib tüüpsõnade paradigma\-funktsioonid vastavateks sektsioonideks, mille ette lisatakse kommentaar selle sõnaliigi sektsiooni algamise kohta. Koodi lugemise hõlbustamiseks järjestatakse tüüpsõnade funktsioonid veel nende nimede järgi tähestikuliselt. 

% Every paradigm is split into two separate functions a high-level dispatch function
% (mkTšiutto, and low-level to mkTšiuttoConcrete). This is to allow a developer to
% use the low-level function for debugging or testing purposes.

% The high-level function takes a string with the chosen lemma form as its input and
% splits it into the technical stem parts, which are then simply delegated to the low-level
% function. An error message is thrown in case the input string is not able to match.
% Üldisem, lemmavormil opereeriv funktsioon eraldab sisestatud muutvormi selle tehnilise tüve osadeks, mis seejärel edastatakse konkreetsele paradigma\-funktsioonile.

% Note that GF’s implementation of regular expressions is non-greedy as opposed to
% many other programming languages. Because of this the expression @(-(_+"t"+_))
% is appended to the last part of the pattern. This expression is automatically created
% and makes the `t` match the last letter t in the wordform.
Võib märkida, et GFi regulaaravaldised töötavad mitte-aplalt (ingl. \textit{non-greedy}), vastupidiselt paljudele teistele programmeerimis\-keeltele. Seetõttu lisatakse regulaar\-avaldisele ploki \mbox{\texttt{@(-(\_+"t"+\_))}}, mis muudab avaldises \textit{t}-tähega ühtimise aplaks, s.o ühtimaks viimase \textit{t}-tähega.

% The names of the variables holding the technical stem parts are taken from the
% paradigm’s name.
Tehnilise tüve muutujate nimetused saadakse tüüpsõna nimetusest ja koosnevad muutujate tähtkoostisest.

% The low-level, or concrete, function is the one that generates the inflection table.
% On the left-hand side of the table is the grammatical features and on the right-hand
% side are the concatenation patterns that instantiates the wordforms.
Konkreetse funktsiooni nimetuse järele lisatakse \textit{Concrete} ja see funktsioon tagastab muutvormide tabeli. Tabeli vasakul poolel on grammatilised tunnused ja paremal poolel on muutvormide mallid.



\FloatBarrier
\subsubsection{Leksikoni moodul}
\label{sec:gf-leksikon}

Grammatical Framework on oma olemuselt mitmekeelne ning iga rakendus peaks täitma nende oma semantika abstraktse grammatikaga. Kuna vormisõnastik ei sisalda hetkel tõlkeid, genereeritakse lihtne sõnaloend.
%GF is a multilingual framework and each application is expected to define their own semantics in an abstract grammar. No attempt is made to include semantic pivots to the Votic resource, instead a simple monolingual word list is generated as a lexicon.

Iga sõnaartikli puhul koostatakse sõnaloendisse kirje, mille nimeks on lemma ja sõnaliigi\-tähis ning väljakutse vastava tüüpsõna paradigma\-funktsioonile. Juhul, kui sõna\-artiklil on mitu tüüpsõna, määratakse need kirje variantideks.
%This lexicon specifies for each entry in the resource its lemma, which is appended with its part of speech and a call to the paradigm function. In the case a lemma has multiple paradigms, each one is declared as variants.

%Rakenduste loomisel võib GF arendaja tugineda genereeritud sõnaloendile.
%The GF developer could use this word list when translating the application-specific vocabulary names of the abstract grammar.

Genereeritud sõnaloendi programm\-koodi illustreeritakse joonisel~\ref{code:gf-lexicon}. % TODO joonis?
%The source code of the generated word list is illustrated in figure~\ref{code:gf-lexikon}.

\begin{figure}[ht]
  \center
  \begin{minted}[frame=single,fontsize=\small,framesep=10pt]{haskell}
fun
  lin hattu_N = mkHattu "hattu"
  lin katto_N = mkKatto "katto"
  lin čiutto_N = mkKatto "čiutto" ;
\end{minted}
\caption{Vadja GF leksikoni genereeritud programmkood.
  \label{code:gf-lexicon}}
\end{figure}


%\subsubsection{Arutelu}
%\label{sec:gf-arutelu}

%Loodud morfoloogiakomponenti on kasutatud interaktiivses vadja-vene-vadja vestmikus.




\FloatBarrier
\subsection{Integreerimine Giella-taristuga}
\label{sec:giella}
% pmst on see 
% cat root.lexc affixes/nouns.lexc stems/nouns.lexc > vot.lexc
% hfst-xfst
% read lexc /tmp/vot.lexc
% define Lexicon
% source paradigms.xfscript
% define Paradigms
% define Speller Lexicon ?+ .o. Paradigms
% 
% ! kusjuures ?+ katab/ühtib käänded
% NB! veel juurde on testide genereerimine!


% mis on Giellatekno infra
Moshagen ja teised (\citeyear[344]{moshagen_building_2013}) on kirjeldanud Giella taristut arendus\-keskkonnana, mis abistab morfoloogia\-kirjelduste ja arvuti\-leksikonide loomist ning muuta need automaatselt ümber paljudeks kasulikeks programmideks, mh õige\-kirja\-kontrollijaks (\textit{speller}), poolitajaks (\textit{hyphenator}) ja grammatika\-kontrollijaks (\textit{grammar checker}). 
% The Giellatekno infrastructure has been characterized in \cite{moshagen_building_2013} to be a \textit{development environment infrastructure} (as opposed to a resource infrastructure), offering a framework for building language-specific analysers and directly turn them into a wide range of useful programs.

Giella taristu on ainus nii suur ja laia\-ulatuslik arendus\-keskkond, mis on mõeldud väikeste keelte keele\-tehnoloogia arendamiseks (\cite[346]{moshagen_building_2013}). Taristus arendatavate keelte hulka kuulub palju lääne\-mere soome ja saami keeli, soome, Lule saami, põhjasaami, lõunasaami, kveeni, Inari saami, isuri, karjala, Kildini saami, liivi, livviko, Pite saami, kolta\-saami, Ume saami, vepsa, võru ja eesti keel  (\cite[349]{moshagen_building_2013}).

Taristu loojad on valinud tugineda reegli\-põhilisele keele\-mudelleerimisele, kuna väikeste keelte puhul on tihti puudu piisavalt suurtest teksti\-kogudest, mida eeldavad statistilised meetodid. Morfoloogia kirjeldamiseks kasutatakse lõplike muundurite formalismi (\textit{Finite State Transducer}), mida on eesti keeles nimetatud ka lõplike automaatide formalismiks (\textit{Finite State Automaton}). Kõrgema keele\-tasandi, süntaksi kirjeldamiseks kasutatakse kitsenduste grammatika formalismi (\textit{Constraint Grammar}). (\cite[346]{moshagen_building_2013})

Magistritöö eesmärk integreerida Giella taristuga ei ole olnud selleks, et seda kasutada arendus\-keskkonnana morfoloogia jaoks. Töö laiemaks eesmärgiks on olnud integreerida vormi\-sõnastik Giella taristu vadja morfoloogia\-kirjelduseks, et saaks alustada vadja süntaksi kirjeldamisega. Integreerimise kitsemaks eesmärgiks on olnud saada Giella taristust üht kasulikku programmi, vadja keele õige\-kirja\-kontrollijat. 
% From the point of view of our work on Votic morphology, the programs of interest are proofing tools and morphological analyzers.

Giella taristu sihtrühmaks on keeleteadlased ja arvuti\-lingvistid, kellel on oskused Unix'i süsteemi käsurida ja tarkvara kasutada (\cite[347]{moshagen_building_2013}). Magistri\-töö võimaldab taristule uue sihtrühma. Kuna magistri\-töö integreerib vadja vormi\-sõnastiku Giella taristusse ja taristu loob selle põhjal õige\-kirja\-kontrollija, on edas\-pidine töö õige\-kirja\-kontrollija kallal võimalik teha vormi\-sõnastiku kaudu. Vormi\-sõnastiku täiendamine ja muutmine ei vaja programmeerimis\-oskusi, mida eeldatakse Giella arendus\-keskkonna siht\-rühmalt. % Magistritöö vormi\-sõnastiku kasutaja

See võimaldab morfoloogia\-komponendi arendamist kõigi poolt, kes oskavad vormisõnastikku muuta ja täiendada. Keele\-teadlane ja arvuti\-lingvist võib arendada kõrgema keele\-tasandi ehk süntaksi mudelleerimist eraldi\-seisvalt morfoloogia arendamisest, mis peaks võimaldama suuremat koos\-töö\-võimet väikse keele kasutajas\-konnaga.
%Taristu toetab keelte normatiivset ja deskriptiivset kirjeldamist rööpselt. 

Kuigi Giella taristu toetab õige\-keelsuslikku informatsiooni, määrates selle abil morfoloogilisi vorme olema kas deskriptiivsed või normatiivsed, ei ole vastavat informatsiooni lisatud magistritöö vadja vormi\-sõnastikku. Seda on võimalik teha hiljem, kui vormi\-sõnastik sisaldab piisavalt suurt sõna\-vara. Õige\-keelsuslikku informatsiooni kasutatakse näiteks Giella taristu loodud õige\-kirja\-kontrollijas, mis pakub ainult normatiivile vastavaid vorme, ent ei määra valeks sellist sõna, mis on deskriptiivses kirjelduses olemas (\cite{moshagen_building_2013}). % vadja keele jaoks ei 

Vormisõnastiku integreerimiseks on vaja luua mitu komponenti Giella taristus: leksikon, tüüpsõnade paradigmade kirjeldused, automaatsete testide kirjeldused ning kooste\-programmi kirjeldus. 
% For the integration with Giellatekno's infrastructure, several components are needed: a lexicon, paradigm descriptions in FST, automatic test declarations, and a Makefile that binds all the components together.

% Leksikoni ja paradigmade kirjeldamiseks kasutatakse Giella taristus lõplike muundurite formalismi (\textit{Finite State Transducer}), mida on eesti keeles nimetatud ka lõplikeks automaatideks (\textit{Finite State Automaton}). 

% Keeletehnoloogilise taristuga Giella integreeritakse selles töös peamiselt selleks, et saada kätte õigekirja\-kontrollija. Giella-taristu koosneb veel võimalustest. Taristut kasutavad peamiselt Giellatekno ja Divvun.

% Integreerimine on jagatav kaheks peamiseks osaks: leksikoni integreerimeerimine ja tüüpsõnamallide integreerimine. Seejärel kirjeldatakse taristu poolt loodud õigekirjakontrollija tööpõhimõtet ja lõpetuseks on arutelu.


\subsubsection{Leksikoni integreerimine}
\label{sec:giella-leksikon}

% Failistruktuurist!

Giella taristus kirjeldatakse tavaliselt nii leksikoni kui ka morfoloogiliste paradigmade koostamis\-reegleid \textit{lexc} formalismiga.

Lexc (ingl \textit{Lexicon Compiler}, ee leksikoni\-koostaja või kompileerija) on eeskätt mõeldud lihtsustamaks leksikograafilisi ülesandeid, nagu leksikoni koostamist, kuigi sellega on võimalik kirjeldada ka morfotaktikat. 
% Ajalooliselt on see seotud Kimmo Koskenniemi kahe-tasandilise.
Alguse on lexc saanud Kimmo Koskenniemi kahetasandilisest morfoloogia\-kirjeldus\-viisist TwoL. 
Lexc on parem-rekursiivne fraasi-fraasi\-struktuuri\-grammatika (\textit{right-recursive phrase-structure grammar}), mis kompileerub kirjelduseks lõplike muundurite formalismis (\cites[]{karttunen_finite-state_1993}[203]{beesley_finite_2003}). 

Töös järgitakse üldiselt Giella taristu faili-hierarhiat ja jagatakse morfoloogiline kirjeldus samal viisil järgmisteks osadeks: \texttt{morphology/stems/nouns.lexc} kannab sõna\-loendit noomeni tüvedega; \texttt{morphology/affixes/nouns.lexc} hoiab noomenite jätkamis\-reegleid; \texttt{morphology/root.lexc} sisaldab kõiki vajalikke eel\-häälestusi ning \texttt{morphology/nouns-paradigms.xfscript} sisaldab noomenite tüüpsõna\-mallide koostamis\-reegleid. Kõik failid koostatakse automaatselt vormi\-sõnastiku andmete põhjal ja on laiendatav ka teistele sõna\-liikidele. Kuivõrd 

Magistritöö vormisõnastiku integreerimiseks on siiski ülevaatlikkuse huvides valitud eraldada tüüpsõna\-mallide kirjeldused leksikonit kirjeldavast formalismist lexc ja kasutada üldiseid lõplikke muundurite asendus\-reegleid (ingl \textit{fst replace rules}). See valik teeb leksikoni ülevaatlikumaks, kuna tüvede\-loendisse jääb ainult kaks välja, lekseemi lemma\-vorm ja tüüpsõna.

Koostatavat leksikoni tüvede loendit näitlikustatakse joonisel~\ref{code:giella-lexicon}. 
% ``Formally, the lexc language is a kind of right-recursive phrase-structure grammar.'' ja ``A lexc description compiles into a standard Xerox finite-state network, either a simple automaton or a transducer.'' (\cite[203]{beesley_finite_2003}).

\spacing{1}
\begin{figure}[H]
  \center
\begin{minted}[frame=single,fontsize=\small,framesep=10pt]{text}
LEXICON nouns
  hattu N_hattu ;
  katto N_hattu ;
  čiutto N_hattu ;
\end{minted}
\caption{Genereeritud programmkood, mis näitlikustab Giella taristu leksikoni struktuuri kolme lekseemiga (\textit{hattu}, \textit{katto} ja \textit{čiutto}), mis jagavad sama tüüp\-sõna (N\_hattu).
  \label{code:giella-lexicon}}
\end{figure}
\spacing{1.5}

Leksikonis kasutatavad tüüpsõnanimed (siin näites \textit{N\_hattu}) viitavad reeglitele, mille abil viiakse leksikoni kirjed kokku nende tüüp\-sõna\-funktsioonidega. Selleks lisatakse lemma\-vormile tüüpsõna\-lipiku. Reeglid on koostatud eraldi faili \texttt{morphology/}\linebreak\texttt{affixes.lexc}, mille struktuuri näitlikustatakse joonisel~\ref{code:giella-affixes}. Tüüpsõna\-lipiku põhjal toimivad tüüpsõna\-mallide reeglid kirjeldatatakse järgmises alaosas.

\spacing{1}
\begin{figure}[H]
  \center
\begin{minted}[frame=single,fontsize=\small,framesep=10pt]{text}
LEXICON N_hattu
  +Paradigm/hattu_N:0 # ;
\end{minted}
\caption{Genereeritud programmkood, mis näitlikustab kasutatud tüüpsõna\-lipikuid, mille abil viiakse leksikon ja tüüp\-sõna\-funktsioonid kokku.
  \label{code:giella-affixes}}
\end{figure}
\spacing{1.5}



% Instead of generating the lexicon straight into FST representation, we use Giellatekno's Sanat XML format. This is achieved by simple XML transformations from the LMF format.
% 
% The Giellatekno infrastructure uses the Sanat XML to generate its own source codes.
% 
% In its essence, the Sanat XML contains the lexical entry's lemma, the name of the paradigm (e.g a FST continuation class) and the technical stem needed by the continuation class.
% 
% What is crucially missing from our implementation is the Finnish translation equivalents used as interlingual pivots in the Giellatekno infrastructure.
% These equivalents will be generated when our work on the Votic morphological dictionary resource has reached the stage of adding them. 
%


\subsubsection{Tüüpsõnafunktsioonide integreerimine}
\label{sec:giella-tüüpsõnad}

Vormisõnastiku tüüpsõnamallide põhjal koostatakse sõnade muutvormide koostamis\-reeglid. Magistritöös on järgitud Forsbergi ja Huldeni (\citeyear{forsberg_learning_2016}) esitatud viisi. % TODO mis viisi?????
Tüüp\-sõna\-malli \textit{asHattuN}, mis järgib leksikonis kasutatud lipikut \textit{N\_hattu}, näitlikustatakse (osaliselt) joonisel~\ref{code:giella-paradigms}. Joonise alguses defineeritud \texttt{Alph} sisaldab vadja kirjakeele tähemärgid. Järgnevad definitsioonid \texttt{asHattuN=var1} ja \texttt{asHattuN=var2} märgivad tüüp\-sõna\-malli muutujaid ja koosnevad mitte-tühjadest tähe\-märkide jadadest (\textit{Alph+}).

Tüüp\-sõna\-malli definitsioni \texttt{asHattuN} esimesed kaks rida tähistavad noomeni muutvormi ainsuse nimetavas käändes, mida märgivad analüüsi lisatavad tähised \texttt{+N}, \texttt{+Sg} ja \texttt{+Nom}. Ainsuse nimetava käände tugeva\-astmelisust on näha asendus\-reeglist \texttt{[{t}:{t}]}, mis asendab tähemärgi \textit{t} sama tähemärgiga \textit{t} (koostatav muutvorm on nt \textit{hattu}).

Järgmised kaks rida tähistavad noomeni muutvormi ainsuse omastavas käändes, mida  märgivad analüüsi lisatavad tähised \texttt{+N}, \texttt{+Sg} ja \texttt{+Gen}. Ainsuse nimetava käände nõrga\-astmelisust on näha asendus\-reeglist \texttt{[0:{t}]}, mis kustutab tähemärgi \textit{t} muutvormist (nt muuvorm \textit{hatu}). Viimased kaks rida tähistavad järelikult noomeni muutvormi ainsuse osastavas käändes, mis on tugevas astmes ning mille lõppu on lisatud tähemärk \textit{a} (nt muutvorm \textit{hattua}).

\spacing{1}
\begin{figure}[H]
  \center
  \begin{minted}[frame=single,fontsize=\small,framesep=10pt]{text}
define Alph "a"|"b"|"c"|"d"|"d̕"|"e"|"f"|"f̕"|"g"|
    "h"|"h̕"|"i"|"j"|"k"|"l"|"l̕"|"m"|"n"|"n̕"|"o"|
    "p"|"p̕"|"r"|"r̕"|"s"|"s̕"|"š"|"t"|"t̕"|"č"|"u"|
    "v"|"v̕"|"y"|"z"|"z̕"|"ž"|"õ"|"ä"|"ö"|"ü";

define asHattuN=var1 Alph+;
define asHattuN=var2 Alph+;

define asHattuN [asHattuN=var1] [{t}:{t}] [asHattuN=var2]
           0:["+Paradigm/hattu_N" "+N" "+Sg" "+Nom"] |
       [asHattuN=var1] [0:{t}] [asHattuN=var2]
           0:["+Paradigm/hattu_N" "+N" "+Sg" "+Gen"] |
       [asHattuN=var1] [{t}:{t}] [asHattuN=var2] [{a}:0]
           0:["+Paradigm/hattu_N" "+N" "+Sg" "+Par"] ;
\end{minted}
\caption{Genereeritud programmkood, mis näitlikustab Giella taristu tüüpsõnamalli reeglite struktuur (piiratud käänete valikuga).
  \label{code:giella-paradigms}}
\end{figure}
\spacing{1.5}



\subsubsection{Automaatsete testide loomine}
\label{sec:giella-testid}

Giella taristu on mõeldud reeglipõhise morfoloogia kirjeldamiseks ja reeglite testimiseks on taristus võimalik luua testid õigete muutvormidega. Kui arendamisel üks test ei lähe läbi annab arendus\-keskkond sellest märku. (\cite[347]{moshagen_building_2013}) %  This becomes essentially a unit-test-like setup (Huizinga and Kolawa 2007): when you write a new inflectional lexicon for a new inflectional class, you can start by spelling out the wanted behaviour in the form of test cases. (347)

Magistritöö vormisõnastiku põhjal tuletatakse programm\-koodi ka testide jaoks. Kuna magistritöö koodi\-genereerimise eesmärk ei ole kasutada Giella taristut arendus\-keskkonnana vaid tuletab põhimõtteliselt õigeid reegleid annab testide mitte-läbimine märku tuletatud koodi adekvaatsuse kohta. Juhul kui testid nurjuvad, annab see märku et koodi\-tuletamisega on midagi valesti.

\subsubsection{Õigekirjakontrollija ehitamine}
\label{sec:giella-õigekirjakontrollija}

Giella arendus\-taristus on automatiseeritud õige\-kirja\-kontrollijate ehitamine morfo\-loogia\-kirjeldustest. Eelnevalt kirjeldatud vadja vormisõnastiku integreerimine Giella taristusse võimaldab vadja keele õige\-kirja\-kontrollija automaatse ehitamise. Vormisõnastiku integreerimise ja õigekirjakontrollija automaatse ehitamise detailides on autorit aidanud terve Giella meeskond, eriti Sjur Moshagen ja Jack Rueter.

Vadja õige\-kirja\-kontrollija töötab kontori\-tarkvaraga OpenOffice ja LibreOffice ning märgib punase joonega valeks kõik need sõnad, mis vormi\-sõnastikus ei sisaldu. Vadja keele õige\-kirja\-kontrollija on kätte\-saadaval internetis\footnote{\url{https://gtsvn.uit.no/langtech/trunk/langs/vot/tools/spellcheckers/}}. 


Automaatselt loodud õige\-kirja\-kontrollija on eesmärgi\-päraselt jäetud lihtsa-\linebreak koeliseks. See märgib kõik sõnad valeks, mis ei sisaldu vormi\-sõnastikus. See on lühi\-ajaliseks kasutamiseks ja mõeldud ärgitama kasutajaid ise pakkuma täiendusi ja sõnaloomet vadja sõnastikku. Ka lihtsate igapäevaste sõnade lisamine vadja vormi\-sõnastikku märgiks suurt edu\-sammu vadja keele taas\-elustamises.



%\newpage
%\section{Arutelu}

%Arutelu struktuur peaks järgima üks-ühele sissejuhatuses väljatoodut, ent sellele siis lisama arutelu (sissejuhatus ainult nentingud).

% ARUTELU POLE, SISSEKIRJUTATUD MUJALE NING KOKKUVÕTE ALL VÕIB MAINIDA MIS OOTUSED OLID JA MIS SAADUD RESULTAAT SELLEST ERINEB

\newpage
\section{Kokkuvõte}
\label{sec:kokkuvõte}

Magistritöös on loodud arvutisüsteem, mille abil on koostatud vadja noomeni morfoloogiline sõnastik ehk vormisõnastik sellisel kujul, et selles kätketud morfoloogiline informatsioon on automaatselt integreeritav kahte keeletehnoloogilisse taristusse (Grammatical Framework ja Giella taristu) ning samuti automaatselt teisendatav erinevateks keele\-tehnoloogilisteks rakendusteks (morfoloogiline analüsaator-süntesaator ja õige\-kirja\-kontrollija). Loodud süsteem ei ole vadja keele spetsiifiline, vaid töötab kõigi keelte jaoks, mille morfoloogia on kirjeldatav sõnade kõigi muutvormide loetlemise teel. % TODO Anette toimetada viimane lause

Koostatud vormisõnastik ei pürgi lõpliku vadja kirjakeele morfoloogia\-kirjelduse poole, vaid loodud süsteemi juures on peetud tähtsaks kasutaja\-sõbralikkust, et vormisõnastik oleks täiendatav ilma programmeerimis\-oskuseta --- %.
%
%Kasutajasõbraliku süsteemiga on autor tahtnud anda vadja keelele kala asemel õnge,
% TODO keeleplaneerimisele? keelekorraldamisele?
et edaspidine vadja keele arendamine võiks toimida teiste eest\-vedamisel, mitte jääda arvutus\-lingvisti või keele\-tehnoloogi taha toppama.

Loodud vadja keele vormisõnastik sisaldab 882~sõna, millest on automaatsel viisil leitud 231~erinevat tüüpsõna. Automaatselt teisendatud õigekirjakontrollija märgib nende sõnade kõiki 28~käändevormi õigeks kirjutatud arvuti kontoritarkvaras. Vormisõnastikku on lihtne muuta ja see ei vaja programmeerimis\-oskusi, muudatused tehakse otse sõna käändevorme muutes või vahetades sõna tüüpsõna. Käände\-vormide muutmisel võib süsteem tuvastada olemas\-oleva tüüpsõna või tekitada uue tüüpsõna. Muudatused viiakse vormisõnastikust automaatselt üle keele\-tehno\-loogilistesse taristutesse ning õige\-kirja\-kontrollijasse.


Magistritöö viies peatükis kirjeldati süsteemi kõiki osi ning vormi\-sõnastiku loomise vajalikke etappe. Peatükis~2 kirjeldati vormi\-sõnastiku sõnavara valikut, käänete valikut ja käände\-vormide moodustamise põhi\-mõtteid. Valikutes järgiti kahte levinud vadja õppe\-materjali.
%Sõnavara valik hõlmas Heinsoo (\citeyear{heinsoo_vadsonakopittoja_2015}) sõnastikus esitatud noomeneid (u~420 sõna), millele lisati Tsvetkovi sõnastikust (\cite{laakso_vatja_1989}) ettesattunud (u~460 sõna, mh~\vadja{asfal̕tti}, \vadja{bibli} ja \vadja{biblioteekkõ}). Kokku on vormi\-sõnastikku koondatud 882 sõna\-artiklit.
%
Valikute tegemine on alati seotud suurema kontekstiga, mistõttu anti ülevaade töö asetusest vadja korpuse planeerimise suhtes, s.o kirja\-viisi määramise, morfoloogia ühtlustamise ja sõnavara moderniseerimise suhtes. Töös kasutati Heinsoo kirjaviisi, mis on teisendatav teise, levinud õppematerjalide kirjaviisiks. Morfoloogiat ühtlustati sel määral, et töös järgiti Tsvetkovi sõnastikus antud Jõgõperä\-murdelisi põhivorme, mida püüti ühtlustada Vaipooli murdele iseloomulike joontega. Sõnavara ei moderniseeritud, kõik vormisõnastikku lisatud sõnad on varem avaldatud. Küll aga lisati rohkem kui poole võrra rohkem sõnu Tsvetkovi sõna\-raamatust.

Peatüki lõpus tutvustati magistritöös kasutatud morfeemi definitsiooni ja kirjeldati selle sarnasust klassikalise paradigmaatilise morfoloogiaga. Morfeemi ei käsitletud siin töös väikseima tähendusliku üksusena, vaid kui sõna mistahes fonoloogilist muutust, mis toob kaasa sõna tähenduse muutuse. Morfeemi definitsioon ühtib töös kasutatud ekstrakt\-morfoloogia metodoloogiaga.


Peatükis~3 tutvustati töös loodud süsteemi kasutaja\-liidest \enquote{Morfologi\-labbet} ehk Morfoloogia\-laborit. Morfoloogia\-labor on Språkbanken'is, Rootsi keele\-ressursside keskuses, loodud eraldiseisev veebi\-rakendus, mis põhineb ekstrakt\-morfoloogial (tutvustatud 4.~peatükis). Ekstrakt\-morfoloogia põhjal suudab morfoloogia\-labor moodustada suvalise sõna kõiki käändevorme selleks valitud olemas\-oleva tüüp\-sõna malli järgi. Uus ja süsteemist puuduv tüüpsõnamall defineeritakse sõna käände\-tabeli sisestamisega süsteemi.

Kasutaja\-liides on oluliselt lihtsustanud ja kiirendanud 882~sõna kõigi 28~käändevormi lisamist vormi\-sõnastikku. Sisestamise peamiseks töökäiguks oli kas valida uuele sisestatavale sõnale õige tüüpsõnamall või tüüpsõna puudumisel tuli sisestatava sõna käände\-tabel moodustada käsitsi. Käsitsi moodustatud ja sisestatud käändetabeli põhjal õppis süsteem uue tüüpsõna malli, mida järgmise sisestatava sõna juures sai tüüp\-sõna\-malliks valida.

Neljandas peatükis tutvustati ekstrakt\-morfoloogia meetodit lähemalt. Näitlikustati kuidas meetod eraldab (ekstraheerib) sõna \vadja{hattu} (ee müts) käände\-tabelist selle tüüp\-sõna\-malli ning kuidas tüüp\-sõna\-malli järgi saab moodustada teise sõna \vadja{čiutto} (ee särk) kõik käände\-vormid. % Tüüpsõna\-mall koostatakse kahes etapis, esimese puhul eraldatakse nn  Seejärel kuidas
%
Käändetabelite kaudu tüüpsõnamallide defineerimine (reeglite programmeerimise asemel) moodustab töö põhimõttelise tuumiku. % TODO XXX järg

Viiendas peatükis esitati kõik vormisõnastikust eraldatud 231~tüüpsõna\-malli ja nende alla kuuluvaid sõnu. Ülevaatlikuma esituse huvides jagati tüüpsõnad Ariste (\citeyear{ariste_grammar_1968}) grammatika käänd\-kondade süsteemi järgi. % Magistri\-töö eesmärk on olnud rakenduslik ja kirjeldav, mistõttu esitatud tüüpsõnade käände\-tabelid ei pürgi lõplikule morfoloogia\-kirjeldusele. %Esitatud tüüpsõnade tabelite parandamine on      % edaspidine töö käib nende põhjal
Võrdluses Viitso ja Ernštreiti liivi keele sõnastikus (\citeyear{viitso_livoki-estiki-tkisonarontoz_2012}) loetletud 242 tüüpsõnaga, tundub saavutatud vadja tüüpsõnade arv mõistlik.

Peatükis~6 tutvustati magistritöö tuumosa, programm\-koodi tuletamist. Programm\-koodi tuletamine toodi esile protsessina, kus vormisõnastikku kätketud teadmist, s.o tüüpsõna\-malle, kasutatakse keskse kirjeldusena, mille põhjal genereeritakse automaatselt programm\-koodi. Kirjeldati kuidas vormi\-sõnastiku sõna\-vara ja tüüp\-sõna\-malle kodeeriti keskse kirjelduse vormingusse, milles lähtuti rahvus\-vahelisest standardist \textit{Lexical Markup Framework} (LMF). Seejärel esitati magistritöö kaks programm\-koodi\-tuletajat ehk koodi\-generaatorit. Mõlemad koodi\-generaatorid olid varustatud tõlke\-tabelitega, mistõttu  oli võimalik toetada kõigi taristute, nii standardi LMFi kui ka siht-taristute terminoloogiaid ja traditsioone. %, nii standardi LMFi kui ka siht-taristute omi.
% lemmavabadus

Programmkoodi generaatorid põhinesid täielikult LMFi kirjeldusele ja ei olnud vadja keele spetsiifilised. Koodi\-generaatorid toetasid ka lemma\-vormi jaoks kasutatud käände valimist.

Esimene koodi\-generaator integreeris vadja vormi\-sõnastiku eri\-ots\-tarbelisesse grammatikate programmeerimis\-keelde \textit{Grammatical Framework} (GF). Selleks genereeriti programm\-koodi GFi vadja morfoloogia- ja leksikoni\-moodulite jaoks. Moodulid on saadaval internetis\footnote{\url{https://github.com/keeleleek/GF-Votic}}.

Selline vormisõnastiku integreerimis\-viis võimaldab tööde jaotust, kus vadja sõnadest ja sõna\-vormidest teadlikul kasutajal on võimalik täiendada vormisõnas\-tikku ning arvuti\-lingvist saab keskenduda GFi rakenduste loomisele. Seni on katsetatud vadja--vene turismi\-vestmiku koostamist, mis on internetis saadaval\footnote{\url{https://github.com/keeleleek/GF-Votic-Phrasebook}}.

Teine koodi\-generaator integreeris vormi\-sõnastiku Giella taristusse. Giella taristu on seni suurim arendus\-keskkond väikeste keelte keele\-tehnoloogia arendamise jaoks. Taristus arendatavate keelte hulgas on 17~lääne\-mere soome ja saami keelt, mille juurde 18. keeleks nüüd vadja keel.

Giella taristu morfoloogia\-kirjeldusi koostatakse reegli\-põhise formalismiga ja taristu siht\-rühmaks on olnud keeleteadlased ja arvuti\-lingvistid. Magistritöös koostatud programm\-koodi\-generaator loob vajaliku kirjelduse vormi\-sõnastiku tüüpsõnamallide põhjal.
Seega on vormi\-sõnastiku integreerimine Giella taristuga loonud taristule uut\-moodi kasutaja\-liidese, mis võimaldab uuel sihtrühmal töötada taristus. Uuelt sihtrühmalt ei ole eeldatud programmeerimis\-oskusi, ainult sõnade ja sõna\-vormide tundmist.


Vadja keele integreerimine Giella taristuga võimaldas kasutada taristu enda automaatikat ja luua vadja keelele õige\-kirja\-kontrollija LibreOffice'i ja OpenOffice'i kontori\-tark\-varale. Esialgu on õige\-kirja\-kontrollija kärbitud lihtsa\-koeliseks ja see märgib õigeteks ainult vormi\-sõnastikus sisalduvaid sõnu ja sõnavorme, kõik üle\-jäänud kontori\-tarkvarasse sisestatud tekst märgitakse punase joonega valeks.

Selline valik tehti innustamaks vadja keele kasutajaid ise pakkuma täiendusi ja sõna\-loomet vormisõnastikku. Ka lihtsate igapäevaste sõnade lisamine vormi\-sõnastikku märgiks suurt edu\-sammu vadja keele taas\-elustamises. Õigekirja\-kontrollija on kättesaadaval internetis\footnote{\url{https://gtsvn.uit.no/langtech/trunk/langs/vot/tools/spellcheckers/}}.

%Vormisõnastiku integreerimine Giella taristuga on loonud uue kasutajaliidese taristule, mis võimaldab uuel sihtrühmal töötada taristus. Uuelt sihtrühmalt ei ole eeldatud programmeerimis\-oskusi.



%
%
%---
%
%% töö tuumik on tüüpsõnakirjeldused
%Töö keskseks osaks on ekstraktmorfoloogiameetodiga saadud tüüpsõnakirjeldused.
%% kirjeldused LMFi ja neist genereeritakse kood
%Kirjeldused kodeeritakse koos sõnastikuga ümber standardsesse vormingusse ja saadud leksikaalse ressursi järgi tuletatakse automaatselt programmkoodi kahe keeletehnoloogilise platvormi jaoks, ja tagatakse seega vadja keele tugi nendes platvormides.
%
%% seetõttu töötab ekstraktmorfoloogia liidesena
%Niivisii kasutatakse ekstraktmorfoloogia meetodit kasutaja\-liidesena, mille abil koostatakse arvutimorfoloogia ainult tüüpsõnade muutvormitabeleid sedastades -- mitte programmeerides.
%
%% kirjeldus kesksel kohal, parandused õigesse kohta
%Magistritöös esitatud töövoog paneb leksikaalse ressursi kesksele kohale ja tuletatud tehnoloogia sellest teiseseks. Uue sõnavara ja vigade parandused tehakse ressursis, mitte mitmes tehnoloogias eraldi.
%
%% standardi kasutamine tagab pikaajalise loetavus
%Kuna nii tüüpsõnade kirjeldused, kui ka ülejäänud sõnastik kodeeritakse rahvusvahelise standardi Lexical Markup Framework vormingusse, tagatakse võimaluse ressursi pikaajaliseks arhiveerimiseks. Leksikaalne ressurss on loetav ja arusaadav palju kauem, kui seda on programmeerimiskood.
%
%% panus dokumentaalsele lingvistikale
%Viimase tõttu püüab magistritöö ühendada arvutuslingvistika ja dokumenteeriva lingvistika valdkondi.
%






%\newpage
% TODO VAATA SEE ÜLE KAS ÜLDSE VAJALIKKI?
%\section{Põhimõisted ja lühendid}
%\label{sec:põhimõisted}
%Siin loetletakse töös kasutatud mõisted ja lühendid koos nende tähendustega.

%\vspace{-4em}
%\spacing{1}
%\glsaddall
%\small{
%  \printglossary[title={},toctitle={}]
%}
%\spacing{1.5}








\newpage
\section{Kirjandus}
\label{sec:kirjandus}
\spacing{1}
\printbibliography[heading=none]
\spacing{1.5}







\newpage
\selectlanguage{english}
\section{Resümee \textit{The use of Extract Morphology for Automatic generation of Language Technology for Votic}}
\subsection*{Summary}


In this master's thesis the author has built a morphological dictionary building tool that requires no programming knowledge from the user, the only knowledge needed is to know the wordforms of a word. The tool extracts morphological patterns from the morphological dictionary and integrates this morphological knowledge into two language technological platforms, the programming language Grammatical Framework (GF) and the Giella infrastructure.

The target language and language community considered in this work has been Votic, a Finnic language with rich morphology. Ethnologue states it as nearly extinct, with a population size of 25 persons in 2007. Nevertheless, a language revitalization program \enquote{Ämmesse Vunukkassaa} has been organized by Heinike Heinsoo every summer since 2009. % No written standard has been developed for Votic.

Integration into the selected platforms enables to get out useful primary tools from them and deliver them to the language community. The tools sought for in this work are a morphological analyzer/synthesizer and a spell checker.

The ambition of the presented tool is to give the small language community the fishing rod instead of just the technological fish. It is seen vital by the author that work on the morphological dictionary can continue also when no computational linguist is present.

The presented tool enables users to add, remove and edit words in the morphological dictionary. This enables them also to change what is included in the automatically generated language technological tools.

Further, the tool allows division of labor. Work on morphology can be done by non-programmers and other applications employing morphological analysis/synthesis can be done by programmers.

In the presented Master's thesis a stand-alone web application \enquote{Morfologilabbet} (developed by Språkbanken, the Swedish Language Bank) has been used as the user interface for inserting 882~nouns into the Votic morphological dictionary. The tool has sped the process of inserting all 24 Votic noun cases' wordforms of all these 882~words. The work flow has mainly been:
\begin{enumerate}[noitemsep]
\item insert any wordform of the word to be added
\item select the appropriate existing paradigm by inspecting inflection tables proposed and ordered by the tool
\item if no appropriate paradigm exists in the system, the full inflection table has been inserted manually
\end{enumerate}

When full inflection tables have been inserted, the Morfologi\-labbet tool uses Extract Morphology's paradigm extraction to create a new paradigm pattern. The created paradigm pattern can be selected as paradigm for next words that are added.

Full inflection tables have been constructed following Tsvetkov's Votic dialect Jõgõperä dictionary, where the main inflection forms are presented. These inflection forms have been altered to correspond more to the Vaipooli dialect, which is the basis for the Votic written standard.

All together 231~paradigm patterns have been found using Extract Morphology's paradigm extraction. This number has been achieved by an iterative process of classifying the extracted paradigm patterns into declension classes following the system presented in Ariste's Votic grammar. When close similarities between paradigm patterns in the same declension has been detected, one paradigm pattern has been chosen to be correct and the other pattern's all wordforms have been changed according to it. For comparison to another Finnic language, a Livonian dictionary reports 242~different paradigms for the Livonian language.

Integration of the Votic morphological dictionary into the two platforms has been accomplished with source code generators. Source code generation is a loose term for any process of generating source code from a central description. The central description describes a \textit{what}-kind of knowledge, whereas the generators transform this knowledge into a runnable program implementation, a sort of \textit{how}-kind of knowledge.

The ISO standard Lexical Markup Framework (LMF) has been used for the central description and encodes all the information recorded by the Morfologi\-labbet tool. The LMF representation of an example lexical entry is shown in figure~\ref{code:lmf-lexicalentry} (p. \pageref{code:lmf-lexicalentry}) and that of an extracted paradigm pattern is shown in figure~\ref{code:lmf-paradigmpattern} (p. \pageref{code:lmf-paradigmpattern}).

Two separate source code generators have been developed. 
A shared architectural feature of both code generators is the use of translation tables for translating terms used in the LMF ontology to corresponding terms used in the host environment. Both terminology for grammatical features and name schemes for paradigm patterns are translated. In this way different terminological traditions have been supported and respected.

The two source code generators have been written in a language independent way, which rely only on information existing in the LMF (such as language name and code). The wordform used as dictionary lemma is also factored out and is configured by specifying its grammatical features.

For integrating the Votic morphological dictionary into the programming language Grammatical Framework, source code generators have been written that generates source code for the Votic lexicon and morphology modules of the GF Resource Grammar Library. The generated source code for the lexicon is shown in figure~\ref{code:gf-lexicon} (p.~\pageref{code:gf-lexicon}) and for the paradigm patterns in figure~\ref{code:gf-morfoloogia} (p.~\ref{code:gf-morfoloogia}). These modules have been made available\footnote{\url{https://github.com/keeleleek/GF-Votic}} and are automatically updated by changes in the dictionary. Based on these modules some initial work has been done on a Votic--Russian GF phrasebook application\footnote{\url{https://github.com/keeleleek/GF-Votic-Phrasebook}}. This initial work has proven the tool developed in this thesis to be useful for quickly adding vocabulary and/or changing incorrect morphological forms.


Integration with the Giella infrastructure has been done by generating source code for a Finite State Transducer (FST). The generated FST consists of three parts. A character listing part that defines all the Unicode characters used in the morphological dictionary is generated automatically based on the dictionary's content. The generated FST code for the Votic lexicon is shown in figure~\ref{code:giella-lexicon} (p.~\pageref{code:giella-lexicon}).

A design decision has been made to use pure FST calculus for the paradigm pattern definitons (something not usually done in the Giella tradition, but encouraged in XFST traditions). Therefore the continuation rules shown in figure~\ref{code:giella-affixes} (p.~\pageref{code:giella-affixes}) are rather sparse. They are simply used to connect the lexicon with the paradigm patterns, shown in figure~\ref{code:giella-paradigms} (p.~\pageref{code:giella-paradigms}).

If the integration towards GF has shown to be fruitful for Votic GF application development, then the integration with the Giella infrastructure has had more direct results for the Votic language community. The Giella infrastructure builds a spell checker for Votic. The spell checker works with LibreOffice and OpenOffice and has been configured in such a way that it marks as incorrect all the words that are not contained in the Votic morphological dictionary. This configuration has been chosen for motivating the Votic language community to contribute to the dictionary and to stimulate new word formation.

The integration worked out in this master's thesis also paves the way to enable a new type of user for the Giella infrastructure. The infrastructure has been characterized by Moshagen (\citeyear{moshagen_building_2013}) as a development environment infrastructure which target audience is linguists and computational linguists that have basic familiarity with Unix systems. The work presented here adds the possibility for the common people who have no Unix systems to work on their language's morphology.




% litsents
% ujuvad joonised ja tabelid _enne_ litsentsilehte
\clearpage
% litsentsileht
\includepdf[pages={1}]{40_lihtlitsents_loputoo_elektroonseks_avaldamiseks_tavaline_002_.pdf}



%\newpage
%\section{Lisad}
%
%Siin on esitatud kõik ekstraheeritud tüüpsõnamallide tabelid.
%
%\input{lmf-paradigms}



\end{document}

% Local Variables:
% TeX-engine: xelatex
% End:
